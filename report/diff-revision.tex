%DIF 1-13c1
%DIF LATEXDIFF DIFFERENCE FILE
%DIF DEL submitted/report/import-benefits.tex   Thu Nov 14 15:40:35 2024
%DIF ADD report/import-benefits.tex             Thu Nov 14 13:09:35 2024
%DIF < \documentclass[5p,9pt]{elsarticle}
%DIF < 
%DIF < \journal{Nature Energy}
%DIF < 
%DIF < % Length – up to 3,000 words, excluding abstract, Methods, references and figure legends.
%DIF < % Abstract – up to 150 words, unreferenced.
%DIF < % Display items – up to 8 items (figures and/or tables).
%DIF < % Article should be divided as follows:
%DIF < %     Introduction (without heading)
%DIF < %     Main text
%DIF < %     Discussion/Conclusions
%DIF < %     Methods. ​
%DIF < % Main text and Methods should be divided by topical subheadings; the Discussion/Conclusions does not contain subheadings.
%DIF -------
\documentclass[1p,11pt]{elsarticle} %DIF > 
%DIF -------

%DIF 15c3
%DIF < % https://twitter.com/mwclimatesci/status/1111773593656918016?s=46&t=p6Kals2l9yy3e5W8zExYbQ
%DIF -------
\journal{Nature Communications} %DIF > 
%DIF -------

% \usepackage{natbib}
%DIF 18c6
%DIF < \bibliographystyle{elsarticle-num}
%DIF -------
\bibliographystyle{naturemag} %DIF > 
%DIF -------
\biboptions{numbers,sort&compress,super}

% format hacks
\usepackage{libertine}
\usepackage{libertinust1math}
\usepackage{geometry}
\geometry{
%DIF 26c14
%DIF <     top=10mm,
%DIF -------
    top=12mm,  %DIF > 
%DIF -------
    bottom=15mm,
%DIF 28-29c16-27
%DIF < 	left=10mm,
%DIF < 	right=10mm,
%DIF -------
	% left=10mm, %DIF > 
	% right=10mm, %DIF > 
} %DIF > 
 %DIF > 
\usepackage{afterpage} %DIF > 
\usepackage{changepage} %DIF > 
 %DIF > 
% Command for full width figures %DIF > 
\newcommand{\fullwidthfigure}[1]{% %DIF > 
  \begin{adjustwidth}{-\dimexpr\marginparwidth+\marginparsep\relax}{-\dimexpr\marginparwidth+\marginparsep\relax} %DIF > 
    #1 %DIF > 
  \end{adjustwidth} %DIF > 
%DIF -------
}

% \usepackage[final]{microtype}
\usepackage{amsmath}
\usepackage{bbold}
\usepackage{graphicx}
\usepackage{eurosym}
\usepackage{mathtools}
\usepackage{url}
\usepackage{booktabs}
\usepackage{epstopdf}
\usepackage{xfrac}
\usepackage{tabularx}
\usepackage{bm}
\usepackage{subcaption}
\usepackage{longtable}
\usepackage{multirow}
\usepackage{threeparttable}
\usepackage[export]{adjustbox}
\usepackage[version=4]{mhchem}
\usepackage[colorlinks]{hyperref}
% \usepackage[parfill]{parskip}
\usepackage[nameinlink,sort&compress,capitalise]{cleveref}
\usepackage[leftcaption,raggedright]{sidecap}
\usepackage[prependcaption,textsize=footnotesize]{todonotes}
\usepackage{blindtext}
% \usepackage{pdflscape}

%DIF 58a56-59
% \usepackage[pagewise]{lineno} %DIF > 
% \usepackage{lineno} %DIF > 
% \linenumbers %DIF > 
 %DIF > 
%DIF -------
\usepackage{xr}
\externaldocument{import-benefits-si}

\urlstyle{sf}

% Adjust the hyphenation penalty
\hyphenpenalty=1000

% Adjust the line breaking tolerance
\tolerance=5000

\renewcommand{\thefootnote}{\Alph{footnote}}

\usepackage{lipsum}

%DIF 73-77c75-79
%DIF < \graphicspath{
%DIF <     {../workflow/notebooks/},
%DIF <     {../workflow/pypsa-eur/resources/},
%DIF <     {../workflow/pypsa-eur/results/},
%DIF < }
%DIF -------
% \graphicspath{ %DIF > 
%     {../workflow/notebooks/}, %DIF > 
%     {../workflow/pypsa-eur/resources/}, %DIF > 
%     {../workflow/pypsa-eur/results/}, %DIF > 
% } %DIF > 
%DIF -------

% \usepackage[
% 	type={CC},
% 	modifier={by},
% 	version={4.0},
% ]{doclicense}

\usepackage{siunitx}
\sisetup{range-units=single, per-mode=symbol}
\DeclareSIUnit\year{a}
\DeclareSIUnit{\tco}{t_{\ce{CO2}}}
\DeclareSIUnit{\sieuro}{\mbox{\euro}}
\DeclareSIUnit{\twh}{{\tera\watt\hour}}
\DeclareSIUnit{\mwh}{{\mega\watt\hour}}
\DeclareSIUnit{\kwh}{{\kilo\watt\hour}}
\newcommand{\co}{\ce{CO2}~}
\def\el{${}_{\textrm{el}}$}
\def\th{${}_{\textrm{th}}$}

%DIF 97c99
%DIF < \newcommand{\bneuro}[1]{#1\,bn\euro{}/a}
%DIF -------
\newcommand{\bneuro}[1]{#1\,bn\euro{}$_\text{2020}$~a$^{-1}$} %DIF > 
%DIF -------

\renewcommand{\ttdefault}{\sfdefault}

\newcommand{\sfigref}[1]{%
    \begingroup%
    \crefname{figure}{Supplementary Fig.}{Supplementary Figs.}%
    \cref{#1}%
    \endgroup%
}

\newcommand{\Sfigref}[1]{%
    \begingroup%
    \Crefname{figure}{Supplementary Fig.}{Supplementary Figs.}%
    \Cref{#1}%
    \endgroup%
}

\newcommand{\stabref}[1]{%
    \begingroup%
    \crefname{table}{Supplementary Table}{Supplementary Tables}%
    \cref{#1}%
    \endgroup%
}

\newcommand{\Stabref}[1]{%
    \begingroup%
    \Crefname{table}{Supplementary Table}{Supplementary Tables}%
    \Cref{#1}%
    \endgroup%
}
%DIF PREAMBLE EXTENSION ADDED BY LATEXDIFF
%DIF UNDERLINE PREAMBLE %DIF PREAMBLE
\RequirePackage[normalem]{ulem} %DIF PREAMBLE
\RequirePackage{color}\definecolor{RED}{rgb}{1,0,0}\definecolor{BLUE}{rgb}{0,0,1} %DIF PREAMBLE
\providecommand{\DIFaddtex}[1]{{\protect\color{blue}\uwave{#1}}} %DIF PREAMBLE
\providecommand{\DIFdeltex}[1]{{\protect\color{red}\sout{#1}}}                      %DIF PREAMBLE
%DIF SAFE PREAMBLE %DIF PREAMBLE
\providecommand{\DIFaddbegin}{} %DIF PREAMBLE
\providecommand{\DIFaddend}{} %DIF PREAMBLE
\providecommand{\DIFdelbegin}{} %DIF PREAMBLE
\providecommand{\DIFdelend}{} %DIF PREAMBLE
\providecommand{\DIFmodbegin}{} %DIF PREAMBLE
\providecommand{\DIFmodend}{} %DIF PREAMBLE
%DIF FLOATSAFE PREAMBLE %DIF PREAMBLE
\providecommand{\DIFaddFL}[1]{\DIFadd{#1}} %DIF PREAMBLE
\providecommand{\DIFdelFL}[1]{\DIFdel{#1}} %DIF PREAMBLE
\providecommand{\DIFaddbeginFL}{} %DIF PREAMBLE
\providecommand{\DIFaddendFL}{} %DIF PREAMBLE
\providecommand{\DIFdelbeginFL}{} %DIF PREAMBLE
\providecommand{\DIFdelendFL}{} %DIF PREAMBLE
%DIF HYPERREF PREAMBLE %DIF PREAMBLE
\providecommand{\DIFadd}[1]{\texorpdfstring{\DIFaddtex{#1}}{#1}} %DIF PREAMBLE
\providecommand{\DIFdel}[1]{\texorpdfstring{\DIFdeltex{#1}}{}} %DIF PREAMBLE
\newcommand{\DIFscaledelfig}{0.5}
%DIF HIGHLIGHTGRAPHICS PREAMBLE %DIF PREAMBLE
\RequirePackage{settobox} %DIF PREAMBLE
\RequirePackage{letltxmacro} %DIF PREAMBLE
\newsavebox{\DIFdelgraphicsbox} %DIF PREAMBLE
\newlength{\DIFdelgraphicswidth} %DIF PREAMBLE
\newlength{\DIFdelgraphicsheight} %DIF PREAMBLE
% store original definition of \includegraphics %DIF PREAMBLE
\LetLtxMacro{\DIFOincludegraphics}{\includegraphics} %DIF PREAMBLE
\newcommand{\DIFaddincludegraphics}[2][]{{\color{blue}\fbox{\DIFOincludegraphics[#1]{#2}}}} %DIF PREAMBLE
\newcommand{\DIFdelincludegraphics}[2][]{% %DIF PREAMBLE
\sbox{\DIFdelgraphicsbox}{\DIFOincludegraphics[#1]{#2}}% %DIF PREAMBLE
\settoboxwidth{\DIFdelgraphicswidth}{\DIFdelgraphicsbox} %DIF PREAMBLE
\settoboxtotalheight{\DIFdelgraphicsheight}{\DIFdelgraphicsbox} %DIF PREAMBLE
\scalebox{\DIFscaledelfig}{% %DIF PREAMBLE
\parbox[b]{\DIFdelgraphicswidth}{\usebox{\DIFdelgraphicsbox}\\[-\baselineskip] \rule{\DIFdelgraphicswidth}{0em}}\llap{\resizebox{\DIFdelgraphicswidth}{\DIFdelgraphicsheight}{% %DIF PREAMBLE
\setlength{\unitlength}{\DIFdelgraphicswidth}% %DIF PREAMBLE
\begin{picture}(1,1)% %DIF PREAMBLE
\thicklines\linethickness{2pt} %DIF PREAMBLE
{\color[rgb]{1,0,0}\put(0,0){\framebox(1,1){}}}% %DIF PREAMBLE
{\color[rgb]{1,0,0}\put(0,0){\line( 1,1){1}}}% %DIF PREAMBLE
{\color[rgb]{1,0,0}\put(0,1){\line(1,-1){1}}}% %DIF PREAMBLE
\end{picture}% %DIF PREAMBLE
}\hspace*{3pt}}} %DIF PREAMBLE
} %DIF PREAMBLE
\LetLtxMacro{\DIFOaddbegin}{\DIFaddbegin} %DIF PREAMBLE
\LetLtxMacro{\DIFOaddend}{\DIFaddend} %DIF PREAMBLE
\LetLtxMacro{\DIFOdelbegin}{\DIFdelbegin} %DIF PREAMBLE
\LetLtxMacro{\DIFOdelend}{\DIFdelend} %DIF PREAMBLE
\DeclareRobustCommand{\DIFaddbegin}{\DIFOaddbegin \let\includegraphics\DIFaddincludegraphics} %DIF PREAMBLE
\DeclareRobustCommand{\DIFaddend}{\DIFOaddend \let\includegraphics\DIFOincludegraphics} %DIF PREAMBLE
\DeclareRobustCommand{\DIFdelbegin}{\DIFOdelbegin \let\includegraphics\DIFdelincludegraphics} %DIF PREAMBLE
\DeclareRobustCommand{\DIFdelend}{\DIFOaddend \let\includegraphics\DIFOincludegraphics} %DIF PREAMBLE
\LetLtxMacro{\DIFOaddbeginFL}{\DIFaddbeginFL} %DIF PREAMBLE
\LetLtxMacro{\DIFOaddendFL}{\DIFaddendFL} %DIF PREAMBLE
\LetLtxMacro{\DIFOdelbeginFL}{\DIFdelbeginFL} %DIF PREAMBLE
\LetLtxMacro{\DIFOdelendFL}{\DIFdelendFL} %DIF PREAMBLE
\DeclareRobustCommand{\DIFaddbeginFL}{\DIFOaddbeginFL \let\includegraphics\DIFaddincludegraphics} %DIF PREAMBLE
\DeclareRobustCommand{\DIFaddendFL}{\DIFOaddendFL \let\includegraphics\DIFOincludegraphics} %DIF PREAMBLE
\DeclareRobustCommand{\DIFdelbeginFL}{\DIFOdelbeginFL \let\includegraphics\DIFdelincludegraphics} %DIF PREAMBLE
\DeclareRobustCommand{\DIFdelendFL}{\DIFOaddendFL \let\includegraphics\DIFOincludegraphics} %DIF PREAMBLE
%DIF COLORLISTINGS PREAMBLE %DIF PREAMBLE
\RequirePackage{listings} %DIF PREAMBLE
\RequirePackage{color} %DIF PREAMBLE
\lstdefinelanguage{DIFcode}{ %DIF PREAMBLE
%DIF DIFCODE_UNDERLINE %DIF PREAMBLE
  moredelim=[il][\color{red}\sout]{\%DIF\ <\ }, %DIF PREAMBLE
  moredelim=[il][\color{blue}\uwave]{\%DIF\ >\ } %DIF PREAMBLE
} %DIF PREAMBLE
\lstdefinestyle{DIFverbatimstyle}{ %DIF PREAMBLE
	language=DIFcode, %DIF PREAMBLE
	basicstyle=\ttfamily, %DIF PREAMBLE
	columns=fullflexible, %DIF PREAMBLE
	keepspaces=true %DIF PREAMBLE
} %DIF PREAMBLE
\lstnewenvironment{DIFverbatim}{\lstset{style=DIFverbatimstyle}}{} %DIF PREAMBLE
\lstnewenvironment{DIFverbatim*}{\lstset{style=DIFverbatimstyle,showspaces=true}}{} %DIF PREAMBLE
%DIF END PREAMBLE EXTENSION ADDED BY LATEXDIFF

\begin{document}


\begin{frontmatter}

	\title{Energy Imports and Infrastructure in a\DIFaddbegin \\\DIFaddend Carbon-Neutral European Energy System}

	\author[tub]{Fabian Neumann\corref{correspondingauthor}}
	\ead{f.neumann@tu-berlin.de}
	\author[pik]{Johannes Hampp}
	\author[tub]{Tom Brown}

	\address[tub]{Department of Digital Transformation in Energy Systems, Institute of Energy Technology,\\Technische Universität Berlin, Fakultät III, Einsteinufer 25 (TA 8), 10587 Berlin, Germany}
	\address[pik]{Potsdam Institute for Climate Impact Research (PIK), Member of the Leibniz Association, P.O.~Box 60 12 03, 14412 Potsdam, Germany}

	\begin{abstract}
		% 150 words
% 5 things a good abstract needs:

% 1. Introduce the topic,
Importing renewable energy to Europe \DIFdelbegin \DIFdel{offers }\DIFdelend \DIFaddbegin \DIFadd{may offer }\DIFaddend many potential benefits,
including reduced energy costs, lower pressure on infrastructure development,
and less land-use within Europe.
% 2. State the unknown,
However, there remain many open questions: on the achievable cost reductions,
how much should be imported, whether the energy vector should be electricity,
hydrogen or hydrogen derivatives like ammonia or steel, and their impact on
Europe's domestic energy infrastructure needs.
% 3. Outline the method used to answer the question,
This study integrates the TRACE global energy supply chain model with the
sector-coupled energy system model for Europe\DIFaddbegin \DIFadd{, }\DIFaddend PyPSA-Eur\DIFdelbegin \DIFdel{to explore }\DIFdelend \DIFaddbegin \DIFadd{, to explore net-zero
emission }\DIFaddend scenarios with varying import volumes, costs, and vectors.
% 4. Preview the findings, and
We find system cost reductions of \DIFdelbegin \DIFdel{1-14\%, depending on assumed import costs}\DIFdelend \DIFaddbegin \DIFadd{1-10\%, within import cost variations of
$\pm20\%$ around our central estimate}\DIFaddend , with diminishing returns for larger
import volumes and a preference for methanol, steel and hydrogen imports.
Keeping some domestic power-to-X production is beneficial for integrating
variable renewables, \DIFdelbegin \DIFdel{utilising }\DIFdelend \DIFaddbegin \DIFadd{leveraging local sustainable carbon sources and utilising
some }\DIFaddend waste heat from fuel synthesis\DIFdelbegin \DIFdel{and leveraging local sustainable carbon sources}\DIFdelend \DIFaddbegin \DIFadd{. Across scenarios, power grid reinforcements
are more stable than hydrogen pipeline expansion}\DIFaddend .
% 5. Tell us what your work teaches us.
Our findings highlight the need for coordinating import strategies with
infrastructure policy and reveal maneuvering space for incorporating non-cost
decision factors.

	\end{abstract}

	% \begin{keyword}
	% 	TODO
	% \end{keyword}

	% \begin{graphicalabstract}
	% \end{graphicalabstract}

	% \begin{highlights}
	% 	\item A
	% 	\item B
	% 	\item C
	% \end{highlights}

\end{frontmatter}

% \listoftodos[TODOs]

% \tableofcontents

% \section*{Introduction}
% \label{sec:intro}

%%% promise of imports %%%

Importing renewable energy to Europe \DIFdelbegin \DIFdel{promises }\DIFdelend \DIFaddbegin \DIFadd{may offer }\DIFaddend several advantages for achieving
a swift energy transition. It \DIFdelbegin \DIFdel{could }\DIFdelend \DIFaddbegin \DIFadd{might }\DIFaddend lower costs, help circumvent the slow
domestic deployment of renewable energy infrastructure and reduce pressure on
land usage in Europe. Many parts of the world have cheap and abundant renewable
energy supply potentials that they could offer to existing or emerging global
energy markets.\DIFdelbegin %DIFDELCMD < \cite{irenaGlobalHydrogen2022,luxSupplyCurves2021,vanderzwaanTimmermansDream2021,fasihiLongTermHydrocarbon2017,reichenbergDeepDecarbonization2022,galvanExportingSunshine2022,armijoFlexibleProduction2020,pfennigGlobalGISbasedPotential2023}
%DIFDELCMD < %%%
\DIFdelend \DIFaddbegin \cite{irenaGlobalHydrogenTrade2022, luxSupplyCurves2021,
vanderzwaanTimmermansDream2021, fasihiLongTermHydrocarbon2017,
reichenbergDeepDecarbonization2022, galvanExportingSunshine2022,
armijoFlexibleProduction2020, pfennigGlobalGISbasedPotential2023} \DIFaddend Partnering
with these regions could help Europe reach its carbon neutrality goals while
stimulating economic development in exporting \DIFdelbegin \DIFdel{countries}\DIFdelend \DIFaddbegin \DIFadd{regions}\DIFaddend .

%%% dangers of imports %%%

However, even if energy imports are economically attractive for Europe, a strong
reliance may not be desirable because of energy security concerns. Awareness of
energy security has risen since Russia \DIFdelbegin \DIFdel{throttled }\DIFdelend \DIFaddbegin \DIFadd{constrained }\DIFaddend fossil gas supplies to Europe
in 2022,\cite{pedersenLongtermImplications2022} at a time when the EU27 imported
around two-thirds of its fossil energy needs.\DIFdelbegin %DIFDELCMD < \cite{eurostatCompleteEnergy2023}
%DIFDELCMD < %%%
\DIFdelend \DIFaddbegin \cite{eurostatCompleteEnergyBalances2023}
\DIFaddend Europe must take care to avoid repeating the mistakes of previous decades when
it became dependent on a small number of exporters with market power and reliant
on rigid pipeline infrastructure.

%%% dependence of energy imports on energy infrastructure %%%

Europe's strategy for clean energy imports will also strongly affect the
requirements for domestic energy infrastructure. Previous research found many
ways to develop a self-sufficient energy
system.\DIFdelbegin %DIFDELCMD < \cite{pickeringDiversityOptions2022,trondleHomemadeImported2019,brownSynergiesSector2018}
%DIFDELCMD < %%%
\DIFdelend \DIFaddbegin \cite{pickeringDiversityOptions2022, trondleHomemadeImported2019,
brownSynergiesSector2018} \DIFaddend To support such scenarios without energy imports into
Europe, reinforcing the European power grid or building a hydrogen network was
often identified as beneficial.\DIFdelbegin %DIFDELCMD < \cite{neumannPotentialRole2023,victoriaSpeedTechnological2022}
%DIFDELCMD < %%%
\DIFdelend \DIFaddbegin \cite{neumannPotentialRoleHydrogen2023,
victoriaSpeedTechnological2022} \DIFaddend However, depending on the volumes \DIFdelbegin \DIFdel{of imports and the energy vectors }\DIFdelend \DIFaddbegin \DIFadd{and vectors of
imports }\DIFaddend (electricity, hydrogen\DIFaddbegin \DIFadd{, }\DIFaddend or hydrogen derivatives) \DIFaddbegin \DIFadd{and levels of industry
migration}\DIFaddend , Europe might not need to expand its hydrogen \DIFdelbegin \DIFdel{transport }\DIFdelend \DIFaddbegin \DIFadd{pipeline }\DIFaddend infrastructure.
Most hydrogen is used to make derivative products (e.g.~, ammonia for
fertilisers\DIFaddbegin \DIFadd{, sponge iron for steel, }\DIFaddend or Fischer-Tropsch fuels for aviation and
shipping).\DIFdelbegin %DIFDELCMD < \cite{neumannPotentialRole2023} %%%
\DIFdelend \DIFaddbegin \cite{neumannPotentialRoleHydrogen2023} \DIFaddend If Europe imported these
products at scale, much of the hydrogen demand would fall away. In consequence,
this would reduce the need for hydrogen transport. However, if hydrogen itself
\DIFdelbegin \DIFdel{is
imported }\DIFdelend \DIFaddbegin \DIFadd{were imported and to be transported to today's industry clusters}\DIFaddend , this would
require a pipeline topology tailored to \DIFdelbegin \DIFdel{accommodate
}\DIFdelend \DIFaddbegin \DIFadd{connecting these to the }\DIFaddend hydrogen
arriving from North Africa or maritime \DIFdelbegin \DIFdel{shipping routes to Northern
}\DIFdelend \DIFaddbegin \DIFadd{entry points across }\DIFaddend Europe.

%%% review of policy strategies %%%

Policy has reflected these different visions for imports in various ways. In
particular, hydrogen imports have recently attracted considerable interest, with
plans of the European Commission\cite{europeancommissionRepowerEUPlan} to import
10~Mt (333~TWh\DIFdelbegin \footnote{\DIFdel{All mass-energy conversion is based on the lower heating
value (LHV). Steel is included in energy terms applying 2.1 kWh/kg as released
by the oxidation of iron.}}%DIFAUXCMD
\addtocounter{footnote}{-1}%DIFAUXCMD
\DIFdelend \DIFaddbegin \DIFadd{$_\text{LHV}$}\DIFaddend ) hydrogen and derivatives by 2030. \DIFaddbegin \DIFadd{New financing
instruments, like the European Hydrogen
Bank}\cite{europeancommissionEuropeanHydrogenBank2024} \DIFadd{or
H2Global}\cite{h2globalfoundationH2Global2024} \DIFadd{are set up to support the scale-up
of green hydrogen imports. }\DIFaddend Desire to import hydrogen and derivative products
is also present in various national
strategies.\cite{corbeauNationalHydrogenStrategies2024} In particular, Germany\DIFdelbegin \DIFdel{seeks }\DIFdelend \DIFaddbegin \DIFadd{'s
new import strategy plans }\DIFaddend to cover up to 70\% of its \DIFdelbegin \DIFdel{hydrogen consumption }\DIFdelend \DIFaddbegin \DIFadd{demand for hydrogen and its
derivatives }\DIFaddend through imports by 2030 and \DIFdelbegin \DIFdel{pursues bilateral partnerships }\DIFdelend \DIFaddbegin \DIFadd{highlights bilateral partnerships as
well as the expansion of import infrastructure as a means }\DIFaddend to accomplish
this.\DIFdelbegin %DIFDELCMD < \cite{bundesministeriumfuerwirtschaftundklimaschutzFortschreibungNationalenWasserstoffstrategie2023}
%DIFDELCMD < %%%
\DIFdelend \DIFaddbegin \cite{germanfederalministryofeconomicaffairsandclimateactionbmwkNationalHydrogenStrategy2023,germanfederalministryofeconomicaffairsandclimateactionbmwkImportStrategyHydrogen2024}
\DIFaddend Conversely, hydrogen roadmaps of
Denmark,\cite{danishministryofclimateenergyandutilitiesRegeringensStrategiPowertoX2021}
Ireland,\cite{departmentoftheenvironmentclimateandcommunicationsgovernmentofirelandNationalHydrogenStrategy2023} 
Spain,\cite{marcoestrategicodeenergiayclimaRutaHidrogenoApuesta2020} and the
United
Kingdom,\cite{ukdepartmentforenergysecurity&netzeroHydrogenStrategyUpdate2023}
recognise these countries' potential to become major exporters of renewable
energy, whereas France's strategy focuses on local hydrogen production to meet
domestic needs.\cite{frenchgovernmentStrategieNationalePour2023} \DIFdelbegin \DIFdel{Additionally, in recent plans by transmission system
operators, }%DIFDELCMD < \cite{entso-eTYNDP2024Project2024}  %%%
\DIFdelend \DIFaddbegin \DIFadd{Beyond direct
energy imports, the Draghi report}\cite{draghiFutureEuropeanCompetitiveness2024}
\DIFadd{also raises broader concerns about European industrial competitiveness and
discusses the benefits of relocating energy-intensive industries to renewable-rich
regions inside Europe. Additionally, }\DIFaddend European grid development
plans\DIFaddbegin \cite{entso-eTYNDP2024Project2024} \DIFaddend reveal renewed enthusiasm for
electricity imports via ultra-long HVDC cables, evolving from early
DESERTEC\cite{desertecfoundationDESERTECSustainableWealth2024} ideas to
contemporary proposals like the Morocco-UK Xlinks
project.\DIFdelbegin %DIFDELCMD < \cite{xlinksMoroccoUKPower2023}
%DIFDELCMD < %%%
\DIFdelend \DIFaddbegin \cite{xlinksMoroccoUKPowerProject2023}
\DIFaddend 

%%% literature review %%%

While many previous academic studies have evaluated the cost of `green'
renewable energy and \DIFaddbegin \DIFadd{energy-intensive }\DIFaddend material imports in the form of
electricity,\cite{lilliestamEnergySecurity2011,triebSolarElectricity2012,lilliestamVulnerabilityTerrorist2014,bogdanovNorthEastAsian2016,benaslaTransitionSustainable2019,reichenbergDeepDecarbonization2022}% (some with reference to the DESERTEC idea),
hydrogen,\DIFdelbegin %DIFDELCMD < \cite{timmerbergHydrogenRenewables2019,ishimotoLargescaleProduction2020,brandleEstimatingLongterm2021,luxSupplyCurves2021,galvanExportingSunshine2022,collisDeterminingProduction2022,galimovaImpactInternational2023,schmitzImplicationsHydrogenImport2024}
%DIFDELCMD < %%%
\DIFdel{ammonia,}%DIFDELCMD < \cite{nayak-lukeTechnoeconomicViability2020,armijoFlexibleProduction2020,galimovaFeasibilityGreen2023}
%DIFDELCMD < %%%
\DIFdel{methane,}%DIFDELCMD < \cite{luxSupplyCurves2021,agoraenergiewendeHydrogenImport2022}
%DIFDELCMD < %%%
\DIFdelend \DIFaddbegin \cite{timmerbergHydrogenRenewables2019,ishimotoLargescaleProduction2020,brandleEstimatingLongterm2021,luxSupplyCurves2021,galvanExportingSunshine2022,collisDeterminingProduction2022,galimovaImpactInternational2023,franzmannGreenHydrogenCostpotentials2023,schmitzImplicationsHydrogenImport2024}
\DIFadd{ammonia,}\cite{nayak-lukeTechnoeconomicViability2020,armijoFlexibleProduction2020,galimovaFeasibilityGreen2023,egererEconomicsGlobalGreen2023}
\DIFadd{methane,}\cite{luxSupplyCurves2021,agoraenergiewendeHydrogenImport2022,carelsSyntheticNaturalGas2024}
\DIFaddend steel,\cite{trollipHowGreen2022a,devlinRegionalSupply2022,lopezDefossilisedSteel2023}
carbon-based
fuels,\cite{fasihiLongTermHydrocarbon2017,sherwinElectrofuelSynthesis2021} or a
broader variety of power-to-X
fuels,\DIFdelbegin %DIFDELCMD < \cite{vanderzwaanTimmermansDream2021,pfennigGlobalGISbased2022,irenaGlobalHydrogen2022,solerEFuelsTechno2022,hamppImportOptions2023,gengeSupplyCosts2023,galimovaGlobalTrading2023a}
%DIFDELCMD < %%%
\DIFdelend \DIFaddbegin \cite{vanderzwaanTimmermansDream2021,pfennigGlobalGISbasedPotential2023,irenaGlobalHydrogenTrade2022,solerEFuelsTechno2022,hamppImportOptions2023,gengeSupplyCostsGreen2023,galimovaGlobalTrading2023a}
\DIFaddend these do not address the interactions of imports with European energy
infrastructure requirements. On the other hand, among studies dealing with the
detailed planning of net-zero energy systems in Europe, some do not consider
energy
imports,\cite{pickeringDiversityOptions2022,brownSynergiesSector2018,victoriaSpeedTechnological2022}
while others only consider hydrogen imports or a limited set of alternative
\DIFaddbegin \DIFadd{endogenously optimised }\DIFaddend import
vectors.\DIFdelbegin %DIFDELCMD < \cite{gilsInteractionHydrogen2021,seckHydrogenDecarbonization2022,wetzelGreenEnergy2023,kountourisUnifiedEuropean2023,neumannPotentialRole2023}
%DIFDELCMD < %%%
\DIFdelend \DIFaddbegin \cite{gilsInteractionHydrogen2021,seckHydrogenDecarbonization2022,wetzelGreenEnergy2023a,neumannPotentialRoleHydrogen2023,fleiterHydrogenInfrastructureFuture2024,kountourisUnifiedEuropeanHydrogen2024}
\DIFaddend Only a few consider at least elementary cost
uncertainties,\cite{frischmuthHydrogenSourcing2022,schmitzImplicationsHydrogenImport2024}
and none investigate a larger range of potential import volumes across subsets
of available import vectors.


%%% main paper idea - scenarios %%%

In this study, we explore the full range between the two poles of complete
self-sufficiency and wide-ranging renewable energy imports into Europe in
scenarios with high shares of wind and solar electricity and net-zero carbon
emissions. We investigate how the infrastructure requirements of a
self-sufficient European energy system that exclusively leverages domestic
resources from the continent may differ from a system that relies on energy
imports from outside of Europe. For our analysis, we integrate an open
\DIFaddbegin \DIFadd{optimisation }\DIFaddend model of global energy supply chains,
TRACE,\cite{hamppImportOptions2023} with a spatially and temporally resolved
sector-coupled open-source energy system \DIFaddbegin \DIFadd{optimisation }\DIFaddend model for Europe,
PyPSA-Eur,\DIFdelbegin %DIFDELCMD < \cite{PyPSAEurSecSectorCoupled} %%%
\DIFdelend \DIFaddbegin \cite{PyPSAEurSecSectorCoupledOpen} \DIFaddend to investigate the impact of imports
on European energy infrastructure needs. We evaluate potential import locations
and costs for different supply vectors, by how much system costs can be reduced
through imports, and how their inclusion affects deployed transport networks\DIFdelbegin \DIFdel{and storage }\DIFdelend \DIFaddbegin \DIFadd{,
storage and backup capacities}\DIFaddend . For this purpose, we perform sensitivity analyses
interpolating between very high levels of imports and no imports at all,
exploring low and high costs for imports to account for associated
uncertainties, and system responses to the exclusion of subsets of import
vectors, in order to \DIFdelbegin \DIFdel{probe the flatness of the solution space.
This allows us to
draw robust policy conclusions from our results.
}
%DIF <  JH Draw connections such that they intersect less, e.g. MA-IE around ES, not
%DIF <  through ES and PT.
\DIFaddbegin \begin{figure}
    \caption{\textbf{Overview of considered import options into Europe.}
        \textit{Panel (a)} shows the regional differences in the cost to deliver
        green methanol to Europe (choropleth layer), the cost composition of
        different import vectors (bar charts), an illustration of the wind and
        solar availability in Morocco, and an illustration of the land
        eligibility analysis for wind turbine placement in the region of Buenos
        Aires in Argentina. \textit{Panel (b)} depicts considered potential
        entry points for energy imports into Europe like the location of
        existing and planned LNG terminals and gas pipeline entry points, the
        lowest costs of hydrogen imports in different European regions
        (choropleth layer), and the considered connections for long-distance
        HVDC import links and hydrogen pipelines from the MENA region, Turkey,
        Ukraine and Central Asia. \textit{Panel (c)} displays the distribution
        and range of import costs for different energy carriers and entry points
        with indications for selected origins from the TRACE model
        (violin charts), i.e.~differences in identically coloured markers are
        due to regional differences in the transport costs to alternative
        entrypoints. These are more variable for liquid hydrogen as transport
        distance is a more substantial cost factor for this import vector.
        Supplementary Fig.~3 shows the world map for lowest 
        hydrogen import costs by pipeline or ship into Europe.}
    \label{fig:options}
\end{figure}
\DIFaddend 

%%% technical discussion of import vectors %%%

As possible import options, we consider electricity by transmission line,
hydrogen as gas by pipeline and liquid by ship, methane as \DIFdelbegin \DIFdel{gas by pipeline and
}\DIFdelend liquid by ship,
liquid ammonia, steel \DIFaddbegin \DIFadd{and its precursor hot briquetted iron (HBI)}\DIFaddend , methanol and
Fischer-Tropsch fuels by ship. Each energy vector has unique characteristics
with regards to its production, \DIFaddbegin \DIFadd{storage, }\DIFaddend transport and consumption\DIFdelbegin \DIFdel{(}%DIFDELCMD < \sfigref{fig:si:balances-a,fig:si:balances-b}%%%
\DIFdel{)}\DIFdelend . Electricity offers
the most flexible usage but is challenging to store and requires variability
management if sourced from wind or solar energy. Hydrogen is easier to store and
transport in large quantities but at the expense of conversion losses and less
versatile applications. Large quantities could be used for backup power and
heat, steel production, \DIFaddbegin \DIFadd{industry feedstocks }\DIFaddend and the domestic synthesis of
shipping and aviation fuels. On the other hand, imported synthetic carbonaceous
fuels like methane, methanol and Fischer-Tropsch fuels could largely substitute
the need for domestic synthesis. There is \DIFdelbegin \DIFdel{much }\DIFdelend more experience with storing and
transporting these fuels and part of the existing infrastructure could
potentially be \DIFdelbegin \DIFdel{leveraged}\DIFdelend \DIFaddbegin \DIFadd{reused or repurposed}\DIFaddend . However, they require a sustainable carbon
source and, particularly for methane, effective carbon management and leakage
prevention.\cite{shirizadehImpactMethaneLeakage2023} Ammonia is similarly easier
to handle than hydrogen but does not require a carbon source. However, it faces
safety and acceptance concerns due to its toxicity and potentially adverse
effects on the global nitrogen
cycle.\cite{bertagniMinimizingImpactsAmmonia2023,wolframUsingAmmoniaShipping2022}
Its demand in Europe is mostly driven by fertiliser usage. Steel \DIFdelbegin \DIFdel{represents }\DIFdelend \DIFaddbegin \DIFadd{and HBI
represent }\DIFaddend the import of energy-intensive materials and \DIFdelbegin \DIFdel{offers low }\DIFdelend \DIFaddbegin \DIFadd{offer low long-distance
}\DIFaddend transport costs.

\DIFdelbegin \DIFdel{Further conversion of imported fuels is also possible once they have arrived in
Europe, e.g.~hydrogen could be used to synthesise carbon-based fuels, ammonia
could be cracked to hydrogen, methane and methanol could be reformed to hydrogen
or combusted for power generation with or without carbon capture. However,
conversion losses can make it less attractive economically to use a high-value
hydrogen derivative merely as a transport and storage vessel only to reconvert
it back to hydrogen or electricity.
}%DIFDELCMD < 

%DIFDELCMD < %%%
\DIFdelend %%% brief methodology PyPSA-Eur %%%

The PyPSA-Eur\DIFdelbegin %DIFDELCMD < \cite{PyPSAEurSecSectorCoupled} %%%
\DIFdelend \DIFaddbegin \cite{PyPSAEurSecSectorCoupledOpen} \DIFaddend model co-optimises the investment
and operation of generation, storage, conversion and transmission
infrastructures in a single linear optimisation problem. The model is further
given the opportunity to relocate \DIFdelbegin \DIFdel{some energy-intensive industries within
Europecapturing a }\DIFdelend \DIFaddbegin \DIFadd{ammonia and primary steel production within
Europe, capturing }\DIFaddend potential renewables pull \DIFdelbegin \DIFdel{effect.}%DIFDELCMD < \cite{verpoortEstimatingRenewables2023,samadiRenewablesPull2023} %%%
\DIFdel{We
resolve 110 }\DIFdelend \DIFaddbegin \DIFadd{effects within Europe and
abroad.}\cite{verpoortImpactGlobalHeterogeneity2024,
samadiRenewablesPullEffect2023,egererIndustryTransformationFossil2024} \DIFadd{We
resolve 115 }\DIFaddend regions comprising the European Union without Cyprus and Malta as
well as the United Kingdom, Norway, Switzerland, Albania, Bosnia and
Herzegovina, Montenegro, North Macedonia, Serbia, and Kosovo. In combination
with a \DIFdelbegin \DIFdel{4-hourly equivalent }\DIFdelend \DIFaddbegin \DIFadd{4-hourly-equivalent }\DIFaddend time resolution for \DIFdelbegin \DIFdel{one year }\DIFdelend \DIFaddbegin \DIFadd{the weather year of 2013}\DIFaddend , grid
bottlenecks, renewable variability, and seasonal storage requirements are
\DIFdelbegin \DIFdel{efficiently
captured. Weather variations between years are not considered for computational
reasons. }\DIFdelend \DIFaddbegin \DIFadd{sufficiently captured. }\DIFaddend The  model includes regional demands from the
electricity, industry, buildings, agriculture and transport sectors,
international shipping and aviation, and non-energy feedstock demands in the
chemicals industry. Transmission infrastructure for electricity, gas and
hydrogen, and candidate entry points like existing and prospective LNG terminals
\DIFdelbegin \DIFdel{and }\DIFdelend \DIFaddbegin \DIFadd{as well as }\DIFaddend cross-continental pipelines are also represented.  \DIFaddbegin \DIFadd{However, no
pathways are modelled in this overnight scenario and the model has perfect
operational foresight. }\DIFaddend We utilize techno-economic assumptions for \DIFdelbegin \DIFdel{2030}%DIFDELCMD < \cite{dea2019}%%%
\DIFdel{, reflecting that infrastructure required for achieving carbon
neutrality must be built well in advance of reaching this goal. While enforcing
}\DIFdelend \DIFaddbegin \DIFadd{2040 and
enforce }\DIFaddend net-zero \DIFdelbegin \DIFdel{emissions for carbon dioxide, we also }\DIFdelend \DIFaddbegin \DIFadd{CO$_2$ emissions and }\DIFaddend limit the annual carbon sequestration \DIFdelbegin \DIFdel{potential }\DIFdelend to
200~Mt$_{\text{CO}_2}$\DIFdelbegin \DIFdel{/a. }\DIFdelend \DIFaddbegin \DIFadd{~a$^{-1}$, similar to the 250~Mt$_{\text{CO}_2}$~a$^{-1}$
highlighted in the EU carbon management
strategy.}\cite{europeancommissionAmbitiousIndustrialCarbon2024} \DIFaddend This suffices to
offset \DIFdelbegin \DIFdel{unabatable }\DIFdelend \DIFaddbegin \DIFadd{unabated }\DIFaddend industrial process emissions \DIFdelbegin \DIFdel{of around 140~Mt$_{\text{CO}_2}$/a and limited }\DIFdelend \DIFaddbegin \DIFadd{and limits the }\DIFaddend use of fossil fuels
beyond that, whose emissions are compensated either through capturing emissions
at source \DIFdelbegin \DIFdel{with a capture rate of 90\% or via }\DIFdelend \DIFaddbegin \DIFadd{or by }\DIFaddend carbon dioxide removal. \DIFdelbegin %DIFDELCMD < 

%DIFDELCMD < %%%
\DIFdelend More details are included in the
\nameref{sec:methods} section.


% \section*{Results}
% \label{sec:results}


\section*{\DIFdelbegin \DIFdel{Cost assessment of energy and material import vectors}\DIFdelend \DIFaddbegin \DIFadd{Results}\DIFaddend }

\DIFaddbegin \subsection*{\DIFadd{Assessment of energy and material import unit costs}}

\DIFaddend %%% brief methodology TRACE %%%


Green fuel and steel import costs seen by the model are based on an extension of
recent research by Hampp et al.,\cite{hamppImportOptions2023} who assessed the
levelised cost of energy exports for different green energy and material supply
chains \DIFdelbegin \DIFdel{in }\DIFdelend \DIFaddbegin \DIFadd{from }\DIFaddend various world regions \DIFdelbegin \DIFdel{(\mbox{%DIFAUXCMD
\cref{fig:options:global}}\hskip0pt%DIFAUXCMD
)}\DIFdelend \DIFaddbegin \DIFadd{to Europe}\DIFaddend . Our selection of exporting \DIFdelbegin \DIFdel{countries comprises Australia, Argentina, Chile, Kazakhstan, Namibia,
Turkey, Ukraine, the
Eastern United States and Canada, mainland China,and the
MENA region. Regional supply cost curves for these countries are developed based on renewable resources, land
availability and prioritised }\DIFdelend \DIFaddbegin \DIFadd{regions
comprises all 53 coloured or dotted regions in \mbox{%DIFAUXCMD
\cref{fig:options:global}}\hskip0pt%DIFAUXCMD
. In the
TRACE optimisation model,}\cite{hamppImportOptions2023} \DIFadd{regional wind and solar
potentials are assessed based on prevailing weather conditions and land
availability while prioritising projected }\DIFaddend domestic demand. \DIFaddbegin \DIFadd{In combination with
the techno-economic modelling of the various fuel-specific supply chains stages
(}\sfigref{fig:si:import-esc-scheme}\DIFadd{), the lowest levelised supply cost for each
carrier, exporter and importer combination are determined for a reference volume
of 500~TWh~a$^{-1}$ (or 100~Mt~a$^{-1}$ of steel/HBI), thus incorporating the
trade-off between import cost and import location (\mbox{%DIFAUXCMD
\cref{fig:options:europe}}\hskip0pt%DIFAUXCMD
).
}\DIFaddend Unlike domestic electrofuel synthesis in Europe, which could use captured CO$_2$
from point sources, direct air capture is assumed to be the only carbon source
of imported fuels. Concepts involving the shipment of captured CO$_2$ from
Europe to exporting \DIFdelbegin \DIFdel{countries }\DIFdelend \DIFaddbegin \DIFadd{regions }\DIFaddend for carbonaceous fuel synthesis \DIFaddbegin \DIFadd{or permanent
sequestration of CO$_2$ captured by direct air capture abroad }\DIFaddend are not
considered.\cite{treeenergysolutionsGreenCycle2024,fonderSyntheticMethaneClosing2024}

%DIF < %% brief methodology TRACE-PyPSA-Eur scoupling %%%
%DIF > %% brief methodology TRACE-PyPSA-Eur coupling %%%

\DIFdelbegin \DIFdel{We use these supply curves to determine the region-specific lowest import cost
for each carrier, thus incorporating the potential trade-off between import cost
and import location (\mbox{%DIFAUXCMD
\cref{fig:options:europe}}\hskip0pt%DIFAUXCMD
). For hydrogen derivatives, the
lowest-cost suppliers are Argentina and Chile for all entry points into Europe. Electricity }\DIFdelend \DIFaddbegin \DIFadd{The import costs for each combination of carrier, exporter and importer are then
included as supply options in the PyPSA-Eur model. Hydrogen and methane can be
imported where there are LNG terminals in operation or under construction or
where pipeline entry points exist (except for entry points from Russia). Due to
higher volatility, electricity }\DIFaddend imports are endogenously optimised, meaning that
the capacities and operation of wind and solar generation\DIFaddbegin \DIFadd{, }\DIFaddend as well as storage in
the respective exporting \DIFdelbegin \DIFdel{countries }\DIFdelend \DIFaddbegin \DIFadd{regions }\DIFaddend and the HVDC transmission lines, are co-planned
with the rest of the \DIFdelbegin \DIFdel{system. Hydrogen and methane can be imported where there are
existing or planned LNG terminals or pipeline entry-points (excluding
connections through Russia). This results in lower hydrogen import costs, where
it can be imported by pipeline. }\DIFdelend \DIFaddbegin \DIFadd{European system. }\DIFaddend Ammonia, carbonaceous fuels\DIFdelbegin \DIFdel{and steel }\DIFdelend \DIFaddbegin \DIFadd{, and ferrous
materials }\DIFaddend are not spatially resolved in the model, assuming they can be
transported within Europe at negligible \DIFdelbegin \DIFdel{additional cost. }\DIFdelend \DIFaddbegin \DIFadd{cost. Thus, their specific import
location is not determined. An import limit of 500~TWh per region for the sum of
all exports is imposed to prevent over-reliance on single exporters.
}\DIFaddend 
\DIFaddbegin \begin{figure}
    \caption{\textbf{Potential for cost reductions with reduced sets of import options.}
        Subsets of available import options are sorted by ascending cost
        reduction potential. Top panel shows profile of total system cost
        savings. Bottom panel shows composition and extent of imports in
        relation to total energy system costs. Percentage numbers in bar plot
        indicate the share of total system costs spent on domestic energy
        infrastructure. Alternative scenarios of this figure with higher and
        lower import cost assumptions are shown in Supplementary Figs.~12 and
        13. }
    \label{fig:sensitivity-bars}
\end{figure}

\subsection*{\DIFadd{Cost savings for fuel and material import combinations}}
\DIFaddend 

In \cref{fig:sensitivity-bars}, we first explore the cost reduction potential of
various energy and material import options. \DIFdelbegin \DIFdel{In the absence of }\DIFdelend \DIFaddbegin \DIFadd{Without }\DIFaddend energy imports, total energy
system costs add up to \DIFdelbegin %DIFDELCMD < \bneuro{815}%%%
\footnote{\DIFdel{All currency values
are given in }%DIFDELCMD < \euro{}%%%
\DIFdel{$_{2020}$.}}%DIFAUXCMD
\addtocounter{footnote}{-1}%DIFAUXCMD
\DIFdelend \DIFaddbegin \bneuro{836}\DIFaddend . By enabling imports from outside \DIFdelbegin \DIFdel{of }\DIFdelend Europe and
considering all import vectors, we find a potential reduction of total energy
system costs by up to \DIFdelbegin %DIFDELCMD < \bneuro{39}%%%
\DIFdelend \DIFaddbegin \bneuro{37}\DIFaddend . This corresponds to a relative reduction of
\DIFdelbegin \DIFdel{4.9}\DIFdelend \DIFaddbegin \DIFadd{4.4}\DIFaddend \%. For cost-optimal imports, around \DIFdelbegin \DIFdel{71}\DIFdelend \DIFaddbegin \DIFadd{77}\DIFaddend \% of these costs are used to develop
domestic energy infrastructure. The remaining \DIFdelbegin \DIFdel{29}\DIFdelend \DIFaddbegin \DIFadd{23}\DIFaddend \% are spent on importing a
volume of \DIFdelbegin \DIFdel{52}\DIFdelend \DIFaddbegin \DIFadd{50}\DIFaddend ~Mt of green steel and around \DIFdelbegin \DIFdel{2700 }\DIFdelend \DIFaddbegin \DIFadd{1498 }\DIFaddend TWh of green energy, which is
\DIFdelbegin \DIFdel{almost a quarter }\DIFdelend \DIFaddbegin \DIFadd{around 13\% }\DIFaddend of the system's total energy supply (\cref{fig:import-shares}). \DIFaddbegin \DIFadd{Our
results show a cost-effective import mix consisting primarily of liquid
carbon-based fuels, hydrogen, and steel imports with small volumes of ammonia
and electricity imports.
}\DIFaddend 

Next, we investigate the impact of restricting the available import options to
subsets of import vectors. We find that if only hydrogen can be imported, cost
savings are reduced to \DIFdelbegin %DIFDELCMD < \bneuro{22} %%%
\DIFdel{(2.8\%). This is because by using }\DIFdelend \DIFaddbegin \bneuro{20} \DIFadd{(2.4\%), with pipeline-based hydrogen imports
being preferred to imports as liquid by ship. By importing a larger volume of
}\DIFaddend hydrogen as an intermediary carrier \DIFaddbegin \DIFadd{(1338~TWh instead of 576~TWh,
}\sfigref{fig:si:import-shares-a}\DIFadd{)}\DIFaddend , low-cost renewable \DIFdelbegin \DIFdel{electricity }\DIFdelend \DIFaddbegin \DIFadd{energy }\DIFaddend from abroad can
still be leveraged for the synthesis of derivative products in Europe. \DIFdelbegin \DIFdel{For this purpose,
pipeline-based hydrogen imports are preferred to ship-based imports as liquid.
When }\DIFdelend \DIFaddbegin \DIFadd{However,
the benefit is reduced as domestic CO$_2$ feedstocks from industrial sources are
depleted.
}

\DIFadd{Conversely, when }\DIFaddend direct hydrogen imports are excluded from the available import
options, cost savings are \DIFdelbegin \DIFdel{similar with }%DIFDELCMD < \bneuro{24} %%%
\DIFdel{(3}\DIFdelend \DIFaddbegin \DIFadd{close to the maximum with }\bneuro{34} \DIFadd{(4.1}\DIFaddend \%). \DIFaddbegin \DIFadd{This
indicates that the benefit of using domestically captured biogenic or fossil
CO$_2$ is similar to tapping into low-cost renewable resources abroad. }\DIFaddend Focusing
imports exclusively on liquid carbonaceous fuels derived from hydrogen,
i.e.~methanol or Fischer-Tropsch fuels, \DIFdelbegin \DIFdel{consistently achieves }\DIFdelend \DIFaddbegin \DIFadd{still achieves high }\DIFaddend cost savings of
\DIFdelbegin %DIFDELCMD < \bneuro{13-20} %%%
\DIFdel{(1.7-2.5\%).
}%DIFDELCMD < 

%DIFDELCMD < %%%
\DIFdel{On the contrary, }\DIFdelend \DIFaddbegin \bneuro{31} \DIFadd{(3.7\%), which is due to the smaller demand or variety of
applications for ammonia, methane, and steel compared to liquid carbonaceous
fuels. Thus, excluding them has a small effect on cost savings. This aligns with
the finding that }\DIFaddend restricting options to only \DIFdelbegin \DIFdel{ammonia or methane }\DIFdelend \DIFaddbegin \DIFadd{methane, ammonia, or ferrous
material }\DIFaddend imports yields negligible \DIFdelbegin \DIFdel{cost savings. Small savings }\DIFdelend \DIFaddbegin \DIFadd{to small cost savings }\DIFaddend below \bneuro{5}
(0.6\%)\DIFdelbegin \DIFdel{can be reached
if only electricity or steel can be imported. This is due to the lower volume
and variety of usage options for ammonia, methane and steel compared to
hydrogen, methanol and Fischer-Tropsch fuels. Furthermore, }\DIFdelend \DIFaddbegin \DIFadd{. Negligible cost savings were also found for }\DIFaddend the direct import of
electricity \DIFaddbegin \DIFadd{as it }\DIFaddend poses more challenges for \DIFdelbegin \DIFdel{system integration . Generally, our results
indicate a preference for methanol, hydrogen and steel imports over electricity
imports, with a mix emerging as the most cost-effective approach.
}%DIFDELCMD < \sfigref{fig:si:subsets} %%%
\DIFdel{show additional insights into how }\DIFdelend \DIFaddbegin \DIFadd{integration into the European
system.
}

\DIFadd{Overall, while }\DIFaddend varying import costs \DIFdelbegin \DIFdel{affect these findings}\DIFdelend \DIFaddbegin \DIFadd{within $\pm 20\%$ affects the magnitude of
attainable cost savings, the relative impact of restricting specific import
options remains broadly consistent
(}\sfigref{fig:si:subsets-higher,fig:si:subsets-lower}\DIFadd{)}\DIFaddend .

\DIFaddbegin \begin{figure}
    \caption{\textbf{Comparison of domestic synthetic production costs and import costs for varying import scenarios.}
        The three panels (a), (b), and (c) refer to different import scenarios.
        In each panel, the \textit{bar charts} show the production-weighted
        average costs of domestic production of steel, hydrogen and its
        derivatives split into its cost and revenue components. These have been
        computed using the marginal prices of the respective inputs and outputs
        for the production volume of each region and snapshot. Capital
        expenditures are distributed to hours in proportion to the production
        volume. Missing bars indicate that no domestic production occured in the
        scenario, e.g.~for the case of methane where all demand is met by
        biogenic and fossil methane and no synthetic production occured
        (cf.~energy balances in supplementary material). All hydrogen is
        produced from electrolysis; i.e.~the model did not choose to produce
        hydrogen via steam methane reforming with or without carbon capture. For
        each bar, the yellow errorbars show the range of time-averaged domestic
        production costs across all regions. The black error bars show the range
        of import costs across all regions. The \textit{maps} on the right
        relate the hydrogen production volume to the weighted cost of domestic
        hydrogen production (left colorbar). Confer Supplementary Fig.~21 for
        information on the domestic cost supply curves.}
    \label{fig:market-values}
\end{figure}
\DIFaddend 

\DIFdelbegin \section*{\DIFdel{Import dynamics for different energy carriers}}
%DIFAUXCMD
\DIFdelend \DIFaddbegin \subsection*{\DIFadd{Import dynamics for different energy carriers}}
\DIFaddend 

%%% state results %%%

\cref{fig:import-shares} outlines which carriers are imported in which
quantities in relation to their total supply under default assumptions when the
vector and volume can be flexibly chosen (\DIFdelbegin \DIFdel{``}\DIFdelend \DIFaddbegin \DIFadd{`}\DIFaddend all imports allowed\DIFdelbegin \DIFdel{'' }\DIFdelend \DIFaddbegin \DIFadd{' }\DIFaddend in
\cref{fig:sensitivity-bars}). In energy terms, cost-optimal imports comprise
around \DIFdelbegin \DIFdel{50\% hydrogen, more than 20\% electricity, and around 20\% of
carbonaceous fuels}\DIFdelend \DIFaddbegin \DIFadd{45\% carbonaceous fuels, 35\% hydrogen, and less than 10\% electricity}\DIFaddend .
Noticeably, all \DIFaddbegin \DIFadd{primary }\DIFaddend crude steel and \DIFdelbegin \DIFdel{methanol for shipping and
industry is imported. Also,
around three-quarters }\DIFdelend \DIFaddbegin \DIFadd{ammonia for fertilizers is imported,
whereby steel imports are preferred over HBI imports. Around half }\DIFaddend of the total
hydrogen supply is imported\DIFdelbegin \DIFdel{. }\DIFdelend \DIFaddbegin \DIFadd{, matching the ratio of the 2030 REPowerEU
targets.}\cite{europeancommissionRepowerEUPlan} \DIFaddend Hydrogen is imported so that it
can be processed into derivative products domestically rather than \DIFaddbegin \DIFadd{used for
}\DIFaddend direct applications for hydrogen. Smaller import shares are observed for
electricity, \DIFdelbegin \DIFdel{Fischer-Tropsch fuels, and ammonia, which are mostly domestically produced}\DIFdelend \DIFaddbegin \DIFadd{which is largely supplied from domestic resources, because of
higher costs and losses in electricity transmission than other import vectors,
and for methane, which is supplied from domestic fossil and biogenic sources
(}\sfigref{fig:si:balances-a}\DIFadd{).
}

%DIF > %% trade flows from origin to destination %%%

\DIFadd{In terms of trade flows (\mbox{%DIFAUXCMD
\cref{fig:import-shares}}\hskip0pt%DIFAUXCMD
), we observe carbonaceous fuel
imports by ship from South America -- leveraging low transport costs of dense
liquid fuels -- as well as ammonia, steel, and hydrogen imports from the Maghreb
region. Hydrogen is mainly received by pipeline in Spain. Moreover, due to its
proximity to Italy, some electricity imports are received by HVDC connections
from Tunisia. While the model suggests trade routes from particular regions, we
aggregate these to broader regions to emphasize that alternative
origin-destination pairs within the regions could often yield similar results
(}\sfigref{fig:si:isc-h2,fig:si:isc-ch4-nh3,fig:si:isc-meoh-ft,fig:si:isc-hbi-St}\DIFadd{)}\DIFaddend .

%%% explanation through market values %%%

To explain the import shares in \cref{fig:import-shares} in more detail, we
compare import costs with average domestic production cost split by cost and
revenue components in \cref{fig:market-values}. First, for the scenario without
imports, imported \DIFdelbegin \DIFdel{fuel appear to be }\DIFdelend \DIFaddbegin \DIFadd{fuels appear }\DIFaddend substantially cheaper than domestic production.
The high demand for hydrogen \DIFdelbegin \DIFdel{and derivative products
(}%DIFDELCMD < \sfigref{fig:si:balances-a,fig:si:balances-b}%%%
\DIFdelend \DIFaddbegin \DIFadd{derivatives (}\sfigref{fig:si:demands}\DIFaddend ) means that
the most attractive domestic potentials for renewable electricity and \DIFaddbegin \DIFadd{captured
}\DIFaddend carbon dioxide have been exhausted. \DIFdelbegin \DIFdel{Power }\DIFdelend \DIFaddbegin \DIFadd{Consequently, power }\DIFaddend from wind and solar
needs to be produced in regions with worse capacity factors\DIFdelbegin \DIFdel{and direct air capture becomes the price-setting technology for
\ce{CO2} as biogenic and industrial sources ($\approx$600~Mt$_\text{\ce{CO2}}$)
are depleted}\DIFdelend .

%%% hydrogen imports lower pressure on domestic supply chain %%%

Part of this gap is closed when hydrogen imports are allowed. By sourcing
cheaper hydrogen from outside Europe, the domestic costs of derivative fuel
synthesis are reduced. \DIFdelbegin \DIFdel{This hybrid approach has the largest effect on
Fischer-Tropsch production due to its higher hydrogen demand compared to
methanolisation and the Haber-Bosch process. Hydrogen imports also decouple the
synthesis from the seasonal variation of domestic hydrogen production costs.
}%DIFDELCMD < 

%DIFDELCMD < %%%
%DIF < %% waste heat integration %%%
%DIFDELCMD < 

%DIFDELCMD < %%%
\DIFdel{The potential for waste heat utilisation from fuel synthesis within Europe adds
further appeal to this hybrid approach. By importing hydrogen rather than the
derivative product, heat supply into district heating networks from synthesis
processes can create an additional revenue stream of up to 10 }%DIFDELCMD < \euro{}%%%
\DIFdel{/MWh, depending on the process. Taking ammonia as example, the levelised cost is 73
}%DIFDELCMD < \euro{}%%%
\DIFdel{/MWh for domestic production compared to 88 }%DIFDELCMD < \euro{}%%%
\DIFdel{/MWh for imported
ammonia.
}%DIFDELCMD < 

%DIFDELCMD < %%%
\DIFdel{The waste heat integration is also the reason why in \mbox{%DIFAUXCMD
\cref{fig:import-shares}}\hskip0pt%DIFAUXCMD
,
with all import vectors allowed, all methanol is imported, whereas
Fischer-Tropsch fuels and ammonia are produced mainly domestically using high
shares of imported hydrogen. Because the thermal discharge from the methanol
synthesis is primarily used for the distillation of the methanol-water output
mix, its waste heat potential is considered much lower compared to
Fischer-Tropsch, Haber-Bosch and Sabatier processes.Therefore, it is less
attractive to retain this part of the value chain within Europe}\DIFdelend \DIFaddbegin \DIFadd{However, the large remaining volume of CO$_2$ handled in
the European system for use and sequestration (}\sfigref{fig:si:balances-b}\DIFadd{)
means that direct air capture is still the price-setting technology for \ce{CO2}
as economic applications for biogenic and industrial carbon capture (i.e.~those
with high full load hours) are depleted}\DIFaddend .

%%% explanations of low cost differences if all imports allowed %%%

With all import vectors allowed, we see minimal cost differences between
domestic production and imports as the supply curves reach equilibrium
\DIFaddbegin \DIFadd{(}\sfigref{fig:si:cost-supply-curves}\DIFadd{)}\DIFaddend . This is because imports of hydrogen and
derivative products lower the strain on the domestic supply \DIFdelbegin \DIFdel{chain}\DIFdelend \DIFaddbegin \DIFadd{curves for hydrogen
and carbon dioxide}\DIFaddend . Thereby, domestic production would only \DIFdelbegin \DIFdel{be ramped }\DIFdelend \DIFaddbegin \DIFadd{ramp }\DIFaddend up where it
competes with imports and associated infrastructure costs. \DIFdelbegin \DIFdel{Such regions
exist in Southern Europe or }\DIFdelend \DIFaddbegin \DIFadd{This was the case for
hydrogen, methanol, and Fischer-Tropsch fuels in }\DIFaddend the British Isles and \DIFdelbegin \DIFdel{, therefore}\DIFdelend \DIFaddbegin \DIFadd{parts of
Southern Europe and Nordic countries (\mbox{%DIFAUXCMD
\cref{fig:market-values}}\hskip0pt%DIFAUXCMD
). Consequently}\DIFaddend ,
not all hydrogen is imported\DIFdelbegin \DIFdel{(}%DIFDELCMD < \sfigref{fig:si:cost-supply-curves}%%%
\DIFdel{)}\DIFdelend \DIFaddbegin \DIFadd{, but some domestic production is retained}\DIFaddend .
\DIFaddbegin \begin{figure}
    \caption{\textbf{Effect of import cost variations on system cost savings and import shares with all vectors allowed.}
    In panel (a), indicated relative import cost changes are applied uniformly
    to all vectors. In panel (b), cost changes are applied uniformly to all
    vectors but electricity imports. In panel (c), cost changes are only applied
    to carbonaceous fuels (methane, methanol and Fischer-Tropsch). Top subpanels
    show potential system cost savings compared to the scenario without imports. Bottom
    subpanels show the share and composition of different import vectors in
    relation to total energy system costs. The information is shown both in
    absolute terms and relative terms compared to the scenario without imports.
    }
    \label{fig:sensitivity-costs}
\end{figure}
\DIFaddend 

\DIFdelbegin \section*{\DIFdel{Sensitivity of potential cost savings to import costs}}
%DIFAUXCMD
\DIFdelend \DIFaddbegin \subsection*{\DIFadd{Sensitivity of potential cost savings to import costs}}
\DIFaddend 

\DIFdelbegin \DIFdel{It should be noted, however, that }\DIFdelend \DIFaddbegin \DIFadd{Thus far, the presented findings originate from a central estimate for the
import cost. However, }\DIFaddend the cost-optimal import mix \DIFdelbegin \DIFdel{also }\DIFdelend strongly depends on the
assumed import costs. This uncertainty is addressed in
\cref{fig:sensitivity-costs}. \cref{fig:sensitivity-costs:A} highlights the
extensive range in potential cost reductions if higher or lower import costs
could be attained and underlines the resulting variance in cost-effective import
mixes. Within \DIFdelbegin \DIFdel{$\pm 20\%$ }\DIFdelend \DIFaddbegin \DIFadd{$\pm 30\%$ }\DIFaddend of the default import costs applied to all carriers\DIFdelbegin \DIFdel{but
electricity}\DIFdelend ,
total cost savings vary between \DIFdelbegin %DIFDELCMD < \bneuro{13} %%%
\DIFdel{(1.6}\DIFdelend \DIFaddbegin \bneuro{2} \DIFadd{(0.3}\DIFaddend \%) and \DIFdelbegin %DIFDELCMD < \bneuro{83}
%DIFDELCMD < %%%
\DIFdel{(10.2}\DIFdelend \DIFaddbegin \bneuro{112} \DIFadd{(13.5}\DIFaddend \%).
Within this range, import volumes vary between \DIFdelbegin \DIFdel{1700 and 3800}\DIFdelend \DIFaddbegin \DIFadd{500 and 2646}\DIFaddend ~TWh. Across most
scenarios, there is a stable role for \DIFdelbegin \DIFdel{methanol, steel, electricity
and hydrogen imports. One significant difference, however, are Fischer-Tropsch
fuel imports starting from cost reductions of 10\% and their absence at cost
increases beyond }\DIFdelend \DIFaddbegin \DIFadd{ammonia and liquid carbonaceous fuel
imports. Within a narrower $\pm 20\%$ range, hydrogen imports also appear in
larger quantities, while steel imports become less attractive with cost
increases of }\DIFaddend 10\% \DIFaddbegin \DIFadd{or more. Electricity imports grow with declining costs}\DIFaddend .

\DIFdelbegin \DIFdel{A breakdown of potential causes for such cost variationssuch as cost of
capital, cost of carbon dioxide and investment costs are presented in
}%DIFDELCMD < \stabref{tab:cost-uncertainty}%%%
\DIFdel{. Some of these only
affect }\DIFdelend \DIFaddbegin \DIFadd{However, not all carriers are equally affected by technology cost variations.
Fuel synthesis technologies do not influence electricity imports, and only
carbon-based fuels are subject to }\DIFaddend the cost of \DIFdelbegin \DIFdel{carbonaceous fuels.
}\DIFdelend \DIFaddbegin \DIFadd{CO$_2$ supply. We find that when
the relative cost variation is not applied to electricity imports
(\mbox{%DIFAUXCMD
\cref{fig:sensitivity-costs:B}}\hskip0pt%DIFAUXCMD
), they remain less attractive than other
vectors, even when those alternative vectors face a 20\% cost rise.
}

\DIFaddend One central assumption \DIFaddbegin \DIFadd{regarding costs }\DIFaddend for carbon-based fuels is that imported
fuels rely \DIFdelbegin \DIFdel{exclusively }\DIFdelend on direct air capture (DAC) as a carbon source. Arguments for this
assumption relate to the potential remoteness of the ideal locations for
renewable fuel production or the absence of industrial point sources in the
exporting \DIFdelbegin \DIFdel{country. On the other hand}\DIFdelend \DIFaddbegin \DIFadd{region. In contrast}\DIFaddend , domestic electrofuels can mostly use less
expensive captured \DIFaddbegin \DIFadd{biogenic or fossil }\DIFaddend carbon dioxide from industrial \DIFdelbegin \DIFdel{point sources
or biogenic origin}\DIFdelend \DIFaddbegin \DIFadd{processes}\DIFaddend .
Therefore, the higher cost for DAC partially cancels out the savings from
utilising better renewable resources abroad. This is one of the reasons why
there is substantial power-to-X production in Europe, even with corresponding
import options. \DIFdelbegin \DIFdel{The }\DIFdelend \DIFaddbegin \DIFadd{However, the }\DIFaddend availability of cheaper \DIFdelbegin \DIFdel{biogenic }\DIFdelend \DIFaddbegin \DIFadd{(biogenic) }\DIFaddend CO$_2$ in
\DIFdelbegin \DIFdel{the
exporting country, for instance, would make importing carbonaceous fuels more
attractive }\DIFdelend \DIFaddbegin \DIFadd{exporting regions would lower costs of carbonaceous fuel imports
}\DIFaddend (\stabref{tab:cost-uncertainty}).

When the relative cost variation \DIFdelbegin \DIFdel{of $\pm 20\%$ }\DIFdelend is only applied to carbon-based fuels
(\DIFdelbegin \DIFdel{\mbox{%DIFAUXCMD
\cref{fig:sensitivity-costs:B}}\hskip0pt%DIFAUXCMD
), }\DIFdelend \DIFaddbegin \DIFadd{\mbox{%DIFAUXCMD
\cref{fig:sensitivity-costs:C}}\hskip0pt%DIFAUXCMD
), reflecting cost uncertainty in carbon
provision, }\DIFaddend hydrogen imports are \DIFdelbegin \DIFdel{increasingly
displaced by methane and }\DIFdelend \DIFaddbegin \DIFadd{quickly displaced by }\DIFaddend Fischer-Tropsch \DIFaddbegin \DIFadd{and
methanol }\DIFaddend imports with falling costs. \DIFdelbegin \DIFdel{However, it
takes a cost increase of more than 10\% for domestic methanol production to
}\DIFdelend \DIFaddbegin \DIFadd{Only when import costs rise by 20\% do
domestically produced liquid hydrocarbons -- derived mainly from imported
hydrogen -- }\DIFaddend become more cost-effective than \DIFdelbegin \DIFdel{methanol imports. This underlines the robust
benefit of importing methanol}\DIFdelend \DIFaddbegin \DIFadd{direct imports. In all three cases
of import cost variations, methane imports become relevant only with substantial
cost reductions of 40\%, replacing biogas and residual fossil gas consumption}\DIFaddend .

\DIFaddbegin \begin{figure}
    \caption{\textbf{Sensitivity of import volume on total system cost and composition.}
        \textit{Left panel:} Solid lines show the total system cost as a
        function of enforced import volumes for higher (brown scale) or lower
        (blue scale) import costs. The dashed lines indicate the corresponding
        shares of the domestic system cost. The red markers denote the maximum
        cost reductions and cost-optimal import volume for given import cost
        levels (extreme points of the curves). The cost alterations are
        uniformly applied to all imports options but direct electricity imports.
        Steel is included in energy terms applying 2.1 kWh~kg$^{-1}$ as released by the
        oxidation of iron. \textit{Right panel:} Shows the composition of the
        total system cost as a function of enforced import volumes for the
        central import cost estimate. The dashed line splits the system costs
        into costs for imports and the domestic system. Cost compositions for
        the alternative import cost scenarios are presented in Supplementary
        Figs.~14-17.}
    \label{fig:sensitivity-volume}
\end{figure}
\DIFaddend 

\DIFdelbegin \section*{\DIFdel{Attainable cost savings for varying import volumes}}
%DIFAUXCMD
\DIFdelend \DIFaddbegin \subsection*{\DIFadd{Attainable cost savings for varying import volumes}}
\DIFaddend 

What is consistent \DIFdelbegin \DIFdel{across all import cost variations }\DIFdelend \DIFaddbegin \DIFadd{for many scenarios with higher or lower import costs }\DIFaddend is the
flat solution space around the respective cost-optimal import volumes.
Increasing or decreasing the total amount of imports \DIFaddbegin \DIFadd{from the optimum }\DIFaddend barely
affects system costs within $\pm 1000$\DIFaddbegin \DIFadd{~}\DIFaddend TWh. This is illustrated in
\cref{fig:sensitivity-volume} \DIFdelbegin \DIFdel{, which shows }\DIFdelend \DIFaddbegin \DIFadd{and extended
}\sfigref{fig:si:volume-higher-1,fig:si:volume-higher-2,fig:si:volume-lower-1,fig:si:volume-lower-2}\DIFadd{,
which show }\DIFaddend the system cost as a function of enforced import volumes and
different import costs \DIFaddbegin \DIFadd{for hydrogen and its derivatives}\DIFaddend . A wide range \DIFaddbegin \DIFadd{of
}\DIFaddend scenarios with import volumes below \DIFdelbegin \DIFdel{5600}\DIFdelend \DIFaddbegin \DIFadd{4100}\DIFaddend ~TWh (\DIFdelbegin \DIFdel{4000~TWh with +}\DIFdelend \DIFaddbegin \DIFadd{2300~TWh for }\DIFaddend 20\% \DIFdelbegin \DIFdel{and 7500~TWh with }\DIFdelend \DIFaddbegin \DIFadd{higher import
costs, 5800~TWh for }\DIFaddend -20\% \DIFaddbegin \DIFadd{lower }\DIFaddend import costs) have lower total energy system
costs than the \DIFdelbegin \DIFdel{scenario
without any imports. These import volumes are roughly twice the }\DIFdelend \DIFaddbegin \DIFadd{no-imports scenario. These ranges of import values are two to
three times as large as the corresponding }\DIFaddend cost-optimal import volumes, which are
indicated by the \DIFdelbegin \DIFdel{black }\DIFdelend \DIFaddbegin \DIFadd{red }\DIFaddend markers in \cref{fig:sensitivity-volume} and correspond to
the bars previously shown in \DIFdelbegin \DIFdel{\mbox{%DIFAUXCMD
\cref{fig:sensitivity-costs:A}}\hskip0pt%DIFAUXCMD
}\DIFdelend \DIFaddbegin \DIFadd{\mbox{%DIFAUXCMD
\cref{fig:sensitivity-costs:B}}\hskip0pt%DIFAUXCMD
}\DIFaddend . Naturally, the
cost-optimal volume of imports increases as their costs decrease, but \DIFdelbegin \DIFdel{the response weakens with
lower import
costs}\DIFdelend \DIFaddbegin \DIFadd{with
noticeably varying slopes for system cost savings per unit of additional
imported energy}\DIFaddend .

As we explore the effect of increasing import volumes on system costs, we find
that already \DIFdelbegin \DIFdel{43\% (36-61}\DIFdelend \DIFaddbegin \DIFadd{56\% (48-80}\DIFaddend \% within $\pm$20\% import costs) of the \DIFdelbegin \DIFdel{4.9\%
(1.6-10.2}\DIFdelend \DIFaddbegin \DIFadd{4.4\%
(1.3-9.0}\DIFaddend \%) total cost benefit \DIFdelbegin \DIFdel{(}%DIFDELCMD < \bneuro{17}%%%
\DIFdel{) }\DIFdelend can be achieved with the first 500 TWh of
imports. This corresponds to \DIFdelbegin \DIFdel{only 18\% (15-29}\DIFdelend \DIFaddbegin \DIFadd{31\% (25-49}\DIFaddend \%) of the cost-optimal import volumes,
highlighting the diminishing \DIFdelbegin \DIFdel{return }\DIFdelend \DIFaddbegin \DIFadd{returns }\DIFaddend of large amounts of energy imports in
Europe. \DIFdelbegin \DIFdel{While importing }\DIFdelend \DIFaddbegin \DIFadd{The initial }\DIFaddend 1000 TWh \DIFdelbegin \DIFdel{already realises 70\%of the maximum
cost
savingswith our default assumptions, this maximum is only obtained for 2800~TWh of imports. For these initial 1000~TWh, }\DIFdelend \DIFaddbegin \DIFadd{realise 90\% (80-100\%) of the highest cost
savings, for which }\DIFaddend primary crude steel and \DIFdelbegin \DIFdel{methanol }\DIFdelend \DIFaddbegin \DIFadd{liquid carbonaceous fuel }\DIFaddend imports are
prioritised, followed by \DIFaddbegin \DIFadd{ammonia and }\DIFaddend hydrogen and, subsequently, \DIFdelbegin \DIFdel{electricity beyond 2000 TWh}\DIFdelend \DIFaddbegin \DIFadd{larger volumes
of electricity beyond cost-optimal import levels}\DIFaddend . Once more than \DIFdelbegin \DIFdel{5000~TWh}\DIFdelend \DIFaddbegin \DIFadd{5500~TWh
(5000-8200~TWh) }\DIFaddend are imported, less than half the total system cost would be
spent on domestic energy infrastructure.

As imports increase, there is a corresponding decrease in the need for domestic
power-to-X (PtX) production and renewable capacities. A large share of the
hydrogen, methanol\DIFdelbegin \DIFdel{and raw }\DIFdelend \DIFaddbegin \DIFadd{, and primary }\DIFaddend steel production is outsourced from Europe,
reducing the need for domestic wind and solar capacities. This trend is further
characterised by the displacement of biogas usage in favour of hydrogen imports
around the \DIFdelbegin \DIFdel{2000}\DIFdelend \DIFaddbegin \DIFadd{4000}\DIFaddend ~TWh mark \DIFaddbegin \DIFadd{(3000-5000~TWh within $\pm$20\% import costs) }\DIFaddend as demand
for domestic \ce{CO2} utilisation drops \DIFdelbegin \DIFdel{. An
increase in the amount of hydrogen imported coincides with an increasing use of }\DIFdelend \DIFaddbegin \DIFadd{and methane use for power and heat
provision is displaced by hydrogen. The increase in }\DIFaddend hydrogen \DIFdelbegin \DIFdel{fuel cells for electricity and central }\DIFdelend \DIFaddbegin \DIFadd{imports results in
the build-out of more hydrogen fuel cell CHPs for power and }\DIFaddend heat supply in
district heating networks\DIFdelbegin \DIFdel{, partially displacing the use of methane}\DIFdelend . Regarding electricity imports from the MENA region,
\cref{fig:sensitivity-volume} reveals a mix of wind and solar power \DIFaddbegin \DIFadd{with some
batteries }\DIFaddend to establish favourable feed-in profiles for the European system
integration and higher utilisation rates for the long-distance HVDC links\DIFdelbegin \DIFdel{with a
}\DIFdelend \DIFaddbegin \DIFadd{. For
instance, for imports of 4000~TWh in \mbox{%DIFAUXCMD
\cref{fig:sensitivity-volume}}\hskip0pt%DIFAUXCMD
, the
}\DIFaddend capacity-weighted average \DIFdelbegin \DIFdel{of 72\%. Utilisation rates are high }\DIFdelend \DIFaddbegin \DIFadd{utilisation rate was 85\%. This is }\DIFaddend because a
considerable share of the \DIFdelbegin \DIFdel{import costs of electricity }\DIFdelend \DIFaddbegin \DIFadd{electricity import costs }\DIFaddend can be attributed to power
transmission.

%DIF <  FN LCOE for solar and wind in MENA are mostly similar potentials are not
%DIF <  constraining DZ: solar 34 €/MWh, wind 32 €/MWh TN: solar 30 €/MWh, wind 38
%DIF <  €/MWh LY: solar 24 €/MWh, wind 37 €/MWh
\DIFdelbegin %DIFDELCMD < 

%DIFDELCMD < %%%
\DIFdelend As import costs are varied, the composition of the domestic system and import
mix \DIFaddbegin \DIFadd{for different import volumes }\DIFaddend is primarily similar
(\DIFdelbegin %DIFDELCMD < \sfigref{fig:si:volume}%%%
\DIFdelend \DIFaddbegin \sfigref{fig:si:volume-higher-2,fig:si:volume-higher-1,fig:si:volume-lower-1,fig:si:volume-lower-2}\DIFaddend ).
The main \DIFdelbegin \DIFdel{differences are a more }\DIFdelend \DIFaddbegin \DIFadd{difference is a less }\DIFaddend prominent role for \DIFdelbegin \DIFdel{Fischer-Tropsch fuel imports with lower import costsand
green methane for high import volumes. It should also be noted that }\DIFdelend \DIFaddbegin \DIFadd{steel imports with higher
import costs. What is furthermore noteworthy is that reducing import costs from
\mbox{-30\%} to \mbox{-50\%} only marginally reduces domestic infrastructure costs, indicating
largely saturated import potentials. Regarding available import options, }\DIFaddend the
windows for cost savings are \DIFdelbegin \DIFdel{much smaller }\DIFdelend \DIFaddbegin \DIFadd{more limited }\DIFaddend if only subsets \DIFdelbegin \DIFdel{of import options }\DIFdelend are available
(\sfigref{fig:si:volume-subsets}). However, up to an import volume of \DIFdelbegin \DIFdel{2000~TWh
}\DIFdelend \DIFaddbegin \DIFadd{1500~TWh
for the central cost estimate}\DIFaddend , excluding electricity imports \DIFdelbegin \DIFdel{would not }\DIFdelend \DIFaddbegin \DIFadd{or constraining
imports to methanol and Fischer-Tropsch fuels only, would not substantially
}\DIFaddend diminish the cost-saving potential\DIFdelbegin \DIFdel{substantially}\DIFdelend .

\DIFaddbegin \begin{figure}
    \caption{\textbf{Layout of European energy infrastructure for different import scenarios.}
        Left column shows the regional electricity supply mix (pies), added HVDC
        and HVAC transmission capacity (lines), and the siting of battery
        storage (choropleth). Right column shows the hydrogen supply (top half
        of pies) and consumption (bottom half of pies), net flow and direction
        of hydrogen in newly built pipelines (lines), and the siting of hydrogen
        storage subject to geological potentials (choropleth). Total volumes of
        transmission expansion are given in TWkm, which is the sum product of
        the capacity and length of individual connections. The half circle in
        the Bay of Biscay indicates the imports of
        hydrogen derivatives that are not spatially resolved: ammonia, steel, HBI, methanol,
        Fischer-Tropsch fuels. Hydrogen imports are shown at the entry points.
        Maps for more scenarios are included in Supplementary Figs.~25 to 27. }
    \label{fig:import-infrastructure}
\end{figure}
\DIFaddend 

\DIFdelbegin \section*{\DIFdel{Interactions of import strategy \& domestic infrastructure}}
%DIFAUXCMD
\DIFdelend \DIFaddbegin \subsection*{\DIFadd{Interactions of import strategy \& domestic infrastructure}}
\DIFaddend 

Across the range of import scenarios analysed, we find that the decision which
import vectors are used strongly affects domestic energy infrastructure needs
(\cref{fig:import-infrastructure}).

%%% self-sufficiency %%%

In the fully self-sufficient European energy supply scenario, we see large \DIFdelbegin \DIFdel{\mbox{power-to-X} }\DIFdelend \DIFaddbegin \DIFadd{PtX
}\DIFaddend production within Europe to cover the demand for hydrogen and hydrogen
derivatives in steelmaking, \DIFaddbegin \DIFadd{fertilizers, }\DIFaddend high-value chemicals, green shipping\DIFaddbegin \DIFadd{,
}\DIFaddend and aviation fuels. Production sites are concentrated \DIFdelbegin \DIFdel{in Southern Europe for
solar-based electrolysis and the broader North Sea region for }\DIFdelend \DIFaddbegin \DIFadd{mainly in and around the
North and Baltic Seas using }\DIFaddend wind-based electrolysis \DIFdelbegin \DIFdel{. The steel and ammonia industries relocate to the periphery of
Europe in
Spain and Scotland, where hydrogen is cheap and abundant}\DIFdelend \DIFaddbegin \DIFadd{and some additional hubs in
Southern Europe using solar-based electrolysis}\DIFaddend . Electricity grid reinforcements\DIFaddbegin \DIFadd{,
representing around 50\% of the current transmission capacity, }\DIFaddend are focused in
Northwestern Europe\DIFdelbegin \DIFdel{and }\DIFdelend \DIFaddbegin \DIFadd{, with numerous }\DIFaddend long-distance HVDC connections\DIFaddbegin \DIFadd{, }\DIFaddend but are
broadly distributed overall.
\DIFdelbegin \DIFdel{Hydrogen }\DIFdelend \DIFaddbegin 

%DIF > %% industry relocation and hydrogen network %%%

\DIFadd{With a total of 57~TWkm, the hydrogen }\DIFaddend pipeline build-out is \DIFdelbegin \DIFdel{strongest in Spain and France to transport hydrogen from the Southern production
hubs to
fuel synthesis sites. Most of these pipelines are used unidirectionally,with bidirectional usage where pipelines link hydrogen production and }\DIFdelend \DIFaddbegin \DIFadd{smaller, mostly
serving regional connections. For several reasons, it is also considerably
smaller than the 204-306~TWkm observed previously in Neumann et
al.~}\cite{neumannPotentialRoleHydrogen2023} \DIFadd{or the European Hydrogen Backbone
reports}\cite{gasforclimateEuropeanHydrogen2022} \DIFadd{which envisioned a similar order
of magnitude. Besides assumed full electrification of heavy-duty road
transport}\cite{} \DIFadd{and assuming low CO$_2$ transport costs from point sources to
}\DIFaddend low-cost \DIFdelbegin \DIFdel{geological storage sites (for instance, between Greece and Italy and Southern
Spain )}\DIFdelend \DIFaddbegin \DIFadd{hydrogen sites,}\cite{hofmannH2CO2Network2024} \DIFadd{one reason is the
considered relocation of steel and ammonia production to where hydrogen is cheap
and abundant, reducing the need to transport hydrogen
(}\sfigref{fig:si:relocation}\DIFadd{). Not considering relocation of primary steel and
ammonia production would result in a slightly larger hydrogen network of 69~TWkm
(}\sfigref{fig:si:infra-b}\DIFadd{), while increasing system costs by }\bneuro{22} \DIFadd{(2.7\%)
in the no-imports scenario. With permitted relocation of ammonia and steel
production, primary steel production shifts to the British Isles and Spain while
ammonia production moves to the Nordic-Baltic region. Both sectors become more
strongly localized, with individual regions capturing a market share surpassing
30\%}\DIFaddend .

%%% flexible imports %%%

Considering imports of renewable electricity, green hydrogen, and electrofuels
substantially alters the \DIFdelbegin \DIFdel{infrastructure buildout }\DIFdelend \DIFaddbegin \DIFadd{magnitude of energy infrastructure }\DIFaddend in Europe. Imports
displace much of the European power-to-X production capacities and,
particularly, domestic solar energy generation in Southern Europe. \DIFaddbegin \DIFadd{Much of the
remaining derivative fuel synthesis in Southern Spain uses imported hydrogen,
assuming the delivery of captured CO$_2$ from other parts of Europe at low
cost.}\cite{hofmannH2CO2Network2024} \DIFaddend In contrast, the British Isles retain some
domestic electrolyser capacities to produce synthetic \DIFdelbegin \DIFdel{methane
locally, also leveraging the Sabatier process's waste heat. The electricity
imports are distributed evenly between the North African countries Algeria, Libya, and Tunisia and across multiple entry points in Spain, France, Italy and Greece. This }\DIFdelend \DIFaddbegin \DIFadd{fuels locally. Electricity
imports of 131~TWh, compared to total imports of 1609~TWh, mainly enter from
Tunisia at multiple nodes in Mallorca, Corsica, Sardinia, Sicily, and mainland
Italy. This distribution }\DIFaddend facilitates grid integration without strong
reinforcement needs \DIFdelbegin \DIFdel{.
Electricity imports are also optimised to achieve higher utilisation rates above
70\% for the HVDC import connections. This is realised by mixing wind and solar
generation for seasonal balancing and using some batteries for short-term
storage (}%DIFDELCMD < \sfigref{fig:si:import-operation}%%%
\DIFdel{)}\DIFdelend \DIFaddbegin \DIFadd{in the Italian peninsula}\DIFaddend .

While the \DIFdelbegin \DIFdel{amount and locations }\DIFdelend \DIFaddbegin \DIFadd{broad regions }\DIFaddend of domestic power grid reinforcements are not
significantly affected by the import of electricity and other fuels, the \DIFdelbegin \DIFdel{extent
of }\DIFdelend \DIFaddbegin \DIFadd{volume
of power grid expansion is reduced by 20\%. The reduction in network
infrastructure is even more pronounced with the hydrogen network; }\DIFaddend the hydrogen
network \DIFdelbegin \DIFdel{is halved and its routing is significantly altered}\DIFdelend \DIFaddbegin \DIFadd{size is reduced by 70\% with many of the North and East European
connections omitted}\DIFaddend . Compared to the self-sufficiency scenario, the cost-benefit
of the hydrogen network shrinks from \DIFdelbegin %DIFDELCMD < \bneuro{11} %%%
\DIFdel{(1.3}\DIFdelend \DIFaddbegin \bneuro{3} \DIFadd{(0.4}\DIFaddend \%) to \DIFdelbegin %DIFDELCMD < \bneuro{3} %%%
\DIFdel{(0.4}\DIFdelend \DIFaddbegin \DIFadd{less than }\bneuro{1}
\DIFadd{(0.1}\DIFaddend \%). This is caused by substantial amounts of \DIFdelbegin \DIFdel{methanol imports that diminish }\DIFdelend \DIFaddbegin \DIFadd{hydrogen derivative imports or
direct processing of imported hydrogen at the entry points, which diminishes }\DIFaddend the
demand for hydrogen in Europe and, hence, the need to transport it. \DIFdelbegin \DIFdel{In combination with the steel
and ammonia industry relocation, longer hydrogen pipeline connections are then predominantly built to meet hydrogen CHP demands to bring electricity and heat to renewables-poor and grid-poor regions in Eastern Europe and Germany. Moreover, the hydrogen network helps absorb inbound hydrogen in South and Southeast Europe, transporting some hydrogen, which is not directly used for
fuel synthesis at the entry points, to neighbouring regions}\DIFdelend \DIFaddbegin \DIFadd{With further
10\% cheaper carbonaceous fuel imports, the hydrogen network would then shrink
to 9~TWkm (}\sfigref{fig:si:infra-c}\DIFadd{).
}

%DIF > %% backup power and power-to-X flexibility value %%%

\DIFadd{Changes in the magnitude of domestic PtX production also affect Europe's backup
capacity needs. Less PtX production means lower wind and solar capacities, which
reduces the amount of available generation in dark wind lulls. Next to energy
storage and demand-side management of electric vehicles and heat pumps, the
operational flexibility of electrolysers and derivative fuel production yields
significant benefits for integrating variable wind and solar feed-in and reduces
reserve capacity requirements. In a theoretical scenario without imports where
all PtX processes must run inflexibly at full capacity, system costs rise by
8.8\%. Between the main scenarios with and without imports, we observe that as
imported fuels displace some flexible domestic power-to-X, domestic thermal
backup capacities increase from 129~GW$_\text{el.}$ (no imports allowed) to
276~GW$_\text{el.}$ (all imports allowed). Instead of curtailing the domestic
production of electrofuels, backup power plants need to be dispatched. Most of
these power plants are CHPs fuelled by fossil gas, providing backup heat
alongside backup power when electricity prices are high during winter
(}\sfigref{fig:si:backup-power}\DIFadd{). The resulting emissions are then compensated
elsewhere in the system through biogenic carbon dioxide removal
(}\sfigref{fig:si:balances-b}\DIFadd{). Spatially, these are distributed across Central
Europe, while batteries provide backup power in Southern Europe
(}\sfigref{fig:si:backup-power-map}\DIFadd{). The model leverages Europe's extensive
power grid to widely distribute centralised backup power, even though, in
reality, individual nations may prefer maintaining domestic reserve capacities}\DIFaddend .

%DIF < %% overarching trends %%%
%DIF > %% waste heat usage potential %%%

A further observation is the high \DIFdelbegin \DIFdel{value of power-to-X production for system
integration and the role of waste heat in the siting of }\DIFdelend \DIFaddbegin \DIFadd{potential value of PtX waste heat and
its role in siting }\DIFaddend fuel synthesis plants (\sfigref{fig:si:infra-b}\DIFdelbegin \DIFdel{). Using the process waste heat }\DIFdelend \DIFaddbegin \DIFadd{,
}\sfigref{fig:si:infra-d}\DIFadd{). Alongside the flexible operation of electrolysis to
integrate variable wind and solar feed-in and the broad availability of
industrial and biogenic carbon sources in Europe, waste heat usage }\DIFaddend in district
heating networks \DIFdelbegin \DIFdel{with seasonal thermal storage generates notable }\DIFdelend \DIFaddbegin \DIFadd{is a potential revenue stream that could make electricity and
hydrogen imports with subsequent domestic conversion more attractive relative to
the direct import of derivative products. Our default assumption that only 25\%
of the waste heat can be utilised stems from potential challenges in co-locating
PtX plants with district heating networks within the 115 model regions. If all
waste heat could be leveraged, notable system }\DIFaddend cost savings of \DIFdelbegin %DIFDELCMD < \bneuro{11-21} %%%
\DIFdel{(1.3-2.6\%) . Consequently, savings are lower when imports
displace domestic PtX infrastructure}\DIFdelend \DIFaddbegin \bneuro{20}
\DIFadd{(2.4\%) could be achieved in the no-imports scenario compared to a scenario
where waste heat is fully vented}\DIFaddend . To realise these benefits, \DIFdelbegin \DIFdel{PtX facilities
}\DIFdelend \DIFaddbegin \DIFadd{Fischer-Tropsch and
Haber-Bosch plants }\DIFaddend tend to be located \DIFdelbegin \DIFdel{in densely populated areas }\DIFdelend \DIFaddbegin \DIFadd{where space heating demand is high
}\DIFaddend (e.g.~Paris or Hamburg) \DIFdelbegin \DIFdel{, which
drives part of the the hydrogen
network . Notably, because of the waste heat
produced in Fischer-Tropsch and Sabatier plants, these tend to locate where
space heating demand is high}\DIFdelend \DIFaddbegin \DIFadd{(}\sfigref{fig:si:infra-b}\DIFadd{), which increases hydrogen
network build-out to 98~TWkm (+72\%) compared to the reference scenario with
25\% waste heat utilisation}\DIFaddend . This is not the case for methanolisation plants,
which have lower waste heat potential.
\DIFdelbegin \DIFdel{Alongside the flexible operation of electrolysis to integrate variable wind and solar feed-in and the broad
availability of industrial and
biogenic carbon sources in Europe, waste heat
usage is a key factor that makes electricity and hydrogen imports with
subsequent domestic conversion more attractive relative to the direct import of derivative products.Infrastructure layouts for further import scenarios are presented in }%DIFDELCMD < \sfigref{fig:si:infra-b} %%%
\DIFdel{to }%DIFDELCMD < \sfigref{fig:si:infra-d}%%%
\DIFdelend \DIFaddbegin 

\subsection*{\DIFadd{Causes of import cost variations and their effect}}


\begin{table}
    \fullwidthfigure{
    \footnotesize
    \centering
    \begin{tabular}{lrrrr}
        \toprule
        Cost factor & Absolute change & Unit & Relative change & Unit\\
        \midrule
        Higher WACC of 12\% abroad (e.g.~high project risk) & +48.8 & \euro{}~MWh$^{-1}$  &
        +38.0 & \% \\
        Higher WACC of 10\% abroad (e.g.~high project risk) & +28.6 & \euro{}~MWh$^{-1}$  &
        +22.3 & \% \\
        Higher WACC of 8\% abroad (e.g.~high project risk) & +9.2 & \euro{}~MWh$^{-1}$  & +7.2
        & \% \\
        Higher direct air capture investment cost abroad (+200\%) & +44.3 & \euro{}~MWh$^{-1}$
        & +34.5 & \% \\
        Higher direct air capture investment cost abroad (+100\%) & +22.3 & \euro{}~MWh$^{-1}$
        & +17.4 & \% \\
        Higher direct air capture investment cost abroad (+50\%) & +11.2 & \euro{}~MWh$^{-1}$
        & +8.7 & \% \\
        Higher electrolysis investment cost abroad (+50\%) & +17.3 & \euro{}~MWh$^{-1}$  &
        +13.5 & \% \\
        % Argentina and Chile not available for export & +10.1 & \euro{}~MWh$^{-1}$  &
        % +9.2 & \% \\
        \midrule
        Lower WACC of 3\% abroad (e.g.~government guarantees) & -33.4 & \euro{}~MWh$^{-1}$  &
        -26.0 & \% \\
        Lower WACC of 5\% abroad (e.g.~government guarantees) & -17.5 & \euro{}~MWh$^{-1}$  &
        -13.6 & \% \\
        Lower WACC of 6\% abroad (e.g.~government guarantees) & -8.9 & \euro{}~MWh$^{-1}$  &
        -6.9 & \% \\
        Lower electrolysis investment cost abroad (-50\%) & -18.4 & \euro{}~MWh$^{-1}$  &
        -14.3 & \% \\
        Sell excess curtailed electricity at 50 \euro{}~MWh$^{-1}$ abroad & -8.3 & \euro{}~MWh$^{-1}$  &
        -6.5 & \% \\
        Sell excess curtailed electricity at 30 \euro{}~MWh$^{-1}$ abroad & -4.6 & \euro{}~MWh$^{-1}$  &
        -3.6 & \% \\
        Sell excess curtailed electricity at 10 \euro{}~MWh$^{-1}$ abroad & -1.5 & \euro{}~MWh$^{-1}$  &
        -1.2 & \% \\
        Buy available biogenic or cycled \ce{CO2} for 50 \euro{}~t$^{-1}$ abroad & -20.1 &
        \euro{}~MWh$^{-1}$  & -15.6 & \% \\
        Buy available biogenic or cycled \ce{CO2} for 75 \euro{}~t$^{-1}$ abroad & -13.7 &
        \euro{}~MWh$^{-1}$  & -10.7 & \% \\
        Buy available biogenic or cycled \ce{CO2} for 100 \euro{}~t$^{-1}$ abroad & -7.2 &
        \euro{}~MWh$^{-1}$  & -5.6 & \% \\
        Availability of geological hydrogen storage at 2.1 \euro{}/kWh
        (reduction by 95.5\%) & -5.1
        & \euro{}~MWh$^{-1}$  & -4.0 & \% \\
        Sell power-to-X waste heat at 10 \euro{}~MWh$^{-1}$ abroad  &
        -7.8 & \euro{}~MWh$^{-1}$  & -6.1 & \% \\
        Sell power-to-X waste heat at 5 \euro{}~MWh$^{-1}$ abroad &
        -6.9 & \euro{}~MWh$^{-1}$  & -5.4 & \% \\
        Highly flexible operation of Fischer-Tropsch synthesis (20\% minimum
        part-load) & -3.6 & \euro{}~MWh$^{-1}$  & -2.8 & \% \\
        \bottomrule
    \end{tabular}
    \caption{\textbf{Examples for potential import cost increases or decreases.}
    The table presents cost sensitivities in absolute and relative terms based
    on the supply chain for producing Fischer-Tropsch fuels in Southern
    Argentina for export to Europe (Portugal). The reference fuel import cost
    for this case is 128.5~\euro{}~MWh$^{-1}$.}
    \label{tab:cost-uncertainty}
    }
\end{table}

\DIFadd{In \mbox{%DIFAUXCMD
\cref{tab:cost-uncertainty}}\hskip0pt%DIFAUXCMD
, we present a breakdown of some potential causes
for import cost variations compared to domestic supply chains relating to
technology costs, financing costs, excess power and heat revenues, fuel
synthesis flexibility, and the availability of geological hydrogen storage and
alternative sources of CO$_2$.
}

\DIFadd{For example, we show that a higher weighted average cost of capital (WACC) than
the uniformly applied 7\%, e.g.~due to higher project financing risks, and lower
WACC, e.g.~due to the government-backing of projects, strongly affect import
costs.}\cite{calcaterraReducingCostCapital2024} \DIFadd{An increase or decrease by just
one percentage point already alters the unit costs by around $\pm$7\%. Likewise,
technology cost variations abroad for electrolysers and DAC units have a strong
influence. Biogenic CO$_2$ -- or fossil CO$_2$ from industrial processes that is
largely cycled between use and synthesis and, hence, not emitted to the
atmosphere -- can reduce the levelised fuel cost by 16\% if it can be provided
for 50~}\euro{}\DIFadd{~t$^{-1}$.
}

\DIFadd{By default, we assume islanded fuel synthesis sites, which causes curtailment
rates of 8\%. If surplus electricity production could be sold and absorbed by
the local power grid in exporting regions, additional cost reductions could be
achieved. Furthermore, process integration with waste heat usage and flexible
operation can also reduce fuel cost by 3-6\%. Import costs are also reduced by
4\% where geological hydrogen storage is available by reducing the need for
flexible \mbox{power-to-X} operation.
}

\DIFadd{In contrast, the cost impact is low if the cheapest exporting region withdraws
from the market. Within a cost premium of 10\% in relation to the lowest cost
exporting region, Chile, ten other regions could step in if these regions were
unavailable for exports (}\sfigref{fig:si:isc-meoh-ft}\DIFadd{)}\DIFaddend .


\section*{Discussion\DIFdelbegin \DIFdel{and conclusions}\DIFdelend }
\label{sec:discussion}

Our analysis offers insights into how renewable energy imports might reduce
overall systems costs and interact with European energy infrastructure. Our
results show that imports of green energy reduce \DIFaddbegin \DIFadd{the }\DIFaddend costs of a carbon-neutral
European energy system by \DIFdelbegin %DIFDELCMD < \bneuro{39} %%%
\DIFdel{(5}\DIFdelend \DIFaddbegin \bneuro{37} \DIFadd{(4.4}\DIFaddend \%), noting, however, that the
uncertainty range is considerable. While we find that some imports are
\DIFdelbegin \DIFdel{robustly
beneficial }\DIFdelend \DIFaddbegin \DIFadd{beneficial within a $\pm$20\% variation of import costs}\DIFaddend , system cost savings
range between 1\% and \DIFdelbegin \DIFdel{14\%depending on the
import costs. What is consistent , however, }\DIFdelend \DIFaddbegin \DIFadd{10\%. However, what is consistent within this range }\DIFaddend are
the diminishing \DIFdelbegin \DIFdel{return }\DIFdelend \DIFaddbegin \DIFadd{returns }\DIFaddend of energy imports for larger quantities, with peak cost
savings below imports of \DIFdelbegin \DIFdel{4000}\DIFdelend \DIFaddbegin \DIFadd{3000}\DIFaddend ~TWh/a \DIFaddbegin \DIFadd{(equivalent to 90~Mt of hydrogen)}\DIFaddend . We also find
that there is value in pursuing some \mbox{power-to-X} production in Europe as a
source of flexibility for wind and solar integration and as a \DIFaddbegin \DIFadd{potential }\DIFaddend source
of waste heat for district heating networks. Another \DIFdelbegin \DIFdel{location
factor in favour of }\DIFdelend \DIFaddbegin \DIFadd{siting factor favouring
}\DIFaddend European \mbox{power-to-X} is the wide availability of sustainable biogenic and
industrial carbon sources, which helps \DIFdelbegin \DIFdel{to }\DIFdelend reduce reliance on \DIFaddbegin \DIFadd{more }\DIFaddend costly direct air
capture.

\DIFaddbegin \DIFadd{In relation to other studies, our assumed import costs for different carriers
mostly conform to the interquartile ranges for 2050 of the meta-study by Genge
et al.~}\cite{gengeSupplyCostsGreen2023} \DIFadd{for imports to Europe. Some higher
costs, e.g.~for ammonia, can be attributed to our updated electrolyser cost
assumptions (950}\euro{}\DIFadd{~kW$_\text{el.}^{-1}$ in 2040), which reflect recent
market developments.}\cite{ieaGlobalHydrogenReview2024} \DIFadd{Also for steel imports,
our central cost estimate of 531~}\euro{}\DIFadd{~t$^{-1}$ for the lowest-cost exporter
is positioned between studies with lower}\cite{lopezDefossilisedSteel2023} \DIFadd{and
higher}\cite{verpoortImpactGlobalHeterogeneity2024} \DIFadd{cost estimates. Among other
studies investigating the relationship between energy imports and the European
energy system, several analyses report lower import shares in the range of
10-20\% of total hydrogen
supply.}\cite{seckHydrogenDecarbonization2022,frischmuthHydrogenSourcingStrategies2022,kountourisUnifiedEuropeanHydrogen2024}
\DIFadd{For instance, Kountouris et al.~}\cite{kountourisUnifiedEuropeanHydrogen2024} \DIFadd{see
limited hydrogen imports of 182~TWh~a$^{-1}$ from the Maghreb region and Ukraine
despite favourable import costs of 33~}\euro{}\DIFadd{~MWh$_{H_2}^{-1}$. Conversely,
Wetzel et al.~}\cite{wetzelGreenEnergy2023a} \DIFadd{find higher import shares of 53\%
for methane and 43\% for hydrogen. The latter closely aligns with our 49\%
import share for hydrogen. Wetzel et al.~}\cite{wetzelGreenEnergy2023a} \DIFadd{find that
imports reduce system costs by 2.8\%, which is also comparable to our 2.4\%
system cost reduction when only direct hydrogen imports are considered. The most
pronounced import dependency we found in Schmitz et
al.}\cite{schmitzImplicationsHydrogenImport2024a}\DIFadd{, with import shares beyond 90\%
for Germany. Results in Kountouris et
al.~}\cite{kountourisUnifiedEuropeanHydrogen2024} \DIFadd{further substantiate our
finding that derivative imports and demand relocation could diminish hydrogen
network benefits.
}

\DIFadd{Several limitations of our study should be noted. First, the optimization
results represent a long-term equilibrium that disregards potential
transition-related infrastructure lock-ins or mid-term ramp-up constraints of
export capacities or domestic infrastructure development. A further limitation
is that our cost-based analysis of imports, which best reflects long-term
bilateral purchase agreements, neglects price impacts of intensifying global
competition for green fuel imports and exports.}\cite{galimovaGlobalTrading2023a}
\DIFadd{Besides unclear market developments, local challenges in exporting regions such
as public acceptance for export-oriented energy
projects}\cite{ishmamMappingLocalGreen2024} \DIFadd{and potential water
scarcity}\cite{franzmannGreenHydrogenCostpotentials2023,terlouwFutureHydrogenEconomies2024}
\DIFadd{to produce large amounts of hydrogen in renewable-rich but arid regions are not
addressed. Furthermore, the model's lack of spatial resolution for CO$_2$ means
that carbonaceous hydrogen derivatives are sited where H$_2$ is cheapest,
implicitly assuming that the CO$_2$ from biogenic or industrial sources can be
transported there. However, such required CO$_2$ pipelines could be built at
relatively low additional system cost.}\cite{hofmannH2CO2Network2024} \DIFadd{In the
context of carbon management, more lenient assumptions on sustainable biofuel
potentials, allowed levels of geological carbon sequestration, or plastic
landfill could alter the results, shifting the system away from synthetic
electrofuels towards more fossil fuel use with carbon capture or carbon dioxide
removal.}\cite{hofmannH2CO2Network2024,millingerDiversityBiomassUsage2023}

\DIFaddend Overall, we find that the import vectors used strongly affect domestic
infrastructure needs. For example, only a smaller hydrogen network would be
required if hydrogen derivatives were largely imported \DIFdelbegin \DIFdel{. This underscores }\DIFdelend \DIFaddbegin \DIFadd{and the domestic ammonia
and steel industry is allowed to relocate. We also identify higher electricity
backup requirements in the absence of large power-to-X flexibilities. These
findings underscore }\DIFaddend the importance of coordination between energy import
strategies and infrastructure policy decisions. Our results present a
quantitative basis for further discussions about the trade-offs between system
cost, carbon neutrality, public acceptance, energy security, infrastructure
buildout\DIFaddbegin \DIFadd{, }\DIFaddend and imports.

The small differences in cost \DIFaddbegin \DIFadd{observed }\DIFaddend between some scenarios \DIFdelbegin \DIFdel{is }\DIFdelend \DIFaddbegin \DIFadd{are }\DIFaddend particularly
relevant because factors other than pure costs might then drive the designs of
import strategies. The relatively limited cost benefit of imports and \DIFdelbegin \DIFdel{value chain
reordering, }\DIFdelend \DIFaddbegin \DIFadd{offshoring
of industrial production }\DIFaddend may speak against \DIFdelbegin \DIFdel{pursuing this avenue. A desire for }\DIFdelend \DIFaddbegin \DIFadd{imports. Concerns about }\DIFaddend energy
sovereignty would motivate more domestic supply and diversified imports. For
instance, \DIFdelbegin \DIFdel{focusing on ship-bourne imports }\DIFdelend \DIFaddbegin \DIFadd{shipborne imports of hydrogen derivatives }\DIFaddend would reduce pipeline
lock-in and mitigate the risks of sudden supply disruptions and abuse of market
power. \DIFdelbegin \DIFdel{Focussing on carriers that are already a globally traded commodity }\DIFdelend \DIFaddbegin \DIFadd{From a practical perspective, it }\DIFaddend may also be more appealing \DIFdelbegin \DIFdel{. Producing renewable energy locally would bring value creation
and jobs to Europe that are currently outsourced to fossil fuel producers
abroad. For similar reasons, the import of intermediary raw products like sponge
iron, which represents the most energy-intensive part of the steel value chain,
could also be a relevant option.
}\DIFdelend \DIFaddbegin \DIFadd{to focus on
carriers that are already globally traded commodities and to prefer
infrastructure offering quick deployment.
}\DIFaddend 

\DIFdelbegin \DIFdel{There is also a social dimension to the import strategy and the question of how
fast the associated infrastructure can get built.
Policymaking }\DIFdelend \DIFaddbegin \DIFadd{Policymakers }\DIFaddend in Europe might prefer \DIFaddbegin \DIFadd{such }\DIFaddend easy-to-implement systems featuring,
for instance, lower domestic infrastructure requirements, reuse of existing
infrastructure, lower technology risk, and reduced land usage for broader public
support than the most cost-effective solution. \DIFaddbegin \DIFadd{Moreover, policies favoring local
energy supply chains and importing intermediary products like sponge iron could
preserve European jobs while outsourcing only the most energy-intensive
processes. }\DIFaddend However, in \DIFdelbegin \DIFdel{outsourcing }\DIFdelend \DIFaddbegin \DIFadd{shifting }\DIFaddend potential land use and infrastructure conflicts
to abroad, \DIFaddbegin \DIFadd{where population densities are often lower, }\DIFaddend potential exporting
countries must weigh the prospect of economic development against internal
social and environmental \DIFdelbegin \DIFdel{issues.}\DIFdelend \DIFaddbegin \DIFadd{concerns, particularly in countries with a history of
colonial exploitation.}\cite{tunnGreenHydrogenTransitions2024} \DIFaddend Ultimately,
Europe's energy strategy must balance cost savings from green energy and
material imports with broader concerns like geopolitics, economic development,
public opinion\DIFaddbegin \DIFadd{, }\DIFaddend and the willingness of potential exporting countries in order to
ensure a swift, secure\DIFaddbegin \DIFadd{, }\DIFaddend and sustainable energy future. Our research shows that
there is maneuvering space to accommodate such non-cost concerns.


\section*{Methods}

\label{sec:methods}



\subsection*{\DIFdelbegin \DIFdel{Modelling }\DIFdelend \DIFaddbegin \DIFadd{PyPSA-Eur: Overview }\DIFaddend of \DIFdelbegin \DIFdel{the }\DIFdelend European energy system \DIFaddbegin \DIFadd{model}\DIFaddend }

For our analysis, we use the European sector-coupled high-resolution energy
system model PyPSA-Eur\cite{horschPyPSAEurOpen2018a} based on the open-source
modelling framework PyPSA\cite{brownPyPSAPython2018} (Python for Power System
Analysis)\DIFdelbegin \DIFdel{in a setup similar to Neumann et al.~}%DIFDELCMD < \cite{neumannPotentialRole2023}%%%
\DIFdelend , covering the energy demands of all sectors including electricity,
heat, transport, industry, agriculture, as well as \DIFdelbegin \DIFdel{international shipping }\DIFdelend \DIFaddbegin \DIFadd{non-energy feedstock demands,
international shipping, }\DIFaddend and aviation. \DIFaddbegin \DIFadd{An overview of considered supply,
consumption, and balancing technologies per carrier is shown in
}\sfigref{fig:si:supply-consumption-options}\DIFadd{.
}\DIFaddend 

The model simultaneously optimises spatially explicit investments and the
operation of generation, storage, conversion\DIFaddbegin \DIFadd{, }\DIFaddend and transmission assets to
minimise total system costs in a \DIFaddbegin \DIFadd{single }\DIFaddend linear optimisation problem, which
\DIFaddbegin \DIFadd{assumes perfect operational foresight and }\DIFaddend is solved with \textit{Gurobi
\DIFaddbegin \DIFadd{v11}\DIFaddend }.\cite{gurobi} \DIFaddbegin \DIFadd{To manage computational complexity, no pathways with multiple
investment periods are calculated, but overnight scenarios targeting net-zero
CO$_2$ emissions. }\DIFaddend The capacity expansion is based on technology cost and
efficiency projections for \DIFdelbegin \DIFdel{the year 2030, many of which are taken from the
technology catalogue of the Danish Energy Agency.}%DIFDELCMD < \cite{DEA} %%%
\DIFdel{Choosing projections
for the year 2030 for a net-zero carbon emission scenarios more likely to be
reached by mid-century acknowledges }\DIFdelend \DIFaddbegin \DIFadd{2040 (see `Data availability'), acknowledging }\DIFaddend that
much of the required infrastructure must be constructed well \DIFdelbegin \DIFdel{in advance of reaching this goal.}\DIFdelend \DIFaddbegin \DIFadd{before reaching
net-zero emissions.
}

\DIFadd{Existing hydro-electric power plants}\cite{gotzensPerformingEnergy2019} \DIFadd{are
included, as well as nuclear power plants built before 1990 or currently under
construction according to Global Energy Monitor's Global Nuclear Plant Tracker
(52~GW total of 106~GW in current
operation).}\cite{globalenergymonitorGlobalNuclearPower2024} \DIFadd{While
hydroelectricity is assumed to be non-extendable due to geographic constraints,
additional nuclear capacities can be expanded where cost-effective. We assume
the existing nuclear fleet is operated inflexibly and apply country-specific
historical availability factors from 2021 to
2023.}\cite{internationalatomicenergyagencyPowerReactorInformation2024}

\DIFadd{Temporally, the model is solved with an uninterrupted 4-hourly equivalent
resolution for a single year, using a segmentation clustering approach
implemented in the }\textit{\DIFadd{tsam}} \DIFadd{toolbox on all time-varying
data.}\cite{hoffmannParetooptimalTemporal2022} \DIFadd{While weather variations between
years are not considered for computational reasons, the chosen weather year 2013
is representative in terms of wind and solar
availability.}\cite{gotskeDesigningSectorcoupledEuropean2024}

\DIFaddend Spatially, the model resolves \DIFdelbegin \DIFdel{110 regionsin Europe}\DIFdelend \DIFaddbegin \DIFadd{115 European
regions}\DIFaddend ,\cite{frysztackiStrongEffect2021} covering the European Union, the
United Kingdom, Norway, Switzerland\DIFaddbegin \DIFadd{, }\DIFaddend and the Balkan countries without Malta and
Cyprus. \DIFdelbegin \DIFdel{Temporally, the model is solved with an uninterrupted
4-hourly equivalent resolutionfor the weather year 2013, using a segmentation
clustering approach implemented in the }\textit{\DIFdel{tsam}}
%DIFAUXCMD
\DIFdel{toolbox. }%DIFDELCMD < \cite{hoffmannParetooptimalTemporal2022} %%%
\DIFdel{In terms of investment periods, no pathway optimisation is conducted,
but a greenfield approach is pursued
except for existing hydro-electricity and transmission infrastructure targeting
net-zero CO$_2$ emissions. }\DIFdelend \DIFaddbegin \DIFadd{For computational reasons, only electricity, heat, and hydrogen are
modelled at high spatial resolution, while oil, methanol, methane, ammonia, and
carbon dioxide are treated as easily transportable without spatial constraints.
Of the total final energy and non-energy demand
(}\sfigref{fig:si:demand_totals}\DIFadd{), only some demands are spatially fixed
(}\sfigref{fig:si:demands}\DIFadd{). These include electricity for residential, industry,
services, and agriculture; heat; electric vehicles; solid biomass for industry;
and naphtha/methanol feedstocks. Hydrogen demand for steel and ammonia
production is also spatially fixed unless these industries can relocate. In this
case, regionally fixed hydrogen demand becomes negligible. Since the model
optimizes the siting and operation of most fuel synthesis units, many demands
are spatially variable (e.g.~electricity demand for electrolysers or hydrogen
demand for methanolisation).
}

\DIFadd{A mathematical description of PyPSA-Eur can be found in Section S12 in Neumann
et al.}\cite{neumannPotentialRoleHydrogen2023}

\subsection*{\DIFadd{Gas and electricity network modelling}}
\DIFaddend 

Networks are considered for electricity, methane\DIFaddbegin \DIFadd{, }\DIFaddend and hydrogen transport.
\DIFdelbegin %DIFDELCMD < \cite{ENTSOE,plutaSciGRIDGas2022a} %%%
\DIFdel{However, different to Neumann et
al.,}%DIFDELCMD < \cite{neumannPotentialRole2023} %%%
\DIFdel{pipeline retrofitting to hydrogen is
disabled for computational reasons such that all hydrogen
pipelinesare assumed
to be newly built.}\DIFdelend \DIFaddbegin \DIFadd{Existing gas pipelines taken from SciGRID\_gas,}\cite{plutaSciGRIDGas2022a} \DIFadd{can
be repurposed to hydrogen in addition to new hydrogen
pipelines.}\cite{neumannPotentialRoleHydrogen2023} \DIFaddend Data on the gas transmission
\DIFdelbegin \DIFdel{system }\DIFdelend \DIFaddbegin \DIFadd{network }\DIFaddend is further supplemented by the locations of fossil gas extraction sites
and gas storage facilities based on SciGRID\_gas,\cite{plutaSciGRIDGas2022a} as
well as investment costs and capacities of \DIFdelbegin \DIFdel{existing and planned LNG terminals }%DIFDELCMD < \cite{instituteforenergyeconomicsandfinancialanalysisEuropeanLNG2023}
%DIFDELCMD < %%%
\DIFdel{Moreover, a carbon dioxide network is not explicitly co-optimised since CO$_2$
is not spatially resolved in this model version.}%DIFDELCMD < \cite{hofmannDesigningCO22023}
%DIFDELCMD < %%%
\DIFdelend \DIFaddbegin \DIFadd{LNG terminals in operation or under
construction from Global Energy Monitor's Europe Gas
Tracker.}\cite{globalenergymonitorEuropeGasTracker2024} \DIFadd{Geological potentials for
hydrogen storage are taken from Caglayan et
al.,}\cite{caglayanTechnicalPotentialSalt2020} \DIFadd{restricting where this low-cost
storage option is available. In modelling gas and hydrogen flows, we incorporate
electricity demands for compression of 1\% and 2\% per 1000km of the transported
energy, respectively.}\cite{gasforclimateEuropeanHydrogen2021} \DIFadd{Existing
high-voltage grid data is taken from
OpenStreetMap.}\cite{xiongModellingHighVoltageGrid2024} \DIFadd{For HVDC transmission
lines, we assume 2\% static losses at the substations and additional losses of
3\% per 1000km. The losses of high-voltage AC transmission lines are estimated
using the piecewise linear approximation from Neumann et
al.,}\cite{neumannAssessmentsLinear2022} \DIFadd{in addition to applying linearised power
flow equations.}\cite{horschLinearOptimal2018} \DIFadd{Up to a maximum capacity increase
of 30\%
%DIF > , we consider dynamic line rating (DLR), leveraging the cooling effect of
wind and low ambient temperatures to exploit existing transmission assets
fully.}\cite{glaumLeveragingExisting2023} \DIFadd{To approximate $N-1$ resilience,
transmission lines may only be used up to 70\% of their rated dynamic
capacity.}\cite{shokrigazafroudiTopologybasedApproximations12022} \DIFadd{To prevent
excessive expansion of single connections, power transmission reinforcements
between two regions are limited to 15 GW, while an upper limit of 50.7 GW is
placed on hydrogen pipelines, which corresponds to three 48-inch
pipelines.}\cite{gasforclimateEuropeanHydrogen2021}
\DIFaddend 

\DIFdelbegin \DIFdel{The overall annual sequestration of CO$_2$ is limited to 200
Mt$_{\text{CO}_2}$/a. This number allows for sequestering the industry's
unabated fossil emissions (e.g.in the cement industry) while minimising
reliance on carbon removal technologies. The carbon management features of the
model trace the carbon cycles through various conversion stages: industrial
emissions,biomass and gas combustion, carbon capture, storage or long-term
sequestration,
direct air capture, electrofuels, recycling, and waste-to-energy
plants.}\DIFdelend \DIFaddbegin \subsection*{\DIFadd{Wind and solar potentials}}
\DIFaddend 

Renewable potentials and time series for wind and solar electricity generation
are calculated with \textit{atlite},\cite{hofmannAtliteLightweight2021}
considering land eligibility constraints like nature reserves\DIFdelbegin \DIFdel{or }\DIFdelend \DIFaddbegin \DIFadd{, excluded land use
types, topography, bathymetry, and }\DIFaddend distance criteria to settlements. Given low
onshore wind expansion in many European countries in recent
years,\cite{ourworldindataInstalledWind2023} \DIFdelbegin \DIFdel{restrictive
onshore wind expansion potentials are applied, using a }\DIFdelend \DIFaddbegin \DIFadd{a deployment density of }\DIFaddend 1.5
MW\DIFdelbegin \DIFdel{/km$^2$ factor for the eligible land area.Geological potentials for hydrogen storage are taken
from Caglayan et al. }%DIFDELCMD < \cite{caglayanTechnicalPotential2020} %%%
\DIFdelend \DIFaddbegin \DIFadd{~km$^{-2}$ is assumed for eligible land for onshore wind
expansion.}\cite{turkovskaLanduseRequirementsSolar2023a} \DIFadd{For reference, this
assumption leads to an onshore wind potential for Germany of 244~GW. The
temporal renewable generation potential for the available area is then assessed
based on reanalysis weather data, ERA5,}\cite{ecmwf} \DIFadd{and satellite observations
for solar irradiation, SARAH-3,}\cite{pfeifrothSurfaceRadiationData2023} \DIFadd{in
combination with standard solar panel and wind turbine models provided by
}\textit{\DIFadd{atlite}}\DIFadd{.
}

\subsection*{\DIFadd{Biomass potentials}}

\DIFaddend Biomass potentials are restricted to residues from agriculture and forestry, as
well as waste and manure, based on the \DIFaddbegin \DIFadd{regional }\DIFaddend medium potentials specified for
\DIFdelbegin \DIFdel{2030 }\DIFdelend \DIFaddbegin \DIFadd{2050 }\DIFaddend in the JRC-ENSPRESO database.\cite{ruizENSPRESOOpen2019} The finite
\DIFaddbegin \DIFadd{sustainable }\DIFaddend biomass resource can be employed for low-temperature heat provision
in industrial applications, biomass boilers\DIFaddbegin \DIFadd{, }\DIFaddend and CHPs, and \DIFaddbegin \DIFadd{(electro-)}\DIFaddend biofuel
production for use in aviation, shipping\DIFaddbegin \DIFadd{, }\DIFaddend and the chemicals industry.
Additionally, we allow biogas upgrading, including \DIFdelbegin \DIFdel{the capture of }\DIFdelend \DIFaddbegin \DIFadd{capturing }\DIFaddend the CO$_2$
contained in biogas\DIFaddbegin \DIFadd{, which unlocks all considered uses of regular methane
(}\sfigref{fig:si:supply-consumption-options}\DIFadd{)}\DIFaddend . The total assumed bioenergy
potentials are \DIFdelbegin \DIFdel{1569~TWhwith a
carbon content corresponding to
546}\DIFdelend \DIFaddbegin \DIFadd{1730~TWh, which splits into 358~TWh/a for biogas and 1,014~TWh/a
for solid biomass. The total carbon content corresponds to
605}\DIFaddend ~Mt$_{\text{CO}_2}$\DIFdelbegin \DIFdel{/a}\DIFdelend \DIFaddbegin \DIFadd{~a$^{-1}$}\DIFaddend , which is not fully available as a feedstock for
fuel synthesis \DIFaddbegin \DIFadd{or sequestration for negative emissions }\DIFaddend due to imperfect capture
rates of up to 90\%.

\DIFdelbegin \DIFdel{Heating supply technologies like heat pumps, electric boilers, gas boilers,
and
combined heat and power (CHP) plants are endogenously optimised separately for
decentral use and central district heating. District heating networks can
further be supplemented with waste heat from various power-to-X processes
(electrolysis, methanation, methanolisation, ammonia synthesis, Fischer-Tropsch
fuel synthesis) .
}\DIFdelend \DIFaddbegin \subsection*{\DIFadd{Carbon management}}

\DIFadd{The carbon management features of the model trace the carbon cycles through
various conversion stages: industrial emissions, biomass and gas combustion,
carbon capture in numerous applications, direct air capture, intermediate
storage, electrofuels, recycling, landfill or long-term sequestration. The
overall annual sequestration of CO$_2$ is limited to
200~Mt$_{\text{CO}_2}$~a$^{-1}$, similar to the 250~Mt$_{\text{CO}_2}$~a$^{-1}$
highlighted in the European Commission's carbon management
strategy.}\cite{europeancommissionAmbitiousIndustrialCarbon2024} \DIFadd{This number
allows for sequestering the industry's unabated fossil emissions (e.g.~in the
cement industry) while minimising reliance on carbon removal technologies. A
carbon dioxide network topology is not co-optimised since CO$_2$ is not
spatially resolved. This means that the location of biogenic or industrial point
sources of CO$_2$ is not a siting factor that this model version considers for
PtX processes, implicitly assuming that the CO$_2$ would be transported there at
low cost. }\cite{hofmannDesigningCO22023,hofmannH2CO2Network2024}

\subsection*{\DIFadd{Transport sector fuel assumptions}}
\DIFaddend 

While the shipping sector is assumed to use methanol as fuel \DIFdelbegin \DIFdel{, }\DIFdelend \DIFaddbegin \DIFadd{given its high
technology-readiness level compared to hydrogen or
ammonia,}\cite{ieaETPCleanEnergy2024} \DIFaddend land-based transport, including heavy-duty
vehicles, is \DIFdelbegin \DIFdel{deemed }\DIFdelend fully electrified in the presented
scenarios.\DIFdelbegin \DIFdel{Aviation can decide to }\DIFdelend \DIFaddbegin \cite{linkRapidlyDecliningCosts2024} \DIFadd{Aviation can }\DIFaddend use green kerosene
derived from Fischer-Tropsch fuels or methanol\DIFdelbegin \DIFdel{. Besides methanol-to-kerosene, further
usage options for methanol have been added.These include
methanol-to-olefins/aromatics for the production of green plastics, methanol-to-power}\DIFdelend \DIFaddbegin \DIFadd{, owing to lower technology
readiness levels of fuel cell or battery-electric
aircraft.}\cite{ieaETPCleanEnergy2024} \DIFadd{Alternative uses for methanol and
Fischer-Tropsch fuels extend beyond transport, including
power-to-methanol,}\DIFaddend \cite{brownUltralongdurationEnergyStorage2023} \DIFdelbegin \DIFdel{in open-cycle gas
turbines or Allam cycle turbines,and steam reforming of methanol with or
without carbon capture.
For the synthesis of electrofuels, we also account for
potential operational restrictions by considering }\DIFdelend \DIFaddbegin \DIFadd{diesel for
agriculture machinery and as feedstock for high-value chemicals.
}

\subsection*{\DIFadd{Technical constraints of synthetic fuel production}}

\DIFadd{We consider potential flexibility restrictions in the synthesis processes to
obtain more realistic operational patterns of green electrofuel synthesis
plants. We apply }\DIFaddend a minimum part load of \DIFdelbegin \DIFdel{30}\DIFdelend \DIFaddbegin \DIFadd{20}\DIFaddend \% for methanolisation and \DIFdelbegin \DIFdel{methanation compared to 70\% for
}\DIFdelend \DIFaddbegin \DIFadd{50\% for
methanation and }\DIFaddend Fischer-Tropsch
synthesis\DIFdelbegin \DIFdel{, both within Europe and abroad.}\DIFdelend \DIFaddbegin \DIFadd{.}\cite{mucciPowerXProcessesBased2023,wentrupDynamicOperationFischerTropsch2022,dieterichPowerliquidSynthesisMethanol2020,mbathaPowermethanolProcessReview2021}
\DIFadd{These `green' options then compete with `blue' and `grey' options, such as steam
methane reforming of fossil gas with or without carbon capture for hydrogen
(}\sfigref{fig:si:supply-consumption-options}\DIFadd{). Some carriers also feature a
biogenic production route (e.g., methane and oil).
}\DIFaddend 

\DIFdelbegin \DIFdel{A further core improvement of the model regards the physical representation of energy transport over long distances. For gas and hydrogen
pipelines, we
incorporate electricity demands for compression of 1\% and 2\% per 1000km
of the transported energy, respectively.
}%DIFDELCMD < \cite{gasforclimateEuropeanHydrogen2021}
%DIFDELCMD < %%%
\DIFdel{For HVDC transmission lines, we assume 2\% static losses at the substations and
additional losses of 3\% per 1000km. The losses of high-voltage AC transmission
lines are estimated using a piecewise linear approximation as proposed in
Neumann et al.,}%DIFDELCMD < \cite{neumannAssessmentsLinear2022} %%%
\DIFdel{in addition to the linearised
power flow equations.}%DIFDELCMD < \cite{horschLinearOptimal2018} %%%
\DIFdel{Up }\DIFdelend \DIFaddbegin \subsection*{\DIFadd{Heating sector modelling and PtX waste heat}}

\DIFadd{Heating supply technologies like heat pumps, electric boilers, gas boilers, and
combined heat and power (CHP) plants are endogenously optimised separately for
decentral use and central district heating. District heating shares of demand
are exogenously set }\DIFaddend to a maximum \DIFdelbegin \DIFdel{capacity
increase of 30\% , we consider dynamic line rating (DLR), leveraging the cooling
effect of wind and low ambient temperatures to exploit existing transmission
assets fully. }%DIFDELCMD < \cite{glaumLeveragingExisting2023} %%%
\DIFdel{To approximate N-1 resilience, transmission lines may only be used up to 70\% of their rated dynamic capacity. To prevent excessive expansion of single connections, the expansion of power
transmission lines between two regionsis limited to 15 GW for HVAC and }\DIFdelend \DIFaddbegin \DIFadd{of 60\% of the total urban heat demand with
sufficiently high population density. Besides the options for long-duration
thermal energy storage, district heating networks can further be supplemented
with waste heat from various power-to-X processes: electrolysis, methanation,
ammonia synthesis, and Fischer-Tropsch fuel synthesis. Because the thermal
discharge from the methanol synthesis is primarily used to distillate the
methanol-water output mix,}\cite{brownUltralongdurationEnergyStorage2023} \DIFadd{its
waste heat potential is not considered for district heat. Here, we assume a
utilizable share of waste heat of }\DIFaddend 25\DIFdelbegin \DIFdel{GW
for HVDC lines,
while a similar constraint of 50.7 GW is placed on hydrogen
pipelines, which corresponds to three parallel 48-inch
pipelines.
}%DIFDELCMD < \cite{gasforclimateEuropeanHydrogen2021}
%DIFDELCMD < %%%
\DIFdelend \DIFaddbegin \DIFadd{\%, considering that within the 115 regions,
only a fraction of fuel synthesis plants might be connected to district heating
systems. In further sensitivity analyses, we explore the effect of no or full
waste heat utilisation.
}\DIFaddend 


\DIFdelbegin \DIFdel{Finally}\DIFdelend \DIFaddbegin \subsection*{\DIFadd{Industry relocation modelling for steel and ammonia production}}

\DIFadd{In some scenarios}\DIFaddend , we also \DIFdelbegin \DIFdel{developed the possibility for }\DIFdelend \DIFaddbegin \DIFadd{allow }\DIFaddend the model to relocate the steel and ammonia
industry within Europe \DIFdelbegin \DIFdel{, mainly to level the playing field between
non-European green steel imports and domestic production . This is achieved by
explicitly modelling the cost, efficiency and operation of hydrogen }\DIFdelend \DIFaddbegin \DIFadd{endogenously. This allows the best sites within Europe to
compete with outsourced production abroad. While this captures some of the most
energy-intensive industry sectors, other sectors, like concrete and alumina
production, are not considered for relocation.
}

\DIFadd{Without relocation of steel and ammonia production allowed, the production
volumes of primary steel, by }\DIFaddend direct iron reduction (\DIFdelbegin \DIFdel{H2-DRI}\DIFdelend \DIFaddbegin \DIFadd{DRI}\DIFaddend ) and electric arc
\DIFdelbegin \DIFdel{furnaces }\DIFdelend \DIFaddbegin \DIFadd{furnace }\DIFaddend (EAF), \DIFdelbegin \DIFdel{which can be sited all over
Europe, and the cost to procure iron ore . We further allow the oversizing of steelmaking plants to allow flexible production in response to
the renewables
supply conditions}\DIFdelend \DIFaddbegin \DIFadd{and ammonia for fertilizers, by Haber-Bosch synthesis, are
spatially fixed. This results in exogenous hydrogen demand per region. Total
production volumes are based on current
levels.}\cite{unitedstatesgeologicalsurveyAmmoniaProductionCountry2022,europeancommission.jointresearchcentre.JRCIDEES2021IntegratedDatabase2024}
\DIFadd{For the spatial distribution, we use data on the existing integrated steelworks
listed in Global Energy Monitor's Global Steel Plant Tracker
}\cite{globalenergymonitorGlobalSteelPlant2024} \DIFadd{and manually collected data on
the location and size of ammonia plants in Europe.
}

\DIFadd{With the relocation of steel and ammonia production allowed, the model
endogenously chooses the regional production volumes of primary steel, HBI, and
ammonia, subject to the availability of cheap hydrogen. Thereby, the regional
capacities and operation of Haber-Bosch, DRI, and EAF plants are co-optimised
with the rest of the system, similar to the siting of Fischer-Tropsch or
methanolisation plants. For DRI and EAF, investment costs and specific
requirements for fuels and iron ore are taken from the Steel Sector Transition
Strategy Model (ST-STSM) of the Mission Possible
Partnership.}\cite{missionpossiblepartnershipSteelSectorTransition2022,missionpossiblepartnershipMakingNetZeroSteel2022}\DIFadd{.
and assume steel can be stored and transported without constraints within
Europe. Relocation costs and local job impacts are excluded from the analysis
due to a lack of data.
}

\DIFadd{For both cases, we assume a rise in the steel recycling rate from 40\% today to
70\% in our carbon-neutral
scenarios.}\cite{materialeconomicsIndustrialTransformation20502019} \DIFadd{We assume
that the electric arc furnaces for secondary steel remain, in proportion, at
current locations and do not relocate}\DIFaddend .

\subsection*{\DIFdelbegin \DIFdel{Modelling of import }\DIFdelend \DIFaddbegin \DIFadd{TRACE: Import }\DIFaddend supply \DIFdelbegin \DIFdel{chains and costs}\DIFdelend \DIFaddbegin \DIFadd{chain modelling}\DIFaddend }

\DIFdelbegin %DIFDELCMD < \begin{figure*}
%DIFDELCMD <     \centering
%DIFDELCMD <     \includegraphics[width=.82\textwidth]{static/graphics/sketch2.drawio-2.pdf}
%DIFDELCMD <     %%%
%DIFDELCMD < \caption{%
{%DIFAUXCMD
\textbf{\DIFdelFL{Schematic overview of the import supply chains.}} %DIFAUXCMD
\DIFdelFL{The
    illustration includes key input-output ratios of the different conversion
    processes and the transport efficiencies for the different import vectors.}}
    %DIFAUXCMD
%DIFDELCMD < \label{fig:import-esc-scheme}
%DIFDELCMD < \end{figure*}
%DIFDELCMD < 

%DIFDELCMD < %%%
\DIFdelend The European energy system model is extended with data from the TRACE model \DIFaddbegin \DIFadd{used
in Hampp et al.}\DIFaddend \cite{hamppImportOptions2023} to assess the \DIFaddbegin \DIFadd{unit }\DIFaddend costs of
different vectors for importing green energy and material \DIFdelbegin \DIFdel{into }\DIFdelend \DIFaddbegin \DIFadd{to entry points in
}\DIFaddend Europe from various world regions. For \DIFdelbegin \DIFdel{each vector, we identify locations with existing or planned import
infrastructure where the respective carrier may enter the European energy
system.
}%DIFDELCMD < 

%DIFDELCMD < %%%
\DIFdel{Starting from the methodology by Hampp et al.}%DIFDELCMD < \cite{hamppImportOptions2023}%%%
\DIFdel{, some
adjustments were made to the
original TRACE model. Namely, land availabilityand
wind and solar time-series are determined using
}\textit{\DIFdel{atlite}}%DIFAUXCMD
%DIFDELCMD < \cite{hofmannAtliteLightweight2021} %%%
\DIFdel{instead of }\textit{\DIFdel{GEGIS}}%DIFAUXCMD
\DIFdel{.}%DIFDELCMD < \cite{mattssonAutopilotEnergy2021} %%%
\DIFdel{Techno-economic assumptions
were aligned with those used in the European model, and steel was included as an
energy-intensive material import vector.The exporting countries comprise
Australia, Argentina, Chile, Kazakhstan, Namibia, Turkey,
Ukraine, }\DIFdelend \DIFaddbegin \DIFadd{consistency with }\DIFaddend the \DIFdelbegin \DIFdel{Eastern
United States and Canada, mainland China, and counties in the MENA region}\DIFdelend \DIFaddbegin \DIFadd{European model, the
techno-economic assumptions were aligned, using the same projections for 2040
(see `Data availability`) and a uniform weighted average cost of capital (WACC)
of 7\%.}\cite{lonerganImprovingRepresentationCost2023} \DIFadd{As possible import
vectors, we consider electricity by transmission lines, hydrogen as gas by
pipelines and as a liquid by ship, methane as a liquid by ship, liquid ammonia,
steel and HBI, methanol and Fischer-Tropsch fuels by ship. Liquid organic
hydrogen carriers (LOHC) are not considered as export vectors due to their lower
technology readiness level (TRL) compared to other
vectors.}\cite{irenaGlobalHydrogenTrade2022}

\DIFadd{Our selection of 53 potential exporting regions broadly comprises countries with
favourable wind and solar resources and large enough potentials for substantial
exports above 500~TWh~a$^{-1}$ in addition to domestic consumption. We exclude
some countries due to political instability (e.g.~Sudan, Somalia, Yemen), using
a Fragile States Index}\cite{thefundforpeaceffpFragileStatesIndex2023} \DIFadd{value of
100 as a threshold, or due to severe imposed sanctions (e.g.~Russia, Iran,
Iraq), following the EU Sanctions
Map.}\cite{estonianpresidencyofthecounciloftheeuEUSanctionsMap2024} \DIFadd{Landlocked
countries without access to seaports or realistic pipeline connections are
excluded. For landlocked regions within pipeline reach, we only exclude
shipborne vectors. Some large countries are split into multiple subregions for
a more differentiated view (e.g.~USA, Argentina, Brazil, and China). The
resulting regions are marked in \mbox{%DIFAUXCMD
\cref{fig:options:global}}\hskip0pt%DIFAUXCMD
}\DIFaddend .

%DIF >  renewable potentials minus domestic demand
\DIFaddbegin 

\DIFaddend To determine the levelised cost of energy for exports, the methodology first
assesses the regional potentials for \DIFdelbegin \DIFdel{wind and solar energy. A regional
electricity cost supply curve is determined }\DIFdelend \DIFaddbegin \DIFadd{solar, onshore, and offshore wind energy.
These potentials and time series are calculated using
}\textit{\DIFadd{atlite}}\DIFaddend ,\DIFdelbegin \DIFdel{from which }\DIFdelend \DIFaddbegin \cite{hofmannAtliteLightweight2021} \DIFadd{applying similar land
eligibility constraints as in PyPSA-Eur (but using other datasets with global
coverage) and applying the same wind turbine and solar panel models to
ERA5}\cite{ecmwf} \DIFadd{weather data for 2013 in eligible regions. Since TRACE
evaluates whole regions without further transmission network resolution, the
renewable potentials and profiles within a region are split into different
resource classes to reduce smoothing effects. We consider 30 classes each for
onshore wind and solar and 10 for offshore wind where applicable. Based on these
calculations, levelised cost of electricity (LCOE) curves can be determined for
each region. A selection of LCOE curves is shown in
}\sfigref{fig:si:cost-supply-curves}\DIFadd{.
}

%DIF >  deducted domestic demand

\DIFadd{In the next step, potentials are reduced by the }\DIFaddend projected future local energy
demand\DIFdelbegin \DIFdel{is subtracted. Thereby}\DIFdelend \DIFaddbegin \DIFadd{, starting with the lowest LCOE resource classes. With this approach}\DIFaddend ,
domestic consumption is prioritised and supplied by the \DIFdelbegin \DIFdel{countries}\DIFdelend \DIFaddbegin \DIFadd{regions}\DIFaddend ' best renewable
resources\DIFaddbegin \DIFadd{, }\DIFaddend even though we do not model the energy transition in exporting
\DIFdelbegin \DIFdel{countries }\DIFdelend \DIFaddbegin \DIFadd{regions }\DIFaddend in detail. \DIFaddbegin \DIFadd{To create the demand projections, we use the
GEGIS}\cite{mattssonAutopilotEnergyModels2021} \DIFadd{tool, which utilises machine
learning on historical time series, weather data, and macro-economic factors to
create artificial electricity demand time series based on population and GDP
growth scenarios following the SSP2 scenario of the Shared Socioeconomic
Pathways.}\cite{riahiSharedSocioeconomicPathways2017} \DIFadd{From these time series we
take the annual total and increase it by a factor of two to account for further
electrification of other sectors, which the GEGIS tool does not consider.
}

%DIF >  conversion pathways and scale

\DIFaddend The remaining wind and solar electricity supply can then be used to produce the
specific energy or material vector \DIFdelbegin \DIFdel{using }\DIFdelend \DIFaddbegin \DIFadd{according to the flow chart of conversion
pathways shown in }\sfigref{fig:si:import-esc-scheme}\DIFadd{. Considered technologies
include }\DIFaddend water electrolysis for \ce{H2}, direct air capture (DAC) for \ce{CO2},
\DIFdelbegin \DIFdel{air separation units (ASU) for \ce{N2}, }\DIFdelend synthesis of methane, methanol, ammonia or Fischer-Tropsch fuels from \ce{H2}
with \ce{CO2} or \ce{N2}, and \ce{H2} direct iron reduction (DRI) \DIFaddbegin \DIFadd{for sponge
iron }\DIFaddend with subsequent processing \DIFaddbegin \DIFadd{to green steel }\DIFaddend in electric arc furnaces (EAF)
\DIFdelbegin \DIFdel{for the processing of iron ore (103.7 }\DIFdelend \DIFaddbegin \DIFadd{from iron ore priced at
97.7~}\DIFaddend \euro{}\DIFdelbegin \DIFdel{/t)
into green steel.}\DIFdelend \DIFaddbegin \DIFadd{~t$^{-1}$.}\cite{missionpossiblepartnershipSteelSectorTransition2022}
\DIFaddend Other CO$_2$ sources than DAC are not considered in the exporting \DIFdelbegin \DIFdel{countries, a notable difference from the European model.
Liquid
organic hydrogen carriers (LOHC) are not consideredas export vector due to their lower technology readiness level (TRL) compared to other
vectors.}%DIFDELCMD < \cite{irenaGlobalHydrogen2022} %%%
\DIFdel{Further details on the energy and
feedstock flow and process efficiencies are outlined in
\mbox{%DIFAUXCMD
\cref{fig:import-esc-scheme}}\hskip0pt%DIFAUXCMD
}\DIFdelend \DIFaddbegin \DIFadd{regions.
Furthermore, while batteries and hydrogen storage in steel tanks are considered,
underground hydrogen storage is excluded due to the uncertain availability of
salt caverns in many potential exporting
regions.}\cite{hevinUndergroundStorageHydrogen2019,hydrogentcp-task42UndergroundHydrogenStorage2023}
\DIFadd{We also assume that the energy supply chains dedicated to exports will be
islanded from the rest of the local energy system, i.e.~that curtailed
electricity or waste heat could not be used locally}\DIFaddend .

For each vector, an annual reference export demand of 500~TWh\DIFdelbegin \DIFdel{(lower heating
value, LHV) }\DIFdelend \DIFaddbegin \DIFadd{$_\text{LHV}$ }\DIFaddend or
100~Mt of steel \DIFaddbegin \DIFadd{and HBI }\DIFaddend is assumed, mirroring large-scale energy and material
infrastructures and export volumes, corresponding to approximately 40\% of
current \DIFaddbegin \DIFadd{European }\DIFaddend LNG
imports\cite{instituteforenergyeconomicsandfinancialanalysisEuropeanLNG2023} and
66\% of European steel
production.\cite{eurofer-theeuropeansteelassociationEuropeanSteel2023} \DIFdelbegin %DIFDELCMD < 

%DIFDELCMD < %%%
%DIF < %% capacity expansion %%%
%DIFDELCMD < 

%DIFDELCMD < %%%
\DIFdel{Based on these supply chain definitions, a capacity expansion optimisation is
performed to determine the cost-optimal combination of infrastructure and
process capacities for all intermediary products and delivering the final
carrier either through pipelines (\ce{H2(g)}, \ce{CH4(g)}) or by ship
(\ce{H2(l)}, \ce{CH4(l)}, \ce{NH3(l)}, \ce{MeOH}, Fischer-Tropsch fuel, and steel).
Exports from each of the regions shown in \mbox{%DIFAUXCMD
\cref{fig:options:global}
}\hskip0pt%DIFAUXCMD
are modelled to each of twelve European import locations }\DIFdelend \DIFaddbegin \DIFadd{Transport
distances are calculated between the exporting regions and the twelve
representative European import locations using the }\textit{\DIFadd{searoute}} \DIFadd{Python
tool}\cite{haliliSeaRoutePython2024} \DIFadd{for shipborne vectors or crow-fly distances
for pipeline or HVDC connections, and modified by a mode-specific detour factor.
The chosen representative import locations are }\DIFaddend based on large \DIFdelbegin \DIFdel{port
locations, determining the levelised costs of energy or steel the European entry point will see for each supply chain}\DIFdelend \DIFaddbegin \DIFadd{ports and LNG
terminals in the United Kingdom, the Netherlands, Poland, Greece, Italy, Spain,
and Portugal, as well as pipeline entry points in Slovakia, Greece, Italy, and
Spain}\DIFaddend . All energy supply chains are assumed to consume their energy vector as
fuel for transport to Europe, except for \DIFaddbegin \DIFadd{HBI and }\DIFaddend steel, which \DIFdelbegin \DIFdel{uses }\DIFdelend \DIFaddbegin \DIFadd{use }\DIFaddend externally
bought methanol as shipping fuel.

%DIF < %% import constraints in European model %%%
%DIF > %% capacity expansion %%%

\DIFdelbegin \DIFdel{The resulting }\DIFdelend \DIFaddbegin \DIFadd{For each combination of carrier, exporter, and importer, a linear capacity
expansion optimisation is performed to determine cost-optimal investments and
the operation of generation, conversion, storage, and transport capacities for
all intermediary products to deliver 500~TWh~a$^{-1}$ (or 100~Mt~a$^{-1}$ for
materials) of the final carrier to Europe. Dividing the total annual system
costs by the targeted annual export volume yields the }\DIFaddend levelised cost of \DIFdelbegin \DIFdel{exported energy
specific to the respective
importing regions is added as a constant marginal import
cost for all
chemical energy carriers and steel. For the import of hydrogen and methane, candidate entry points are identified based on where existing and prospective
}\DIFdelend \DIFaddbegin \DIFadd{energy
or material as seen by the European entry point. To match the multi-hourly
resolution used for the European model, the TRACE model was configured to use a
3-hourly resolution for 2013, resulting in similar balancing requirements.
Considering the reference export volume of 500~TWh~a$^{-1}$ (or 100~Mt~a$^{-1}$
for materials), the resulting levelised cost curves of imports for different
import vectors and exporting regions are presented for the respective
lowest-cost entry point to Europe in
}\sfigref{fig:si:isc-h2,fig:si:isc-ch4-nh3,fig:si:isc-meoh-ft,fig:si:isc-hbi-St}\DIFadd{.
The curves show the varying cost composition of the country-carrier pairs. In
this step, each import vector combination of carrier, exporter, and importer is
optimised separately. Further constraints, like constraints on total export
volumes per country, are imposed in the coupling to the European model.
}

\DIFadd{A mathematical description of TRACE can be found in Section S3 in Hampp et
al.}\cite{hamppImportOptions2023}

\subsection*{\DIFadd{Coupling of import options to European model}}

\DIFadd{The resulting levelised unit cost for each combination of carrier, exporter, and
reference importer is then used as an exogenous input to the European model. For
each candidate entry point in the 115 European model regions, we match the
closest reference import location from TRACE and add the corresponding import
cost curve as a supply option
(}\sfigref{fig:si:isc-h2,fig:si:isc-ch4-nh3,fig:si:isc-meoh-ft,fig:si:isc-hbi-St}\DIFadd{).
Moreover, we limit energy exports from any one exporting region to Europe for
the sum of all carriers to 500~TWh~a$^{-1}$. This is to both prevent a single
country from dominating the import mix and be consistent with the target export
volume assumed in TRACE. Beyond that, the decision about the origin,
destination, vector, volume, and timing of imports is largely endogenous to
PyPSA-Eur.
}

%DIF > %% entry points %%%

\DIFadd{However, imports may be further restricted by the expansion of domestic import
infrastructure. For each vector, we identify locations where the respective
carrier may enter the European energy system by considering where }\DIFaddend LNG terminals
and cross-continental pipelines are located \DIFdelbegin \DIFdel{. This includes new LNG import terminals
in Europe in response to ambitions to phase out Russian gas
supply in 2022.
}%DIFDELCMD < \cite{instituteforenergyeconomicsandfinancialanalysisEuropeanLNG2023} %%%
\DIFdel{To
achieve
}\DIFdelend \DIFaddbegin \DIFadd{(\mbox{%DIFAUXCMD
\cref{fig:options:europe}}\hskip0pt%DIFAUXCMD
). For
hydrogen imports by pipeline, imports must be near-constant, varying between
90-100\% of peak imports, aligning with the high pipeline utilisation rates
observed in the TRACE model. For methane imports by ship, existing LNG terminals
reported in Global Energy Monitor's Europe Gas
Tracker}\cite{globalenergymonitorEuropeGasTracker2024} \DIFadd{can be used. For hydrogen
by ship, new terminals can be built in regions where LNG terminals exist. To
ensure }\DIFaddend regional diversity in potential gas and hydrogen imports and avoid
vulnerable singular import locations, we allow the expansion beyond the reported
capacities only up to a factor of 2.5, taking the median value of reported
investment costs for LNG terminals.\cite{GlobalGas2022} A \DIFdelbegin \DIFdel{surcharge }\DIFdelend \DIFaddbegin \DIFadd{premium }\DIFaddend of 20\% is
added for hydrogen import terminals due to the lack of practical experience \DIFdelbegin \DIFdel{. Carbonaceous }\DIFdelend \DIFaddbegin \DIFadd{with
them. For electricity, the capacity and operational patterns of the HVDC links
can be endogenously optimised. Imports for carbonaceous }\DIFaddend fuels, ammonia, \DIFdelbegin \DIFdel{and steel imports }\DIFdelend \DIFaddbegin \DIFadd{HBI, and
steel }\DIFaddend are not spatially \DIFdelbegin \DIFdel{resolved due to }\DIFdelend \DIFaddbegin \DIFadd{allocated to specific ports, given }\DIFaddend their low transport
costs \DIFdelbegin \DIFdel{and, therefore, are not constrained by the availability of suitable entry
points.
To present energy and material imports in a common unit, the embodied
energy in steel is approximated with the 2.1 kWh/kg released in
iron oxide
reduction, i.e.~energy released by combustion. }%DIFDELCMD < \cite{kuhnIronRecyclable2022}
%DIFDELCMD < %%%
\DIFdelend \DIFaddbegin \DIFadd{relative to value. Port capacities are assumed unconstrained since these
commodities, particularly carbonaceous fuels, are comparable to the large fossil
oil volumes currently handled at European ports.
}

%DIF > %% reconversion %%%

\DIFadd{Further conversion of imported fuels is also possible once they have arrived in
Europe, e.g.~hydrogen could be used to synthesise carbon-based fuels, ammonia
could be cracked to hydrogen, methane could be reformed to hydrogen, and methane
or methanol could be combusted for power generation. However, conversion losses
can make it less attractive economically to use a high-value hydrogen derivative
merely as a transport and storage vessel only to reconvert it back to hydrogen
or electricity.
}\DIFaddend 

%%% special handling of electricity imports %%%

\DIFdelbegin \DIFdel{Owing to the variability of wind and solar electricity, the }\DIFdelend \DIFaddbegin \DIFadd{The }\DIFaddend supply chain of electricity imports is endogenously optimised with the rest
of the European system rather than using a constant levelised cost of
\DIFdelbegin \DIFdel{exported electricity . This
comprises the optimisation }\DIFdelend \DIFaddbegin \DIFadd{electricity for each export region. This is because, owing to the greater
challenge of storing electricity, the hourly variability }\DIFaddend of wind and solar
\DIFaddbegin \DIFadd{electricity leads to higher price variability than hydrogen and its derivatives,
and the intake needs to be more closely coordinated with the European power
grid. The endogenous optimisation comprises wind and solar }\DIFaddend capacities, batteries
and hydrogen storage in steel tanks, and the size and operation of HVDC link
\DIFdelbegin \DIFdel{connection into
}\DIFdelend \DIFaddbegin \DIFadd{connections to }\DIFaddend Europe based on the \DIFdelbegin \DIFdel{availability time series in neighbouring countries }\DIFdelend \DIFaddbegin \DIFadd{renewable capacity factor time series }\DIFaddend as
illustrated in \cref{fig:options:europe}. \DIFdelbegin \DIFdel{Underground hydrogen storage options
are not considered due to the limited availability of salt caverns in many of
the best renewable resource regions in the countries that are considered
exporting.}%DIFDELCMD < \cite{hevinUndergroundStorage2019} %%%
\DIFdel{We also assume that the energy
supply chains dedicated to exports will be islanded from the rest of the local
energy system. }\DIFdelend Europe's connection options with
exporting \DIFdelbegin \DIFdel{countries }\DIFdelend \DIFaddbegin \DIFadd{regions }\DIFaddend are confined to the \DIFdelbegin \DIFdel{5}\DIFdelend \DIFaddbegin \DIFadd{4}\DIFaddend \% nearest regions, with additional
ultra-long distance connection options to Ireland, Cornwall\DIFaddbegin \DIFadd{, }\DIFaddend and Brittany
following the vision of the Xlinks project between Morocco and the United
Kingdom.\DIFdelbegin %DIFDELCMD < \cite{xlinksMoroccoUKPower2023}
%DIFDELCMD < %%%
\DIFdelend \DIFaddbegin \cite{xlinksMoroccoUKPowerProject2023} \DIFaddend Connections through Russia or
Belarus are excluded\DIFdelbegin \DIFdel{, and thus}\DIFdelend \DIFaddbegin \DIFadd{. In addition to excluded entry points}\DIFaddend , some connections
\DIFaddbegin \DIFadd{from Central Asia }\DIFaddend are affected by additional detours beyond the regularly
applied detour factor of 125\% of the as-the-crow-flies distance. Similar to
intra-European HVDC transmission, a 3\% loss per 1000km and a 2\% converter
station loss are assumed.

%DIF < %% more detailed results %%%
\DIFaddbegin \vspace{1em}
\DIFaddend 

\DIFdelbegin \DIFdel{As illustrated in \mbox{%DIFAUXCMD
\cref{fig:options}}\hskip0pt%DIFAUXCMD
, for imports of hydrogen by pipeline, nearby countries like Algeria and Egypt emerged as lowest cost exporters (ca.~57
}%DIFDELCMD < \euro{}%%%
\DIFdel{/MWh). Importing hydrogen by ship is substantially more expensive due to
liquefaction and evaporation losses. Algeria could offer supply through this
vector at 84 }%DIFDELCMD < \euro{}%%%
\DIFdel{/MWh. For all other hydrogen derivatives, Argentina and
Chile offer the lowest cost imports between 88 }\DIFdelend \DIFaddbegin \DIFadd{Finally, we note that all mass-energy conversion is based on the lower heating
value (LHV). To present energy }\DIFaddend and \DIFdelbegin \DIFdel{110~}%DIFDELCMD < \euro{}%%%
\DIFdel{/MWh or
501~}\DIFdelend \DIFaddbegin \DIFadd{material imports in a common unit, the
embodied energy in steel is approximated with the 2.1 kWh~kg$^{-1}$ released in
iron oxide reduction, i.e.~energy released by
combustion.}\cite{kuhnIronRecyclable2022} \DIFadd{All currency values are given in
}\DIFaddend \euro{}\DIFdelbegin \DIFdel{/t for steel. Methanol is found to be cheaper than the Fischer-Tropsch route because it is assumed to be more flexible (30\% minimum
part load compared to 70\% for
Fischer-Tropsch). }%DIFDELCMD < \cite{brownUltralongdurationEnergy2023} %%%
\DIFdel{The lower process
flexibility shifts the energy mix towards solar electricity and causes higher
levels of curtailment, increasing costs. The transport costs of \ce{CH4(l)} are
lower than for \ce{H2(l)} since the
liquefaction consumes less energy and
individual ships can carry more energy with \ce{CH4(l)}. Pipeline imports of
\ce{CH4(g)} were also considered, but costs were higher than for \ce{CH4(l)}
shipping under the assumption that new pipelines would have to be built.
Consequently, the model preferred LNG imports over pipeline imports}\DIFdelend \DIFaddbegin \DIFadd{$_{2020}$}\DIFaddend .


%DIF <  \section*{Acknowledgements}
\DIFaddbegin \section*{\DIFadd{Acknowledgements}}
\DIFaddend 

%DIF <  XY gratefully acknowledge funding from XY under grant no. 00000.
\DIFaddbegin \DIFadd{J.H. gratefully acknowledges funding from the Kopernikus-Ariadne project (FKZ
03SFK5A and 03SFK5A0-2) by the German Federal Ministry of Education and
Research (}\textit{\DIFadd{Bundesministerium für Bildung und Forschung, BMBF}}\DIFadd{).
}\DIFaddend 

% \section*{License}
% \doclicenseLongText
% \doclicenseIcon

\section*{Author contributions}

% following https://casrai.org/credit/

\textbf{F.N.}:
Conceptualization --
Data curation --
Formal Analysis --
% Funding acquisition --
Investigation --
Methodology --
% Project administration --
% Resources --
Software --
% Supervision --
Validation --
Visualization --
Writing - original draft --
% Writing - review \& editing
\textbf{J.H.}:
Conceptualization --
Data curation --
Formal Analysis --
% Funding acquisition --
Investigation --
Methodology --
% Project administration --
% Resources --
Software --
% Supervision --
Validation --
Visualization --
% Writing - original draft --
Writing - review \& editing
\textbf{T.B.}:
Conceptualization --
% Data curation --
Formal Analysis --
Funding acquisition --
Investigation --
Methodology --
Project administration --
% Resources --
Software --
Supervision --
Validation --
% Visualization --
% Writing - original draft --
Writing - review \& editing

\section*{Declaration of interests}

The authors declare no competing interests.

\section*{Data \DIFdelbegin \DIFdel{and code }\DIFdelend availability}

A dataset of the model results \DIFdelbegin \DIFdel{will be made available on }%DIFDELCMD < \url{zenodo} %%%
\DIFdel{after peer-review.
}\DIFdelend \DIFaddbegin \DIFadd{is available on Zenodo under
}\url{https://doi.org/10.5281/zenodo.14041768} \DIFadd{(released upon publication). Data
on techno-economic assumptions can be found at
}\url{https://github.com/PyPSA/technology-data/releases/tag/import-benefits-v2}\DIFadd{.
}

\section*{\DIFadd{Code availability}}

\DIFaddend The code to reproduce the experiments is available at
\url{https://github.com/fneum/import-benefits} \DIFaddbegin \DIFadd{(Version v2). We also refer to
the documentation of PyPSA-Eur at }\url{https://pypsa-eur.readthedocs.io} \DIFadd{for
more details}\DIFaddend .

\renewcommand{\ttdefault}{\sfdefault}
% \bibliography{import-benefits}
\bibliography{/home/fneum/zotero-bibtex.bib}


\newcounter{LastMainFigure} 
\setcounter{LastMainFigure}{\value{figure}}

\end{document}
