
\section*{Limitations}

potential water scarcity to produce large amounts of hydrogen in arid regions

no pathways (e.g. when imports will be available, how quickly supporting infrastructure can scale up )

ignoring interactions in the industry sector between steel production, chemicals industry and heat integration

potential repurcusssions and local transition burden of moving industry not captured

only some industries are allowed to relocate (those where we allow imports)
- yes: steel, ammonia, chemicals (indirectly)
- no: concrete, alumina

and this does not factor in relocation costs

import costs are not time-dependent and assumed to be available on demand (may underestimate some intermediate storage requirement of imports)

optimistic on district heating and therefore use of PtX waste heat almost anywhere

global competition make prices and volumes difficult to predict

hydrogen network \& comparison to Joule paper
\newpage
\section*{Supplementary Figures}
\begin{figure}[!htb]
    \footnotesize
    (a) 10\% higher import costs \\
    \includegraphics[width=\textwidth]{20231025-zecm/sensitivity-bars-p10pc.pdf} \\
    (b) 10\% lower import costs \\
    \includegraphics[width=\textwidth]{20231025-zecm/sensitivity-bars-m10pc.pdf} \\
    (c) 20\% lower import costs \\
    \includegraphics[width=\textwidth]{20231025-zecm/sensitivity-bars-m20pc.pdf} \\
    \caption{\textbf{Potential for cost reductions with reduced sets of import options for varying import costs.}}
    \label{fig:si:subsets}
\end{figure}

\begin{figure*}
    \includegraphics[width=\textwidth]{20231025-zecm/sensitivity-import-volume-AC+H21.2+CH41.2+NH31.2+FT1.2+MeOH1.2+St1.2.pdf}
    \includegraphics[width=0.49\textwidth, trim=0cm 0cm 12.8cm 0cm, clip]{20231025-zecm/sensitivity-import-volume-AC+H21.1+CH41.1+NH31.1+FT1.1+MeOH1.1+St1.1.pdf}
    \includegraphics[width=0.49\textwidth, trim=0cm 0cm 12.8cm 0cm, clip]{20231025-zecm/sensitivity-import-volume-AC+H20.9+CH40.9+NH30.9+FT0.9+MeOH0.9+St0.9.pdf} \\
    \includegraphics[width=0.49\textwidth, trim=0cm 0cm 12.8cm 0cm, clip]{20231025-zecm/sensitivity-import-volume-AC+H20.8+CH40.8+NH30.8+FT0.8+MeOH0.8+St0.8.pdf}
    \includegraphics[width=0.49\textwidth, trim=0cm 0cm 12.8cm 0cm, clip]{20231025-zecm/sensitivity-import-volume-AC+H20.7+CH40.7+NH30.7+FT0.7+MeOH0.7+St0.7.pdf}
    \caption{\textbf{Sensitivity of import volume on total system cost and composition for varying import costs.}}
    \label{fig:si:volume}
\end{figure*}


\begin{figure*}
    \includegraphics[width=\textwidth]{20231025-zecm/sensitivity-lines.pdf}
    \caption{\textbf{Sensitivity of import volume on total system cost with subsets of import vectors allowed.}}
    \label{fig:si:volume-subsets}
\end{figure*}


\begin{figure*}
    \footnotesize
    \begin{tabular}{cc}
        (a) only electricity imports & (b) only hydrogen imports \\
        \includegraphics[width=0.49\textwidth]{20231025-zecm/graphics/import_shares/s_110_lvopt__Co2L0-2190SEG-T-H-B-I-S-A-onwind+p0.5-imp+AC_2050.pdf} & 
        \includegraphics[width=0.49\textwidth]{20231025-zecm/graphics/import_shares/s_110_lvopt__Co2L0-2190SEG-T-H-B-I-S-A-onwind+p0.5-imp+H2_2050.pdf} \\
        (c) -10\% import costs (all carriers but electricity) & (d) -10\% import cost (only carbonaceous fuels) \\
        \includegraphics[width=0.49\textwidth]{20231025-zecm/graphics/import_shares/s_110_lvopt__Co2L0-2190SEG-T-H-B-I-S-A-onwind+p0.5-imp+AC+H20.9+CH40.9+NH30.9+FT0.9+MeOH0.9+St0.9_2050.pdf} & 
        \includegraphics[width=0.49\textwidth]{20231025-zecm/graphics/import_shares/s_110_lvopt__Co2L0-2190SEG-T-H-B-I-S-A-onwind+p0.5-imp+AC+H2+CH40.9+NH3+FT0.9+MeOH0.9+St_2050.pdf} \\
    \end{tabular}
    \caption{\textbf{Import shares and mix for different import scenarios.}}
    \label{fig:si:import-shares}
\end{figure*}

\begin{figure*}
    \includegraphics[width=\textwidth]{20231025-zecm/infrastructure-map-2x3-B.pdf}
    \caption{\textbf{Layout of European energy infrastructure for different import scenarios.} Role of PtX waste heat, hydrogen network, and electricity imports.
    Left column shows the regional electricity supply mix (pies), added HVDC and HVAC transmission capacity (lines), and the siting of battery storage (choropleth).
        Right column shows the hydrogen supply (top half of pies) and consumption (bottom half of pies), net flow and direction of hydrogen in newly built pipelines (lines), and the siting of hydrogen storage subject to geological potentials (choropleth).
        Total volumes of transmission expansion are given in TWkm, which is the sum product of the capacity and length of individual connections.
    }
    \label{fig:si:infra-b}
\end{figure*}

\begin{figure*}
    \includegraphics[width=\textwidth]{20231025-zecm/infrastructure-map-2x3-C.pdf}
    \caption{\textbf{Layout of European energy infrastructure for different import scenarios.} Sensitivities of infrastructure to import costs.
    Left column shows the regional electricity supply mix (pies), added HVDC and HVAC transmission capacity (lines), and the siting of battery storage (choropleth).
        Right column shows the hydrogen supply (top half of pies) and consumption (bottom half of pies), net flow and direction of hydrogen in newly built pipelines (lines), and the siting of hydrogen storage subject to geological potentials (choropleth).
        Total volumes of transmission expansion are given in TWkm, which is the sum product of the capacity and length of individual connections.
    }
    \label{fig:si:infra-c}
\end{figure*}

\begin{figure*}
    \includegraphics[width=\textwidth]{20231025-zecm/infrastructure-map-2x3-D.pdf}
    \caption{\textbf{Layout of European energy infrastructure for different import scenarios.} Role of industry relocation and focus on carbonaceous fuel imports.
    Left column shows the regional electricity supply mix (pies), added HVDC and HVAC transmission capacity (lines), and the siting of battery storage (choropleth).
        Right column shows the hydrogen supply (top half of pies) and consumption (bottom half of pies), net flow and direction of hydrogen in newly built pipelines (lines), and the siting of hydrogen storage subject to geological potentials (choropleth).
        Total volumes of transmission expansion are given in TWkm, which is the sum product of the capacity and length of individual connections.
    }
    \label{fig:si:infra-d}
\end{figure*}



\begin{figure*}
    \centering
    \footnotesize
    (a) average utilisation rate of import HVDC links \\
    \includegraphics[width=\textwidth]{20231025-zecm/graphics/heatmap_timeseries/s_110_lvopt__Co2L0-2190SEG-T-H-B-I-S-A-onwind+p0.5-imp_2050/ts-heatmap-utilisation_rate-import_hvdc-to-elec.pdf} \\
    (b) average utilisation rate of import hydrogen pipelines \\
    \includegraphics[width=\textwidth]{20231025-zecm/graphics/heatmap_timeseries/s_110_lvopt__Co2L0-2190SEG-T-H-B-I-S-A-onwind+p0.5-imp_2050/ts-heatmap-utilisation_rate-import_pipeline-h2.pdf}
    \caption{\textbf{Temporal usage pattern of electricity and hydrogen storage.}
    The average utilisation rate is XX\% for import HVDC links and XX\% for
    hydrogen pipelines. For hydrogen import pipelines, a clear seasonal pattern
    with higher utilisation in winter is visible, demonstrated by an average utilisation rate of XX\% from November to April and XX\% from May to October.
    For other energy or material
    imports than hydrogen and electricity, the timing of imports is not
    informatively captured due to problem degeneracy caused by negligible
    storage costs of carbonaceous fuels and steel.}
    \label{fig:si:import-operation}
\end{figure*}

\begin{figure*}
    \centering
    \footnotesize
    (a) without imports \\
    \includegraphics[width=\textwidth]{20231025-zecm/graphics/balance_timeseries/s_110_lvopt__Co2L0-2190SEG-T-H-B-I-S-A-onwind+p0.5_2050/ts-balance-electricity-D.pdf} \\
    (b) with imports \\
    \includegraphics[width=\textwidth]{20231025-zecm/graphics/balance_timeseries/s_110_lvopt__Co2L0-2190SEG-T-H-B-I-S-A-onwind+p0.5-imp_2050/ts-balance-electricity-D.pdf}
    \caption{\textbf{Energy balance time series for electricity with and without imports.} Resampled to daily averages. Positive numbers indicate supply, negative numbers indicate consumption.}
    \label{fig:si:balance-elec}
\end{figure*}

\begin{figure*}
    \centering
    \footnotesize
    (a) without imports \\
    \includegraphics[width=\textwidth]{20231025-zecm/graphics/balance_timeseries/s_110_lvopt__Co2L0-2190SEG-T-H-B-I-S-A-onwind+p0.5_2050/ts-balance-heat-D.pdf} \\
    (b) with imports \\
    \includegraphics[width=\textwidth]{20231025-zecm/graphics/balance_timeseries/s_110_lvopt__Co2L0-2190SEG-T-H-B-I-S-A-onwind+p0.5-imp_2050/ts-balance-heat-D.pdf}
    \caption{\textbf{Energy balance time series for heat with and without imports.} Resampled to daily averages. Positive numbers indicate supply, negative numbers indicate consumption.}
    \label{fig:si:balance-heat}
\end{figure*}

\begin{figure*}
    \centering
    \footnotesize
    (a) without imports \\
    \includegraphics[width=\textwidth]{20231025-zecm/graphics/balance_timeseries/s_110_lvopt__Co2L0-2190SEG-T-H-B-I-S-A-onwind+p0.5_2050/ts-balance-hydrogen-D.pdf} \\
    (b) with imports \\
    \includegraphics[width=\textwidth]{20231025-zecm/graphics/balance_timeseries/s_110_lvopt__Co2L0-2190SEG-T-H-B-I-S-A-onwind+p0.5-imp_2050/ts-balance-hydrogen-D.pdf}
    \caption{\textbf{Energy balance time series for hydrogen with and without imports.} Resampled to daily averages. Positive numbers indicate supply, negative numbers indicate consumption.}
    \label{fig:si:balance-h2}
\end{figure*}

\begin{figure*}
    \centering
    \footnotesize
    (a) gas transmission network \\
    \includegraphics[width=.95\textwidth]{20231025-zecm/graphics/gas-network-unclustered.pdf} \\
    (b) electricity transmission network \\
    \includegraphics[width=.95\textwidth]{20231025-zecm/graphics/power-network-unclustered.pdf}
    \caption{\textbf{Gas and electricity transmission network data.} For gas
    transmission, the map shows pipelines sized and colored by rated capacity,
    fossil gas extraction sites, storage locations, pipeline entrypoints, and
    LNG terminals. The data comes from SciGRID\_gas and is supplemented with
    data from Global Energy Monitor. For power transmission, the map shows
    existing transmission lines at and above 220~kV taken from the ENTSO-E
    Transmission System Map (\url{https://www.entsoe.eu/data/map/}),
    supplemented with planned TYNDP projects (\url{https://tyndp.entsoe.eu/}).}
    \label{fig:si:networks-raw}
\end{figure*}