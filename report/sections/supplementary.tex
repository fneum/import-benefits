
\newpage

\begin{figure*}[!htb]
    \centering
    \includegraphics[width=1.0\textwidth]{static/graphics/sketch2-1.pdf}
    \caption{\textbf{Schematic overview of the import supply chains.} The
    illustration includes key input-output ratios of the different conversion
    processes and the transport efficiencies for the different import vectors.}
    \label{fig:si:import-esc-scheme}
\end{figure*}

\begin{figure}[!htb]
    \includegraphics[page=1,width=\textwidth, trim=0cm 1.5cm 3cm 0cm, clip]{../workflow/notebooks/flowchart.pdf} \\
    \includegraphics[page=2,width=\textwidth, trim=0cm 5cm 3cm 0cm, clip]{../workflow/notebooks/flowchart.pdf}
    \caption{\textbf{Overview of supply and consumption options per carrier.}
    Each technology inherits the resolution of the highest resolved carrier it
    connects to (i.e.~any technology that consumes electricity exists as
    investment option for each of the 115 regions).}
    \label{fig:si:supply-consumption-options}
\end{figure}

\begin{figure}[!htb]
    \includegraphics[width=\textwidth]{../workflow/pypsa-eur/resources/20240826-z1/graphics/import_world_map_hydrogen.pdf} \\
    \caption{\textbf{Overview of lowest direct hydrogen import costs into Europe
    per exporting country.} Supplement to Figure 1. Dotted areas indicate that
    hydrogen export from the respective region is cheaper by pipeline than by
    ship.}
    \label{fig:si:worlmap-h2}
\end{figure}


\begin{figure}[!htb]
    \includegraphics[width=\textwidth]{../workflow/notebooks/20240826-z1/fixed-demand.pdf} \\
    \caption{\textbf{Overview of spatially fixed demands, when steel and ammonia
    industry can relocate.} In this scenario, there is virtually no spatially
    fixed hydrogen demand.}
    \label{fig:si:demands}
\end{figure}

\begin{figure}[!htb]
    \includegraphics[width=\textwidth]{../workflow/pypsa-eur/results/20240826-z1/graphics/fixed_demand/s_115_lvopt___2050.png.pdf} \\
    \caption{\textbf{Overview of exogenous final energy and non-energy demand totals.}
    The energy content of steel is given as 2.1 MWh per tonne.
    }
    \label{fig:si:demand_totals}
\end{figure}


\begin{figure}[!htb]
    \includegraphics[width=\textwidth]{../workflow/notebooks/20240826-z1/industry-relocation.pdf} \\
    \caption{\textbf{Relocation patterns of steel and ammonia production in
    scenario without imports.} All values are given relative to the total
    production volume. The left column shows the original regional distribution
    of steel and ammonia production volumes. The centre column shows the
    endogenously optimized allocation of steel and ammonia production. The right
    column shows the absolute change in the regions' total production share.
    Much of the steel production moves to Spain, Scotland and Ireland. A
    substantial share of ammonia production relocates to Finland.}
    \label{fig:si:relocation}
\end{figure}

\begin{figure}[!htb]
    \footnotesize
    \begin{tabular}{cc}
        (a) hydrogen by pipeline & (b) hydrogen by ship \\
        \includegraphics[width=0.49\textwidth]{../workflow/pypsa-eur/resources/20240826-z1/graphics/import_supply_curve_pipeline-h2} &
        \includegraphics[width=0.49\textwidth]{../workflow/pypsa-eur/resources/20240826-z1/graphics/import_supply_curve_shipping-lh2.pdf} \\
    \end{tabular}
    \caption{\textbf{Calculated import cost supply curve for hydrogen by pipeline and by
    ship.} The levelised cost for each region have been calculated for an annual
    export volume of 500~TWh. Some landlocked regions in Central Asia have been
    excluded from ship-based exports. Some countries have been excluded from
    pipeline-based exports due to unrealistic distances to bridge.}
    \label{fig:si:isc-h2}
\end{figure}

\begin{figure}[!htb]
    \footnotesize
    \begin{tabular}{cc}
        (a) methane by ship & (b) ammonia by ship \\
        \includegraphics[width=0.49\textwidth]{../workflow/pypsa-eur/resources/20240826-z1/graphics/import_supply_curve_shipping-lch4.pdf} &
        \includegraphics[width=0.49\textwidth]{../workflow/pypsa-eur/resources/20240826-z1/graphics/import_supply_curve_shipping-lnh3.pdf} \\
    \end{tabular}
    \caption{\textbf{Calculated import cost supply curve for methane and ammonia
    by ship.} The levelised cost for each region have been calculated for an
    annual export volume of 500~TWh. Some landlocked regions in Central Asia
    have been excluded from ship-based exports.}
    \label{fig:si:isc-ch4-nh3}
\end{figure}

\begin{figure}[!htb]
    \footnotesize
    \begin{tabular}{cc}
        (a) methanol by ship & (b) Fischer-Tropsch fuel by ship \\
        \includegraphics[width=0.49\textwidth]{../workflow/pypsa-eur/resources/20240826-z1/graphics/import_supply_curve_shipping-meoh.pdf} &
        \includegraphics[width=0.49\textwidth]{../workflow/pypsa-eur/resources/20240826-z1/graphics/import_supply_curve_shipping-ftfuel.pdf} \\
    \end{tabular}
    \caption{\textbf{Calculated import cost supply curve for methanol and Fischer-Tropsch by ship.} The levelised cost for each region have been calculated for an
    annual export volume of 500~TWh. Some landlocked regions in Central Asia
    have been excluded from ship-based exports.}
    \label{fig:si:isc-meoh-ft}
\end{figure}

\begin{figure}[!htb]
    \footnotesize
    \begin{tabular}{cc}
        (a) HBI by ship & (b) steel by ship \\
        \includegraphics[width=0.49\textwidth]{../workflow/pypsa-eur/resources/20240826-z1/graphics/import_supply_curve_shipping-hbi.pdf} &
        \includegraphics[width=0.49\textwidth]{../workflow/pypsa-eur/resources/20240826-z1/graphics/import_supply_curve_shipping-steel.pdf} \\
    \end{tabular}
    \caption{\textbf{Calculated import cost supply curve for HBI and steel by
    ship.} The levelised cost for each region have been calculated for an annual
    export volume of 100~Mt. Some landlocked regions in Central Asia have been
    excluded from ship-based exports.}
    \label{fig:si:isc-hbi-St}
\end{figure}

\begin{figure}[!htb]
    \footnotesize
    \centering
    \includegraphics[width=0.49\textwidth]{../workflow/trace/results/figures/default_2040_shipping-lh2_AR-South_T0500_supply-curves.pdf}
    \includegraphics[width=0.49\textwidth]{../workflow/trace/results/figures/default_2040_shipping-lh2_AU-West_T0500_supply-curves.pdf} \\
    \includegraphics[width=0.49\textwidth]{../workflow/trace/results/figures/default_2040_shipping-lh2_CA-East_T0500_supply-curves.pdf}
    \includegraphics[width=0.49\textwidth]{../workflow/trace/results/figures/default_2040_shipping-lh2_TN_T0500_supply-curves.pdf} \\
    \includegraphics[width=0.49\textwidth]{../workflow/trace/results/figures/default_2040_shipping-lh2_EG_T0500_supply-curves.pdf}
    \includegraphics[width=0.49\textwidth]{../workflow/trace/results/figures/default_2040_shipping-lh2_MA_T0500_supply-curves.pdf} \\
    \includegraphics[width=0.49\textwidth]{../workflow/trace/results/figures/default_2040_shipping-lh2_MR_T0500_supply-curves.pdf}
    \includegraphics[width=0.49\textwidth]{../workflow/trace/results/figures/default_2040_shipping-lh2_US-South_T0500_supply-curves.pdf} \\
    \includegraphics[width=0.49\textwidth]{../workflow/trace/results/figures/default_2040_shipping-lh2_EH_T0500_supply-curves.pdf}
    \includegraphics[width=0.49\textwidth]{../workflow/trace/results/figures/default_2040_shipping-lh2_NA_T0500_supply-curves.pdf} \\

    \caption{\textbf{Levelised cost supply curve of electricity in selected
    exporting regions.} Shows potential and levelised cost of electricity for
    each resource class of onshore wind (blue), offshore wind (light blue), and solar (yellow) in ascending order.}
    \label{fig:si:lcoe-curve}
\end{figure}

\begin{figure}[!htb]
    \footnotesize
    (a) 10\% higher import costs \\
    \includegraphics[width=\textwidth]{../workflow/notebooks/20240826-z1/sensitivity-bars-p10pc.pdf} \\
    (b) 20\% higher import costs \\
    \includegraphics[width=\textwidth]{../workflow/notebooks/20240826-z1/sensitivity-bars-p20pc.pdf} \\
    \caption{\textbf{Potential for cost reductions with reduced sets of import
    options for higher import costs.} This figure includes the sensitivity with
    10\% and 20\% higher import costs for all fuels but electricity.}
    \label{fig:si:subsets-higher}
\end{figure}




\begin{figure}[!htb]
    \footnotesize
    (a) 10\% lower import costs \\
    \includegraphics[width=\textwidth]{../workflow/notebooks/20240826-z1/sensitivity-bars-m10pc.pdf} \\
    (b) 20\% lower import costs \\
    \includegraphics[width=\textwidth]{../workflow/notebooks/20240826-z1/sensitivity-bars-m20pc.pdf} \\
    \caption{\textbf{Potential for cost reductions with reduced sets of import options for lower import costs.} This figure includes the sensitivity with
    10\% and 20\% lower import costs for all fuels but electricity.}
    \label{fig:si:subsets-lower}
\end{figure}

\begin{figure*}
    \small
    (a) 50\% higher import costs \\
    \includegraphics[width=\textwidth]{../workflow/notebooks/20240826-z1/sensitivity-import-volume-AC+H21.5+CH41.5+NH31.5+FT1.5+MeOH1.5+HBI1.5+St1.5.pdf} \\
    (b) 30\% higher import costs \\
    \includegraphics[width=\textwidth]{../workflow/notebooks/20240826-z1/sensitivity-import-volume-AC+H21.3+CH41.3+NH31.3+FT1.3+MeOH1.3+HBI1.3+St1.3.pdf} \\
    \caption{\textbf{Sensitivity of import volume on total system cost and composition for varying import costs.} This figure includes the sensitivity with
    50\% and 30\% higher import costs for all fuels but electricity.}
    \label{fig:si:volume-higher-2}
\end{figure*}

\begin{figure*}
    \small
    (a) 20\% higher import costs \\
    \includegraphics[width=\textwidth]{../workflow/notebooks/20240826-z1/sensitivity-import-volume-AC+H21.2+CH41.2+NH31.2+FT1.2+MeOH1.2+HBI1.2+St1.2.pdf} \\
    (b) 10\% higher import costs \\
    \includegraphics[width=\textwidth]{../workflow/notebooks/20240826-z1/sensitivity-import-volume-AC+H21.1+CH41.1+NH31.1+FT1.1+MeOH1.1+HBI1.1+St1.1.pdf} \\
    \caption{\textbf{Sensitivity of import volume on total system cost and composition for varying import costs.} This figure includes the sensitivity with
    10\% and 20\% higher import costs for all fuels but electricity.}
    \label{fig:si:volume-higher-1}
\end{figure*}

\begin{figure*}
    \small
    (a) 10\% lower import costs \\
    \includegraphics[width=\textwidth]{../workflow/notebooks/20240826-z1/sensitivity-import-volume-AC+H20.9+CH40.9+NH30.9+FT0.9+MeOH0.9+HBI0.9+St0.9.pdf} \\
    (b) 20\% lower import costs \\
    \includegraphics[width=\textwidth]{../workflow/notebooks/20240826-z1/sensitivity-import-volume-AC+H20.8+CH40.8+NH30.8+FT0.8+MeOH0.8+HBI0.8+St0.8.pdf} \\
    \caption{\textbf{Sensitivity of import volume on total system cost and composition for varying import costs.} This figure includes the sensitivity with
    10\% and 20\% lower import costs for all fuels but electricity.}
    \label{fig:si:volume-lower-1}
\end{figure*}

\begin{figure*}
    \small
    (a) 30\% lower import costs \\
    \includegraphics[width=\textwidth]{../workflow/notebooks/20240826-z1/sensitivity-import-volume-AC+H20.7+CH40.7+NH30.7+FT0.7+MeOH0.7+HBI0.7+St0.7.pdf} \\
    (b) 50\% lower import costs \\
    \includegraphics[width=\textwidth]{../workflow/notebooks/20240826-z1/sensitivity-import-volume-AC+H20.5+CH40.5+NH30.5+FT0.5+MeOH0.5+HBI0.5+St0.5.pdf} \\
    \caption{\textbf{Sensitivity of import volume on total system cost and composition for varying import costs.} This figure includes the sensitivity with
    30\% and 50\% higher import costs for all fuels but electricity.}
    \label{fig:si:volume-lower-2}
\end{figure*}


\begin{figure*}
    \includegraphics[width=\textwidth]{../workflow/notebooks/20240826-z1/sensitivity-lines.pdf}
    \caption{\textbf{Sensitivity of import volume on total system cost with
    subsets of import vectors available.} Supplement to Figure 6. The volume of imports is exogenously
    forced for these runs and coloured lines represent certain restrictions in available import vectors.}
    \label{fig:si:volume-subsets}
\end{figure*}

% \begin{figure*}
%     \centering
%     \includegraphics[width=0.75\textwidth]{../workflow/notebooks/20240826-z1/sensitivity-bars-all-wAC.pdf}
%     \caption{\textbf{Effect of import cost variations of all vectors (including
%     electricity) on cost savings and import shares with all vectors allowed.}
%     Supplement to Figure 5.}
%     \label{fig:si:sensitivity-wac}
% \end{figure*}

\begin{figure*}
    \centering
    (a) only electricity imports \\
    \includegraphics[width=0.57\textwidth]{../workflow/pypsa-eur/results/20240826-z1/graphics/import_shares/s_115_lvopt__imp+AC_2050.pdf}
    \includegraphics[width=0.41\textwidth]{../workflow/pypsa-eur/results/20240826-z1/graphics/import_sankey/s_115_lvopt__imp+AC_2050.pdf} \\

    (b) only hydrogen imports \\
    \includegraphics[width=0.57\textwidth]{../workflow/pypsa-eur/results/20240826-z1/graphics/import_shares/s_115_lvopt__imp+H2_2050.pdf}
    \includegraphics[width=0.41\textwidth]{../workflow/pypsa-eur/results/20240826-z1/graphics/import_sankey/s_115_lvopt__imp+H2_2050.pdf} \\
    \caption{\textbf{Import shares, mix and trade flows for import scenarios with restricted
    import vectors.} For only electricity imports (a) and only hydrogen imports
    (b). Supplement to Figure 3.}
    \label{fig:si:import-shares-a}
\end{figure*}

\begin{figure*}
    \centering
    (a) 10\% lower import costs (all carriers but electricity) \\
    \includegraphics[width=0.57\textwidth]{../workflow/pypsa-eur/results/20240826-z1/graphics/import_shares/s_115_lvopt__imp+AC+H20.9+CH40.9+NH30.9+FT0.9+MeOH0.9+HBI0.9+St0.9_2050.pdf}
    \includegraphics[width=0.41\textwidth]{../workflow/pypsa-eur/results/20240826-z1/graphics/import_sankey/s_115_lvopt__imp+AC+H20.9+CH40.9+NH30.9+FT0.9+MeOH0.9+HBI0.9+St0.9_2050.pdf} \\

    (b) 10\% lower import cost (only carbonaceous fuels) \\
    \includegraphics[width=0.57\textwidth]{../workflow/pypsa-eur/results/20240826-z1/graphics//import_shares/s_115_lvopt__imp+AC+H2+CH40.9+NH3+FT0.9+MeOH0.9+HBI+St_2050.pdf}
    \includegraphics[width=0.41\textwidth]{../workflow/pypsa-eur/results/20240826-z1/graphics//import_sankey/s_115_lvopt__imp+AC+H2+CH40.9+NH3+FT0.9+MeOH0.9+HBI+St_2050.pdf} \\
    \caption{\textbf{Import shares, mix and trade flows for import scenarios with 10\% lower
    costs.} For all carries but electricity (a) and only carbonaceous fuels (b).
    Supplement to Figure 3.}
    \label{fig:si:import-shares-b}
\end{figure*}

\begin{figure*}
    \footnotesize
    \begin{tabular}{ccc}
        (a) H$_2$ / no imports allowed & (b) H$_2$ / only hydrogen imports & (c) H$_2$ / all imports allowed \\
        \includegraphics[width=0.325\textwidth]{../workflow/notebooks/20240826-z1/csc-h2-noimp} &
        \includegraphics[width=0.325\textwidth]{../workflow/notebooks/20240826-z1/csc-h2-imp+H2} &
        \includegraphics[width=0.325\textwidth]{../workflow/notebooks/20240826-z1/csc-h2-imp} \\
        (d) MeOH / no imports allowed & (e) MeOH / only hydrogen imports & (f) MeOH / all imports allowed \\
        \includegraphics[width=0.325\textwidth]{../workflow/notebooks/20240826-z1/csc-meoh-noimp} &
        \includegraphics[width=0.325\textwidth]{../workflow/notebooks/20240826-z1/csc-meoh-imp+H2} &
        \includegraphics[width=0.325\textwidth]{../workflow/notebooks/20240826-z1/csc-meoh-imp} \\
        (g) Fischer-Tropsch / no imports allowed & (h) Fischer-Tropsch / only hydrogen imports & (i) Fischer-Tropsch / all imports allowed \\
        \includegraphics[width=0.325\textwidth]{../workflow/notebooks/20240826-z1/csc-ftf-noimp} &
        \includegraphics[width=0.325\textwidth]{../workflow/notebooks/20240826-z1/csc-ftf-imp+H2} &
        \includegraphics[width=0.325\textwidth]{../workflow/notebooks/20240826-z1/csc-ftf-imp} \\
    \end{tabular}
    \caption{\textbf{Domestic cost supply curves for different import scenarios and carriers.}
        The cost supply curves are built using sorted spatio-temporal market
        values with corresponding production volumes per region and snapshot.
        Dotted lines show the import cost supply curves (in steps of 500~TWh) of
        the respective carriers as reference.}
    \label{fig:si:cost-supply-curves}
\end{figure*}

\begin{figure*}
    \footnotesize
    (a) no imports allowed \\
    \includegraphics[width=\textwidth]{../workflow/notebooks/20240826-z1/market-value-ts-noimp.pdf} \\
    (b) only hydrogen imports allowed \\
    \includegraphics[width=\textwidth]{../workflow/notebooks/20240826-z1/market-value-ts-imp+H2.pdf} \\
    (c) all imports allowed \\
    \includegraphics[width=\textwidth]{../workflow/notebooks/20240826-z1/market-value-ts-imp.pdf}
    \caption{\textbf{Temporal variations of domestic hydrogen and
    Fischer-Tropsch production costs for different import scenarios.} Supplement to Figure 4. Dotted lines show
    the minimum import costs as a reference. The solid lines show the
    production-weighted average costs across all regions. The shaded areas show
    the regional range between lowest and highest production cost in any of the
    115 model regions.}
    \label{fig:si:market-value-ts}
\end{figure*}



\begin{figure*}
    \includegraphics[width=0.8\textwidth]{../workflow/notebooks/20240826-z1/balances-electricity.pdf}
    \includegraphics[width=0.8\textwidth]{../workflow/notebooks/20240826-z1/balances-heat.pdf}
    \includegraphics[width=0.8\textwidth]{../workflow/notebooks/20240826-z1/balances-H2.pdf}
    \includegraphics[width=0.8\textwidth]{../workflow/notebooks/20240826-z1/balances-gas.pdf}
    \caption{\textbf{Energy balances for three import scenarios for the carriers electricity, heat, hydrogen and gas.}
    }
    \label{fig:si:balances-a}
\end{figure*}

\begin{figure*}
    \includegraphics[width=0.8\textwidth]{../workflow/notebooks/20240826-z1/balances-NH3.pdf}
    \includegraphics[width=0.8\textwidth]{../workflow/notebooks/20240826-z1/balances-methanol.pdf}
    \includegraphics[width=0.8\textwidth]{../workflow/notebooks/20240826-z1/balances-oil.pdf}
    \includegraphics[width=0.8\textwidth]{../workflow/notebooks/20240826-z1/balances-co2.pdf}
    \includegraphics[width=0.8\textwidth]{../workflow/notebooks/20240826-z1/balances-co2 stored.pdf}
    \caption{\textbf{
        Energy balances for three import scenarios for the carriers
        ammonia, methanol, and oil, as well as stored and atmospheric carbon dioxide.
    }
    }
    \label{fig:si:balances-b}
\end{figure*}


\begin{figure*}
    \includegraphics[width=\textwidth]{../workflow/notebooks/20240826-z1/infrastructure-map-2x3-B.pdf}
    \caption{\textbf{Layout of European energy infrastructure for different
    import scenarios.} Infrastructure build-out with full power-to-X waste heat
    availability in domestic scenario (first row), without steel and ammonia
    industry relocation (second row), and only electricity imports (third row).
    Left column shows the regional electricity supply mix (pies), added HVDC and
    HVAC transmission capacity (lines), and the siting of battery storage
    (choropleth). Right column shows the hydrogen supply (top half of pies) and
    consumption (bottom half of pies), net flow and direction of hydrogen in
    newly built and retrofitted pipelines (lines), and the siting of hydrogen
    storage subject to geological potentials (choropleth). Total volumes of
    transmission expansion are given in TWkm, which is the sum product of the
    capacity and length of individual connections. }
    \label{fig:si:infra-b}
\end{figure*}

\begin{figure*}
    \includegraphics[width=\textwidth]{../workflow/notebooks/20240826-z1/infrastructure-map-2x3-C.pdf}
    \caption{\textbf{Layout of European energy infrastructure for different import scenarios.} Sensitivities of infrastructure build-out to import costs.
    Left column shows the regional electricity supply mix (pies), added HVDC and HVAC transmission capacity (lines), and the siting of battery storage (choropleth).
        Right column shows the hydrogen supply (top half of pies) and consumption (bottom half of pies), net flow and direction of hydrogen in newly built and retrofitted pipelines (lines), and the siting of hydrogen storage subject to geological potentials (choropleth).
        Total volumes of transmission expansion are given in TWkm, which is the sum product of the capacity and length of individual connections.
    }
    \label{fig:si:infra-c}
\end{figure*}

\begin{figure*}
    \includegraphics[width=\textwidth]{../workflow/notebooks/20240826-z1/infrastructure-map-2x3-D.pdf}
    \caption{\textbf{Layout of European energy infrastructure for different
    import scenarios.}  Infrastructure build-out with no power-to-X flexibility
    (first row), without hydrogen imports or pipeline network (second row), only
    hydrogen derivative imports (third row). Left column shows the regional
    electricity supply mix (pies), added HVDC and HVAC transmission capacity
    (lines), and the siting of battery storage (choropleth). Right column shows
    the hydrogen supply (top half of pies) and consumption (bottom half of
    pies), net flow and direction of hydrogen in newly built and retrofitted
    pipelines (lines), and the siting of hydrogen storage subject to geological
    potentials (choropleth). Total volumes of transmission expansion are given
    in TWkm, which is the sum product of the capacity and length of individual
    connections. }
    \label{fig:si:infra-d}
\end{figure*}

\begin{figure*}
    \centering
    \footnotesize
    (a) average utilisation rate of import HVDC links \\
    \includegraphics[width=\textwidth]{../workflow/pypsa-eur/results/20240826-z1/graphics/heatmap_timeseries/s_115_lvopt__imp_2050/ts-heatmap-utilisation_rate-import_hvdc-to-elec.pdf} \\
    (b) average utilisation rate of import hydrogen pipelines \\
    \includegraphics[width=\textwidth]{../workflow/pypsa-eur/results/20240826-z1/graphics/heatmap_timeseries/s_115_lvopt__imp_2050/ts-heatmap-utilisation_rate-import_infrastructure_pipeline-h2.pdf}
    \caption{\textbf{Temporal usage pattern of electricity and hydrogen storage.}
    The capacity-weighted average utilisation rate is 75+\% for import HVDC links
    and 90+\% for hydrogen pipelines. For hydrogen import pipelines, a clear
    seasonal pattern with higher utilisation in winter is visible. For other
    energy or material imports than hydrogen and electricity, the timing of
    imports is not informatively captured due to problem degeneracy caused by
    negligible storage costs of carbonaceous fuels and steel.}
    \label{fig:si:import-operation}
\end{figure*}

\begin{figure*}
    \centering
    \footnotesize
    (a) state of charge of hydrogen storage \\
    \includegraphics[width=\textwidth]{../workflow/pypsa-eur/results/20240826-z1/graphics/heatmap_timeseries/s_115_lvopt___2050/ts-heatmap-soc-h2_store.pdf} \\
    (b) state of charge profile of thermal storage in district heating \\
    \includegraphics[width=\textwidth]{../workflow/pypsa-eur/results/20240826-z1/graphics/heatmap_timeseries/s_115_lvopt___2050/ts-heatmap-soc-urban_central_water_tanks.pdf}
    \caption{\textbf{State-of-charge profile of long-duration energy storage in
    the domestic scenario without imports.} Hydrogen storage is used
    synoptically, whereas district heating storage is used seasonally.}
    \label{fig:si:storage-operation}
\end{figure*}

\begin{figure*}
    \centering
    \footnotesize
    (a) operation of district heating gas CHPs without imports  \\
    \includegraphics[width=0.95\textwidth]{../workflow/pypsa-eur/results/20240826-z1/graphics/heatmap_timeseries/s_115_lvopt___2050/ts-heatmap-utilisation_rate-urban_central_gas_chp.pdf} \\
    (b) operation of district heating gas CHPs with imports \\
    \includegraphics[width=0.95\textwidth]{../workflow/pypsa-eur/results/20240826-z1/graphics/heatmap_timeseries/s_115_lvopt__imp_2050/ts-heatmap-utilisation_rate-urban_central_gas_chp.pdf}
    (c) load-weighted average electricity price with imports \\
    \includegraphics[width=0.95\textwidth]{../workflow/pypsa-eur/results/20240826-z1/graphics/heatmap_timeseries/s_115_lvopt__imp_2050/ts-heatmap-marginal_price-ac.pdf}
    \caption{\textbf{Temporal usage pattern of backup power/heat in relation to
    import scenario.} In both cases, gas CHPs were the main backup option,
    running for few days during winter when prices are higher. Backup power
    plant operation is higher when imports displace power-to-X flexibilities.}
    \label{fig:si:backup-power}
\end{figure*}

\begin{figure*}
    \centering
    \footnotesize
    (a) no imports allowed  \\
    \includegraphics[height=0.45\textheight]{../workflow/pypsa-eur/results/20240826-z1/graphics/backup_map/s_115_lvopt___2050.pdf} \\
    (b) all imports allowed \\
    \includegraphics[height=0.45\textheight]{../workflow/pypsa-eur/results/20240826-z1/graphics/backup_map/s_115_lvopt__imp_2050.pdf}
    \caption{\textbf{Spatial distribution of backup power for scenarios with all
    imports allowed and no imports.} Batteries are concentrated in Southern
    Europe. Gas-fired combined heat and power plants are distributed across
    Central Europe where electricity prices are higher, with lower build-out
    when no imports are allowed as domestic power-to-X flexibility reduces the
    need for backup capacities. Instead of firing up reserve power plants, the
    production of power-to-X plants is curtailed. Power transmission
    infrastructure distributes backup capacities across Europe where there are none.
    Blue lines represent HVAC lines, rosa lines represent HVDC links.}
    \label{fig:si:backup-power-map}
\end{figure*}


\begin{figure*}
    \centering
    \footnotesize
    % (a) electrolysis  \\
    \includegraphics[width=\textwidth]{../workflow/pypsa-eur/results/20240826-z1/graphics/heatmap_timeseries/s_115_lvopt___2050/ts-heatmap-utilisation_rate-h2_electrolysis.pdf} \\
    % (b) Fischer-Tropsch synthesis \\
    \includegraphics[width=\textwidth]{../workflow/pypsa-eur/results/20240826-z1/graphics/heatmap_timeseries/s_115_lvopt__imp_2050/ts-heatmap-utilisation_rate-fischer-tropsch.pdf}
    % (c) methanol synthesis \\
    \includegraphics[width=\textwidth]{../workflow/pypsa-eur/results/20240826-z1/graphics/heatmap_timeseries/s_115_lvopt__imp_2050/ts-heatmap-utilisation_rate-methanolisation.pdf}
    \caption{\textbf{Temporal usage patterns of selected power-to-X technologies
    in scenario without imports.} Electrolysis clearly reacts to the
    availability of wind and solar electricity, despite high unit investment
    costs of 950\euro/kW$_e$. Fischer-Tropsch runs more steadily, reducing
    production over much of the challenging winter months to its minimum
    part-load of 50\%. The methanolisation process can be used more flexibly
    with a minimum part load of 20\%.}
    \label{fig:si:power-to-x}
\end{figure*}


\begin{figure*}
    \centering
    \footnotesize
    (a) without imports \\
    \includegraphics[width=\textwidth]{../workflow/pypsa-eur/results/20240826-z1/graphics/balance_timeseries/s_115_lvopt___2050/ts-balance-electricity-D.pdf} \\
    (b) with imports \\
    \includegraphics[width=\textwidth]{../workflow/pypsa-eur/results/20240826-z1/graphics/balance_timeseries/s_115_lvopt__imp_2050/ts-balance-electricity-D.pdf}
    \caption{\textbf{Energy balance time series for electricity with and without imports.} Resampled to daily averages. Positive numbers indicate supply, negative numbers indicate consumption.}
    \label{fig:si:balance-elec}
\end{figure*}

\begin{figure*}
    \centering
    \footnotesize
    (a) without imports \\
    \includegraphics[width=\textwidth]{../workflow/pypsa-eur/results/20240826-z1/graphics/balance_timeseries/s_115_lvopt___2050/ts-balance-heat-D.pdf} \\
    (b) with imports \\
    \includegraphics[width=\textwidth]{../workflow/pypsa-eur/results/20240826-z1/graphics/balance_timeseries/s_115_lvopt__imp_2050/ts-balance-heat-D.pdf}
    \caption{\textbf{Energy balance time series for heat with and without imports.} Resampled to daily averages. Positive numbers indicate supply, negative numbers indicate consumption.}
    \label{fig:si:balance-heat}
\end{figure*}

\begin{figure*}
    \centering
    \footnotesize
    (a) without imports \\
    \includegraphics[width=\textwidth]{../workflow/pypsa-eur/results/20240826-z1/graphics/balance_timeseries/s_115_lvopt___2050/ts-balance-hydrogen-D.pdf} \\
    (b) with imports \\
    \includegraphics[width=\textwidth]{../workflow/pypsa-eur/results/20240826-z1/graphics/balance_timeseries/s_115_lvopt__imp_2050/ts-balance-hydrogen-D.pdf}
    \caption{\textbf{Energy balance time series for hydrogen with and without imports.} Resampled to daily averages. Positive numbers indicate supply, negative numbers indicate consumption.}
    \label{fig:si:balance-h2}
\end{figure*}

\begin{figure*}
    \centering
    \footnotesize
    (a) gas transmission network \\
    \includegraphics[width=.87\textwidth, trim=1cm 5cm 1.5cm 5cm, clip]{../workflow/pypsa-eur/resources/20240826-z1/graphics/gas-network-unclustered.pdf} \\
    (b) electricity transmission network \\
    \includegraphics[width=.82\textwidth]{../workflow/pypsa-eur/resources/20240826-z1/graphics/power-network-unclustered.pdf}
    \caption{\textbf{Gas and electricity transmission network data.} For gas
    transmission, the map shows pipelines sized and colored by rated capacity,
    fossil gas extraction sites, storage locations, pipeline entrypoints, and
    LNG terminals. The data comes from SciGRID\_gas and is supplemented with
    data from Global Energy Monitor. For power transmission, the map shows
    existing transmission lines at and above 220~kV taken from OpenStreetMap (\url{https://www.openstreetmap.org/}),
    supplemented with planned TYNDP projects (\url{https://tyndp.entsoe.eu/}).}
    \label{fig:si:networks-raw}
\end{figure*}

\begin{figure*}
    \footnotesize
    (a) unclustered raw data from Caglayan et al. (2020) \\
    \begin{center}
        \includegraphics[width=0.65\textwidth]{../workflow/pypsa-eur/resources/20240826-z1/graphics/salt-caverns}\\
    \end{center}
    (b) clustered near-shore cavern storage potential \\
    \begin{center}
        \includegraphics[width=0.75\textwidth, trim=0cm 2cm 0cm 3cm, clip]{../workflow/pypsa-eur/resources/20240826-z1/graphics/salt-caverns-s-115-nearshore}
    \end{center}
    \caption{\textbf{Locations considered for geologica hydrogen storage in salt
    caverns.} Only near-shore caverns are considered to minimize environmental
    impact of brine disposal.}
    \label{fig:si:hydrogen-caverns}
\end{figure*}


\begin{figure*}
    \centering
    \includegraphics[width=\textwidth]{../workflow/pypsa-eur/resources/20240826-z1/graphics/industrial-sites.pdf}
    \caption{\textbf{Considered locations of industrial production sites by
    sector.} Marker size scales proportionally to the emissions of the
    respective site. This data is used for the spatial distribution of
    industrial energy and feedstock demands.}
    \label{fig:si:industrial-sites}
\end{figure*}
