
\section{Supplementary Notes}

\subsection{Mathematical Model Formulation}
\label{sec:si:math}

This section describes the mathematical formulation of the energy system
optimisation model PyPSA-Eur used in this study. This section is a reprint of a
previous model description in Neumann et al.
\cite{neumannPotentialRoleHydrogen2023}, which was updated and amended to
explain how steel and ammonia industry relocation and the endogenous siting of
hydrogen consumers for fuel synthesis and other applications are modelled. In
the configuration used in this study, the model does not consider pathway
optimisation (i.e.~no sequences of investments), but searches for a cost-optimal
layout corresponding to a given \co emission reduction level and assumes perfect
operational foresight for the weather year (2013) based on which capacities are
optimised (`overnight scenario').

The objective is to minimise the total annual energy system costs of the energy
system that comprises both investment costs and operational expenditures of
generation, storage, transmission and conversion infrastructure. To express both
as annual costs, we use the annuity factor $(1-(1+\tau)^{-n}) / \tau$ that, like
a mortgage, converts the upfront investment of an asset to annual payments
considering its lifetime $n$ and cost of capital $\tau$. Thus, the objective
includes on one hand the annualised capital costs $c_*$ for investments at bus
$i$ in generator capacity $G_{i,r}\in\R^+$ of technology $r$ (e.g. primary
energy supply of gas, oil, nuclear, biomass), storage energy capacity
$E_{i,s}\in\R^+$ of technology $s$ (e.g.~battery, heat, or hydrogen storage),
electricity transmission line capacities $P_{\ell}\in\R^+$, and energy
conversion (e.g.~power-to-heat, electrolysis, Fischer-Tropsch synthesis) and
transport (e.g.~hydrogen/gas pipelines) capacities $F_k\in\R^+$ (`links'), as
well as the variable operating costs $o_*$ for generator dispatch
$g_{i,r,t}\in\R^+$ and link dispatch $f_{k,t}\in\R^+$ on the other:
\begin{align}
  \label{eq:objective}
  \min_{G,E,P,F,g} \quad &\left[\sum_{i,r} c_{i,r}\cdot G_{i,r} + \sum_{i,s} c_{i,s}\cdot E_{i,s} + \sum_{\ell}c_{\ell}\cdot P_{\ell}+ \sum_{k}c_{k}\cdot F_k +\right. \\
  & \left.  \sum_{t} w_t \cdot \left( \sum_{i,r} o_{i,r} \cdot g_{i,r,t} + \sum_k o_k \cdot f_{k,t} \right) \right].
\end{align}
Thereby, the representative time snapshots $t$ are weighted by the time span
$w_t$ such that their total duration adds up to one year; \mbox{$\sum_{t}
w_t=365\cdot 24\text{h}=8760\text{h}$}. A bus $i$ represents both a regional
scope and an energy carrier. Represented carriers include electricity, heat
(various subdivisions), hydrogen, ammonia, methane, methanol, oil, hot
briquetted iron, steel, and carbon dioxide. Spatially resolved carriers are
electricity, heat, hydrogen and methane (e.g.~\mbox{$i=$`region\_id hydrogen'}).
For all other carriers, assuming negligible transport costs of the respective
fuels, the model aggregates the regions to a single bus per carrier
(e.g.~$i=$`Europe methanol').


% constraints

In addition to the cost-minimising objective function, we further impose a set
of linear constraints that define limits on (i) the capacities of generation,
storage, conversion and transmission infrastructure from geographical and
technical potentials, (ii) the availability of variable renewable energy
sources, performance of heat pumps and carrier-specific demands for each
location and time step (iii) the limit for \co emissions or transmission
expansion, (iv) storage consistency equations, and (v) a multi-period linearised
optimal power flow (LOPF) formulation and energy balance constraints for each
region, carrier and time step. Overall, this results in a large linear problem
(LP).

The capacities of generation, storage, conversion and transmission
infrastructure are constrained from above by their installable potentials and
from below by any existing components:
\begin{align}
  \underline{G}_{i,r}  &  & \leq &  & G_{i,r}  &  & \leq &  & \overline{G}_{i,r}  & \qquad\forall i, r \label{eq:genlimit} \\
  \underline{E}_{i,s}  &  & \leq &  & E_{i,s}  &  & \leq &  & \overline{E}_{i,s}  & \qquad\forall i, s \\
  \underline{P}_{\ell} &  & \leq &  & P_{\ell} &  & \leq &  & \overline{P}_{\ell} & \qquad\forall \ell \\
  \underline{F}_{k} &  & \leq &  & F_{k} &  & \leq &  & \overline{F}_{k} & \qquad\forall k
\end{align}

Moreover, the dispatch of generators and links may not only be constrained by
their rated capacity but also by the weather-dependent availability of variable
renewable energy or must-run conditions. This can be expressed as a time- and
location-dependent availability factor
$\overline{g}_{i,r,t}$/$\overline{f}_{k,t}$  and must-run factor
$\underline{g}_{i,r,t}$/$\underline{f}_{k,t}$ (e.g. for certain power-to-X processes),
given per unit of the nominal capacity:
\begin{align}
    \underline{g}_{i,r,t}  G_{i,r} &  & \leq &  & g_{i,r,t} &  & \leq &  & \overline{g}_{i,r,t} G_{i,r} & \qquad\forall i, r, t \\
    \underline{f}_{k,t}  F_{k} &  & \leq &  & f_{k,t} &  & \leq &  & \overline{f}_{k,t} F_{i,r} & \qquad\forall k, t
\end{align}
The parameter $\underline{f}_{k,t}$ can also be used to define whether a link is
bidirectional or unidirectional. For instance, for HVDC links
$\underline{f}_{k,t}=-1$ would allow lossless power flows in either direction.
On the other hand, a heat resistor has $\underline{f}_{k,t}=0$ since it can only
convert electricity to heat, not the other way around.

The energy levels $e_{i,s,t}$ of all stores are constrained by their energy capacity
\begin{align}
  0 &  & \leq &  & e_{i,s,t} &  & \leq &  & E_{i,s} & \qquad\forall i, s, t,
\end{align}
and have to be consistent with the dispatch variable $h_{i,s,t}\in\R$ in all
hours
\begin{align}
  e_{i,s,t} =\: & \eta_{i,s,0}^{w_t} \cdot e_{i,s,t-1} + w_t \cdot h_{i,s,t}, \label{eq:stoe}
\end{align}
where $\eta_{i,s,0}$ denotes the standing loss. Furthermore, the storage energy
levels are either assumed to be cyclic or given an initial state of charge,
\begin{align}
  e_{i,s,0} = e_{i,s,T} \qquad\forall i, s,\\
  \text{or} \qquad e_{i,s,0} = e_{i,s,\text{initial}} \qquad\forall i, s.
\end{align}

The modelling of hydroelectricity storage deviates from regular storage to
additionally account for natural inflow and spillage of water. We also assume
fixed power ratings $H_{i,s}$ for hydroelectricity storage. The dispatch of
hydroelectricity storage units is split into two positive variables; one each
for charging $h_{i,s,t}^+$ and discharging $h_{i,s,t}^-$, and limited by
$H_{i,s}$.
\begin{align}
  0 &  & \leq &  & h_{i,s,t}^+ &  & \leq &  & H_{i,s} & \qquad\forall i, s, t \label{eq:sto1} \\
  0 &  & \leq &  & h_{i,s,t}^- &  & \leq &  & H_{i,s} & \qquad\forall i, s, t \label{eq:sto2}
\end{align}
The energy levels $e_{i,s,t}$ of all hydroelectric storage also have to match
the dispatch across all hours
\begin{align}
  e_{i,s,t} =\: & \eta_{i,s,0}^{w_t} \cdot e_{i,s,t-1} + w_t \cdot h_{i,s,t}^\text{inflow} - w_t \cdot h_{i,s,t}^\text{spillage} & \quad\forall i, s, t \nonumber \\
                & + \eta_{i,s,+} \cdot w_t \cdot h_{i,s,t}^+ - \eta_{i,s,-}^{-1} \cdot w_t \cdot h_{i,s,t}^-, \label{eq:stoe-2}
\end{align}
whereby hydropower storage units can additionally have a charging efficiency
$\eta_{i,s,+}$, a discharging efficiency $\eta_{i,s,-}$, natural inflow
$h_{i,s,t}^\text{inflow}$ and spillage $h_{i,s,t}^\text{spillage}$, besides the
standing loss $\eta_{i,s,0}$.

The nodal balance constraint for supply and demand (Kirchoff's current law for
electricity buses, or simply energy balances for all other carriers) requires
local generators and storage units, incoming or outgoing energy flows
$p_{\ell,t}$ of incident transmission lines $\ell$, and incoming or outgoing
energy flows $f_{k,t}$ of pipelines and converters $k$ (e.g. electrolysis, heat pumps,
Fischer-Tropsch synthesis, hydrogen turbines, electric arc furnace) to balance the
perfectly inelastic, exogenously given demand $d_{i,t}$ at each location $i$ and
snapshot $t$
\begin{align}
    \sum_r g_{i,r,t} + \sum_s \left(h_{i,s,t}^- - h_{i,s,t}^+ \right) + \sum_s h_{i,s,t} + \sum_\ell K_{i\ell} p_{\ell,t} + \sum_k L_{ikt} f_{k,t} = d_{i,t}\quad \forall i,t, \label{eq:balance}
\end{align}
where $K_{i\ell}$ is the incidence matrix of the electricity network with
non-zero values $-1$ if line $\ell$ starts at node $i$ and $1$ if it ends at
node $i$. $L_{ikt}$ is the pipeline and conversion incidence matrix of the network with non-zero
values $-1$ if link $k$ starts at node $i$ and $\eta_{i,k,t}$ if one of its
terminal buses is node $i$. For a link with more than two outputs (e.g. CHP
converts gas to heat and electricity in a fixed ratio), the respective column of
the conversion incidence matrix has more than two non-zero entries. These entries may
also be negative to denote additional inputs rather than multiple outputs. For
instance, modelling Fischer-Tropsch synthesis requires hydrogen and carbon
dioxide as inputs to produce liquid hydrocarbons (`oil'). The operational
decision variable $f_{k,t}$ is always associated with one input; additional
inputs would be modelled as outputs with a negative entry for $\eta_{i,k,t}$.
These entries may also contain unit conversions (e.g. for Fischer-Tropsch
synthesis from carbon dioxide input in tonnes to oil output in MWh). Moreover,
the factor $\eta_{i,k,t}$ can also be time-dependent and greater than one for
certain technologies (e.g. for heat pumps converting electricity and ambient
heat to hot water), which can result in additional primary energy input to the
model scope (ambient heat).

By modelling the conversion between different energy carriers in this way, i.e.
as links with capacity $F_k$, dispatch $f_{k,t}$ and conversion incidence matrix
$L_{ikt}$, the model can endogenously decide some consumption patterns at bus
$i$ in addition to the exogenously given demands $d_{i,t}$. Noting that bus $i$
represents a region/carrier combination (e.g.~`region\_id~hydrogen' or
`Europe~methanol'), $k$ would conversely represent a region/technology combination
(e.g.~`region\_id electrolyser'). While the positive terms of $\sum_k L_{ikt}
f_{k,t}$ in the nodal balance constraints, \cref{eq:balance}, represent
endogenous supply at bus $i$, the negative terms represent endogenous
consumption (e.g.~consumption of electricity for electrolysis, or hydrogen for
methanolisation in a region). What consumption and supply patterns are possible
then depends on where the model decides to build capacity $F_k$ (e.g.~of
electrolysis in a region). For our relocation modelling of steel and ammonia
production, we keep $F_k$ as free variables for Haber-Bosch, direct iron
reduction and electric arc furnace links for each region, effectively allowing
greenfield siting of these industries. In scenarios without relocation, we fix
$F_k$ to the current regional capacities of these technologies such that no new
sites for steel and ammonia production can be developed.

The effective power flows $p_{\ell,t}$ are limited by their nominal capacities $P_\ell$ minus losses $\psi_\ell$.
\begin{align}
	|p_{\ell,t}| \leq \overline{p}_{\ell} P_{\ell} - \psi_\ell & \qquad\forall \ell, t,
	\label{eq:cap}
\end{align}
where $\overline{p}_\ell$ acts as an additional per-unit security margin on the line capacity
to allow a buffer for the failure of single circuits ($N-1$ criterion) and reactive power flows.

Kirchoff's voltage law (KVL) imposes further constraints on the flow of AC
transmission lines and there are several ways to formulate KVL with large
impacts on performance. Here, we use linearised load flow assumptions, where the
voltage angle difference around every closed cycle in the electricity
transmission network must add up to zero. Using a cycle basis $C_{\ell c}$ of
the network graph where the independent cycles $c$ are expressed as directed
linear combinations of lines $\ell$, we can write
KVL as
\begin{align}
    \sum_\ell C_{\ell c} \cdot x_\ell \cdot p_{\ell,t} = 0 \qquad\forall c,t
    \label{eq:kvl}
\end{align}
where $x_\ell$ is the series inductive reactance of line $\ell$.

The AC transmission losses $\psi_\ell$ are approximated as a tangent-based linear approximation
of the loss parabola $\psi_\ell = r_\ell p_\ell^2$, where $r_\ell$ is the
resistance, following Neumann et al.~\cite{neumannAssessmentsLinear2022}:
\begin{align}
    0 \leq \psi_\ell \leq r_\ell (\overline{p}_\ell \overline{P}_\ell)^2 & \qquad\forall \ell \\
    \psi_\ell \geq   m_k\cdot p_\ell + a_k & \qquad\forall \ell, k=1,\dots,n \\
	\psi_\ell \geq - m_k\cdot p_\ell + a_k & \qquad\forall \ell, k=1,\dots,n
\end{align}
For each segment $k$ of the total $n$ segments, we derive the slope $m_k$ and
offset $a_k$ in the following way:
\begin{align}
	\psi_\ell(k) &= r_\ell \left(\frac{k}{n}\cdot \overline{p}_\ell \overline{P}_\ell\right)^2 \\
	m_k &= \frac{\text{d} \psi_\ell(k) }{\text{d}k} = 2 r_\ell \left(\frac{k}{n}\cdot \overline{p}_\ell \overline{P}_\ell\right) \\
	a_k &= \psi_\ell(k) - m_k \left(\frac{k}{n}\cdot\overline{p}_\ell \overline{P}_\ell\right).
\end{align}
The losses also extend the left-hand side of the nodal balance constraints in
\cref{eq:balance} by the term $-0.5 \cdot |K_{i\ell}| \cdot \psi_\ell$, splitting
the losses equally between both connection points.

Finally, we add a constraint so that the total CO$_2$ emissions net out to zero
over the year. The emissions are determined from the difference between final
and initial `filling level' of the store used to represent CO$_2$ in the
atmosphere.
\begin{align}
	e_{\text{CO}_2,\text{atmosphere},t=T} - e_{\text{CO}_2,\text{atmosphere},t=0}  \leq 0.
\end{align}

\clearpage
\section{Supplementary Figures}

\begin{figure*}[!htb]
    \centering
    \includegraphics[width=1.0\textwidth]{static/graphics/sketch2-1.pdf}
    \caption{\textbf{Schematic overview of the import supply chains.} The
    illustration includes key input-output ratios of the different conversion
    processes and the transport efficiencies for the different import vectors.}
    \label{fig:si:import-esc-scheme}
\end{figure*}

\begin{figure}[!htb]
    \includegraphics[page=1,width=\textwidth, trim=0cm 1.5cm 3cm 0cm, clip]{../workflow/notebooks/flowchart.pdf} \\
    \includegraphics[page=2,width=\textwidth, trim=0cm 5cm 3cm 0cm, clip]{../workflow/notebooks/flowchart.pdf}
    \caption{\textbf{Overview of supply and consumption options per carrier.}
    Each technology inherits the resolution of the highest resolved carrier it
    connects to (i.e.~any technology that consumes electricity exists as
    investment option for each of the 115 regions).}
    \label{fig:si:supply-consumption-options}
\end{figure}

\begin{table}[!htb]
    \footnotesize
    \setlength{\extrarowheight}{4pt}
    \caption{\textbf{Overview of scenarios.} Overall, 352 scenario runs have
    been computed. The estimate for a single re-run of all scenarios is in the
    order of 50,000 CPU-hours. Figure numbers with the prefix `S' denote
    supplementary figures. $^*$ Some combinations of import carriers and volumes
    have been omitted in case they exceeded the model's ability to absorb
    these.}
    \label{tab:scenarios}
    \begin{tabular}{>{\raggedright\arraybackslash}p{3cm}>{\raggedright\arraybackslash}p{4cm}>{\raggedright\arraybackslash}p{1.8cm}>{\raggedright\arraybackslash}p{2cm}>{\raggedright\arraybackslash}p{1.5cm}>{\raggedright\arraybackslash}p{1.3cm}>{\raggedright\arraybackslash}p{1cm}p{0.5cm}}
    \toprule
    Scenario group & Import carriers & Import volume [TWh] & Import costs & Technology assumptions year & St. / NH$_3$ relocation & Figures & Runs \\
    \midrule
    \textbf{Subsets of import carriers} & none, all, methane, electricity, ammonia, HBI/steel, hydrogen, methanol, FT, methanol/FT, only carbonaceous fuels, hydrogen derivatives, hydrogen derivatives and steel, all but electricity and steel, all but electricity & any & default ($\pm$0\%) & 2030, 2040, 2050 & yes, no (for 2040) & \ref{fig:sensitivity-bars}, \ref{fig:import-shares}, \ref{fig:market-values}, \ref{fig:import-infrastructure}, S\ref{fig:si:relocation}, S\ref{fig:si:technology-projection}, S\ref{fig:si:import-shares-a}, S\ref{fig:si:cost-supply-curves}, S\ref{fig:si:market-value-ts}, S\ref{fig:si:balances-a}, S\ref{fig:si:balances-b}, S\ref{fig:si:infra-b}, S\ref{fig:si:infra-d}, S\ref{fig:si:infra-e}ff. & 60 \\
    \textbf{Subsets of import carriers (import cost variation)} & all, methane, electricity, ammonia, HBI/steel, hydrogen, methanol, FT, methanol/FT, only carbonaceous fuels, hydrogen derivatives, hydrogen derivatives and steel, all but electricity and steel, all but electricity & any & -20\%, -10\%, +10\%, +20\% & 2040 & yes & S\ref{fig:si:subsets-higher}, S\ref{fig:si:subsets-lower}, S\ref{fig:si:import-shares-b}, S\ref{fig:si:infra-c} & 56 \\
    \textbf{Import cost variation (all import carriers)} & all & any & -50\%, -40\%, -30\%, -20\%, -10\%, +10\%, +20\%, +30\%, +50\% & 2030, 2040, 2050 & yes & \ref{fig:sensitivity-costs} & 27 \\
    \textbf{Import cost variation (all import carriers but electricity)} & all & any & -50\%, -40\%, -30\%, -20\%, -10\%, +10\%, +20\%, +30\%, +50\% & 2030, 2040, 2050 & yes & \ref{fig:sensitivity-costs} & 27 \\
    \textbf{Import cost variation (only carbonaceous import carriers)} & all & any & -50\%, -40\%, -30\%, -20\%, -10\%, +10\%, +20\%, +30\%, +50\% & 2030, 2040, 2050 & yes & \ref{fig:sensitivity-costs}, S\ref{fig:si:infra-c} & 27 \\
    \textbf{Import cost variation (only electricity)} & all & any & -30\%, -20\%, -10\% & 2040 & yes & -- & 3 \\
    \textbf{Import volumes (subsets of import carriers)} & all, electricity, hydrogen, methane, ammonia, methanol/FT, all but electricity & 500, 1000, 1500, 2000, 3000, 4000, 5000, 6000, 7000, 8000, 9000, 10000* & default ($\pm$0\%) & 2040 & Yes & \ref{fig:sensitivity-volume}, S\ref{fig:si:volume-subsets} & 47 \\
    \textbf{Import volumes (import cost variation)} & all & 500, 1000, 1500, 2000, 3000, 4000, 5000, 6000, 7000, 8000, 9000, 10000 & -50\%, -30\%, -20\%, -10\%, +10\%, +20\%, +30\%, +50\% & 2040 & yes & \ref{fig:sensitivity-volume}, S\ref{fig:si:volume-higher-2}, S\ref{fig:si:volume-higher-1}, S\ref{fig:si:volume-lower-1}, S\ref{fig:si:volume-lower-2},  & 96 \\
    \textbf{Sensitivity: No hydrogen network} & none, all, all but hydrogen & any & default ($\pm$0\%) & 2040 & yes & S\ref{fig:si:infra-d} & 3 \\
    \textbf{Sensitivity: no PtX waste heat} & none, all & any & default  ($\pm$0\%)& 2040 & yes & -- & 2 \\
    \textbf{Sensitivity: all PtX waste heat} & none, all & any & default ($\pm$0\%) & 2040 & yes & S\ref{fig:si:infra-b} & 2 \\
    \textbf{Sensitivity: PtX flexibility} & none, all & any & default ($\pm$0\%) & 2040 & yes & S\ref{fig:si:infra-d} & 2 \\
    \bottomrule
    \end{tabular}
\end{table}


\begin{figure}[!htb]
    \includegraphics[width=\textwidth]{../workflow/pypsa-eur/resources/20240826-z1/graphics/import_world_map_hydrogen.pdf} \\
    \caption{\textbf{Overview of lowest direct hydrogen import costs into Europe
    per exporting country.} Supplement to Figure 1. Dotted areas indicate that
    hydrogen export from the respective region is cheaper by pipeline than by
    ship. Maps made with Natural Earth.}
    \label{fig:si:worlmap-h2}
\end{figure}


\begin{figure}[!htb]
    \includegraphics[width=\textwidth]{../workflow/notebooks/20240826-z1/fixed-demand.pdf} \\
    \caption{\textbf{Overview of spatially fixed demands, when steel and ammonia
    industry can relocate.} In this scenario, there is virtually no spatially
    fixed hydrogen demand. Maps made with Natural Earth.}
    \label{fig:si:demands}
\end{figure}

\begin{figure}[!htb]
    \includegraphics[width=\textwidth]{../workflow/pypsa-eur/results/20240826-z1/graphics/fixed_demand/s_115_lvopt___2050.png.pdf} \\
    \caption{\textbf{Overview of exogenous final energy and non-energy demand totals.}
    The energy content of steel is given as 2.1 MWh per tonne.
    }
    \label{fig:si:demand_totals}
\end{figure}


\begin{figure}[!htb]
    \includegraphics[width=\textwidth]{../workflow/notebooks/20240826-z1/industry-relocation.pdf} \\
    \caption{\textbf{Relocation patterns of steel and ammonia production in
    scenario without imports.} All values are given relative to the total
    production volume. The left column shows the original regional distribution
    of steel and ammonia production volumes. The centre column shows the
    endogenously optimized allocation of steel and ammonia production. The right
    column shows the absolute change in the regions' total production share.
    Much of the steel production moves to Spain, Scotland and Ireland. A
    substantial share of ammonia production relocates to Finland. Maps made with Natural Earth.}
    \label{fig:si:relocation}
\end{figure}

\begin{figure}[!htb]
    \footnotesize
    \begin{tabular}{cc}
        (a) hydrogen by pipeline & (b) hydrogen by ship \\
        \includegraphics[width=0.49\textwidth]{../workflow/pypsa-eur/resources/20240826-z1/graphics/import_supply_curve_pipeline-h2} &
        \includegraphics[width=0.49\textwidth]{../workflow/pypsa-eur/resources/20240826-z1/graphics/import_supply_curve_shipping-lh2.pdf} \\
    \end{tabular}
    \caption{\textbf{Calculated import cost supply curve for hydrogen by pipeline and by
    ship.} The levelised cost for each region have been calculated for an annual
    export volume of 500~TWh. Some landlocked regions in Central Asia have been
    excluded from ship-based exports. Some countries have been excluded from
    pipeline-based exports due to unrealistic distances to bridge.}
    \label{fig:si:isc-h2}
\end{figure}

\begin{figure}[!htb]
    \footnotesize
    \begin{tabular}{cc}
        (a) methane by ship & (b) ammonia by ship \\
        \includegraphics[width=0.49\textwidth]{../workflow/pypsa-eur/resources/20240826-z1/graphics/import_supply_curve_shipping-lch4.pdf} &
        \includegraphics[width=0.49\textwidth]{../workflow/pypsa-eur/resources/20240826-z1/graphics/import_supply_curve_shipping-lnh3.pdf} \\
    \end{tabular}
    \caption{\textbf{Calculated import cost supply curve for methane and ammonia
    by ship.} The levelised cost for each region have been calculated for an
    annual export volume of 500~TWh. Some landlocked regions in Central Asia
    have been excluded from ship-based exports.}
    \label{fig:si:isc-ch4-nh3}
\end{figure}

\begin{figure}[!htb]
    \footnotesize
    \begin{tabular}{cc}
        (a) methanol by ship & (b) Fischer-Tropsch fuel by ship \\
        \includegraphics[width=0.49\textwidth]{../workflow/pypsa-eur/resources/20240826-z1/graphics/import_supply_curve_shipping-meoh.pdf} &
        \includegraphics[width=0.49\textwidth]{../workflow/pypsa-eur/resources/20240826-z1/graphics/import_supply_curve_shipping-ftfuel.pdf} \\
    \end{tabular}
    \caption{\textbf{Calculated import cost supply curve for methanol and Fischer-Tropsch by ship.} The levelised cost for each region have been calculated for an
    annual export volume of 500~TWh. Some landlocked regions in Central Asia
    have been excluded from ship-based exports.}
    \label{fig:si:isc-meoh-ft}
\end{figure}

\begin{figure}[!htb]
    \footnotesize
    \begin{tabular}{cc}
        (a) HBI by ship & (b) steel by ship \\
        \includegraphics[width=0.49\textwidth]{../workflow/pypsa-eur/resources/20240826-z1/graphics/import_supply_curve_shipping-hbi.pdf} &
        \includegraphics[width=0.49\textwidth]{../workflow/pypsa-eur/resources/20240826-z1/graphics/import_supply_curve_shipping-steel.pdf} \\
    \end{tabular}
    \caption{\textbf{Calculated import cost supply curve for HBI and steel by
    ship.} The levelised cost for each region have been calculated for an annual
    export volume of 100~Mt. Some landlocked regions in Central Asia have been
    excluded from ship-based exports.}
    \label{fig:si:isc-hbi-St}
\end{figure}

\begin{figure}[!htb]
    \footnotesize
    \centering
    \includegraphics[width=0.49\textwidth]{../workflow/trace/results/figures/default_2040_shipping-lh2_AR-South_T0500_supply-curves.pdf}
    \includegraphics[width=0.49\textwidth]{../workflow/trace/results/figures/default_2040_shipping-lh2_AU-West_T0500_supply-curves.pdf} \\
    \includegraphics[width=0.49\textwidth]{../workflow/trace/results/figures/default_2040_shipping-lh2_CA-East_T0500_supply-curves.pdf}
    \includegraphics[width=0.49\textwidth]{../workflow/trace/results/figures/default_2040_shipping-lh2_TN_T0500_supply-curves.pdf} \\
    \includegraphics[width=0.49\textwidth]{../workflow/trace/results/figures/default_2040_shipping-lh2_EG_T0500_supply-curves.pdf}
    \includegraphics[width=0.49\textwidth]{../workflow/trace/results/figures/default_2040_shipping-lh2_MA_T0500_supply-curves.pdf} \\
    \includegraphics[width=0.49\textwidth]{../workflow/trace/results/figures/default_2040_shipping-lh2_MR_T0500_supply-curves.pdf}
    \includegraphics[width=0.49\textwidth]{../workflow/trace/results/figures/default_2040_shipping-lh2_US-South_T0500_supply-curves.pdf} \\
    \includegraphics[width=0.49\textwidth]{../workflow/trace/results/figures/default_2040_shipping-lh2_EH_T0500_supply-curves.pdf}
    \includegraphics[width=0.49\textwidth]{../workflow/trace/results/figures/default_2040_shipping-lh2_NA_T0500_supply-curves.pdf} \\

    \caption{\textbf{Levelised cost supply curve of electricity in selected
    exporting regions.} Shows potential and levelised cost of electricity for
    each resource class of onshore wind (blue), offshore wind (light blue), and solar (yellow) in ascending order.}
    \label{fig:si:lcoe-curve}
\end{figure}

\begin{figure}[!htb]
    \includegraphics[width=\textwidth]{../workflow/notebooks/20240826-z1/technology-year-sensitivity.pdf} \\
    \caption{\textbf{Impact of technology development assumptions on total energy system
    cost with and without imports.} Left colums show total system cost.
    Right columns show difference in total system cost compared to technology
    assumptions for 2040. Black lines show net cost difference in absolute and
    relative terms.}
    \label{fig:si:technology-projection}
\end{figure}

\begin{figure}[!htb]
    \footnotesize
    (a) 10\% higher import costs \\
    \includegraphics[width=\textwidth]{../workflow/notebooks/20240826-z1/sensitivity-bars-p10pc.pdf} \\
    (b) 20\% higher import costs \\
    \includegraphics[width=\textwidth]{../workflow/notebooks/20240826-z1/sensitivity-bars-p20pc.pdf} \\
    \caption{\textbf{Potential for cost reductions with reduced sets of import
    options for higher import costs.} This figure includes the sensitivity with
    10\% and 20\% higher import costs for all fuels but electricity.}
    \label{fig:si:subsets-higher}
\end{figure}




\begin{figure}[!htb]
    \footnotesize
    (a) 10\% lower import costs \\
    \includegraphics[width=\textwidth]{../workflow/notebooks/20240826-z1/sensitivity-bars-m10pc.pdf} \\
    (b) 20\% lower import costs \\
    \includegraphics[width=\textwidth]{../workflow/notebooks/20240826-z1/sensitivity-bars-m20pc.pdf} \\
    \caption{\textbf{Potential for cost reductions with reduced sets of import options for lower import costs.} This figure includes the sensitivity with
    10\% and 20\% lower import costs for all fuels but electricity.}
    \label{fig:si:subsets-lower}
\end{figure}

\begin{figure*}
    \small
    (a) 50\% higher import costs \\
    \includegraphics[width=\textwidth]{../workflow/notebooks/20240826-z1/sensitivity-import-volume-AC+H21.5+CH41.5+NH31.5+FT1.5+MeOH1.5+HBI1.5+St1.5.pdf} \\
    (b) 30\% higher import costs \\
    \includegraphics[width=\textwidth]{../workflow/notebooks/20240826-z1/sensitivity-import-volume-AC+H21.3+CH41.3+NH31.3+FT1.3+MeOH1.3+HBI1.3+St1.3.pdf} \\
    \caption{\textbf{Sensitivity of import volume on total system cost and composition for varying import costs.} This figure includes the sensitivity with
    50\% and 30\% higher import costs for all fuels but electricity.}
    \label{fig:si:volume-higher-2}
\end{figure*}

\begin{figure*}
    \small
    (a) 20\% higher import costs \\
    \includegraphics[width=\textwidth]{../workflow/notebooks/20240826-z1/sensitivity-import-volume-AC+H21.2+CH41.2+NH31.2+FT1.2+MeOH1.2+HBI1.2+St1.2.pdf} \\
    (b) 10\% higher import costs \\
    \includegraphics[width=\textwidth]{../workflow/notebooks/20240826-z1/sensitivity-import-volume-AC+H21.1+CH41.1+NH31.1+FT1.1+MeOH1.1+HBI1.1+St1.1.pdf} \\
    \caption{\textbf{Sensitivity of import volume on total system cost and composition for varying import costs.} This figure includes the sensitivity with
    10\% and 20\% higher import costs for all fuels but electricity.}
    \label{fig:si:volume-higher-1}
\end{figure*}

\begin{figure*}
    \small
    (a) 10\% lower import costs \\
    \includegraphics[width=\textwidth]{../workflow/notebooks/20240826-z1/sensitivity-import-volume-AC+H20.9+CH40.9+NH30.9+FT0.9+MeOH0.9+HBI0.9+St0.9.pdf} \\
    (b) 20\% lower import costs \\
    \includegraphics[width=\textwidth]{../workflow/notebooks/20240826-z1/sensitivity-import-volume-AC+H20.8+CH40.8+NH30.8+FT0.8+MeOH0.8+HBI0.8+St0.8.pdf} \\
    \caption{\textbf{Sensitivity of import volume on total system cost and composition for varying import costs.} This figure includes the sensitivity with
    10\% and 20\% lower import costs for all fuels but electricity.}
    \label{fig:si:volume-lower-1}
\end{figure*}

\begin{figure*}
    \small
    (a) 30\% lower import costs \\
    \includegraphics[width=\textwidth]{../workflow/notebooks/20240826-z1/sensitivity-import-volume-AC+H20.7+CH40.7+NH30.7+FT0.7+MeOH0.7+HBI0.7+St0.7.pdf} \\
    (b) 50\% lower import costs \\
    \includegraphics[width=\textwidth]{../workflow/notebooks/20240826-z1/sensitivity-import-volume-AC+H20.5+CH40.5+NH30.5+FT0.5+MeOH0.5+HBI0.5+St0.5.pdf} \\
    \caption{\textbf{Sensitivity of import volume on total system cost and composition for varying import costs.} This figure includes the sensitivity with
    30\% and 50\% higher import costs for all fuels but electricity.}
    \label{fig:si:volume-lower-2}
\end{figure*}


\begin{figure*}
    \includegraphics[width=\textwidth]{../workflow/notebooks/20240826-z1/sensitivity-lines.pdf}
    \caption{\textbf{Sensitivity of import volume on total system cost with
    subsets of import vectors available.} Supplement to Figure 6. The volume of imports is exogenously
    forced for these runs and coloured lines represent certain restrictions in available import vectors.}
    \label{fig:si:volume-subsets}
\end{figure*}

% \begin{figure*}
%     \centering
%     \includegraphics[width=0.75\textwidth]{../workflow/notebooks/20240826-z1/sensitivity-bars-all-wAC.pdf}
%     \caption{\textbf{Effect of import cost variations of all vectors (including
%     electricity) on cost savings and import shares with all vectors allowed.}
%     Supplement to Figure 5.}
%     \label{fig:si:sensitivity-wac}
% \end{figure*}

\begin{figure*}
    \centering
    (a) only electricity imports \\
    \includegraphics[width=0.57\textwidth]{../workflow/pypsa-eur/results/20240826-z1/graphics/import_shares/s_115_lvopt__imp+AC_2050.pdf}
    \includegraphics[width=0.41\textwidth]{../workflow/pypsa-eur/results/20240826-z1/graphics/import_sankey/s_115_lvopt__imp+AC_2050.pdf} \\

    (b) only hydrogen imports \\
    \includegraphics[width=0.57\textwidth]{../workflow/pypsa-eur/results/20240826-z1/graphics/import_shares/s_115_lvopt__imp+H2_2050.pdf}
    \includegraphics[width=0.41\textwidth]{../workflow/pypsa-eur/results/20240826-z1/graphics/import_sankey/s_115_lvopt__imp+H2_2050.pdf} \\
    \caption{\textbf{Import shares, mix and trade flows for import scenarios with restricted
    import vectors.} For only electricity imports (a) and only hydrogen imports
    (b). Supplement to Figure 3.}
    \label{fig:si:import-shares-a}
\end{figure*}

\begin{figure*}
    \centering
    (a) 10\% lower import costs (all carriers but electricity) \\
    \includegraphics[width=0.57\textwidth]{../workflow/pypsa-eur/results/20240826-z1/graphics/import_shares/s_115_lvopt__imp+AC+H20.9+CH40.9+NH30.9+FT0.9+MeOH0.9+HBI0.9+St0.9_2050.pdf}
    \includegraphics[width=0.41\textwidth]{../workflow/pypsa-eur/results/20240826-z1/graphics/import_sankey/s_115_lvopt__imp+AC+H20.9+CH40.9+NH30.9+FT0.9+MeOH0.9+HBI0.9+St0.9_2050.pdf} \\

    (b) 10\% lower import cost (only carbonaceous fuels) \\
    \includegraphics[width=0.57\textwidth]{../workflow/pypsa-eur/results/20240826-z1/graphics//import_shares/s_115_lvopt__imp+AC+H2+CH40.9+NH3+FT0.9+MeOH0.9+HBI+St_2050.pdf}
    \includegraphics[width=0.41\textwidth]{../workflow/pypsa-eur/results/20240826-z1/graphics//import_sankey/s_115_lvopt__imp+AC+H2+CH40.9+NH3+FT0.9+MeOH0.9+HBI+St_2050.pdf} \\
    \caption{\textbf{Import shares, mix and trade flows for import scenarios with 10\% lower
    costs.} For all carries but electricity (a) and only carbonaceous fuels (b).
    Supplement to Figure 3.}
    \label{fig:si:import-shares-b}
\end{figure*}

\begin{figure*}
    \footnotesize
    \begin{tabular}{ccc}
        (a) H$_2$ / no imports allowed & (b) H$_2$ / only hydrogen imports & (c) H$_2$ / all imports allowed \\
        \includegraphics[width=0.325\textwidth]{../workflow/notebooks/20240826-z1/csc-h2-noimp} &
        \includegraphics[width=0.325\textwidth]{../workflow/notebooks/20240826-z1/csc-h2-imp+H2} &
        \includegraphics[width=0.325\textwidth]{../workflow/notebooks/20240826-z1/csc-h2-imp} \\
        (d) MeOH / no imports allowed & (e) MeOH / only hydrogen imports & (f) MeOH / all imports allowed \\
        \includegraphics[width=0.325\textwidth]{../workflow/notebooks/20240826-z1/csc-meoh-noimp} &
        \includegraphics[width=0.325\textwidth]{../workflow/notebooks/20240826-z1/csc-meoh-imp+H2} &
        \includegraphics[width=0.325\textwidth]{../workflow/notebooks/20240826-z1/csc-meoh-imp} \\
        (g) Fischer-Tropsch / no imports allowed & (h) Fischer-Tropsch / only hydrogen imports & (i) Fischer-Tropsch / all imports allowed \\
        \includegraphics[width=0.325\textwidth]{../workflow/notebooks/20240826-z1/csc-ftf-noimp} &
        \includegraphics[width=0.325\textwidth]{../workflow/notebooks/20240826-z1/csc-ftf-imp+H2} &
        \includegraphics[width=0.325\textwidth]{../workflow/notebooks/20240826-z1/csc-ftf-imp} \\
    \end{tabular}
    \caption{\textbf{Domestic cost supply curves for different import scenarios and carriers.}
        The cost supply curves are built using sorted spatio-temporal market
        values with corresponding production volumes per region and snapshot.
        Dotted lines show the import cost supply curves (in steps of 500~TWh) of
        the respective carriers as reference.}
    \label{fig:si:cost-supply-curves}
\end{figure*}

\begin{figure*}
    \footnotesize
    (a) no imports allowed \\
    \includegraphics[width=\textwidth]{../workflow/notebooks/20240826-z1/market-value-ts-noimp.pdf} \\
    (b) only hydrogen imports allowed \\
    \includegraphics[width=\textwidth]{../workflow/notebooks/20240826-z1/market-value-ts-imp+H2.pdf} \\
    (c) all imports allowed \\
    \includegraphics[width=\textwidth]{../workflow/notebooks/20240826-z1/market-value-ts-imp.pdf}
    \caption{\textbf{Temporal variations of domestic hydrogen and
    Fischer-Tropsch production costs for different import scenarios.} Supplement to Figure 4. Dotted lines show
    the minimum import costs as a reference. The solid lines show the
    production-weighted average costs across all regions. The shaded areas show
    the regional range between lowest and highest production cost in any of the
    115 model regions.}
    \label{fig:si:market-value-ts}
\end{figure*}



\begin{figure*}
    \includegraphics[width=0.8\textwidth]{../workflow/notebooks/20240826-z1/balances-electricity.pdf}
    \includegraphics[width=0.8\textwidth]{../workflow/notebooks/20240826-z1/balances-heat.pdf}
    \includegraphics[width=0.8\textwidth]{../workflow/notebooks/20240826-z1/balances-H2.pdf}
    \includegraphics[width=0.8\textwidth]{../workflow/notebooks/20240826-z1/balances-gas.pdf}
    \caption{\textbf{Energy balances for three import scenarios for the carriers electricity, heat, hydrogen and gas.}
    }
    \label{fig:si:balances-a}
\end{figure*}

\begin{figure*}
    \includegraphics[width=0.8\textwidth]{../workflow/notebooks/20240826-z1/balances-NH3.pdf}
    \includegraphics[width=0.8\textwidth]{../workflow/notebooks/20240826-z1/balances-methanol.pdf}
    \includegraphics[width=0.8\textwidth]{../workflow/notebooks/20240826-z1/balances-oil.pdf}
    \includegraphics[width=0.8\textwidth]{../workflow/notebooks/20240826-z1/balances-co2.pdf}
    \includegraphics[width=0.8\textwidth]{../workflow/notebooks/20240826-z1/balances-co2 stored.pdf}
    \caption{\textbf{
        Energy balances for three import scenarios for the carriers
        ammonia, methanol, and oil, as well as stored and atmospheric carbon dioxide.
    }
    }
    \label{fig:si:balances-b}
\end{figure*}


\begin{figure*}
    \includegraphics[width=\textwidth]{../workflow/notebooks/20240826-z1/infrastructure-map-2x3-B.pdf}
    \caption{\textbf{Layout of European energy infrastructure for different
    import scenarios.} Infrastructure build-out with full power-to-X waste heat
    availability in domestic scenario (first row), without steel and ammonia
    industry relocation (second row), and only electricity imports (third row).
    Left column shows the regional electricity supply mix (pies), added HVDC and
    HVAC transmission capacity (lines), and the siting of battery storage
    (choropleth). Right column shows the hydrogen supply (top half of pies) and
    consumption (bottom half of pies), net flow and direction of hydrogen in
    newly built and retrofitted pipelines (lines), and the siting of hydrogen
    storage subject to geological potentials (choropleth). Total volumes of
    transmission expansion are given in TWkm, which is the sum product of the
    capacity and length of individual connections.  Maps made with Natural Earth.}
    \label{fig:si:infra-b}
\end{figure*}

\begin{figure*}
    \includegraphics[width=\textwidth]{../workflow/notebooks/20240826-z1/infrastructure-map-2x3-C.pdf}
    \caption{\textbf{Layout of European energy infrastructure for different import scenarios.} Sensitivities of infrastructure build-out to import costs.
    Left column shows the regional electricity supply mix (pies), added HVDC and HVAC transmission capacity (lines), and the siting of battery storage (choropleth).
        Right column shows the hydrogen supply (top half of pies) and consumption (bottom half of pies), net flow and direction of hydrogen in newly built and retrofitted pipelines (lines), and the siting of hydrogen storage subject to geological potentials (choropleth).
        Total volumes of transmission expansion are given in TWkm, which is the sum product of the capacity and length of individual connections.
        Maps made with Natural Earth.
    }
    \label{fig:si:infra-c}
\end{figure*}

\begin{figure*}
    \includegraphics[width=\textwidth]{../workflow/notebooks/20240826-z1/infrastructure-map-2x3-D.pdf}
    \caption{\textbf{Layout of European energy infrastructure for different
    import scenarios.}  Infrastructure build-out with no power-to-X flexibility
    (first row), without hydrogen imports or pipeline network (second row), only
    hydrogen derivative imports (third row). Left column shows the regional
    electricity supply mix (pies), added HVDC and HVAC transmission capacity
    (lines), and the siting of battery storage (choropleth). Right column shows
    the hydrogen supply (top half of pies) and consumption (bottom half of
    pies), net flow and direction of hydrogen in newly built and retrofitted
    pipelines (lines), and the siting of hydrogen storage subject to geological
    potentials (choropleth). Total volumes of transmission expansion are given
    in TWkm, which is the sum product of the capacity and length of individual
    connections.  Maps made with Natural Earth.}
    \label{fig:si:infra-d}
\end{figure*}


\begin{figure*}
    \includegraphics[width=\textwidth]{../workflow/notebooks/20240826-z1/infrastructure-map-2x3-E.pdf}
    \caption{\textbf{Layout of European energy infrastructure for different
    import scenarios.}  Infrastructure build-out with technology assumptions for
    2050 and no imports (first row), with technology assumptions for 2030 and no
    imports (second row), with technology assumptions for 2050 and all imports
    allowed (third row). Left column shows the regional electricity supply mix
    (pies), added HVDC and HVAC transmission capacity (lines), and the siting of
    battery storage (choropleth). Right column shows the hydrogen supply (top
    half of pies) and consumption (bottom half of pies), net flow and direction
    of hydrogen in newly built and retrofitted pipelines (lines), and the siting
    of hydrogen storage subject to geological potentials (choropleth). Total
    volumes of transmission expansion are given in TWkm, which is the sum
    product of the capacity and length of individual connections.  Maps made with Natural Earth.}
    \label{fig:si:infra-e}
\end{figure*}

\begin{figure*}
    \centering
    \footnotesize
    (a) average utilisation rate of import HVDC links \\
    \includegraphics[width=\textwidth]{../workflow/pypsa-eur/results/20240826-z1/graphics/heatmap_timeseries/s_115_lvopt__imp_2050/ts-heatmap-utilisation_rate-import_hvdc-to-elec.pdf} \\
    (b) average utilisation rate of import hydrogen pipelines \\
    \includegraphics[width=\textwidth]{../workflow/pypsa-eur/results/20240826-z1/graphics/heatmap_timeseries/s_115_lvopt__imp_2050/ts-heatmap-utilisation_rate-import_infrastructure_pipeline-h2.pdf}
    \caption{\textbf{Temporal usage pattern of electricity and hydrogen storage.}
    The capacity-weighted average utilisation rate is 75+\% for import HVDC links
    and 90+\% for hydrogen pipelines. For hydrogen import pipelines, a clear
    seasonal pattern with higher utilisation in winter is visible. For other
    energy or material imports than hydrogen and electricity, the timing of
    imports is not informatively captured due to problem degeneracy caused by
    negligible storage costs of carbonaceous fuels and steel.}
    \label{fig:si:import-operation}
\end{figure*}

\begin{figure*}
    \centering
    \footnotesize
    (a) state of charge of hydrogen storage \\
    \includegraphics[width=\textwidth]{../workflow/pypsa-eur/results/20240826-z1/graphics/heatmap_timeseries/s_115_lvopt___2050/ts-heatmap-soc-h2_store.pdf} \\
    (b) state of charge profile of thermal storage in district heating \\
    \includegraphics[width=\textwidth]{../workflow/pypsa-eur/results/20240826-z1/graphics/heatmap_timeseries/s_115_lvopt___2050/ts-heatmap-soc-urban_central_water_tanks.pdf}
    \caption{\textbf{State-of-charge profile of long-duration energy storage in
    the domestic scenario without imports.} Hydrogen storage is used
    synoptically, whereas district heating storage is used seasonally.}
    \label{fig:si:storage-operation}
\end{figure*}

\begin{figure*}
    \centering
    \footnotesize
    (a) operation of district heating gas CHPs without imports  \\
    \includegraphics[width=0.95\textwidth]{../workflow/pypsa-eur/results/20240826-z1/graphics/heatmap_timeseries/s_115_lvopt___2050/ts-heatmap-utilisation_rate-urban_central_gas_chp.pdf} \\
    (b) operation of district heating gas CHPs with imports \\
    \includegraphics[width=0.95\textwidth]{../workflow/pypsa-eur/results/20240826-z1/graphics/heatmap_timeseries/s_115_lvopt__imp_2050/ts-heatmap-utilisation_rate-urban_central_gas_chp.pdf}
    (c) load-weighted average electricity price with imports \\
    \includegraphics[width=0.95\textwidth]{../workflow/pypsa-eur/results/20240826-z1/graphics/heatmap_timeseries/s_115_lvopt__imp_2050/ts-heatmap-marginal_price-ac.pdf}
    \caption{\textbf{Temporal usage pattern of backup power/heat in relation to
    import scenario.} In both cases, gas CHPs were the main backup option,
    running for few days during winter when prices are higher. Backup power
    plant operation is higher when imports displace power-to-X flexibilities.}
    \label{fig:si:backup-power}
\end{figure*}

\begin{figure*}
    \centering
    \footnotesize
    (a) no imports allowed  \\
    \includegraphics[height=0.45\textheight]{../workflow/pypsa-eur/results/20240826-z1/graphics/backup_map/s_115_lvopt___2050.pdf} \\
    (b) all imports allowed \\
    \includegraphics[height=0.45\textheight]{../workflow/pypsa-eur/results/20240826-z1/graphics/backup_map/s_115_lvopt__imp_2050.pdf}
    \caption{\textbf{Spatial distribution of backup power for scenarios with all
    imports allowed and no imports.} Batteries are concentrated in Southern
    Europe. Gas-fired combined heat and power plants are distributed across
    Central Europe where electricity prices are higher, with lower build-out
    when no imports are allowed as domestic power-to-X flexibility reduces the
    need for backup capacities. Instead of firing up reserve power plants, the
    production of power-to-X plants is curtailed. Power transmission
    infrastructure distributes backup capacities across Europe where there are none.
    Blue lines represent HVAC lines, rosa lines represent HVDC links.  Maps made with Natural Earth.}
    \label{fig:si:backup-power-map}
\end{figure*}


\begin{figure*}
    \centering
    \footnotesize
    % (a) electrolysis  \\
    \includegraphics[width=\textwidth]{../workflow/pypsa-eur/results/20240826-z1/graphics/heatmap_timeseries/s_115_lvopt___2050/ts-heatmap-utilisation_rate-h2_electrolysis.pdf} \\
    % (b) Fischer-Tropsch synthesis \\
    \includegraphics[width=\textwidth]{../workflow/pypsa-eur/results/20240826-z1/graphics/heatmap_timeseries/s_115_lvopt__imp_2050/ts-heatmap-utilisation_rate-fischer-tropsch.pdf}
    % (c) methanol synthesis \\
    \includegraphics[width=\textwidth]{../workflow/pypsa-eur/results/20240826-z1/graphics/heatmap_timeseries/s_115_lvopt__imp_2050/ts-heatmap-utilisation_rate-methanolisation.pdf}
    \caption{\textbf{Temporal usage patterns of selected power-to-X technologies
    in scenario without imports.} Electrolysis clearly reacts to the
    availability of wind and solar electricity, despite high unit investment
    costs of 950\euro/kW$_e$. Fischer-Tropsch runs more steadily, reducing
    production over much of the challenging winter months to its minimum
    part-load of 50\%. The methanolisation process can be used more flexibly
    with a minimum part load of 20\%.}
    \label{fig:si:power-to-x}
\end{figure*}


\begin{figure*}
    \centering
    \footnotesize
    (a) without imports \\
    \includegraphics[width=\textwidth]{../workflow/pypsa-eur/results/20240826-z1/graphics/balance_timeseries/s_115_lvopt___2050/ts-balance-electricity-D.pdf} \\
    (b) with imports \\
    \includegraphics[width=\textwidth]{../workflow/pypsa-eur/results/20240826-z1/graphics/balance_timeseries/s_115_lvopt__imp_2050/ts-balance-electricity-D.pdf}
    \caption{\textbf{Energy balance time series for electricity with and without imports.} Resampled to daily averages. Positive numbers indicate supply, negative numbers indicate consumption.}
    \label{fig:si:balance-elec}
\end{figure*}

\begin{figure*}
    \centering
    \footnotesize
    (a) without imports \\
    \includegraphics[width=\textwidth]{../workflow/pypsa-eur/results/20240826-z1/graphics/balance_timeseries/s_115_lvopt___2050/ts-balance-heat-D.pdf} \\
    (b) with imports \\
    \includegraphics[width=\textwidth]{../workflow/pypsa-eur/results/20240826-z1/graphics/balance_timeseries/s_115_lvopt__imp_2050/ts-balance-heat-D.pdf}
    \caption{\textbf{Energy balance time series for heat with and without imports.} Resampled to daily averages. Positive numbers indicate supply, negative numbers indicate consumption.}
    \label{fig:si:balance-heat}
\end{figure*}

\begin{figure*}
    \centering
    \footnotesize
    (a) without imports \\
    \includegraphics[width=\textwidth]{../workflow/pypsa-eur/results/20240826-z1/graphics/balance_timeseries/s_115_lvopt___2050/ts-balance-hydrogen-D.pdf} \\
    (b) with imports \\
    \includegraphics[width=\textwidth]{../workflow/pypsa-eur/results/20240826-z1/graphics/balance_timeseries/s_115_lvopt__imp_2050/ts-balance-hydrogen-D.pdf}
    \caption{\textbf{Energy balance time series for hydrogen with and without imports.} Resampled to daily averages. Positive numbers indicate supply, negative numbers indicate consumption.}
    \label{fig:si:balance-h2}
\end{figure*}

\begin{figure*}
    \centering
    \footnotesize
    (a) gas transmission network \\
    \includegraphics[width=.85\textwidth, trim=1cm 5cm 1.5cm 5cm, clip]{../workflow/pypsa-eur/resources/20240826-z1/graphics/gas-network-unclustered.pdf} \\
    (b) electricity transmission network \\
    \includegraphics[width=.8\textwidth]{../workflow/pypsa-eur/resources/20240826-z1/graphics/power-network-unclustered.pdf}
    \caption{\textbf{Gas and electricity transmission network data.} For gas
    transmission, the map shows pipelines sized and colored by rated capacity,
    fossil gas extraction sites, storage locations, pipeline entrypoints, and
    LNG terminals. The data comes from SciGRID\_gas and is supplemented with
    data from Global Energy Monitor. For power transmission, the map shows
    existing transmission lines at and above 220~kV taken from OpenStreetMap (\url{https://www.openstreetmap.org/}),
    supplemented with planned TYNDP projects (\url{https://tyndp.entsoe.eu/}). Maps made with Natural Earth.}
    \label{fig:si:networks-raw}
\end{figure*}

\begin{figure*}
    \footnotesize
    (a) unclustered raw data from Caglayan et al. (2020) \\
    \begin{center}
        \includegraphics[width=0.65\textwidth]{../workflow/pypsa-eur/resources/20240826-z1/graphics/salt-caverns}\\
    \end{center}
    (b) clustered near-shore cavern storage potential \\
    \begin{center}
        \includegraphics[width=0.75\textwidth, trim=0cm 2cm 0cm 3cm, clip]{../workflow/pypsa-eur/resources/20240826-z1/graphics/salt-caverns-s-115-nearshore}
    \end{center}
    \caption{\textbf{Locations considered for geological hydrogen storage in
    salt caverns.} Data based on Caglayan et
    al.~\cite{caglayanTechnicalPotentialSalt2020}. Only near-shore caverns are
    considered to minimize environmental impact of brine disposal. Maps made
    with Natural Earth.}
    \label{fig:si:hydrogen-caverns}
\end{figure*}


\begin{figure*}
    \centering
    \includegraphics[width=\textwidth]{../workflow/pypsa-eur/resources/20240826-z1/graphics/industrial-sites.pdf}
    \caption{\textbf{Considered locations of industrial production sites by
    sector.} Data based on Manz and
    Fleiter (2018). Marker size scales
    proportionally to the emissions of the respective site. This data is used
    for the spatial distribution of industrial energy and feedstock demands.
    Maps made with Natural Earth.}
    \label{fig:si:industrial-sites}
\end{figure*}

