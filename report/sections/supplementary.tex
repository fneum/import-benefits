
\section*{Limitations}

Several limitations of the study should be noted. The model is a simplified
representation of the energy system and does not capture all aspects of the
real-world transition. First of all, the optimization results represent a
long-term equilibrium that does not account for the transition process from the
mostly fossil-based energy system to a carbon-neutral one. This means that the
model does not capture constraints regards neither the mid-term ramp-up of green
energy export capacities nor the speed at which supporting infrastructure could
scale up. Furthermore, global competition betweeen potential exporting
countries, as well as between importing countries, make import volumes and
prices more difficult to predict than our purely cost-based analysis can
provide. Besides unclear market developments, local issues in exporting regions
such as potential water scarcity to produce large amounts of hydrogen in
renewable-rich but arid countries is not addressed. Moreover, the fact that we
assume that import costs are not time-dependent and available on demand may
underestimate some intermediate storage requirements for imports at
entry-points, especially for hydrogen.

In terms of industry relocation within Europe, our modelling is constrained to
the migration of steel, ammonia and chemicals. Other sectors like concrete and
alumina production is not considered for relocation. We make this choice so that
relocation within Europe can compete with imports from abroad, which equates a
migration of the industry branch out of the European value chain. Potential
repurcussions on local jobs and relocatoin costs are not captured by the model.
Finally, it should be acknowledged that our assumptions about the broad
availability of district heating networks across Europe to absorb
\mbox{power-to-X} waste heat are relatively progressive as it is not certain
that central heating will become available in regions which have not based their
heating supply on district heating so far.

\newpage

\section*{Causes and severity of import cost uncertainty}

In \cref{tab:cost-uncertainty}, we vary some of the techno-economic assumptions
for evaluating green fuel supply chains in the exporting countries to justify
the range of import cost deviations from the defaults. These relate to
technology costs, financing costs, excess power and heat handling, fuel
synthesis flexibility, and the availability of geological hydrogen storage and
alternative sources of CO$_2$. For all the following sensitivities, it should be
noted that they are not additive.

A higher weighted average cost of capital (WACC) than the uniformly applied 7\%,
e.g.~due to higher project risks, and lower WACC, e.g.~due to the
government-backing of a project, have a substantial effect on the import cost
calculations; an increase or decrease by just one percentage point already
alters the costs per unit of energy by more than
7\%.
 
Likewise, a failure to achieve the anticipated cost reductions for electrolysers
and DAC systems would also result in far-reaching cost increases for green
energy imports, especially if the fuel contains carbon. The availability of
biogenic CO$_2$ (or fossil CO$_2$ from industrial processes that is largely
cycled between use and synthesis and, hence, not emitted to the atmosphere) can
reduce the green fuel cost by 20\% if it can be provided for 60 \euro{}/t and by
10\% if made available for 100 \euro{}/t.

The default assumptions for export supply chains assume islanded fuel synthesis
sites that are not connected to the local electricity system. The isolation
drives the system into high curtailment rates of 8\%. If surplus electricity
production could be sold and absorbed by the local power grid, considerable cost
reductions could be achieved (refer to \cref{tab:cost-uncertainty}).

Besides integration with the local energy system, process integration using
waste heat streams from power-to-X plants for direct air capture and flexible
Fischer-Tropsch synthesis similar to methanolisation can also reduce fuel cost
by 3-5\% each. Conditions that would allow for geological hydrogen storage
reduce the need for flexible synthesis plant operation and could reduce import
costs by more than 7\%. However, even though many potential export countries
possess geological hydrogen storage potential, suitable storage sites are not
always co-located with the countries' best renewable potentials.

Finally, cost rises can also be expected if the most competitive exporting
countries are not offering to export green energy. Argentina and Chile have a
margin of 10 \euro{}/MWh over the next cheapest exporting country
(i.e.~Australia, Algeria and Libya with 120 \euro{}/MWh). If these countries
were unavailable for import, costs would rise by almost 10\%.

\begin{table*}
    \footnotesize
    \centering
    \begin{tabular}{lrrrr}
        \toprule
        Factor & Change & Unit & Change & Unit\\
        \midrule
        higher WACC of 12\% (e.g.~high project risk) & +43.1 & \euro{}/MWh  &
        +39.3 & \% \\
        higher WACC of 10\% (e.g.~high project risk) & +25.3 & \euro{}/MWh  &
        +23.0 & \% \\
        higher WACC of 8\% (e.g.~high project risk) & +8.2 & \euro{}/MWh  & +7.4
        & \% \\
        higher direct air capture investment cost (+200\%) & +55.8 & \euro{}/MWh
        & +50.8 & \% \\
        higher direct air capture investment cost (+100\%) & +28.1 & \euro{}/MWh
        & +25.6 & \% \\
        higher direct air capture investment cost (+50\%) & +14.1 & \euro{}/MWh
        & +12.9 & \% \\
        higher direct air capture investment cost (+25\%) & +7.1 & \euro{}/MWh &
        +6.5 & \% \\
        higher electrolysis investment cost (+200\%) & +29.2 & \euro{}/MWh  &
        +26.6 & \% \\
        higher electrolysis investment cost (+100\%) & +16.7 & \euro{}/MWh  &
        +15.2 & \% \\
        higher electrolysis investment cost (+50\%) & +9.0 & \euro{}/MWh  & +8.2
        & \% \\
        higher electrolysis investment cost (+25\%) & +4.7 & \euro{}/MWh  & +4.3
        & \% \\
        Argentina and Chile not available for export & +10.1 & \euro{}/MWh  &
        +9.2 & \% \\
        \midrule
        lower WACC of 3\% (e.g.~government guarantees) & -29.5 & \euro{}/MWh  &
        -26.8 & \% \\
        lower WACC of 5\% (e.g.~government guarantees) & -15.5 & \euro{}/MWh  &
        -14.1 & \% \\
        lower WACC of 6\% (e.g.~government guarantees) & -8.0 & \euro{}/MWh  &
        -7.2 & \% \\
        sell excess curtailed electricity at 40 \euro{}/MWh & -24.7 & \euro{}/MWh  &
        -22.6 & \% \\
        sell excess curtailed electricity at 30 \euro{}/MWh & -15.6 & \euro{}/MWh  &
        -14.2 & \% \\
        sell excess curtailed electricity at 20 \euro{}/MWh & -8.0 & \euro{}/MWh  &
        -7.2 & \% \\
        option to use available biogenic or cycled \ce{CO2} for 60 \euro{}/t & -21.7 &
        \euro{}/MWh  & -19.7 & \% \\
        option to use available biogenic or cycled \ce{CO2} for 80 \euro{}/t & -16.1 &
        \euro{}/MWh  & -14.7 & \% \\
        option to use available biogenic or cycled \ce{CO2} for 100 \euro{}/t & -10.6 &
        \euro{}/MWh  & -9.7 & \% \\
        option to build geological hydrogen storage at 2.4 \euro{}/kWh
        (reduction by 95\%) & -8.2 & \euro{}/MWh  & -7.4 & \% \\
        option to use power-to-X waste heat streams for direct air capture &
        -3.8 & \euro{}/MWh  & -3.4 & \% \\
        highly flexible operation of fuel synthesis plant (20\% minimum
        part-load instead of 70\%) & -5.4 & \euro{}/MWh  & -4.9 & \% \\
        \bottomrule
    \end{tabular}
    \caption{\textbf{Examples for potential import cost increases or decreases.}
    The table presents cost sensitivities in absolute and relative terms based
    on the supply chain for producing Fischer-Tropsch fuels in Argentina for
    export to Europe. The reference fuel import cost for this case is 109.8
    \euro{}/MWh. Responses to changes in the input assumptions are not
    additive.}
    \label{tab:cost-uncertainty}
\end{table*}

\newpage


\section*{Supplementary Figures}
\begin{figure}[!htb]
    \footnotesize
    (a) 10\% higher import costs \\
    \includegraphics[width=\textwidth]{20231025-zecm/sensitivity-bars-p10pc.pdf} \\
    (b) 10\% lower import costs \\
    \includegraphics[width=\textwidth]{20231025-zecm/sensitivity-bars-m10pc.pdf} \\
    (c) 20\% lower import costs \\
    \includegraphics[width=\textwidth]{20231025-zecm/sensitivity-bars-m20pc.pdf} \\
    \caption{\textbf{Potential for cost reductions with reduced sets of import options for varying import costs.}}
    \label{fig:si:subsets}
\end{figure}

\begin{figure*}
    \small
    (a) import costs +20\% \\
    \includegraphics[width=\textwidth]{20231025-zecm/sensitivity-import-volume-AC+H21.2+CH41.2+NH31.2+FT1.2+MeOH1.2+St1.2.pdf}
    \begin{tabular}{ll}
        (b) import costs +10\% & (c) import costs -10\% \\
        \includegraphics[width=0.495\textwidth, trim=0cm 0cm 12.8cm 0cm, clip]{20231025-zecm/sensitivity-import-volume-AC+H21.1+CH41.1+NH31.1+FT1.1+MeOH1.1+St1.1.pdf} &
        \includegraphics[width=0.495\textwidth, trim=0cm 0cm 12.8cm 0cm, clip]{20231025-zecm/sensitivity-import-volume-AC+H20.9+CH40.9+NH30.9+FT0.9+MeOH0.9+St0.9.pdf} \\
        (d) import costs -20\% & (e) import costs -30\% \\
        \includegraphics[width=0.495\textwidth, trim=0cm 0cm 12.8cm 0cm, clip]{20231025-zecm/sensitivity-import-volume-AC+H20.8+CH40.8+NH30.8+FT0.8+MeOH0.8+St0.8.pdf} &
        \includegraphics[width=0.495\textwidth, trim=0cm 0cm 12.8cm 0cm, clip]{20231025-zecm/sensitivity-import-volume-AC+H20.7+CH40.7+NH30.7+FT0.7+MeOH0.7+St0.7.pdf}
    \end{tabular}
    \caption{\textbf{Sensitivity of import volume on total system cost and composition for varying import costs.}}
    \label{fig:si:volume}
\end{figure*}


\begin{figure*}
    \includegraphics[width=\textwidth]{20231025-zecm/sensitivity-lines.pdf}
    \caption{\textbf{Sensitivity of import volume on total system cost with subsets of import vectors available.}}
    \label{fig:si:volume-subsets}
\end{figure*}

\begin{figure*}
    \footnotesize
    \begin{tabular}{cc}
        (a) only electricity imports & (b) only hydrogen imports \\
        \includegraphics[width=0.49\textwidth]{20231025-zecm/graphics/import_shares/s_110_lvopt__Co2L0-2190SEG-T-H-B-I-S-A-onwind+p0.5-imp+AC_2050.pdf} &
        \includegraphics[width=0.49\textwidth]{20231025-zecm/graphics/import_shares/s_110_lvopt__Co2L0-2190SEG-T-H-B-I-S-A-onwind+p0.5-imp+H2_2050.pdf} \\
        (c) -10\% import costs (all carriers but electricity) & (d) -10\% import cost (only carbonaceous fuels) \\
        \includegraphics[width=0.49\textwidth]{20231025-zecm/graphics/import_shares/s_110_lvopt__Co2L0-2190SEG-T-H-B-I-S-A-onwind+p0.5-imp+AC+H20.9+CH40.9+NH30.9+FT0.9+MeOH0.9+St0.9_2050.pdf} &
        \includegraphics[width=0.49\textwidth]{20231025-zecm/graphics/import_shares/s_110_lvopt__Co2L0-2190SEG-T-H-B-I-S-A-onwind+p0.5-imp+AC+H2+CH40.9+NH3+FT0.9+MeOH0.9+St_2050.pdf} \\
    \end{tabular}
    \caption{\textbf{Import shares and mix for different import scenarios.}}
    \label{fig:si:import-shares}
\end{figure*}

\begin{figure*}
    \footnotesize
    \begin{tabular}{ccc}
        (a) H$_2$ / no imports allowed & (b) H$_2$ / only hydrogen imports & (c) H$_2$ / all imports allowed \\
        \includegraphics[width=0.325\textwidth]{20231025-zecm/csc-h2-noimp} &
        \includegraphics[width=0.325\textwidth]{20231025-zecm/csc-h2-imp+H2} &
        \includegraphics[width=0.325\textwidth]{20231025-zecm/csc-h2-imp} \\
        (d) MeOH / no imports allowed & (e) MeOH / only hydrogen imports & (f) MeOH / all imports allowed \\
        \includegraphics[width=0.325\textwidth]{20231025-zecm/csc-meoh-noimp} &
        \includegraphics[width=0.325\textwidth]{20231025-zecm/csc-meoh-imp+H2} &
        \includegraphics[width=0.325\textwidth]{20231025-zecm/csc-meoh-imp} \\
        (g) Fischer-Tropsch / no imports allowed & (h) Fischer-Tropsch / only hydrogen imports & (i) Fischer-Tropsch / all imports allowed \\
        \includegraphics[width=0.325\textwidth]{20231025-zecm/csc-ftf-noimp} &
        \includegraphics[width=0.325\textwidth]{20231025-zecm/csc-ftf-imp+H2} &
        \includegraphics[width=0.325\textwidth]{20231025-zecm/csc-ftf-imp} \\
        (j) CH$_4$ / no imports allowed & (k) CH$_4$ / only hydrogen imports & (l) CH$_4$ / all imports allowed \\
        \includegraphics[width=0.325\textwidth]{20231025-zecm/csc-sab-noimp} &
        \includegraphics[width=0.325\textwidth]{20231025-zecm/csc-sab-imp+H2} &
        \includegraphics[width=0.325\textwidth]{20231025-zecm/csc-sab-imp} \\
    \end{tabular}
    \caption{\textbf{Domestic cost supply curves for different import scenarios and carriers.}
        The cost supply curves are built using sorted spatio-temporal market
        values with corresponding production volumes per region and snapshot. If
        the domestic supply curve is missing, no domestic production occured in
        the scenario. Shaded areas or dotted lines show import cost ranges of
        the respective carriers as reference.}
    \label{fig:si:cost-supply-curves}
\end{figure*}


\begin{figure*}
    \includegraphics[width=0.8\textwidth]{20231025-zecm/balances-electricity.pdf}
    \includegraphics[width=0.8\textwidth]{20231025-zecm/balances-heat.pdf}
    \includegraphics[width=0.8\textwidth]{20231025-zecm/balances-H2.pdf}
    \includegraphics[width=0.8\textwidth]{20231025-zecm/balances-gas.pdf}
    \caption{\textbf{Energy balances for three import scenarios for the carriers electricity, heat, hydrogen and gas.}
    }
    \label{fig:si:balances-a}
\end{figure*}

\begin{figure*}
    \includegraphics[width=0.8\textwidth]{20231025-zecm/balances-NH3.pdf}
    \includegraphics[width=0.8\textwidth]{20231025-zecm/balances-methanol.pdf}
    \includegraphics[width=0.8\textwidth]{20231025-zecm/balances-oil.pdf}
    \includegraphics[width=0.8\textwidth]{20231025-zecm/balances-co2.pdf}
    \includegraphics[width=0.8\textwidth]{20231025-zecm/balances-co2 stored.pdf}
    \caption{\textbf{
        Energy balances for three import scenarios for the carriers
        ammonia, methanol, and oil, as well as stored and atmospheric carbon dioxide.
    }
    }
    \label{fig:si:balances-b}
\end{figure*}


\begin{figure*}
    \includegraphics[width=\textwidth]{20231025-zecm/infrastructure-map-2x3-B.pdf}
    \caption{\textbf{Layout of European energy infrastructure for different import scenarios.} Role of PtX waste heat, hydrogen network, and electricity imports.
    Left column shows the regional electricity supply mix (pies), added HVDC and HVAC transmission capacity (lines), and the siting of battery storage (choropleth).
        Right column shows the hydrogen supply (top half of pies) and consumption (bottom half of pies), net flow and direction of hydrogen in newly built pipelines (lines), and the siting of hydrogen storage subject to geological potentials (choropleth).
        Total volumes of transmission expansion are given in TWkm, which is the sum product of the capacity and length of individual connections.
    }
    \label{fig:si:infra-b}
\end{figure*}

\begin{figure*}
    \includegraphics[width=\textwidth]{20231025-zecm/infrastructure-map-2x3-C.pdf}
    \caption{\textbf{Layout of European energy infrastructure for different import scenarios.} Sensitivities of infrastructure to import costs.
    Left column shows the regional electricity supply mix (pies), added HVDC and HVAC transmission capacity (lines), and the siting of battery storage (choropleth).
        Right column shows the hydrogen supply (top half of pies) and consumption (bottom half of pies), net flow and direction of hydrogen in newly built pipelines (lines), and the siting of hydrogen storage subject to geological potentials (choropleth).
        Total volumes of transmission expansion are given in TWkm, which is the sum product of the capacity and length of individual connections.
    }
    \label{fig:si:infra-c}
\end{figure*}

\begin{figure*}
    \includegraphics[width=\textwidth]{20231025-zecm/infrastructure-map-2x3-D.pdf}
    \caption{\textbf{Layout of European energy infrastructure for different import scenarios.} Role of industry relocation and focus on carbonaceous fuel imports.
    Left column shows the regional electricity supply mix (pies), added HVDC and HVAC transmission capacity (lines), and the siting of battery storage (choropleth).
        Right column shows the hydrogen supply (top half of pies) and consumption (bottom half of pies), net flow and direction of hydrogen in newly built pipelines (lines), and the siting of hydrogen storage subject to geological potentials (choropleth).
        Total volumes of transmission expansion are given in TWkm, which is the sum product of the capacity and length of individual connections.
    }
    \label{fig:si:infra-d}
\end{figure*}



\begin{figure*}
    \centering
    \footnotesize
    (a) average utilisation rate of import HVDC links \\
    \includegraphics[width=\textwidth]{20231025-zecm/graphics/heatmap_timeseries/s_110_lvopt__Co2L0-2190SEG-T-H-B-I-S-A-onwind+p0.5-imp_2050/ts-heatmap-utilisation_rate-import_hvdc-to-elec.pdf} \\
    (b) average utilisation rate of import hydrogen pipelines \\
    \includegraphics[width=\textwidth]{20231025-zecm/graphics/heatmap_timeseries/s_110_lvopt__Co2L0-2190SEG-T-H-B-I-S-A-onwind+p0.5-imp_2050/ts-heatmap-utilisation_rate-import_pipeline-h2.pdf}
    \caption{\textbf{Temporal usage pattern of electricity and hydrogen storage.}
    The capacity-weighted average utilisation rate is 72\% for import HVDC links and 45\% for
    hydrogen pipelines. For hydrogen import pipelines, a clear seasonal pattern
    with higher utilisation in winter is visible, demonstrated by an average utilisation rate of 56\% from November to April and 35\% from May to October.
    For other energy or material
    imports than hydrogen and electricity, the timing of imports is not
    informatively captured due to problem degeneracy caused by negligible
    storage costs of carbonaceous fuels and steel.}
    \label{fig:si:import-operation}
\end{figure*}

\begin{figure*}
    \centering
    \footnotesize
    (a) without imports \\
    \includegraphics[width=\textwidth]{20231025-zecm/graphics/balance_timeseries/s_110_lvopt__Co2L0-2190SEG-T-H-B-I-S-A-onwind+p0.5_2050/ts-balance-electricity-D.pdf} \\
    (b) with imports \\
    \includegraphics[width=\textwidth]{20231025-zecm/graphics/balance_timeseries/s_110_lvopt__Co2L0-2190SEG-T-H-B-I-S-A-onwind+p0.5-imp_2050/ts-balance-electricity-D.pdf}
    \caption{\textbf{Energy balance time series for electricity with and without imports.} Resampled to daily averages. Positive numbers indicate supply, negative numbers indicate consumption.}
    \label{fig:si:balance-elec}
\end{figure*}

\begin{figure*}
    \centering
    \footnotesize
    (a) without imports \\
    \includegraphics[width=\textwidth]{20231025-zecm/graphics/balance_timeseries/s_110_lvopt__Co2L0-2190SEG-T-H-B-I-S-A-onwind+p0.5_2050/ts-balance-heat-D.pdf} \\
    (b) with imports \\
    \includegraphics[width=\textwidth]{20231025-zecm/graphics/balance_timeseries/s_110_lvopt__Co2L0-2190SEG-T-H-B-I-S-A-onwind+p0.5-imp_2050/ts-balance-heat-D.pdf}
    \caption{\textbf{Energy balance time series for heat with and without imports.} Resampled to daily averages. Positive numbers indicate supply, negative numbers indicate consumption.}
    \label{fig:si:balance-heat}
\end{figure*}

\begin{figure*}
    \centering
    \footnotesize
    (a) without imports \\
    \includegraphics[width=\textwidth]{20231025-zecm/graphics/balance_timeseries/s_110_lvopt__Co2L0-2190SEG-T-H-B-I-S-A-onwind+p0.5_2050/ts-balance-hydrogen-D.pdf} \\
    (b) with imports \\
    \includegraphics[width=\textwidth]{20231025-zecm/graphics/balance_timeseries/s_110_lvopt__Co2L0-2190SEG-T-H-B-I-S-A-onwind+p0.5-imp_2050/ts-balance-hydrogen-D.pdf}
    \caption{\textbf{Energy balance time series for hydrogen with and without imports.} Resampled to daily averages. Positive numbers indicate supply, negative numbers indicate consumption.}
    \label{fig:si:balance-h2}
\end{figure*}

\begin{figure*}
    \centering
    \footnotesize
    (a) gas transmission network \\
    \includegraphics[width=.95\textwidth]{20231025-zecm/graphics/gas-network-unclustered.pdf} \\
    (b) electricity transmission network \\
    \includegraphics[width=.95\textwidth]{20231025-zecm/graphics/power-network-unclustered.pdf}
    \caption{\textbf{Gas and electricity transmission network data.} For gas
    transmission, the map shows pipelines sized and colored by rated capacity,
    fossil gas extraction sites, storage locations, pipeline entrypoints, and
    LNG terminals. The data comes from SciGRID\_gas and is supplemented with
    data from Global Energy Monitor. For power transmission, the map shows
    existing transmission lines at and above 220~kV taken from the ENTSO-E
    Transmission System Map (\url{https://www.entsoe.eu/data/map/}),
    supplemented with planned TYNDP projects (\url{https://tyndp.entsoe.eu/}).}
    \label{fig:si:networks-raw}
\end{figure*}
