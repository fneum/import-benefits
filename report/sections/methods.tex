

\subsection*{PyPSA-Eur: Overview of European energy system model}

For our analysis, we use the European sector-coupled high-resolution energy
system model PyPSA-Eur\cite{horschPyPSAEurOpen2018a} based on the open-source
modelling framework PyPSA\cite{brownPyPSAPython2018} (Python for Power System
Analysis), covering the energy demands of all sectors including electricity,
heat, transport, industry, agriculture, as well as non-energy feedstock demands,
international shipping and aviation. An overview of considered supply,
consumption and balancing technologies per carrier is shown in
\sfigref{fig:si:supply-consumption-options}.

The model simultaneously optimises spatially explicit investments and the 
operation of generation, storage, conversion and transmission assets to minimise
total system costs in a single linear optimisation problem, which assumes
perfect operational foresight and is solved with \textit{Gurobi 
v11}.\cite{gurobi} To manage computational complexity, no pathways with multiple
investment periods are calculated but overnights scenarios targeting net-zero
CO$_2$ emissions. The capacity expansion is based on technology cost and
efficiency projections for the year 2040 (see `Data availability'),
acknowledging that much of the required infrastructure must be constructed well
in advance of reaching net-zero emissions.

Existing hydro-electric power plants\cite{gotzensPerformingEnergy2019} are
included as well as nuclear power plants built before 1990 or currently under
construction from Global Energy Monitor's Global Nuclear Plant Tracker (52~GW
total of 106~GW in current
operation).\cite{globalenergymonitorGlobalNuclearPower2024} While
hydro-electricity is assumed to be non-extendable due to geographic constraints,
additional nuclear capacities can be expanded where cost-effective. We assume
that the existing nuclear fleet is operated inflexibly and apply
country-specific historical availability factors from the 2021-2023
period.\cite{internationalatomicenergyagencyPowerReactorInformation2024}

Temporally, the model is solved with an uninterrupted 4-hourly equivalent
resolution for a single year, using a segmentation clustering approach
implemented in the \textit{tsam} toolbox on all time-varying
data.\cite{hoffmannParetooptimalTemporal2022} While weather variations between
years are not considered for computational reasons, the chosen weather year 2013
is representative in terms of wind and solar
availability.\cite{gotskeDesigningSectorcoupledEuropean2024}

Spatially, the model resolves 115 regions in
Europe,\cite{frysztackiStrongEffect2021} covering the European Union, the United
Kingdom, Norway, Switzerland and the Balkan countries without Malta and Cyprus.
For computational reasons, only electricity, heat and hydrogen are modelled at
high spatial resolution, while oil, methanol, methane, ammonia and carbon
dioxide are treated as easily transportable without spatial constraints. Of the
total final energy and non-energy demand (\sfigref{fig:si:demand_totals}), only
some demands are spatially fixed (\sfigref{fig:si:demands}). These include
electricity for residential, industry, services, agriculture; heat; electric
vehicles; solid biomass for industry; and naphtha/methanol feedstocks. Hydrogen
demand for steel and ammonia production is also spatially fixed, unless these
industries can relocate, in which case regionally fixed hydrogen demand becomes
negligible. Since the model optimizes the siting and operation of most fuel
synthesis units, many demands are spatially variable (e.g.~electricity demand
for electrolysers or hydrogen demand for methanolisation).

A mathematical description of PyPSA-Eur can be found in
Section S12 in Neumann et al.\cite{neumannPotentialRoleHydrogen2023}

\subsection*{Gas and electricity network modelling}

Networks are considered for electricity, methane and hydrogen transport.
Existing gas pipelines taken from SciGRID\_gas,\cite{plutaSciGRIDGas2022a} can
be repurposed to hydrogen in addition to new hydrogen
pipelines.\cite{neumannPotentialRoleHydrogen2023} Data on the gas transmission
network is further supplemented by the locations of fossil gas extraction sites
and gas storage facilities based on SciGRID\_gas,\cite{plutaSciGRIDGas2022a} as
well as investment costs and capacities of LNG terminals in operation or under
construction from Global Energy Monitor's Europe Gas
Tracker.\cite{globalenergymonitorEuropeGasTracker2024} Geological potentials for
hydrogen storage are taken from Caglayan et
al.,\cite{caglayanTechnicalPotentialSalt2020} restricting where this low-cost
storage options is available. In modelling gas and hydrogen flows, we
incorporate electricity demands for compression of 1\% and 2\% per 1000km of the
transported energy, respectively.\cite{gasforclimateEuropeanHydrogen2021}
Existing high-voltage grid data is taken from
OpenStreetMap.\cite{xiongModellingHighVoltageGrid2024} For HVDC transmission
lines, we assume 2\% static losses at the substations and additional losses of
3\% per 1000km. The losses of high-voltage AC transmission lines are estimated
using the piecewise linear approximation from Neumann et
al.,\cite{neumannAssessmentsLinear2022} in addition to linearised power flow
equations.\cite{horschLinearOptimal2018} Up to a maximum capacity increase of
30\%, we consider dynamic line rating (DLR), leveraging the cooling effect of
wind and low ambient temperatures to exploit existing transmission assets
fully.\cite{glaumLeveragingExisting2023} To approximate $N-1$ resilience,
transmission lines may only be used up to 70\% of their rated dynamic
capacity.\cite{shokrigazafroudiTopologybasedApproximations12022} To prevent
excessive expansion of single connections, power transmission reinforcements
between two regions are limited to 15 GW, while an upper limit of 50.7 GW is
placed on hydrogen pipelines, which corresponds to three 48-inch
pipelines.\cite{gasforclimateEuropeanHydrogen2021}

\subsection*{Wind and solar potentials}

Renewable potentials and time series for wind and solar electricity generation
are calculated with \textit{atlite},\cite{hofmannAtliteLightweight2021}
considering land eligibility constraints like nature reserves, excluded land use
types, topography, bathymetry, and distance criteria to settlements. Given low
onshore wind expansion in many European countries in recent
years,\cite{ourworldindataInstalledWind2023} a deployment density of 1.5
MW~km$^{-2}$ is assumed for eligible land for onshore wind
expansion.\cite{turkovskaLanduseRequirementsSolar2023a} For reference, this
assumption leads to an onshore wind potential for Germany of 244~GW. The
temporal renewable generation potential for the available area is then assessed
based on reanalysis weather data, ERA5,\cite{ecmwf} and satellite observations
for solar irradiation, SARAH-3,\cite{pfeifrothSurfaceRadiationData2023} in
combination with standard solar panel and wind turbine models provided by
\textit{atlite}.

\subsection*{Biomass potentials}

Biomass potentials are restricted to residues from agriculture and forestry, as
well as waste and manure, based on the regional medium potentials specified for
2050 in the JRC-ENSPRESO database.\cite{ruizENSPRESOOpen2019} The finite biomass
resource can be employed for low-temperature heat provision in industrial
applications, biomass boilers and CHPs, and (electro-)biofuel production for use
in aviation, shipping and the chemicals industry. Additionally, we allow biogas
upgrading, including the capture of the CO$_2$ contained in biogas, which
unlocks all considered uses of regular methane
(\sfigref{fig:si:supply-consumption-options}). The total assumed bioenergy
potentials are 1730~TWh, which splits into 358~TWh/a for biogas and 1,014~TWh/a
for solid biomass. The total carbon content corresponds to
605~Mt$_{\text{CO}_2}$~a$^{-1}$, which is not fully available as a feedstock for
fuel synthesis or seuqestration for negative emissions due to imperfect capture
rates of up to 90\%.

\subsection*{Carbon management}

The carbon management features of the model trace the carbon cycles through
various conversion stages: industrial emissions, biomass and gas combustion,
carbon capture in numerous applications, direct air capture, intermediate
storage, electrofuels, recycling, landfill or  long-term sequestration. The
overall annual sequestration of CO$_2$ is limited to
200~Mt$_{\text{CO}_2}$~a$^{-1}$, similar to the 250~Mt$_{\text{CO}_2}$~a$^{-1}$
highlighted in the European Commission's carbon management
strategy.\cite{europeancommissionAmbitiousIndustrialCarbon2024} This number
allows for sequestering the industry's unabated fossil emissions (e.g.~in the
cement industry) while minimising reliance on carbon removal technologies. A
carbon dioxide network topology is not co-optimised since CO$_2$ is not
spatially resolved. This means that the location of biogenic or industrial point
sources of CO$_2$ is not a siting factor that this model version considers for
PtX processes, implicitly assuming that the CO$_2$ would be transported there at
low cost.
\cite{hofmannDesigningCO22023,hofmannH2CO2Network2024}

\subsection*{Transport sector fuel assumptions}

While the shipping sector is assumed to use methanol as fuel given its high
technology-readiness level compared to hydrogen or
ammonia,\cite{ieaETPCleanEnergy2024} land-based transport, including heavy-duty
vehicles, is fully electrified in the presented
scenarios.\cite{linkRapidlyDecliningCosts2024} Aviation can decide to use green
kerosene derived from Fischer-Tropsch fuels or methanol, owing to lower
technology readiness levels of fuel cell or battery-electric
aircrafts.\cite{ieaETPCleanEnergy2024} Alternative uses for methanol and
Fischer-Tropsch fuels extend beyond transport, including
power-to-methanol,\cite{brownUltralongdurationEnergyStorage2023} diesel for
agriculture machinery and as feedstock for high-value chemicals.

\subsection*{Technical constraints of synthetic fuel production}

To obtain more realistic operational patterns of green electrofuel synthesis
plants, we consider potential flexibility restrictions in the synthesis
processes. We apply a minimum part load of 20\% for methanolisation and 50\% for
methanation and Fischer-Tropsch
synthesis.\cite{mucciPowerXProcessesBased2023,wentrupDynamicOperationFischerTropsch2022,dieterichPowerliquidSynthesisMethanol2020,mbathaPowermethanolProcessReview2021}
These `green' options then compete with `blue' and `grey' options, such as steam
methane reforming of fossil gas with or without carbon capture for hydrogen
(\sfigref{fig:si:supply-consumption-options}). Some carriers also feature a
biogenic production route (e.g. methane and oil).

\subsection*{Heating sector modelling and PtX waste heat}


Heating supply technologies like heat pumps, electric boilers, gas boilers, and
combined heat and power (CHP) plants are endogenously optimised separately for
decentral use and central district heating. District heating shares of demand
are exogenously set to a maximum  of 60\% of the total urban heat demand with
sufficiently high population density. Besides the options for long-duration
thermal energy storage, district heating networks can further be supplemented
with waste heat from various power-to-X processes; electrolysis, methanation,
ammonia synthesis, Fischer-Tropsch fuel synthesis. Because the thermal discharge
from the methanol synthesis is primarily used for the distillation of the
methanol-water output mix,\cite{brownUltralongdurationEnergyStorage2023} its
waste heat potential is not considered for district heat. Here, we assume a
utilizable share of waste heat of 25\% considering that within the 115 regions
only a fraction of fuel synthesis plants might be connected to district heating
systems. In sensitivity analyses, we explore the effect of no or full waste heat
utilisation.


\subsection*{Industry relocation modelling for steel and ammonia production}

In some scenarios, we also allow the model to endogenously relocate the steel
and ammonia industry within Europe. This is to allow the best sites within
Europe to compete with outsourced production abroad. While this captures some of
the most energy-intensive industry sectors, other sectors like concrete and
alumina production is not considered for relocation.

Without relocation of steel and ammonia production allowed, the production volumes of primary steel, by
direct iron reduction (DRI) and electric arc furnace (EAF), and ammonia for
fertilizers, by Haber-Bosch synthesis, are spatially-fixed. This results in
exogenous hydrogen demand per region. Total production volumes are based on
current
levels.\cite{unitedstatesgeologicalsurveyAmmoniaProductionCountry2022,europeancommission.jointresearchcentre.JRCIDEES2021IntegratedDatabase2024}
For the spatial distribution, we use data on the existing integrated steelworks
listed in Global Energy Monitor's Global Steel Plant Tracker
\cite{globalenergymonitorGlobalSteelPlant2024} and manually collected data on
the location and size of ammonia plants in Europe.

With relocation of steel and ammonia production allowed, the model endogenously chooses the regional
production volumes of primary steel, HBI and ammonia, subject to the
availability of cheap hydrogen. Thereby, the regional capacities and operation
of Haber-Bosch, DRI and EAF plants are co-optimised with the rest of the system,
similar to the siting of Fischer-Tropsch or methanolisation plants. For DRI and
EAF, investment costs and specific requirements for fuels and iron ore are taken
from the Steel Sector Transition Strategy Model (ST-STSM) of the Mission
Possible
Partnership.\cite{missionpossiblepartnershipSteelSectorTransition2022,missionpossiblepartnershipMakingNetZeroSteel2022}.
and assume steel can be stored and transported without constraints within
Europe. Relocation costs and local job impacts are excluded from the analysis
due to lack of data.

For both cases, we assume a rise in the steel recycling rate from 40\% today to
70\% in our carbon-neutral
scenarios.\cite{materialeconomicsIndustrialTransformation20502019} We assume that
the electric arc furnaces for secondary steel remain, in proportion, at current
locations and do not relocate.

\subsection*{TRACE: Import supply chain modelling}

The European energy system model is extended with data from the TRACE model used
in Hampp et al.\cite{hamppImportOptions2023} to assess the unit costs of
different vectors for importing green energy and material to entry-points in
Europe from various world regions. For consistency with the European model, the
techno-economic assumptions were aligned, using the same projections for the
year 2040 (see `Data availability`) and a uniform weighted average cost of
capital (WACC) of 7\%.\cite{lonerganImprovingRepresentationCost2023} As possible
import vectors, we consider electricity by transmission line, hydrogen as gas by
pipeline and liquid by ship, methane as liquid by ship, liquid ammonia, steel
and HBI, methanol and Fischer-Tropsch fuels by ship. Liquid organic hydrogen
carriers (LOHC) are not considered as export vector due to their lower
technology readiness level (TRL) compared to other
vectors.\cite{irenaGlobalHydrogenTrade2022}

Our selection of 53 potential exporting regions broadly comprises countries with
favourable wind and solar resources and large enough potentials for substantial
exports above 500~TWh~a$^{-1}$ in addition to domestic consumption. We exclude
some countries due to political instability (e.g.~Sudan, Somalia, Yemen), using
a Fragile States Index\cite{thefundforpeaceffpFragileStatesIndex2023} value of 100 as a threshold, or due
to severe imposed sanctions (e.g.~Russia, Iran, Iraq), following the EU
Sanctions Map.\cite{estonianpresidencyofthecounciloftheeuEUSanctionsMap2024}
Landlocked countries outside the reach of realistic pipeline transport are also
excluded due to lack of access to ports. For landlocked regions within pipeline
reach, we only exclude ship-bourne vectors. Some large countries are split into
multiple subregions for a more differentiated view (e.g.~USA, Argentina, Brazil
and China). The resulting regions are marked in \cref{fig:options:global}.

% renewable potentials minus domestic demand

To determine the levelised cost of energy for exports, the methodology first
assesses the regional potentials for solar, onshore and offshore wind energy.
These potentials and time series are calculated using
\textit{atlite}\cite{hofmannAtliteLightweight2021}, applying similar land
eligibility constraints as in PyPSA-Eur, albeit using other datasets with global
coverage, and the same wind turbine and solar panel models to ERA5\cite{ecmwf}
weather data for 2013 in eligible regions. Since TRACE evaluates whole regions
without further network resolution, the renewable potentials and profiles are
split into different resources classes to reduce smoothing effects. We consider
30 classes each for onshore wind and solar, and 10 classes for offshore wind
where applicable. Based on these calculations, levelised cost of electricity
(LCOE) curves can be determined for each region. A selection of LCOE curves is
shown in \sfigref{fig:si:cost-supply-curves}.

% deducted domestic demand

In the next step, potentials are reduced by the projected future local energy
demand, starting with the lowest LCOE resource classes. With this approach,
domestic consumption is prioritised and supplied by the regions' best renewable
resources even though we do not model the energy transition in exporting regions
in detail. To create the demand projections, we use the
GEGIS\cite{mattssonAutopilotEnergyModels2021} tool, which utilises machine
learning on historic time-series, weather data and marco-economic factors to
create artificial electricity demand time-series based on population and GDP
growth scenarios following the SSP2 scenario of the Shared Socioeconomic
Pathways.\cite{riahiSharedSocioeconomicPathways2017} From these time-series we
take the annual total and increase it by a factor of two to account for further
electrification of other sectors, which the GEGIS tool does not consider.

% conversion pathways and scale

The remaining wind and solar electricity supply can then be used to produce the
specific energy or material vector according to the flow chart of conversion
pathways shown in \sfigref{fig:si:import-esc-scheme}. Considered technologies
include water electrolysis for \ce{H2}, direct air capture (DAC) for \ce{CO2},
synthesis of methane, methanol, ammonia or Fischer-Tropsch fuels from \ce{H2}
with \ce{CO2} or \ce{N2}, and \ce{H2} direct iron reduction (DRI) for sponge
iron with subsequent processing to green steel in electric arc furnaces (EAF)
from iron ore priced at
97.7~\euro{}~t$^{-1}$.\cite{missionpossiblepartnershipSteelSectorTransition2022}
Other CO$_2$ sources than DAC are not considered in the exporting regions.
Furthermore, while batteries and hydrogen storage in steel tanks is considered,
underground hydrogen storage is excluded due to uncertain availability of salt
caverns in many of the potential exporting
regions.\cite{hevinUndergroundStorageHydrogen2019,hydrogentcp-task42UndergroundHydrogenStorage2023}
We also assume that the energy supply chains dedicated to exports will be
islanded from the rest of the local energy system, i.e.~that curtailed
electricity or waste heat could not be used locally.

For each vector, an annual reference export demand of 500~TWh$_\text{LHV}$ or
100~Mt of steel and HBI is assumed, mirroring large-scale energy and material
infrastructures and export volumes, corresponding to approximately 40\% of
current LNG
imports\cite{instituteforenergyeconomicsandfinancialanalysisEuropeanLNG2023} and
66\% of European steel
production.\cite{eurofer-theeuropeansteelassociationEuropeanSteel2023} Transport
distances are calculated between the exporting regions and the twelve
representative European import locations using the \textit{searoute} Python tool
for ship-bourne vectors or crow-fly distances for pipeline or HVDC connections.
The representative import locations are large ports and LNG terminals in the
United Kingdom, the Netherlands, Poland, Greece, Italy, Spain and Portugal and
pipeline entry-points in Slovakia, Greece, Italy and Spain. All energy supply
chains are assumed to consume their energy vector as fuel for transport to
Europe, except for steel, which uses externally bought methanol as shipping
fuel.

%%% capacity expansion %%%

For each combination of carrier, exporter and importer, a linear capacity
expansion optimisation is performed to determine cost-optimal investments and
the operation of generation, conversion, storage and transport capacities for
all intermediary products to deliver 500~TWh~a$^{-1}$ (or 100~Mt~a$^{-1}$ for
materials) of the final carrier to Europe. Dividing the total annual system
costs by the targeted annual export volume yields the levelised cost of energy
as seen by the European entry point. To match the multi-hourly resolution used
for the European model, the TRACE model was configured to use a 3-hourly
resolution for the year 2013, resulting in similar balancing requirements. The
resulting levelised cost curves of energy imports for different import vectors
and exporting regions are presented for the respective lowest-cost entry point
to Europe in
\sfigref{fig:si:isc-h2,fig:si:isc-ch4-nh3,fig:si:isc-meoh-ft,fig:si:isc-hbi-St}
with assumed step width of 500~TWh~a$^{-1}$ (or 100~Mt~a$^{-1}$ for materials).
The curves show the varying cost composition of the country-carrier pairs.

A mathematical description of TRACE can be found in Section
S3 in Hampp et al.\cite{hamppImportOptions2023}

% - electrolysis costs \cite{ieaGlobalHydrogenReview2024}

\subsection*{Coupling of import options to European model}

The resulting levelised unit cost for each combination of carrier, exporter and
reference importer is then used as an exogenous input to the European model. For
each candidate entry point, we match the closest reference import location from
TRACE and add the corresponding import cost curve as supply option
(\sfigref{fig:si:isc-h2,fig:si:isc-ch4-nh3,fig:si:isc-meoh-ft,fig:si:isc-hbi-St}).
Moreover, we limit energy exports from any one exporting region to Europe for
the sum of all carriers to 500~TWh~a$^{-1}$. This is to both prevent a single
country from dominating the import mix and be consistent with the target export
volume assumed in TRACE. Beyond that, the decision about the origin,
destination, vector, volume, and timing of imports is largely endogenous to
PyPSA-Eur.

%%% entry points %%%

However, imports may be further restricted by the level of domestic import
infrastructure expansion. For each vector, we identify locations where the
respective carrier may enter the European energy system by considering where LNG
terminals and cross-continental pipelines are located
(\cref{fig:options:europe}). For hydrogen imports by pipeline, imports have to
be near-constant, varying between 90-100\% of peak imports. For methane imports
by ship, existing LNG terminals reported in Global Energy Monitor's Europe Gas
Tracker\cite{globalenergymonitorEuropeGasTracker2024} can be used. For hydrogen
by ship, new terminals can be built in regions where LNG terminals exist. To
ensure regional diversity in potential gas and hydrogen imports and avoid
vulnerable singular import locations, we allow the expansion beyond the reported
capacities only up to a factor of 2.5, taking the median value of reported
investment costs for LNG terminals.\cite{GlobalGas2022} A premium of 20\% is
added for hydrogen import terminals due to the lack of practical experience with
them. For electricity, the capacity and operational patterns of the HVDC links
can be endogenously optimised. Imports for carbonaceous fuels, ammonia, HBI, and
steel are not spatially allocated to specific ports, given their low transport
costs relative to value. Port capacities are assumed unconstrained since these
commodities, particularly carbonaceous fuels, are comparable to the large fossil
oil volumes currently handled at European ports.

%%% reconversion %%%

Further conversion of imported fuels is also possible once they have arrived in
Europe, e.g.~hydrogen could be used to synthesise carbon-based fuels, ammonia
could be cracked to hydrogen, methane could be reformed to hydrogen, and methane
or methanol could be combusted for power generation. However, conversion losses
can make it less attractive economically to use a high-value hydrogen derivative
merely as a transport and storage vessel only to reconvert it back to hydrogen
or electricity.

%%% special handling of electricity imports %%%

The supply chain of electricity imports is endogenously optimised with the rest
of the European system rather than using a constant levelised cost of
electricity for each export region. This is because, owing to the greater
challenge to store electricity, the hourly variability of wind and solar
electricity leads to higher price variability compared to hydrogen and its
derivatives and the intake needs to be more closely coordinated with the
European power grid. The endogenous optimisation comprises wind and solar
capacities, batteries and hydrogen storage in steel tanks, and the size and
operation of HVDC link connections to Europe based on the renewable capacity
factor time series as illustrated in \cref{fig:options:europe}. Europe's
connection options with exporting regions are confined to the 4\% nearest
regions, with additional ultra-long distance connection options to Ireland,
Cornwall and Brittany following the vision of the Xlinks project between Morocco
and the United Kingdom.\cite{xlinksMoroccoUKPowerProject2023} Connections through
Russia or Belarus are excluded, and in addition to excluded entry points, some
connections from Central Asia are affected by additional detours beyond the
regularly applied detour factor of 125\% of the as-the-crow-flies distance.
Similar to intra-European HVDC transmission, a 3\% loss per 1000km and a 2\%
converter station loss are assumed.

\vspace{1em}

Finally, we note that all mass-energy conversion is based on the lower heating
value (LHV). To present energy and material imports in a common unit, the
embodied energy in steel is approximated with the 2.1 kWh~kg$^{-1}$ released in
iron oxide reduction, i.e.~energy released by
combustion.\cite{kuhnIronRecyclable2022} All currency values are given in
\euro{}$_{2020}$.
