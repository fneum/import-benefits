

\subsection*{Modelling of the European energy system}

For our analysis, we use the European sector-coupled high-resolution energy
system model PyPSA-Eur\cite{horschPyPSAEurOpen2018a} based on the open-source
modelling framework PyPSA\cite{brownPyPSAPython2018} (Python for Power System
Analysis) in a setup similar to Neumann et al.~\cite{neumannPotentialRole2023},
covering the energy demands of all sectors including electricity, heat,
transport, industry, agriculture, as well as international shipping and
aviation.

The model simultaneously optimises spatially explicit investments and the
operation of generation, storage, conversion and transmission assets to minimise
total system costs in a linear optimisation problem, which is solved with
\textit{Gurobi}.\cite{gurobi} The capacity expansion is based on technology cost
and efficiency projections for the year 2030, many of which are taken from the
technology catalogue of the Danish Energy Agency.\cite{DEA} Choosing projections
for the year 2030 for a net-zero carbon emission scenarios more likely to be
reached by mid-century acknowledges that much of the required infrastructure
must be constructed well in advance of reaching this goal. Spatially, the model
resolves 110 regions in Europe,\cite{frysztackiStrongEffect2021} covering the
European Union, the United Kingdom, Norway, Switzerland and the Balkan countries
without Malta and Cyprus. Temporally, the model is solved with an uninterrupted
4-hourly equivalent resolution for the weather year 2013, using a segmentation
clustering approach implemented in the \textit{tsam}
toolbox.\cite{hoffmannParetooptimalTemporal2022} In terms of investment periods,
no pathway optimisation is conducted, but a greenfield approach is pursued
except for existing hydro-electricity and transmission infrastructure targeting
net-zero CO$_2$ emissions.

Networks are considered for electricity, methane and hydrogen
transport.\cite{ENTSOE,plutaSciGRIDGas2022a} However, other than in Neumann et
al.,\cite{neumannPotentialRole2023} pipeline retrofitting to hydrogen is
disabled for computational reasons. Data on the gas transmission system is
further supplemented by the locations of fossil gas extraction sites and gas
storage facilities based on SciGRID\_gas,\cite{plutaSciGRIDGas2022a} as well as
investment costs and capacities of existing and planned LNG
terminals\cite{instituteforenergyeconomicsandfinancialanalysisEuropeanLNG2023}
Moreover, a carbon dioxide network is not explicitly co-optimised since as CO$_2$ is
not spatially resolved in this model version.\cite{hofmannDesigningCO22023}

The overall annual sequestration of CO$_2$ is limited to 200
Mt$_{\text{CO}_2}$/a. This number allows for sequestering the industry's
unabated fossil emissions (e.g. in the cement industry) while minimising
reliance on carbon removal technologies. The carbon management features of the
model trace the carbon cycles through various conversion stages: industrial
emissions, biomass and gas combustion, carbon capture, storage or long-term
sequestration, direct air capture, electrofuels, recycling, and waste-to-energy
plants.

Renewable potentials and time series for wind and solar electricity generation
are calculated with \textit{atlite},\cite{hofmannAtliteLightweight2021}
considering land eligibility constraints like nature reserves or distance
criteria to settlements. Given low onshore wind expansion in many European
countries in recent years,\cite{ourworldindataInstalledWind2023} restrictive
onshore wind expansion potentials are applied, using a 1.5 MW/km$^2$ factor for
the eligible land area. Geological potentials for hydrogen storage are taken
from Caglayan et al.\cite{caglayanTechnicalPotential2020} Biomass potentials are
restricted to residues from agriculture and forestry, as well as waste and
manure, based on the medium potentials specified for 2030 in the JRC-ENSPRESO
database.\cite{ruizENSPRESOOpen2019} The finite biomass resource can be employed
for low-temperature heat provision in industrial applications, biomass boilers
and CHPs, and biofuel production for use in aviation, shipping and the chemicals
industry. Additionally, we allow biogas upgrading, including the capture of the
CO$_2$ contained in biogas. The total assumed bioenergy potentials are 1569~TWh
with a carbon content corresponding to 546~Mt$_{\text{CO}_2}$/a.

Heating supply technologies like heat pumps, electric boilers, gas boilers, and
combined heat and power (CHP) plants are endogenously optimised separately for
decentral use and central district heating. District heating networks can
further be supplemented with waste heat from various power-to-X processes
(electrolysis, methanation, methanolisation, ammonia synthesis, Fischer-Tropsch
fuel synthesis).

While the shipping sector is assumed to use methanol as fuel, land-based
transport, including heavy-duty vehicles, is deemed fully electrified in the
presented scenarios. Aviation can decide to use green kerosene derived from
Fischer-Tropsch fuels or methanol. Besides methanol-to-kerosene, further
usage options for methanol have been added.
These include
methanol-to-olefins/aromatics for the production of green plastics,
methanol-to-power\cite{brownUltralongdurationEnergyStorage2023} in open-cycle gas
turbines or Allam cycle turbines, and steam reforming of methanol with or
without carbon capture. For the synthesis of electrofuels, we also account for
potential operational restrictions by considering a minimum part load of 30\%
for methanolisation and methanation compared to 70\% for Fischer-Tropsch
synthesis, both within Europe and abroad.

A further core improvement of the model regards the physical representation of
energy transport over long distances. For gas and hydrogen pipelines, we
incorporate electricity demands for compression of 1\% and 2\% per 1000km
of the transported energy, respectively.\cite{gasforclimateEuropeanHydrogen2021}
For HVDC transmission lines, we assume 2\% static losses at the substations and
additional losses of 3\% per 1000km. The losses of high-voltage AC transmission
lines are estimated using a piecewise linear approximation as proposed in
Neumann et al.,\cite{neumannAssessmentsLinear2022} in addition to the linearised
power flow equations.\cite{horschLinearOptimal2018} Up to a maximum capacity
increase of 30\%, we consider dynamic line rating (DLR), leveraging the cooling
effect of wind and low ambient temperatures to exploit existing transmission
assets fully.\cite{glaumLeveragingExisting2023} To approximate N-1 resilience,
transmission lines may only be used up to 70\% of their rated dynamic capacity.
To prevent excessive expansion of single connections, the expansion of power
transmission lines between two regions is limited to 15 GW for HVAC and 25 GW
for HVDC lines, while a similar constraint of 50.7 GW is placed on hydrogen
pipelines, which corresponds to three parallel 48-inch
pipelines.\cite{gasforclimateEuropeanHydrogen2021}

Finally, we also developed the possibility for the model to relocate the steel
and ammonia industry within Europe, mainly to level the playing field between
non-European green steel imports and domestic production. This is achieved by
explicitly modelling the cost, efficiency and operation of hydrogen direct iron
reduction (H2-DRI) and electric arc furnaces (EAF), which can be sited all over
Europe, and the cost to procure iron ore. We further allow the oversizing of
steelmaking plants to allow flexible production in response to the renewables
supply conditions.

\subsection*{Modelling of import supply chains and costs}

\begin{figure*}
    \centering
    \includegraphics[width=.82\textwidth]{static/graphics/sketch2.drawio-2.pdf}
    \caption{\textbf{Schematic overview of the import supply chains.} The
    illustration includes key input-output ratios of the different conversion
    processes and the transport efficiencies for the different import vectors.}
    \label{fig:import-esc-scheme}
\end{figure*}

The European energy system model is extended with data from the TRACE model
\cite{hamppImportOptions2023} to assess the costs of different vectors for
importing green energy and material into Europe from various world regions. For
each vector, we identify locations with existing or planned import
infrastructure where the respective carrier may enter the European energy
system.

Starting from the methodology by Hampp et al.\cite{hamppImportOptions2023}, some
adjustments were made to the original TRACE model. Namely, land availability and
wind and solar time-series are determined using
\textit{atlite}\cite{hofmannAtliteLightweight2021} instead of
\textit{GEGIS},\cite{mattssonAutopilotEnergy2021}. Techno-economic assumptions
were aligned with those used in the European model, and steel was included as an
energy-intensive material import vector. The exporting countries comprise
Australia, Argentina, Chile, Kazakhstan, Namibia, Turkey, Ukraine, the Eastern
United States and Canada, mainland China, and counties in the MENA region.

To determine the levelised cost of energy for exports, the methodology first
assesses the regional potentials for wind and solar energy. A regional
electricity cost supply curve is determined, from which projected future energy
demand is subtracted. Thereby, domestic consumption is prioritised and supplied
by the countries' best renewable resources even though we do not model the
energy transition in exporting countries in detail. The remaining wind and solar
electricity supply can then be used to produce the specific energy or material
vector using water electrolysis for \ce{H2}, direct air capture (DAC) for
\ce{CO2}, air separation units (ASU) for \ce{N2}, synthesis of methane,
methanol, ammonia or Fischer-Tropsch fuels from \ce{H2} with \ce{CO2} or
\ce{N2}, and \ce{H2} direct iron reduction (DRI) with subsequent processing in
electric arc furnaces (EAF) for the processing of iron ore (103.7 \euro{}/t)
into green steel. Other CO$_2$ sources than DAC are not considered in the
exporting countries, a notable difference from the European model. Liquid
organic hydrogen carriers (LOHC) are not considered as export vector due to
their lower technology readiness level (TRL) compared to other
vectors.\cite{irenaGlobalHydrogen2022} Further details on the energy and
feedstock flow and process efficiencies are outlined in
\cref{fig:import-esc-scheme}.

For each vector, an annual reference import demand of 500~TWh (lower heating
value, LHV) or 100~Mt of steel is assumed, mirroring large-scale energy and
material infrastructures and export volumes, corresponding to approximately 40\%
of current LNG
imports\cite{instituteforenergyeconomicsandfinancialanalysisEuropeanLNG2023} and
66\% of European steel
production.\cite{eurofer-theeuropeansteelassociationEuropeanSteel2023}

%%% capacity expansion %%%

Based on these supply chain definitions, a capacity expansion optimisation is
performed to determine the cost-optimal combination of infrastructure and
process capacities for all intermediary products and delivering the final
carrier either through pipelines (\ce{H2(g)}, \ce{CH4(g)}) or by ship
(\ce{H2(l)}, \ce{CH4(l)}, \ce{NH3(l)}, \ce{MeOH}, Fischer-Tropsch fuel,
and steel). Exports from each of the regions shown in \cref{fig:options:global}
are modelled to each of twelve European import locations based on large port
locations, determining the levelised costs of energy or steel the European entry
point will see for each supply chain. All energy supply chains are assumed to
consume their energy vector as fuel for transport to Europe, except for steel,
which uses externally bought methanol as shipping fuel.

%%% import constraints in European model %%%

The resulting levelised cost of exported energy specific to the respective
importing regions is added as a constant marginal import cost for all
chemical energy carriers and steel. For the import of hydrogen and methane,
candidate entry points are identified based on where existing and prospective
LNG terminals and cross-continental pipelines are located. This includes new LNG
import terminals in Europe in response to ambitions to phase out Russian gas
supply in 2022.
\cite{instituteforenergyeconomicsandfinancialanalysisEuropeanLNG2023} To achieve
regional diversity in potential gas and hydrogen imports and avoid vulnerable
singular import locations, we allow the expansion beyond the reported capacities
only up to a factor of 2.5, taking the median value of reported investment costs
for LNG terminals.\cite{GlobalGas2022} A surcharge of 20\% is added for hydrogen
import terminals due to the lack of practical experience. Carbonaceous fuels,
ammonia, and steel imports are not spatially resolved due to their low transport
costs and, therefore, are not constrained by the availability of suitable entry
points. To present energy and material imports in a common unit, the embodied
energy in steel is approximated with the 2.1 kWh/kg released in iron oxide
reduction, i.e.~energy released by combustion.\cite{kuhnIronRecyclable2022}

%%% special handling of electricity imports %%%

Owing to the variability of wind and solar electricity, the supply chain of
electricity imports is endogenously optimised with the rest of the European
system rather than using a constant levelised cost of exported electricity. This
comprises the optimisation of wind and solar capacities, batteries and hydrogen
storage in steel tanks, and the size and operation of HVDC link connection into
Europe based on the availability time series in neighbouring countries as
illustrated in \cref{fig:options:europe}. Underground hydrogen storage options
are not considered due to the limited availability of salt caverns in many of
the best renewable resource regions in the countries that are considered
exporting.\cite{hevinUndergroundStorage2019} We also assume that the energy
supply chains dedicated to exports will be islanded from the rest of the local
energy system. Europe's connection options with exporting countries are confined
to the 5\% nearest regions, with additional ultra-long distance connection
options to Ireland, Cornwall and Brittany following the vision of the Xlinks
project between Morocco and the United Kingdom.\cite{xlinksMoroccoUKPower2023}
Connections through Russia or Belarus are excluded, and thus, some connections
are affected by additional detours beyond the regularly applied detour factor of
125\% of the as-the-crow-flies distance. Similar to intra-European HVDC
transmission, a 3\% loss per 1000km and a 2\% converter station loss are assumed.

%%% more detailed results %%%

As illustrated in \cref{fig:options}, for imports of hydrogen by pipeline,
nearby countries like Algeria and Egypt emerged as lowest cost exporters (ca.~57
\euro{}/MWh). Importing hydrogen by ship is substantially more expensive due to
liquefaction and evaporation losses. Algeria could offer supply through this
vector at 84 \euro{}/MWh. For all other hydrogen derivatives, Argentina and
Chile offer the lowest cost imports between 88 and 110~\euro{}/MWh or
501~\euro{}/t for steel. Methanol is found to be cheaper than the
Fischer-Tropsch route because it is assumed to be more flexible (30\% minimum
part load compared to 70\% for
Fischer-Tropsch).\cite{brownUltralongdurationEnergy2023} The lower process
flexibility shifts the energy mix towards solar electricity and causes higher
levels of curtailment, increasing costs. The transport costs of \ce{CH4(l)} are
lower than for \ce{H2(l)} since the liquefaction consumes less energy and
individual ships can carry more energy with \ce{CH4(l)}. Pipeline imports of
\ce{CH4(g)} were also considered, but costs were higher than for \ce{CH4(l)}
shipping under the assumption that new pipelines would have to be built.
Consequently, the model preferred LNG imports over pipeline imports.
