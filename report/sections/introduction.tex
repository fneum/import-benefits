% promise of imports

The transformation of the European energy system towards climate neutrality
demands unrivalled technological change. Whereas the development of renewable
energy sources in Europe and supporting measures like reinforcing the
electricity grid do not always meet the level of acceptance required for a swift
transition, other parts of the world have cheap and abundant renewable energy
supply potentials to offer to global energy
markets.\cite{irenaGlobalHydrogen2022,luxSupplyCurves2021,vanderzwaanTimmermansDream2021,fasihiLongTermHydrocarbon2017,reichenbergDeepDecarbonization2022,galvanExportingSunshine2022,armijoFlexibleProduction2020,pfennigGlobalGISbased2022}
These regions could become key partners for a cost-effective and socially
accepted energy transition in Europe, especially when the production of large
quantities of domestic green fuels and materials falters.

% repulsion of imports

However, even if energy imports are economically attractive, a strong dependence
on them may not be preferred due to energy security concerns, as European
countries recently experienced owing to their reliance on Russian natural
gas.\cite{pedersenLongtermImplications2022} In 2021, the EU27 nations sourced
around two-thirds of their energy needs through
imports,\cite{eurostatCompleteEnergy2023}, and accordingly, much of the European
energy infrastructure is built around the imports of fossil oil and gas.
Continued energy imports and associated infrastructure might tie European energy
supply to a small number of exporters or markets with market power. These risks
must be weighed against the potential benefits of decreasing energy supply
costs, reducing land usage in Europe and increasing energy security by supplying
storable fuels that can mitigate energy droughts for systems with high shares of
weather-dependent energy supply.

% feasibility of no imports

The transition to a system that exploits the best wind and solar sites across
the continent would offer many ways to develop a self-sufficient system without
imports.\cite{pickeringDiversityOptions2022,trondleHomemadeImported2019,brownSynergiesSector2018}
For instance, reinforcing the power grid or building a hydrogen network that may
repurpose parts of the gas network was consistently identified as beneficial in
such autarkic
scenarios.\cite{neumannPotentialRole2023,victoriaSpeedTechnological2022}
However, what energy infrastructure is needed might strongly depend on the
levels of clean energy imported. For instance, scaling up Europe's energy
transmission infrastructure may not be necessary. Since most hydrogen would be
used to produce synthetic fuels (e.g.~, high-value chemicals, aviation and
shipping) and steel,\cite{neumannPotentialRole2023} if these derivatives were
imported at scale, much of the hydrogen demand would fall away, hence, reducing
the need for hydrogen transport. Even if there were a high demand for direct
hydrogen imports, the optimised topology would differ if it needed to absorb
inbound hydrogen from North Africa rather than domestic production.

% policy strategies

% TODO more recent news / strategies?

In particular, hydrogen imports have recently attracted considerable interest,
with plans of the European Commission under
\mbox{REPowerEU}\cite{europeancommissionRepowerEUPlan} to import 10~Mt (333~TWh)
hydrogen and derivatives by 2030 and reflections of this vision in the German
national hydrogen
strategy.\cite{bundesministeriumfuerwirtschaftundklimaschutzFortschreibungNationalen2023}

% literature review of studies focusing on the cost of energy imports
% literature review focusing on the European energy system with/without considering imports

% TODO stronger novelty statement?

While many previous academic studies have evaluated the cost of green energy and material imports in the form of
electricity,\cite{lilliestamEnergySecurity2011,triebSolarElectricity2012,lilliestamVulnerabilityTerrorist2014,bogdanovNorthEastAsian2016,benaslaTransitionSustainable2019,reichenbergDeepDecarbonization2022}% (some with reference to the DESERTEC idea),
hydrogen,\cite{timmerbergHydrogenRenewables2019,ishimotoLargescaleProduction2020,brandleEstimatingLongterm2021,luxSupplyCurves2021,galvanExportingSunshine2022,collisDeterminingProduction2022,galimovaImpactInternational2023}
ammonia,\cite{nayak-lukeTechnoeconomicViability2020,armijoFlexibleProduction2020,galimovaFeasibilityGreen2023}
methane,\cite{luxSupplyCurves2021,agoraenergiewendeHydrogenImport2022}
steel,\cite{trollipHowGreen2022a,devlinRegionalSupply2022,lopezDefossilisedSteel2023}
carbon-based
fuels,\cite{fasihiLongTermHydrocarbon2017,sherwinElectrofuelSynthesis2021} or a
broader variety of power-to-X
fuels,\cite{vanderzwaanTimmermansDream2021,pfennigGlobalGISbased2022,irenaGlobalHydrogen2022,solerEFuelsTechno2022,hamppImportOptions2023,gengeSupplyCosts2023,galimovaGlobalTrading2023a}
these do not address interactions of imports with European energy infrastructure
requirements. On the other hand, among studies dealing with the detailed
planning of net-zero energy systems in Europe, some do not consider energy
imports,\cite{pickeringDiversityOptions2022,brownSynergiesSector2018,victoriaSpeedTechnological2022}
while others only consider hydrogen imports or a limited set of alternative
import
vectors.\cite{gilsInteractionHydrogen2021,seckHydrogenDecarbonization2022,wetzelGreenEnergy2023,kountourisUnifiedEuropean2023,neumannPotentialRole2023}
Only few consider at least elementary cost
uncertainties,\cite{frischmuthHydrogenSourcing2022} and none investigate a
larger range of potential import volumes across subsets of available import
vectors.


% main paper idea - scenarios

In this contribution, we explore the full range between the two poles of
complete self-sufficiency and wide-ranging energy imports into Europe in
scenarios with high shares of wind and solar electricity and net-zero carbon
emissions. We investigate how the infrastructure requirements of a
self-sufficient European energy system that exclusively leverages domestic
resources from the continent may differ from a system that relies on energy
imports from outside of Europe. For our analysis, we integrate an open model of
global energy supply chains, TRACE,\cite{hamppImportOptions2023} with a
spatially and temporally resolved sector-coupled open-source energy system
model, PyPSA-Eur,\cite{PyPSAEurSecSectorCoupled} to investigate the impact of
imports on European energy infrastructure needs. We evaluate potential import
locations and costs for different supply vectors, the economic impetus for such
imports, and how their inclusion affects deployed transport networks and
storage. For this purpose, we perform sensitivity analyses interpolating between
very high levels of imports and no imports at all, low and high costs for
imports to account for associated uncertainties, and system responses to the
exclusion of subsets of import vectors, all to probe the flatness of the
solution space.


\begin{figure*} 
    \begin{subfigure}[t]{\textwidth}
        \caption{global perspective}
        \label{fig:options:global}
        \includegraphics[width=\textwidth]{20231025-zecm/graphics/import_world_map.pdf}
    \end{subfigure}
    \begin{subfigure}[t]{\textwidth}
        \caption{European perspective}
        \label{fig:options:europe}
        \centering
        \includegraphics[width=\textwidth]{20231025-zecm/graphics/import_options_s_110_lvopt__Co2L0-2190SEG-T-H-B-I-S-A-onwind+p0.5-imp_2050.pdf}
    \end{subfigure}
    \caption{\textbf{Overview of considered import options.}
        \textit{Panel (a)} shows the regional differences in the cost to deliver
        green methanol to Europe (choropleth layer), the cost composition of
        different import vectors (bar charts), an illustration of the wind and
        solar availability in Morocco, and an illustration of the land
        eligibility analysis for wind turbine development in the region of
        Buenos Aires in Argentina. \textit{Panel (b)} depicts potential entry
        points for energy imports into Europe like the location of existing and
        planned LNG terminals and gas pipeline entry points, the costs of
        hydrogen imports in different European regions (choropleth layer), the
        considered connections for long-distance HVDC import links from the MENA
        region, Kazakhstan, Turkey and Ukraine, and the distribution and range
        of import costs for different energy carriers and entry points with
        indications for selected countries of origin (violin charts). }
    \label{fig:options}
\end{figure*}

% TODO give sources for carrier statements

% discussion of different import vectors

As possible import options, we consider electricity by transmission line,
hydrogen as gas by pipeline and liquid by ship, methane as gas by pipeline and
liquid by ship, ammonia as liquid by ship, and steel, methanol and
Fischer-Tropsch fuels by ship. Each vector not only varies in European demand
levels but also presents unique characteristics. Electricity offers the most
flexible usage but is challenging to store and requires variability management
if sourced from wind or solar energy. Hydrogen is easier to store and transport
in large quantities but at the expense of conversion losses and less versatile
applications. Synthetic carbonaceous fuels like methane, methanol and
Fischer-Tropsch fuels are easy to store and transport and could benefit from
existing infrastructure. However, these require a sustainable carbon source and,
particularly for methane, effective leakage prevention. Ammonia is similarly
straightforward to handle and, while not needing a carbon source, faces
acceptance issues due to its toxicity. Steel represents the import of
energy-intensive materials and offers low transport costs compared to its other
cost factors. Further conversion of imported fuels is also possible once they
have arrived in Europe, e.g.~hydrogen could be used to synthesise carbon-based
fuels, and methane could be converted to hydrogen. However, conversion losses
make it less likely to use a high-value hydrogen derivative merely as a
transport vessel only to reconvert it back to hydrogen or electricity.

% brief methodology of TRACE and PyPSA-Eur

The PyPSA-Eur\cite{PyPSAEurSecSectorCoupled} model co-optimises the investment
and operation of generation, storage, conversion and transmission
infrastructures, as well as the relocation of some
industries,\cite{verpoortEstimatingRenewables2023,samadiRenewablesPull2023} in a
single linear optimisation problem. We resolve 110 regions and use a 4-hourly
equivalent time resolution for one year. Thereby, grid bottlenecks, renewable
variability, and seasonal storage requirements are efficiently captured. The
model includes regional demands from the electricity, industry, buildings,
agriculture and transport sectors, international shipping and aviation, and
non-energy feedstock demands in the chemicals industry. Transmission
infrastructure for electricity, gas and hydrogen and candidate entry points like
existing and prospective LNG terminals and cross-continental pipelines are also
represented. In the scenarios shown below, we enforce net-zero emissions for
carbon dioxide, take technology and assumptions for the year 2030,\cite{dea2019}
and limit the carbon sequestration potential to 200~Mt$_{\text{CO}_2}$/a, which
suffices to offset unabatable industrial process emissions. 

Green fuel and steel import costs seen by the model are based on an extension of
recent research by Hampp et al.,\cite{hamppImportOptions2023} who assessed the
levelised cost of energy exports for different green energy and material supply
chains in various world regions (\cref{fig:options:global}). Our selection of
exporting countries comprises Australia, Argentina, Chile, Kazakhstan, Namibia,
Turkey, Ukraine, the Eastern United States and Canada, mainland China, and the
MENA region. Regional supply cost curves for these countries are developed based
on renewable resources, land availability and prioritised domestic demand. Other
than domestic electrofuel synthesis in Europe, which could use captured CO$_2$
from point sources, direct air capture is assumed to be the sole carbon source
of imported fuels. For hydrogen derivatives, the lowest-cost suppliers are
Argentina and Chile.

We use these supply curves to determine the region-specific lowest import cost
for each carrier, thus incorporating the potential trade-off between import cost
and import location (\cref{fig:options:europe}). Electricity imports are
endogenously optimised, meaning that the capacities and operation of wind and
solar generation as well as storage in the respective exporting countries and
the HVDC transmission lines, are co-planned with the rest of the system.
Hydrogen and methane can be imported where there are existing or planned LNG
terminals or pipeline entry-points (excluding connections through Russia). This
results in lower hydrogen import costs, where it can be imported by pipeline.
Imports of ammonia, carbonaceous fuels and steel are not spatially resolved,
assuming they can be transported within Europe at low cost. 

More details are included in the \nameref{sec:methods} section.

% unused text snippets

% , alongside equal amounts of domestic hydrogen production (compared to hydrogen imports REPowerEU)

% repel renewal of such dependencies with green fuels and goods

% Furthermore, green hydrogen could offer
% a replacement for hydrogen from fossil sources as a chemical feedstock in the
% future.

% It is not
% a question of technical feasibility as the renewable potential to fully satisfy
% its own energy demands would be sufficient.

% how the different import scenarios affect the energy infrastructure inside
% Europe.

% which import routes are selected, as well as directions and magnitudes
% of energy flows which have to be supported.