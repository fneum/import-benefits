%%% promise of imports %%%

Importing renewable energy to Europe promises several advantages for achieving a
swift energy transition. It could lower costs, help circumvent the slow domestic
deployment of renewable energy infrastructure and reduce pressure on land usage
in Europe. Many parts of the world have cheap and abundant renewable energy
supply potentials they could offer to existing or emerging global energy
markets.\cite{irenaGlobalHydrogen2022,luxSupplyCurves2021,vanderzwaanTimmermansDream2021,fasihiLongTermHydrocarbon2017,reichenbergDeepDecarbonization2022,galvanExportingSunshine2022,armijoFlexibleProduction2020,pfennigGlobalGISbasedPotential2023}
Partnering with these regions could help Europe reach its carbon neutrality
goals while stimulating economic development in exporting countries.

%%% dangers of imports %%%

However, even if energy imports are economically attractive for Europe, a strong
reliance may not be desired due to energy security concerns. Awareness of energy
security has risen since Russia throttled gas supplies to Europe in
2022,\cite{pedersenLongtermImplications2022} at a time when the EU27 imported
around two-thirds of its energy needs.\cite{eurostatCompleteEnergy2023}
Accordingly, much of the European energy infrastructure is built around the
imports of fossil oil and gas. Continued energy imports and associated rigid
infrastructure might tie European energy supply to a small number of exporters
with market power. Such lock-in risks must be weighed against the potential
benefits of imports.

%%% dependence of energy imports on energy infrastructure %%%

Europe's strategy for clean energy imports will also strongly affect the
requirements for domestic energy infrastructure. Previous research found many
ways to develop a self-sufficient energy
system.\cite{pickeringDiversityOptions2022,trondleHomemadeImported2019,brownSynergiesSector2018}
To support such scenarios without energy imports into Europe, reinforcing the
power grid or building a hydrogen network was often identified as
constructive.\cite{neumannPotentialRole2023,victoriaSpeedTechnological2022}
However, depending on the vectors and volumes of imports, Europe might not need
to expand its hydrogen transport infrastructure. Most hydrogen is used to make
derivative products (e.g.~, Fischer-Tropsch fuels or methanol for high-value
chemicals, aviation and shipping or ammonia for
fertilisers).\cite{neumannPotentialRole2023} If Europe imported these products
at scale, much of the hydrogen demand would fall away. In consequence, this
would reduce the need for hydrogen transport. Furthermore, substantial direct
hydrogen imports would require a different pipeline network topology, tailored
to accommodate hydrogen arriving from North Africa or maritime shipping routes
to Northern Europe.

%%% review of policy strategies %%%

Policy has reflected these different visions for imports in various ways. In
particular, hydrogen imports have recently attracted considerable interest, with
plans of the European Commission under
\mbox{REPowerEU}\cite{europeancommissionRepowerEUPlan} to import 10~Mt
(333~TWh\footnote{All mass-energy conversion is based on the lower heating value
(LHV). Steel is included in energy terms applying 2.1 kWh/kg as released by the
oxidation of iron.}) hydrogen and derivatives by 2030. Desire to import hydrogen
and derivative products is also present in various national
strategies.\cite{corbeauNationalHydrogenStrategies2024} In particular, Germany
seeks to cover up to 70\% of its hydrogen consumption through imports by 2030
and pursues bilateral partnerships to accomplish
this.\cite{bundesministeriumfuerwirtschaftundklimaschutzFortschreibungNationalenWasserstoffstrategie2023}
Conversely, hydrogen roadmaps of
Denmark,\cite{danishministryofclimateenergyandutilitiesRegeringensStrategiPowertoX2021}
Ireland,\cite{departmentoftheenvironmentclimateandcommunicationsgovernmentofirelandNationalHydrogenStrategy2023}
Spain,\cite{marcoestrategicodeenergiayclimaRutaHidrogenoApuesta2020} and the
United
Kingdom,\cite{ukdepartmentforenergysecurity&netzeroHydrogenStrategyUpdate2023}
recognise these countries' potential to become major exporters of renewable
energy, whereas France's strategy focuses on local hydrogen production to meet
domestic needs.\cite{frenchgovernmentStrategieNationalePour2023} Additionally,
in the recent TYNDP 2024,\cite{entso-eTYNDP2024Project2024}  European grid
development plans reveal renewed enthusiasm for electricity imports via
ultra-long HVDC cables, evolving from early
DESERTEC\cite{desertecfoundationDESERTECSustainableWealth2024} ideas to
contemporary proposals like the Morocco-UK Xlinks
project.\cite{xlinksMoroccoUKPower2023}

%%% literature review %%%

While many previous academic studies have evaluated the cost of green energy and
material imports in the form of
electricity,\cite{lilliestamEnergySecurity2011,triebSolarElectricity2012,lilliestamVulnerabilityTerrorist2014,bogdanovNorthEastAsian2016,benaslaTransitionSustainable2019,reichenbergDeepDecarbonization2022}% (some with reference to the DESERTEC idea),
hydrogen,\cite{timmerbergHydrogenRenewables2019,ishimotoLargescaleProduction2020,brandleEstimatingLongterm2021,luxSupplyCurves2021,galvanExportingSunshine2022,collisDeterminingProduction2022,galimovaImpactInternational2023,schmitzImplicationsHydrogenImport2024}
ammonia,\cite{nayak-lukeTechnoeconomicViability2020,armijoFlexibleProduction2020,galimovaFeasibilityGreen2023}
methane,\cite{luxSupplyCurves2021,agoraenergiewendeHydrogenImport2022}
steel,\cite{trollipHowGreen2022a,devlinRegionalSupply2022,lopezDefossilisedSteel2023}
carbon-based
fuels,\cite{fasihiLongTermHydrocarbon2017,sherwinElectrofuelSynthesis2021} or a
broader variety of power-to-X
fuels,\cite{vanderzwaanTimmermansDream2021,pfennigGlobalGISbased2022,irenaGlobalHydrogen2022,solerEFuelsTechno2022,hamppImportOptions2023,gengeSupplyCosts2023,galimovaGlobalTrading2023a}
these do not address the interactions of imports with European energy
infrastructure requirements. On the other hand, among studies dealing with the
detailed planning of net-zero energy systems in Europe, some do not consider
energy
imports,\cite{pickeringDiversityOptions2022,brownSynergiesSector2018,victoriaSpeedTechnological2022}
while others only consider hydrogen imports or a limited set of alternative
import
vectors.\cite{gilsInteractionHydrogen2021,seckHydrogenDecarbonization2022,wetzelGreenEnergy2023,kountourisUnifiedEuropean2023,neumannPotentialRole2023}
Only a few consider at least elementary cost
uncertainties,\cite{frischmuthHydrogenSourcing2022,schmitzImplicationsHydrogenImport2024} and none investigate a
larger range of potential import volumes across subsets of available import
vectors.


%%% main paper idea - scenarios %%%

In this study, we explore the full range between the two poles of complete
self-sufficiency and wide-ranging renewable energy imports into Europe in
scenarios with high shares of wind and solar electricity and net-zero carbon
emissions. We investigate how the infrastructure requirements of a
self-sufficient European energy system that exclusively leverages domestic
resources from the continent may differ from a system that relies on energy
imports from outside of Europe. For our analysis, we integrate an open model of
global energy supply chains, TRACE,\cite{hamppImportOptions2023} with a
spatially and temporally resolved sector-coupled open-source energy system
model, PyPSA-Eur,\cite{PyPSAEurSecSectorCoupled} to investigate the impact of
imports on European energy infrastructure needs. We evaluate potential import
locations and costs for different supply vectors, by how much system costs can
be reduced through imports, and how their inclusion affects deployed transport
networks and storage. For this purpose, we perform sensitivity analyses
interpolating between very high levels of imports and no imports at all,
exploring low and high costs for imports to account for associated
uncertainties, and system responses to the exclusion of subsets of import
vectors, all to probe the flatness of the solution space.


\begin{figure*}
    \begin{subfigure}[t]{\textwidth}
        \caption{Global perspective for energy imports into Europe}
        \label{fig:options:global}
        \includegraphics[width=\textwidth]{20231025-zecm/graphics/import_world_map.pdf}
    \end{subfigure}
    \begin{subfigure}[t]{0.7\textwidth}
        \caption{European perspective for inbound energy imports}
        \label{fig:options:europe}
        \centering
        \includegraphics[width=\textwidth]{20231025-zecm/graphics/import_options_s_110_lvopt__Co2L0-2190SEG-T-H-B-I-S-A-onwind+p0.5-imp_2050.pdf}
    \end{subfigure}
    \begin{subfigure}[t]{0.3\textwidth}
        \caption{Cost distribution by origin, entrypoint and carrier}
        \label{fig:options:distribution}
        \centering
        \includegraphics[width=\textwidth]{20231025-zecm/graphics/import_options_s_110_lvopt__Co2L0-2190SEG-T-H-B-I-S-A-onwind+p0.5-imp_2050-distribution.pdf}
    \end{subfigure}
    \caption{\textbf{Overview of considered import options.}
        \textit{Panel (a)} shows the regional differences in the cost to deliver
        green methanol to Europe (choropleth layer), the cost composition of
        different import vectors (bar charts), an illustration of the wind and
        solar availability in Morocco, and an illustration of the land
        eligibility analysis for wind turbine placement in the region of Buenos
        Aires in Argentina. \textit{Panel (b)} depicts considered potential
        entry points for energy imports into Europe like the location of
        existing and planned LNG terminals and gas pipeline entry points, the
        lowest costs of hydrogen imports in different European regions
        (choropleth layer), and the considered connections for long-distance
        HVDC import links from the MENA region, Kazakhstan, Turkey and Ukraine.
        \textit{Panel (c)} displays the distribution and range of import costs
        for different energy carriers and entry points with indications for
        selected countries of origin from the TRACE model (violin charts),
        i.e.~differences in identically coloured markers are due to regional
        differences in the transport costs to entrypoints.}
    \label{fig:options}
\end{figure*}

% JH Draw connections such that they intersect less, e.g. MA-IE around ES, not
% through ES and PT.

%%% technical discussion of import vectors %%%

As possible import options, we consider electricity by transmission line,
hydrogen as gas by pipeline and liquid by ship, methane as gas by pipeline and
liquid by ship, liquid ammonia, steel, methanol and Fischer-Tropsch fuels by
ship. Each energy vector has unique characteristics with regards to its
production, transport and consumption
(\sfigref{fig:si:balances-a,fig:si:balances-b}). Electricity offers the most
flexible usage but is challenging to store and requires variability management
if sourced from wind or solar energy. Hydrogen is easier to store and transport
in large quantities but at the expense of conversion losses and less versatile
applications. Large quantities could be used for backup power and heat, steel
production, and the domestic synthesis of shipping and aviation fuels. Synthetic
carbonaceous fuels like methane, methanol and Fischer-Tropsch fuels could
largely substitute the need for domestic synthesis. There is much more
experience with storing and transporting these fuels and part of the existing
infrastructure could potentially be leveraged. However, they require a
sustainable carbon source and, particularly for methane, effective carbon
management and leakage prevention.\cite{shirizadehImpactMethaneLeakage2023}
Ammonia is similarly easier to handle than hydrogen but does not require a
carbon source. However, it faces safety and acceptance concerns due to its
toxicity and potentially adverse effects on the global nitrogen
cycle.\cite{bertagniMinimizingImpactsAmmonia2023,wolframUsingAmmoniaShipping2022}
Its demand in Europe is mostly driven by fertiliser usage. Steel represents the
import of energy-intensive materials and offers low transport costs.

Further conversion of imported fuels is also possible once they have arrived in
Europe, e.g.~hydrogen could be used to synthesise carbon-based fuels, ammonia
could be cracked to hydrogen, methane and methanol could be reformed to hydrogen
or combusted for power generation with or without carbon capture. However,
conversion losses can make it less attractive economically to use a high-value
hydrogen derivative merely as a transport and storage vessel only to reconvert
it back to hydrogen or electricity.

%%% brief methodology PyPSA-Eur %%%

The PyPSA-Eur\cite{PyPSAEurSecSectorCoupled} model co-optimises the investment
and operation of generation, storage, conversion and transmission
infrastructures, as well as the relocation of some industries within
Europe,\cite{verpoortEstimatingRenewables2023,samadiRenewablesPull2023} in a
single linear optimisation problem. We resolve 110 regions comprising the
European Union without Cyprus and Malta as well as the United Kingdom, Norway,
Switzerland, Albania, Bosnia and Herzegovina, Montenegro, North Macedonia,
Serbia, and Kosovo. In combination with a 4-hourly equivalent time resolution
for one year, grid bottlenecks, renewable variability, and seasonal storage
requirements are efficiently captured. Weather variations between years are not
considered for computational reasons. The model includes regional demands from
the electricity, industry, buildings, agriculture and transport sectors,
international shipping and aviation, and non-energy feedstock demands in the
chemicals industry. Transmission infrastructure for electricity, gas and
hydrogen and candidate entry points like existing and prospective LNG terminals
and cross-continental pipelines are also represented. We utilize techno-economic
assumptions for 2030\cite{dea2019}, reflecting that infrastructure required for
achieving carbon neutrality must be built well in advance of reaching this goal.
While enforcing net-zero emissions for carbon dioxide, we also limit the annual
carbon sequestration potential to 200~Mt$_{\text{CO}_2}$/a. This suffices to
offset unabatable industrial process emissions of around
140~Mt$_{\text{CO}_2}$/a and limited use of fossil fuels beyond that, either
through capturing emissions at source or via carbon dioxide removal.

More details are included in the \nameref{sec:methods} section.
