%%% promise of imports %%%

Importing renewable energy to Europe may offer several advantages for achieving
a swift energy transition. It might lower costs, help circumvent the slow
domestic deployment of renewable energy infrastructure and reduce pressure on
land usage in Europe. Many parts of the world have cheap and abundant renewable
energy supply potentials that they could offer to existing or emerging global
energy markets.\cite{irenaGlobalHydrogenTrade2022, luxSupplyCurves2021,
vanderzwaanTimmermansDream2021, fasihiLongTermHydrocarbon2017,
reichenbergDeepDecarbonization2022, galvanExportingSunshine2022,
armijoFlexibleProduction2020, pfennigGlobalGISbasedPotential2023} Partnering
with these regions could help Europe reach its carbon neutrality goals while
stimulating economic development in exporting regions.

%%% dangers of imports %%%

However, even if energy imports are economically attractive for Europe, a strong
reliance may not be desirable because of energy security concerns. Awareness of
energy security has risen since Russia constrained fossil gas supplies to Europe
in 2022,\cite{pedersenLongtermImplications2022} at a time when the EU27 imported
around two-thirds of its fossil energy needs.\cite{eurostatCompleteEnergyBalances2023}
Europe must take care to avoid repeating the mistakes of previous decades when
it became dependent on a small number of exporters with market power and reliant
on rigid pipeline infrastructure.

%%% dependence of energy imports on energy infrastructure %%%

Europe's strategy for clean energy imports will also strongly affect the
requirements for domestic energy infrastructure. Previous research found many
ways to develop a self-sufficient energy
system.\cite{pickeringDiversityOptions2022, trondleHomemadeImported2019,
brownSynergiesSector2018} To support such scenarios without energy imports into
Europe, reinforcing the European power grid or building a hydrogen network was
often identified as beneficial.\cite{neumannPotentialRoleHydrogen2023,
victoriaSpeedTechnological2022} However, depending on the volumes and vectors of
imports (electricity, hydrogen, or hydrogen derivatives) and levels of industry
migration, Europe might not need to expand its hydrogen pipeline infrastructure.
Most hydrogen is used to make derivative products (e.g.~, ammonia for
fertilisers, sponge iron for steel, or Fischer-Tropsch fuels for aviation and
shipping).\cite{neumannPotentialRoleHydrogen2023} If Europe imported these
products at scale, much of the hydrogen demand would fall away. In consequence,
this would reduce the need for hydrogen transport. However, if hydrogen itself
were imported and to be transported to today's industry clusters, this would
require a pipeline topology tailored to connecting these to the hydrogen
arriving from North Africa or maritime entry points across Europe.

%%% review of policy strategies %%%

Policy has reflected these different visions for imports in various ways. In
particular, hydrogen imports have recently attracted considerable interest, with
plans of the European Commission\cite{europeancommissionRepowerEUPlan} to import
10~Mt (333~TWh$_\text{LHV}$) hydrogen and derivatives by 2030. New financing
instruments, like the European Hydrogen
Bank\cite{europeancommissionEuropeanHydrogenBank2024} or
H2Global\cite{h2globalfoundationH2Global2024} are set up to support the scale-up
of green hydrogen imports. Desire to import hydrogen and derivative products
is also present in various national
strategies.\cite{corbeauNationalHydrogenStrategies2024} In particular, Germany's
new import strategy plans to cover up to 70\% of its demand for hydrogen and its
derivatives through imports by 2030 and highlights bilateral partnerships as
well as the expansion of import infrastructure as a means to accomplish
this.\cite{germanfederalministryofeconomicaffairsandclimateactionbmwkNationalHydrogenStrategy2023,germanfederalministryofeconomicaffairsandclimateactionbmwkImportStrategyHydrogen2024}
Conversely, hydrogen roadmaps of
Denmark,\cite{danishministryofclimateenergyandutilitiesRegeringensStrategiPowertoX2021}
Ireland,\cite{departmentoftheenvironmentclimateandcommunicationsgovernmentofirelandNationalHydrogenStrategy2023}
Spain,\cite{marcoestrategicodeenergiayclimaRutaHidrogenoApuesta2020} and the
United
Kingdom,\cite{ukdepartmentforenergysecurity&netzeroHydrogenStrategyUpdate2023}
recognise these countries' potential to become major exporters of renewable
energy, whereas France's strategy focuses on local hydrogen production to meet
domestic needs.\cite{frenchgovernmentStrategieNationalePour2023} Beyond direct
energy imports, the Draghi report\cite{draghiFutureEuropeanCompetitiveness2024}
also raises broader concerns about European industrial competitiveness and
discusses the benefits of relocating energy-intensive industries to renewable-rich
regions inside Europe. Additionally, European grid development
plans\cite{entso-eTYNDP2024Project2024} reveal renewed enthusiasm for
electricity imports via ultra-long HVDC cables, evolving from early
DESERTEC\cite{desertecfoundationDESERTECSustainableWealth2024} ideas to
contemporary proposals like the Morocco-UK Xlinks
project.\cite{xlinksMoroccoUKPowerProject2023}

%%% literature review %%%

While many previous academic studies have evaluated the cost of `green'
renewable energy and energy-intensive material imports in the form of
electricity,\cite{lilliestamEnergySecurity2011,triebSolarElectricity2012,lilliestamVulnerabilityTerrorist2014,bogdanovNorthEastAsian2016,benaslaTransitionSustainable2019,reichenbergDeepDecarbonization2022}% (some with reference to the DESERTEC idea),
hydrogen,\cite{timmerbergHydrogenRenewables2019,ishimotoLargescaleProduction2020,brandleEstimatingLongterm2021,luxSupplyCurves2021,galvanExportingSunshine2022,collisDeterminingProduction2022,galimovaImpactInternational2023,franzmannGreenHydrogenCostpotentials2023,schmitzImplicationsHydrogenImport2024}
ammonia,\cite{nayak-lukeTechnoeconomicViability2020,armijoFlexibleProduction2020,galimovaFeasibilityGreen2023,egererEconomicsGlobalGreen2023}
methane,\cite{luxSupplyCurves2021,agoraenergiewendeHydrogenImport2022,carelsSyntheticNaturalGas2024}
steel,\cite{trollipHowGreen2022a,devlinRegionalSupply2022,lopezDefossilisedSteel2023}
carbon-based
fuels,\cite{fasihiLongTermHydrocarbon2017,sherwinElectrofuelSynthesis2021} or a
broader variety of power-to-X
fuels,\cite{vanderzwaanTimmermansDream2021,pfennigGlobalGISbasedPotential2023,irenaGlobalHydrogenTrade2022,solerEFuelsTechno2022,hamppImportOptions2023,gengeSupplyCostsGreen2023,galimovaGlobalTrading2023a}
these do not address the interactions of imports with European energy
infrastructure requirements. On the other hand, among studies dealing with the
detailed planning of net-zero energy systems in Europe, some do not consider
energy
imports,\cite{pickeringDiversityOptions2022,brownSynergiesSector2018,victoriaSpeedTechnological2022}
while others only consider hydrogen imports or a limited set of alternative
endogenously optimised import
vectors.\cite{gilsInteractionHydrogen2021,seckHydrogenDecarbonization2022,wetzelGreenEnergy2023a,neumannPotentialRoleHydrogen2023,fleiterHydrogenInfrastructureFuture2024,kountourisUnifiedEuropeanHydrogen2024}
Only a few consider at least elementary cost
uncertainties,\cite{frischmuthHydrogenSourcing2022,schmitzImplicationsHydrogenImport2024}
and none investigate a larger range of potential import volumes across subsets
of available import vectors.


%%% main paper idea - scenarios %%%

In this study, we explore the full range between the two poles of complete
self-sufficiency and wide-ranging renewable energy imports into Europe in
scenarios with high shares of wind and solar electricity and net-zero carbon
emissions. We investigate how the infrastructure requirements of a
self-sufficient European energy system that exclusively leverages domestic
resources from the continent may differ from a system that relies on energy
imports from outside of Europe. For our analysis, we integrate an open
optimisation model of global energy supply chains,
TRACE,\cite{hamppImportOptions2023} with a spatially and temporally resolved
sector-coupled open-source energy system optimisation model for Europe,
PyPSA-Eur,\cite{PyPSAEurSecSectorCoupledOpen} to investigate the impact of imports
on European energy infrastructure needs. We evaluate potential import locations
and costs for different supply vectors, by how much system costs can be reduced
through imports, and how their inclusion affects deployed transport networks,
storage and backup capacities. For this purpose, we perform sensitivity analyses
interpolating between very high levels of imports and no imports at all,
exploring low and high costs for imports to account for associated
uncertainties, and system responses to the exclusion of subsets of import
vectors, in order to identify the cost-effective maneuvering space.


\begin{figure}
    \fullwidthfigure{
    \begin{subfigure}[t]{\linewidth}
        \caption{Global perspective for energy imports into Europe}
        \label{fig:options:global}
        \includegraphics[width=\linewidth]{../workflow/pypsa-eur/resources/20240826-z1/graphics/import_world_map.pdf}
    \end{subfigure}
    \begin{subfigure}[t]{0.7\linewidth}
        \caption{European perspective for inbound energy imports}
        \label{fig:options:europe}
        \centering
        \includegraphics[width=\linewidth]{../workflow/pypsa-eur/results/20240826-z1/graphics/import_options_s_115_lvopt__imp_2050.pdf}
    \end{subfigure}
    \begin{subfigure}[t]{0.3\linewidth}
        \caption{Cost distribution by origin and vector}
        \label{fig:options:distribution}
        \centering
        \includegraphics[width=\linewidth]{../workflow/pypsa-eur/results/20240826-z1/graphics/import_options_s_115_lvopt__imp_2050-distribution.pdf}
    \end{subfigure}
    \caption{\textbf{Overview of considered import options into Europe.}
        \textit{Panel (a)} shows the regional differences in the cost to deliver
        green methanol to Europe (choropleth layer), the cost composition of
        different import vectors (bar charts), an illustration of the wind and
        solar availability in Morocco, and an illustration of the land
        eligibility analysis for wind turbine placement in the region of Buenos
        Aires in Argentina. \textit{Panel (b)} depicts considered potential
        entry points for energy imports into Europe like the location of
        existing and planned LNG terminals and gas pipeline entry points, the
        lowest costs of hydrogen imports in different European regions
        (choropleth layer), and the considered connections for long-distance
        HVDC import links and hydrogen pipelines from the MENA region, Turkey,
        Ukraine and Central Asia. \textit{Panel (c)} displays the distribution
        and range of import costs for different energy carriers and entry points
        with indications for selected origins from the TRACE model
        (violin charts), i.e.~differences in identically coloured markers are
        due to regional differences in the transport costs to alternative
        entrypoints. These are more variable for liquid hydrogen as transport
        distance is a more substantial cost factor for this import vector.
        Supplementary Fig.~3 shows the world map for lowest
        hydrogen import costs by pipeline or ship into Europe.}
    \label{fig:options}
    }
\end{figure}

%%% technical discussion of import vectors %%%

As possible import options, we consider electricity by transmission line,
hydrogen as gas by pipeline and liquid by ship, methane as liquid by ship,
liquid ammonia, steel and its precursor hot briquetted iron (HBI), methanol and
Fischer-Tropsch fuels by ship. Each energy vector has unique characteristics
with regards to its production, storage, transport and consumption. Electricity offers
the most flexible usage but is challenging to store and requires variability
management if sourced from wind or solar energy. Hydrogen is easier to store and
transport in large quantities but at the expense of conversion losses and less
versatile applications. Large quantities could be used for backup power and
heat, steel production, industry feedstocks and the domestic synthesis of
shipping and aviation fuels. On the other hand, imported synthetic carbonaceous
fuels like methane, methanol and Fischer-Tropsch fuels could largely substitute
the need for domestic synthesis. There is more experience with storing and
transporting these fuels and part of the existing infrastructure could
potentially be reused or repurposed. However, they require a sustainable carbon
source and, particularly for methane, effective carbon management and leakage
prevention.\cite{shirizadehImpactMethaneLeakage2023} Ammonia is similarly easier
to handle than hydrogen but does not require a carbon source. However, it faces
safety and acceptance concerns due to its toxicity and potentially adverse
effects on the global nitrogen
cycle.\cite{bertagniMinimizingImpactsAmmonia2023,wolframUsingAmmoniaShipping2022}
Its demand in Europe is mostly driven by fertiliser usage. Steel and HBI
represent the import of energy-intensive materials and offer low long-distance
transport costs.

%%% brief methodology PyPSA-Eur %%%

The PyPSA-Eur\cite{PyPSAEurSecSectorCoupledOpen} model co-optimises the investment
and operation of generation, storage, conversion and transmission
infrastructures in a single linear optimisation problem. The model is further
given the opportunity to relocate ammonia and primary steel production within
Europe, capturing potential renewables pull effects within Europe and
abroad.\cite{verpoortImpactGlobalHeterogeneity2024,
samadiRenewablesPullEffect2023,egererIndustryTransformationFossil2024} We
resolve 115 regions comprising the European Union without Cyprus and Malta as
well as the United Kingdom, Norway, Switzerland, Albania, Bosnia and
Herzegovina, Montenegro, North Macedonia, Serbia, and Kosovo. In combination
with a 4-hourly-equivalent time resolution for the weather year of 2013, grid
bottlenecks, renewable variability, and seasonal storage requirements are
sufficiently captured. The  model includes regional demands from the
electricity, industry, buildings, agriculture and transport sectors,
international shipping and aviation, and non-energy feedstock demands in the
chemicals industry. Transmission infrastructure for electricity, gas and
hydrogen, and candidate entry points like existing and prospective LNG terminals
as well as cross-continental pipelines are also represented.  However, no
pathways are modelled in this overnight scenario and the model has perfect
operational foresight. We utilize techno-economic assumptions for 2040 and
enforce net-zero CO$_2$ emissions and limit the annual carbon sequestration to
200~Mt$_{\text{CO}_2}$~a$^{-1}$, similar to the 250~Mt$_{\text{CO}_2}$~a$^{-1}$
highlighted in the EU carbon management
strategy.\cite{europeancommissionAmbitiousIndustrialCarbon2024} This suffices to
offset unabated industrial process emissions and limits the use of fossil fuels
beyond that, whose emissions are compensated either through capturing emissions
at source or by carbon dioxide removal. More details are included in the
\nameref{sec:methods} section.
