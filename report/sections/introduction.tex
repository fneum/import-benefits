% promise of imports

% JH Finde die Schiene "acceptance"/"socially accepted" hier problematisch:
% Das ist die Europäische Perspektive, local acceptance in den exporting countries würde damit außen vorgelassen.
% Aspekt: "to offer to global energy markets": Not for all commodities global markets exist yet, and even where they exist they are regionally diverse.
% Maybe we could instead go with motivation:
% * diversification / imports for robustness
% * limited RES potentials, imports nowadays seem cheaper than doing some things locally (which we will partially disprove in the paper when synergies through integration are utilised)

The transformation of the European energy system towards climate neutrality
demands unrivalled technological change. 
% TB Importing renewable energy to Europe offers several advantages: cheaper energy
% from regions with abundant solar and wind, as well as less pressure in Europe
% on land and infrastructure. Regions of the world with cheap and abundant.
Whereas the development of renewable
energy sources in Europe and supporting measures like reinforcing the
electricity grid do not always meet the level of acceptance required for a swift
transition, other parts of the world have cheap and abundant renewable energy
supply potentials to offer to global energy
markets.\cite{irenaGlobalHydrogen2022,luxSupplyCurves2021,vanderzwaanTimmermansDream2021,fasihiLongTermHydrocarbon2017,reichenbergDeepDecarbonization2022,galvanExportingSunshine2022,armijoFlexibleProduction2020,pfennigGlobalGISbased2022}
These regions could become key partners for a cost-effective and socially
acceptable energy transition in Europe, especially when the production of large
quantities of domestic green fuels and materials falters.

% repulsion of imports

However, even if energy imports are economically attractive, a strong dependence
on them may not be preferred due to energy security concerns.  Awareness of
energy security has risen since Russia throttled gas supplies to Europe in
2022.\cite{pedersenLongtermImplications2022} In 2021, the EU27 nations sourced
around two-thirds of their energy needs through
imports,\cite{eurostatCompleteEnergy2023}, and accordingly, much of the European
energy infrastructure is built around the imports of fossil oil and gas.
Continued energy imports and associated infrastructure might tie European energy
supply to a small number of exporters with market power. These risks
must be weighed against the potential benefits of decreasing energy supply
costs, reducing land usage in Europe and increasing energy security by supplying
storable fuels that can mitigate energy droughts for systems with high shares of
weather-dependent energy supply.
% TB too many ideas in this sentence --short + focused

% feasibility of no imports

The transition to a system that exploits the best wind and solar sites across
the continent would offer many ways to develop a self-sufficient system without
imports.\cite{pickeringDiversityOptions2022,trondleHomemadeImported2019,brownSynergiesSector2018}
For instance, reinforcing the power grid or building a hydrogen network was
consistently identified as beneficial in such autarkic
scenarios.\cite{neumannPotentialRole2023,victoriaSpeedTechnological2022}
% JH maybe emphasize: autark as in "Europe autark for itself", not autarky on national level
% JH Context for sentences seems to be missing, both are coming out of nowhere.
However, what energy infrastructure is needed might strongly depend on the
levels of clean energy imported. For instance, scaling up Europe's energy
transmission infrastructure may not be necessary. Since most hydrogen would be
used to produce synthetic fuels (e.g.~, high-value chemicals, fertilisers,
aviation and shipping) and steel,\cite{neumannPotentialRole2023} if these
derivatives were imported at scale, much of the hydrogen demand would fall away,
hence, reducing the need for hydrogen transport. Even if there were a high
demand for direct hydrogen imports, the optimised pipeline network topology
would differ if it needed to absorb inbound hydrogen from North Africa or
sea-bound transport to Northern Europe rather than domestic production.

% TB we need a uniform language here; where is ammonia?
% - molecules: FT, NH3, MeOH, CH4, H2
% % - fuels: FT, MeOH, CH4, H2
% - materials: Steel, MeOH/FT ????

% policy strategies

% TODO more recent news / strategies?

% TB Which idea does this paragraph belong to? Maybe a policy paragraph?
% "Policy has refelcted these different visions / has sided on the side of imports /...."

In particular, hydrogen imports have recently attracted considerable interest,
with plans of the European Commission under
\mbox{REPowerEU}\cite{europeancommissionRepowerEUPlan} to import 10~Mt
(333~TWh\footnote{All mass-energy conversion in this paper is based on the lower
heating value.}) hydrogen and derivatives by 2030 and reflections of this vision
in the German national hydrogen
strategy.\cite{bundesministeriumfuerwirtschaftundklimaschutzFortschreibungNationalen2023}

% TB reflections - Germans might say EU policy reflects German NWS (which came first?)
% JH why emphasis on DE H2 strategy, could also be others? I'd say remove side sentence

% literature review of studies focusing on the cost of energy imports
% literature review focusing on the European energy system with/without considering imports

% TODO stronger novelty statement?

While many previous academic studies have evaluated the cost of green energy and material imports in the form of
electricity,\cite{lilliestamEnergySecurity2011,triebSolarElectricity2012,lilliestamVulnerabilityTerrorist2014,bogdanovNorthEastAsian2016,benaslaTransitionSustainable2019,reichenbergDeepDecarbonization2022}% (some with reference to the DESERTEC idea),
hydrogen,\cite{timmerbergHydrogenRenewables2019,ishimotoLargescaleProduction2020,brandleEstimatingLongterm2021,luxSupplyCurves2021,galvanExportingSunshine2022,collisDeterminingProduction2022,galimovaImpactInternational2023}
ammonia,\cite{nayak-lukeTechnoeconomicViability2020,armijoFlexibleProduction2020,galimovaFeasibilityGreen2023}
methane,\cite{luxSupplyCurves2021,agoraenergiewendeHydrogenImport2022}
steel,\cite{trollipHowGreen2022a,devlinRegionalSupply2022,lopezDefossilisedSteel2023}
carbon-based
fuels,\cite{fasihiLongTermHydrocarbon2017,sherwinElectrofuelSynthesis2021} or a
broader variety of power-to-X
fuels,\cite{vanderzwaanTimmermansDream2021,pfennigGlobalGISbased2022,irenaGlobalHydrogen2022,solerEFuelsTechno2022,hamppImportOptions2023,gengeSupplyCosts2023,galimovaGlobalTrading2023a}
these do not address the interactions of imports with European energy infrastructure
requirements. On the other hand, among studies dealing with the detailed
planning of net-zero energy systems in Europe, some do not consider energy
imports,\cite{pickeringDiversityOptions2022,brownSynergiesSector2018,victoriaSpeedTechnological2022}
while others only consider hydrogen imports or a limited set of alternative
import
vectors.\cite{gilsInteractionHydrogen2021,seckHydrogenDecarbonization2022,wetzelGreenEnergy2023,kountourisUnifiedEuropean2023,neumannPotentialRole2023}
Only a few consider at least elementary cost
uncertainties,\cite{frischmuthHydrogenSourcing2022} and none investigate a
larger range of potential import volumes across subsets of available import
vectors.


% main paper idea - scenarios

In this contribution, we explore the full range between the two poles of
complete self-sufficiency and wide-ranging energy imports into Europe in
scenarios with high shares of wind and solar electricity and net-zero carbon
emissions. We investigate how the infrastructure requirements of a
self-sufficient European energy system that exclusively leverages domestic
resources from the continent may differ from a system that relies on energy
imports from outside of Europe. For our analysis, we integrate an open model of
global energy supply chains, TRACE,\cite{hamppImportOptions2023} with a
spatially and temporally resolved sector-coupled open-source energy system
model, PyPSA-Eur,\cite{PyPSAEurSecSectorCoupled} to investigate the impact of
imports on European energy infrastructure needs. We evaluate potential import
locations and costs for different supply vectors, the economic impetus
% TB partially....I mean we don't consider full value chains or macro-economic effects
% JH Somewhere around here we should add that the imports are also RES-based and net-zero
for such
imports, and how their inclusion affects deployed transport networks and
storage. For this purpose, we perform sensitivity analyses interpolating between
very high levels of imports and no imports at all, exploring low and high costs for
imports to account for associated uncertainties, and system responses to the
exclusion of subsets of import vectors, all to probe the flatness of the
solution space.


\begin{figure*} 
    \begin{subfigure}[t]{\textwidth}
        \caption{Global perspective}
        \label{fig:options:global}
        \includegraphics[width=\textwidth]{20231025-zecm/graphics/import_world_map.pdf}
    \end{subfigure}
    \begin{subfigure}[t]{\textwidth}
        \caption{European perspective}
        \label{fig:options:europe}
        \centering
        \includegraphics[width=\textwidth]{20231025-zecm/graphics/import_options_s_110_lvopt__Co2L0-2190SEG-T-H-B-I-S-A-onwind+p0.5-imp_2050.pdf}
    \end{subfigure}
    \caption{\textbf{Overview of considered import options.}
        \textit{Panel (a)} shows the regional differences in the cost to deliver
        green methanol to Europe (choropleth layer), the cost composition of
        different import vectors (bar charts), an illustration of the wind and
        solar availability in Morocco, and an illustration of the land
        eligibility analysis for wind turbine placement in the region of Buenos
        Aires in Argentina. \textit{Panel (b)} depicts considered potential
        entry points for energy imports into Europe like the location of
        existing and planned LNG terminals and gas pipeline entry points, the
        lowest costs of hydrogen imports in different European regions
        (choropleth layer), the considered connections for long-distance HVDC
        import links from the MENA region, Kazakhstan, Turkey and Ukraine, and
        the distribution and range of import costs for different energy carriers
        and entry points with indications for selected countries of origin from
        the TRACE model (violin charts). }
    \label{fig:options}
\end{figure*}

% TB I found arrows in panel a) hard to read, particularly in Argentina The
% import cost violin plot was nice, but it floats under the legend, then I
% wasn't sure which graphic the legend refers to - can we give it its own panel?

% JH (a)
% Title for capacity factor subplot
% SA extend title: for import to Europe
% AU extend title "steel imports from Australia to Europe"

% JH (b)
% LNG and pipeline are very difficult to disthinguish. Can we used different colors? or maybe hatching?
% Draw connections such that they intersect less, e.g. MA-IE around ES, not through ES and PT.
% Should put this as Panel (c), because it is an independent figure/content on its own

% TODO give sources for carrier statements

% discussion of different import vectors

As possible import options, we consider electricity by transmission line,
hydrogen as gas by pipeline and liquid by ship, methane as gas by pipeline and
liquid by ship, liquid ammonia, steel, methanol and
Fischer-Tropsch fuels by ship. Each vector not only varies in European demand
levels but also presents unique characteristics.
% TB This sentence was clumsy
% TB "Each energy vector has unique characteristics with regards to its production, transport and consumption."
% TB I felt in general the demand for the vectors was not really explained anywhere - can NH3 be cracked? MeOH reformed? FT converted back to H2?
Electricity offers the most
flexible usage but is challenging to store and requires variability management
if sourced from wind or solar energy. Hydrogen is easier to store and transport
in large quantities but at the expense of conversion losses and less versatile
applications. Synthetic carbonaceous fuels like methane, methanol and
Fischer-Tropsch fuels are easy to store and transport and could benefit from
existing infrastructure.
% JH "easier". LNG is not easy, but easier than H2 (l)
However, they require a sustainable carbon source and,
particularly for methane, effective leakage prevention. Ammonia is similarly
easier to handle and, while not needing a carbon source, faces
acceptance issues due to its toxicity.
% TB safe handling is more of a problem than acceptance I think
% JH similarly easier to handle than h2 and does not require a carbon source, however face safety and acceptance concerns due to its toxicity and potential adverse effects on the global nitrogen cycle.
% https://www.pnas.org/doi/10.1073/pnas.2311728120
% https://www.nature.com/articles/s41560-022-01124-4
Steel represents the import of energy-intensive materials and offers low
transport costs compared to its other cost factors. Further conversion of
imported fuels is also possible once they have arrived in Europe, e.g.~hydrogen
could be used to synthesise carbon-based fuels, and methane could be converted
to hydrogen. However, conversion losses make it less attractive to use a
high-value hydrogen derivative merely as a transport vessel only to reconvert it
back to hydrogen or electricity.
% JH  Still attractive if you can get rid of a lot of energy storage capacities
% by converting carbon-based fuel back to electricity.

% brief methodology of TRACE and PyPSA-Eur

The PyPSA-Eur\cite{PyPSAEurSecSectorCoupled} model co-optimises the investment
and operation of generation, storage, conversion and transmission
infrastructures, as well as the relocation of some industries within
Europe,\cite{verpoortEstimatingRenewables2023,samadiRenewablesPull2023} in a
single linear optimisation problem. We resolve 110 regions
% TB maybe mention here it include Europe + some neighbours? also that we don't
% consider local demand for electricity for it? but we do for the imported fuel
% supply curves?
and use a 4-hourly
equivalent time resolution for one year. Thereby, grid bottlenecks, renewable
variability, and seasonal storage requirements are efficiently captured.
% JH Because not all weather-dependent variability is captured with single weather year.
The
model includes regional demands from the electricity, industry, buildings,
agriculture and transport sectors, international shipping and aviation, and
non-energy feedstock demands in the chemicals industry. Transmission
infrastructure for electricity, gas and hydrogen and candidate entry points like
existing and prospective LNG terminals and cross-continental pipelines are also
represented. In the scenarios shown below, we enforce net-zero emissions for
carbon dioxide, take technology and assumptions for the year 2030,\cite{dea2019}
and limit the carbon sequestration potential to 200~Mt$_{\text{CO}_2}$/a, which
suffices to offset unabatable industrial process emissions. 
% TB 2030 and 200 MtCO2/a need a brief explanation - otherwise reviews will ask

Green fuel and steel import costs seen by the model are based on an extension of
recent research by Hampp et al.,\cite{hamppImportOptions2023} who assessed the
levelised cost of energy exports for different green energy and material supply
chains in various world regions (\cref{fig:options:global}). Our selection of
exporting countries comprises Australia, Argentina, Chile, Kazakhstan, Namibia,
Turkey, Ukraine, the Eastern United States and Canada, mainland China, and the
MENA region. Regional supply cost curves for these countries are developed based
on renewable resources, land availability and prioritised domestic demand.
Unlike domestic electrofuel synthesis in Europe, which could use captured CO$_2$
from point sources, direct air capture is assumed to be the sole carbon source
of imported fuels.
% JH Maybe add a note, that no imports of fuels with successive CO2 capture and
% shipping back of the CO2 (cycling) are considered, like the concept proposed
% by TES and or what Fraunhofer Chile is working on.

We use these supply curves to determine the region-specific lowest import cost
for each carrier, thus incorporating the potential trade-off between import cost
and import location (\cref{fig:options:europe}). For hydrogen derivatives, the
lowest-cost suppliers are Argentina and Chile.
% JH at all entry points into Europe? At which costs?
Electricity imports are
endogenously optimised, meaning that the capacities and operation of wind and
solar generation as well as storage in the respective exporting countries and
the HVDC transmission lines, are co-planned with the rest of the system.
Hydrogen and methane can be imported where there are existing or planned LNG
terminals or pipeline entry-points (excluding connections through Russia). This
results in lower hydrogen import costs, where it can be imported by pipeline.
Imports of ammonia, carbonaceous fuels and steel are not spatially resolved,
assuming they can be transported within Europe at negligible cost.
% JH maybe add: that demand isn't spatially resolved either for these.
% "without additional" cost, right?

More details are included in the \nameref{sec:methods} section.

% unused text snippets

% , alongside equal amounts of domestic hydrogen production (compared to hydrogen imports REPowerEU)

% repel renewal of such dependencies with green fuels and goods

% Furthermore, green hydrogen could offer
% a replacement for hydrogen from fossil sources as a chemical feedstock in the
% future.

% It is not
% a question of technical feasibility as the renewable potential to fully satisfy
% its own energy demands would be sufficient.

% how the different import scenarios affect the energy infrastructure inside
% Europe.

% which import routes are selected, as well as directions and magnitudes
% of energy flows which have to be supported.