% promise of imports

The transformation of the European energy system towards climate-neutrality
demands unrivalled technological change. Whereas the development of renewables
energy sources in Europe and supporting measures like reinforcing the
electricity grid do not always meet the level of acceptance required for a swift
transition, other parts of the world have cheap and abundant renewable energy
supply potentials to offer to global energy
markets.\cite{irenaGlobalHydrogen2022,luxSupplyCurves2021,vanderzwaanTimmermansDream2021,fasihiLongTermHydrocarbon2017,reichenbergDeepDecarbonization2022,galvanExportingSunshine2022,armijoFlexibleProduction2020,pfennigGlobalGISbased2022}
These regions could become key partners for a cost-effective and socially
accepted energy transition in Europe, especially when the production of
large quantities of domestic green fuels and materials falters.

% repulsion of imports

However, even if energy imports are economically attractive, a strong dependence
on energy imports may not be preferred as it can be detrimental to energy
security, as Europe recently experienced owing to its reliance on Russian fossil
energy.\cite{pedersenLongtermImplications2022} Energy imports and associated
infrastructure might tie European energy supply to a small number of exporters
or markets who then have market power. These risks must be weighed against the
potential benefits of decreasing energy supply costs, reducing land usage in
Europe and increasing energy security by supplying storable fuels that can
mitigate energy droughts for systems with high shares weather-dependent energy
supply.

% repel renewal of such dependencies with green fuels and goods

% feasibility of no imports

The transition to a system that exploits the
best wind and solar sites across the continent would offer many ways to develop
a self-sufficient system without
imports.\cite{pickeringDiversityOptions2022,brownSynergiesSector2018} It is not
a question of technical feasibility. Sufficient renewable potential to fully satisfy its own energy demands. Developing transmission infrastructure,
like reinforcing the power grid and building a hydrogen network that partially
repurposes an increasingly unused gas network is consistently beneficial in such
autarkic
scenarios.\cite{nneumannPotentialRole2023,wetzelGreenEnergy2022,victoriaSpeedTechnological2022}
This aligns with the vision of a \textit{European Hydrogen Backbone} of the
European gas industry.
\cite{gasforclimateEuropeanHydrogen2020,gasforclimateEuropeanHydrogen2022}

This self-sufficient scenarios much different from today's energy system. EU27
countries import around 90\% of their energy supply, almost exclusively fossil
(eurostat). And European energy infrastructure is built around the imports of
fossil oil and gas. Hence, what infrastructure is needed to support the system
depends on the levels of clean energy imported. This relates to the geographic
distribution of domestic energy resources tapped, which import routes are
selected, as well as directions and magnitudes of energy flows which have to be
supported.

For instance, scaling up Europe's energy transmission infrastructure may not be
necessary if imports of renewable energy carriers are considered. Since most
hydrogen would be used to produce synthetic fuels (e.g.~for high-value
chemicals, aviation and shipping), steel and ammonia,\cite{neumannBenefitsHydrogen2022a} if these
hydrogen derivatives were imported at scale, much of the hydrogen demand would
fall away. This would reduce the need for hydrogen transport infrastructure.
Even if there is high demand for direct hydrogen imports, the optimised topology
of a hydrogen network might differ significantly as new import locations need to
be connected rather than domestic production. The network's role could change
from distributing energy from North Sea hydrogen production hubs to
incorporating inbound hydrogen pipelines from North Africa.

% main paper idea

In this contribution, we explore the full range between the two poles of full
self-sufficiency and wide-ranging energy imports into Europe in scenarios with
high shares of wind and solar electricity and net-zero carbon emissions. We
investigate how the infrastructure requirements of a self-sufficent European
energy system that exclusively leverages domestic resources from the continent
may differ from a system that relies on energy imports from outside of Europe.
For our analysis, we integrate a model of global energy supply chains by Hampp
et al.\cite{hamppImportOptions2023} with a spatially and temporally resolved
sector-coupled energy system model, PyPSA-Eur, to investigate the impact of
imports on European energy infrastructure needs. We evaluate potential import
locations and costs for different supply vectors, the economic impetus for such
imports, and how their inclusion affects deployed transport networks and
storage. For this purpose, we perform sensitivity analyses interpolating between
very high levels of imports and no imports at all, low and high costs for
imports to account for associated uncertainties, and system responses to
limitign certain import vectors.

% literature review of studies focusing on the cost of energy imports

TODO


% literature review focusing on the European energy system with/without considering imports

some do not consider energy imports at all (or only fossil implicitly)

some only consider hydrogen imports






\section*{Overview of approach}

% TODO give sources for commercial examples and academic calculations

% discussion of different import vectors

As possible import options we consider imports of electricity, hydrogen,
methane, methanol, ammonia, steel and Fischer-Tropsch fuels. Each of these
carriers has different characteristics which leads to trade-offs regarding how
and where they may be imported.

Electricity, the most versatile carrier, is challenging to store and requires
variability management if directly sourced from renewable sources. Hydrogen is
easier to store and transport in large quantities than electricity but at the
expense of conversion losses and less versatile usage.

Hydrogen has also attracted considerable interest with plans of the European
Commission under \mbox{REPowerEU}\cite{europeancommissionRepowerEUPlan} to
import 10~Mt (333~TWh) hydrogen and derivatives by 2030, alongside equal amounts
of domestic hydrogen production. Furthermore, green hydrogen could offer a
replacement for hydrogen from fossil sources as a chemical feedstock in the
future.

Synthetic, green methane could benefit from existing infrastructure but requires
a sustainable carbon source and leakage prevention.

Ammonia does not require a carbon source and is simple and cheap to store and
transport over long distances. However, it suffers from acceptance problems due
to its toxicity and lower energy density.

Steel is low-cost transport, can import intermediate products like hot briquetted iron (HBI)

Finally, synthetic green Fischer-Tropsch fuels and other liquid carbonaceous
fuels like methanol are easy to store, transport and reuse existing
infrastructure, but the synthesis is energy-intensive due to high conversion
losses. Like methane, a sustainable carbon source is required. 


For each energy carrier we identify locations with existing or planned import
infrastructure where the respective carrier may enter the European energy
system. We consider import options for electricity by transmission line,
hydrogen as gas by pipeline and as liquid by ship, methane as gas by pipeline
and as liquid by ship, ammonia as liquid by ship, and other fuels and materials
by ship. Further conversion of imported fuels is also possible once fuels have
arrived in Europe, e.g.~hydrogen can be used for the synthesis of carbon-based
fuels and methane can be converted to hydrogen.

But generally, limited demands for each carrier in model

conversion losses make it less likely to import high-value hydrogen derivatives only to reconvert (e.g. methanol as mere transport carrier)

Moreover, we compute scenarios where only a subset of carriers can be imported
to probe the flatness of the near-optimal solution space and assess how the
different import scenarios affect the energy infrastructure inside Europe.

\begin{figure*} 
    \begin{subfigure}[t]{\textwidth}
        \caption{global perspective}
        \label{fig:options:global}
        \includegraphics[width=\textwidth]{20231025-zecm/graphics/import_world_map.pdf}
    \end{subfigure}
    \begin{subfigure}[t]{\textwidth}
        \caption{European perspective}
        \label{fig:options:europe}
        \centering
        \includegraphics[width=\textwidth]{20231025-zecm/graphics/import_options_s_110_lvopt__Co2L0-2190SEG-T-H-B-I-S-A-onwind+p0.5-imp_2050.pdf}
    \end{subfigure}
    \caption{\textbf{Overview of considered import options.}
        \textit{Panel A} shows the regional differences in the cost to deliver
        green methanol to Europe (choropleth layer), the cost composition of
        different import vectors (bar charts), an illustration of the wind and
        solar availability in Morocco, and an illustration of the land
        eligibility analysis for wind turbine development in the region of
        Buenos Aires in Argentina. \textit{Panel B} depicts potential entry
        points for energy imports into Europe like the location of existing and
        planned LNG terminals and gas pipeline entry points, the costs of
        hydrogen imports in different European regions (choropleth layer), the
        considered connections for long-distance HVDC import links from the MENA
        region, Kazakhstan, Turkey and Ukraine, and the distribution and range
        of import costs for different energy carriers and entry points with
        indications for selected countries of origin (violin charts). }
    \label{fig:options}
\end{figure*}



% methodology of TRACE and PyPSA-Eur

To model the European energy system, we use the open energy system optimisation
model PyPSA-Eur,\cite{PyPSAEurSecSectorCoupled} that combines a fully
sector-coupled approach with high spatial and temporal resolution and detailed
transmission infrastructure representation. The model co-optimises the
investment and operation of generation, storage, conversion and transmission
infrastructures in a single linear optimisation problem. It covers 110
individual regions and uses a 4-hourly equivalent time resolution for a full
year. With these settings, the model is detailed enough to capture existing
gridr bottlenecks and the variability of renewables and requirements for
seasonal storage. The model includes regional demands from the electricity,
industry, buildings, agriculture and transport sectors, including international
shipping and aviation as well as non-energy feedstock demands in the chemicals
industry. Furthermore, the model covers transmission infrastructure for
electricity, gas and hydrogen as well as candidate entry points for energy
imports like existing and prospective LNG terminals and cross-continental
pipelines. 

In the scenarios shown below we enforce net-zero emissions for carbon dioxide,
allow the reinforcement today's power grid infrastructure, take technology and
assumptions for the year 2030\cite{dea2019}, and limit the carbon sequestration
potential to 200~Mt$_{CO_2}$/a. This is sufficient to sequester unabated fossil
process emissions in industry, but limits the system's reliance on carbon
removal technologies. Detailed carbon management is central to the model, as it
tracks the carbon cycles between industrial process emissions, the combustion of
biomass and gas, carbon capture, direct air capture, synfuel production,
sequestration, recycling and waste-to-energy conversion. 

Fuel and material import costs seen by our energy system model are based on recent research
by Hampp et al.\cite{hamppImportOptions2021}, who assessed the cost importing
energy across different global green energy supply chains for the afromentioned
energy carriers from various regions of the world. Hampp et
al.\cite{hamppImportOptions2021} developed regional supply cost curves based on
the countries' renewable resources and land availability. To determine the
levelised cost of energy for exports, any domestic electricity demand was
assumed to be supplied with the countries' cheapest potentials. Other than
domestic electrofuel synthesis in Europe, which could use captured CO$_2$ from
point sources, direct air capture is assumed as the sole carbon source of import
fuels. For hydrogen derivatives, the cheapest suppliers are Argentina and Chile.

- low cost differences for carbonaceous fuels between considered exporting countries
- many new LNG import terminals in Europe in response to phasing out Russian gas supply in 2022

We use these supply curves to determine for each energy carrier and model entry
point the region-specific lowest import cost, thus, incoporating the potential
trade-off between import cost and import location. Our selection of exporting
countries comprises Australia, Argentina, Chile, Kazakhstan, Namibia, Turkey,
Ukraine, the Eastern United States and Canada, mainland China, and the MENA
region. Electricity imports are endogenously optimised, meaning that the
capacities and operation of wind and solar generation as well as storage in the
respective exporting countries and the HVDC transmission lines to Europe (with
losses) are co-planned with the rest of the system. Hydrogen and methane can be
imported where there are existing or planned LNG terminals or pipeline
entry-points (excluding connections through Russia). The resulting hydrogen
import costs are much lower in Southern and Eastern Europe where it can be
imported via pipeline rather than by ship. The imports of ammonia, liquid
hydrocarbons and steel are not spatially resolved, assuming they can be
transported within Europe at low cost.

For more details, see \nameref{sec:methods} section.