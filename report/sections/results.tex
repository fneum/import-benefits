
\section*{Cost assessment of energy and material import vectors}

%%% brief methodology TRACE %%%

Green fuel and steel import costs seen by the model are based on an extension of
recent research by Hampp et al.,\cite{hamppImportOptions2023} who assessed the
levelised cost of energy exports for different green energy and material supply
chains in various world regions (\cref{fig:options:global}). Our selection of
exporting countries comprises Australia, Argentina, Chile, Kazakhstan, Namibia,
Turkey, Ukraine, the Eastern United States and Canada, mainland China, and the
MENA region. Regional supply cost curves for these countries are developed based
on renewable resources, land availability and prioritised domestic demand.
Unlike domestic electrofuel synthesis in Europe, which could use captured CO$_2$
from point sources, direct air capture is assumed to be the only carbon source
of imported fuels. Concepts involving the shipment of captured CO$_2$
from Europe to exporting countries for carbonaceous fuel synthesis are not
considered.\cite{treeenergysolutionsGreenCycle2024,fonderSyntheticMethaneClosing2024}

%%% brief methodology TRACE-PyPSA-Eur scoupling %%%

We use these supply curves to determine the region-specific lowest import cost
for each carrier, thus incorporating the potential trade-off between import cost
and import location (\cref{fig:options:europe}). For hydrogen derivatives, the
lowest-cost suppliers are Argentina and Chile for all entry points into Europe.
Electricity imports are endogenously optimised, meaning that the capacities and
operation of wind and solar generation as well as storage in the respective
exporting countries and the HVDC transmission lines, are co-planned with the
rest of the system. Hydrogen and methane can be imported where there are
existing or planned LNG terminals or pipeline entry-points (excluding
connections through Russia). This results in lower hydrogen import costs, where
it can be imported by pipeline. Ammonia, carbonaceous fuels and steel are not
spatially resolved in the model, assuming they can be transported within Europe
at negligible additional cost.

\begin{figure*}
    \includegraphics[width=\textwidth]{20231025-zecm/sensitivity-bars.pdf}
    \caption{\textbf{Potential for cost reductions with reduced sets of import options.}
        Subsets of available import options are sorted by ascending cost
        reduction potential. Top panel shows profile of total cost savings.
        Bottom panel shows composition and extent of imports in relation to
        total energy system costs. Percentage numbers in bar plot indicate the
        share of total system costs spent on domestic energy infrastructure.
        Alternative versions of this figure with higher and lower import cost
        assumptions are included in the supplementary material. }
    \label{fig:sensitivity-bars}
\end{figure*}

\section*{Cost savings for fuel and material import combinations}

In \cref{fig:sensitivity-bars}, we first explore the cost reduction potential of
various energy and material import options. In the absence of energy imports,
total energy system costs add up to \bneuro{815}\footnote{All currency values
are given in \euro{}$_{2020}$.}. By enabling imports from outside of Europe and
considering all import vectors, we find a potential reduction of total energy
system costs by up to \bneuro{39}. This corresponds to a relative reduction of
4.9\%. For cost-optimal imports, around 71\% of these costs are used to develop
domestic energy infrastructure. The remaining 29\% are spent on importing a
volume of 52~Mt of green steel and around 2700 TWh of green energy, which is
almost a quarter of the system's total energy supply (\cref{fig:import-shares}).

Next, we investigate the impact of restricting the available import options to
subsets of import vectors. We find that if only hydrogen can be imported, cost
savings are reduced to \bneuro{22} (2.8\%). This is because by using hydrogen as
an intermediary carrier, low-cost renewable electricity from abroad can still be
leveraged for the synthesis of derivative products in Europe. For this purpose,
pipeline-based hydrogen imports are preferred to ship-based imports as liquid.
When direct hydrogen imports are excluded from the available import options,
cost savings are similar with \bneuro{24} (3\%). Focusing imports exclusively on
liquid carbonaceous fuels derived from hydrogen, i.e.~methanol or Fischer-Tropsch
fuels, consistently achieves cost savings of \bneuro{13-20} (1.7-2.5\%).

On the contrary, restricting options to only ammonia or methane imports yields
negligible cost savings. Small savings below \bneuro{5} (0.6\%) can be reached
if only electricity or steel can be imported. This is due to the lower volume
and variety of usage options for ammonia, methane and steel compared to
hydrogen, methanol and Fischer-Tropsch fuels. Furthermore, the direct import of
electricity poses more challenges for system integration. Generally, our results
indicate a preference for methanol, hydrogen and steel imports over electricity
imports, with a mix emerging as the most cost-effective approach.
\sfigref{fig:si:subsets} show additional insights into how varying import costs
affect these findings.

\begin{figure}%[!htb]
    \includegraphics[width=\columnwidth]{20231025-zecm/graphics/import_shares/s_110_lvopt__Co2L0-2190SEG-T-H-B-I-S-A-onwind+p0.5-imp_2050.pdf}
    \caption{\textbf{Shares of imports and domestic production by carrier and optimised import carrier mix for import scenario with flexible carrier choice.}
        The figure also shows total supply for each carrier. Import shares for
        further import scenarios are included in the supplementary material.
        Steel is included in energy terms applying 2.1 kWh/kg as released by the
        oxidation of iron. }
    \label{fig:import-shares}
\end{figure}

\begin{figure*}
    \centering
    \footnotesize
    (a) no imports allowed \\
    \includegraphics[width=\textwidth]{20231025-zecm/market-values-noimp.pdf} \\
    (b) only hydrogen imports allowed \\
    \includegraphics[width=\textwidth, trim=0cm 0cm 0cm 1.5cm,
    clip]{20231025-zecm/market-values-imp+H2.pdf} \\
    (c) all imports allowed \\
    \includegraphics[width=\textwidth, trim=0cm 0cm 0cm 1.5cm, clip]{20231025-zecm/market-values-imp.pdf}
    \caption{\textbf{Comparison of domestic synthetic production costs and import costs for varying import scenarios.}
        The three panels (a), (b), and (c) refer to different import scenarios.
        In each panel, the \textit{bar charts} show the production-weighted
        average costs of domestic production of steel, hydrogen and its
        derivatives split into its cost and revenue components. These have been
        computed using the marginal prices of the respective inputs and outputs
        for the production volume of each region and snapshot. Capital
        expenditures are distributed to hours in proportion to the production
        volume. Missing bars indicate that no domestic production occured in the
        scenario, e.g.~for the case of methane where all demand is met by
        biogenic and fossil methane and no synthetic production occured
        (cf.~energy balances in supplementary material). All hydrogen is
        produced from electrolysis; i.e.~the model did not choose to produce
        hydrogen via steam methane reforming with or without carbon capture. For
        each bar, the yellow errorbars show the range of time-averaged domestic
        production costs across all regions. The black error bars show the range
        of import costs across all regions. The \textit{maps} relate the
        hydrogen production volume to the weighted cost of domestic hydrogen
        production (left colorbar). The \textit{time series} indicate the
        variance of the domestic production cost over time for hydrogen and
        Fischer-Tropsch fuel (FTF) including the regional spectrum as shaded
        area.}
    \label{fig:market-values}
\end{figure*}

\section*{Import dynamics for different energy carriers}

%%% state results %%%

\cref{fig:import-shares} outlines which carriers are imported in which
quantities in relation to their total supply under default assumptions when the
vector and volume can be flexibly chosen (``all imports allowed'' in
\cref{fig:sensitivity-bars}). In energy terms, cost-optimal imports comprise
around 50\% hydrogen, more than 20\% electricity, and around 20\% of
carbonaceous fuels. Noticeably, all crude steel and methanol for shipping and
industry is imported. Also, around three-quarters of the total hydrogen supply
is imported. Hydrogen is imported so that it can be processed into derivative
products domestically rather than direct applications for hydrogen. Smaller
import shares are observed for electricity, Fischer-Tropsch fuels, and ammonia,
which are mostly domestically produced.

%%% explanation through market values %%%

To explain the import shares in \cref{fig:import-shares} in more detail, we
compare import costs with average domestic production cost split by cost and
revenue components in \cref{fig:market-values}. First, for the scenario without
imports, imported fuel appear to be substantially cheaper than domestic
production. The high demand for hydrogen and derivative products
(\sfigref{fig:si:balances-a,fig:si:balances-b}) means that the most attractive
domestic potentials for renewable electricity and carbon dioxide have been
exhausted. Power from wind and solar needs to be produced in regions with worse
capacity factors and direct air capture becomes the price-setting technology for
\ce{CO2} as biogenic and industrial sources ($\approx$600~Mt$_\text{\ce{CO2}}$)
are depleted.

%%% hydrogen imports lower pressure on domestic supply chain %%%

Part of this gap is closed when hydrogen imports are allowed. By sourcing
cheaper hydrogen from outside Europe, the domestic costs of derivative fuel
synthesis are reduced. This hybrid approach has the largest effect on
Fischer-Tropsch production due to its higher hydrogen demand compared to
methanolisation and the Haber-Bosch process. Hydrogen imports also decouple the
synthesis from the seasonal variation of domestic hydrogen production costs.

%%% waste heat integration %%%

The potential for waste heat utilisation from fuel synthesis within Europe adds
further appeal to this hybrid approach. By importing hydrogen rather than the
derivative product, heat supply into district heating networks from synthesis
processes can create an additional revenue stream of up to 10 \euro{}/MWh,
depending on the process. Taking ammonia as example, the levelised cost is 73
\euro{}/MWh for domestic production compared to 88 \euro{}/MWh for imported
ammonia.

The waste heat integration is also the reason why in \cref{fig:import-shares},
with all import vectors allowed, all methanol is imported, whereas
Fischer-Tropsch fuels and ammonia are produced mainly domestically using high
shares of imported hydrogen. Because the thermal discharge from the methanol
synthesis is primarily used for the distillation of the methanol-water output
mix, its waste heat potential is considered much lower compared to
Fischer-Tropsch, Haber-Bosch and Sabatier processes. Therefore, it is less
attractive to retain this part of the value chain within Europe.

%%% explanations of low cost differences if all imports allowed %%%

With all import vectors allowed, we see minimal cost differences between
domestic production and imports as the supply curves reach equilibrium. This is
because imports of hydrogen and derivative products lower the strain on the
domestic supply chain. Thereby, domestic production would only be ramped up
where it competes with imports and associated infrastructure costs. Such regions
exist in Southern Europe or the British Isles and, therefore, not all hydrogen
is imported (\sfigref{fig:si:cost-supply-curves}).

\begin{figure*}
    \begin{subfigure}[t]{\columnwidth}
        \caption{cost reductions applied to all carriers but electricity}
        \label{fig:sensitivity-costs:A}
        \includegraphics[width=\columnwidth]{20231025-zecm/sensitivity-bars-all.pdf}
    \end{subfigure}
    \begin{subfigure}[t]{\columnwidth}
        \caption{cost reductions only applied to carbonaceous fuels}
        \label{fig:sensitivity-costs:B}
        \includegraphics[width=\columnwidth]{20231025-zecm/sensitivity-bars-all-C.pdf}
    \end{subfigure}
    \caption{\textbf{Effect of import cost variations on cost savings and import shares with all vectors allowed.}
    In panel (a), indicated relative cost changes are applied uniformly to all
    vectors but electricity imports. In panel (b), cost changes are only applied
    to carbonaceous fuels (methane, methanol and Fischer-Tropsch). Top subpanels
    show potential cost savings compared to the scenario without imports. Bottom
    subpanels show the share and composition of different import vectors in
    relation to total energy system costs. The information is shown both in
    absolute terms and relative terms compared to the scenario without imports.
    }
    \label{fig:sensitivity-costs}
\end{figure*}

\section*{Sensitivity of potential cost savings to import costs}

It should be noted, however, that the cost-optimal import mix also strongly
depends on the assumed import costs. This uncertainty is addressed in
\cref{fig:sensitivity-costs}. \cref{fig:sensitivity-costs:A} highlights the
extensive range in potential cost reductions if higher or lower import costs
could be attained and underlines the resulting variance in cost-effective import
mixes. Within $\pm 20\%$ of the default import costs applied to all carriers but
electricity, total cost savings vary between \bneuro{13} (1.6\%) and \bneuro{83}
(10.2\%). Within this range, import volumes vary between 1700 and 3800~TWh.
Across most scenarios, there is a stable role for methanol, steel, electricity
and hydrogen imports. One significant difference, however, are Fischer-Tropsch
fuel imports starting from cost reductions of 10\% and their absence at cost
increases beyond 10\%.

A breakdown of potential causes for such cost variations such as cost of
capital, cost of carbon dioxide and investment costs are presented in
\stabref{tab:cost-uncertainty}. Some of these only affect the cost of
carbonaceous fuels. One central assumption for carbon-based fuels is that
imported fuels rely exclusively on direct air capture (DAC) as a carbon source.
Arguments for this assumption relate to the potential remoteness of the ideal
locations for renewable fuel production or the absence of industrial point
sources in the exporting country. On the other hand, domestic electrofuels can
mostly use less expensive captured carbon dioxide from industrial point sources
or biogenic origin. Therefore, the higher cost for DAC partially cancels out the
savings from utilising better renewable resources abroad. This is one of the
reasons why there is substantial power-to-X production in Europe, even with
corresponding import options. The availability of cheaper biogenic CO$_2$ in the
exporting country, for instance, would make importing carbonaceous fuels more
attractive (\stabref{tab:cost-uncertainty}).

When the relative cost variation of $\pm 20\%$ is only applied to carbon-based
fuels (\cref{fig:sensitivity-costs:B}), hydrogen imports are increasingly
displaced by methane and Fischer-Tropsch imports with falling costs. However, it
takes a cost increase of more than 10\% for domestic methanol production to
become more cost-effective than methanol imports. This underlines the robust
benefit of importing methanol.

% FN import volumes in cost optimum +20%: 1705 TWh +10%: 2391 TWh +-0%: 2800 TWh
% -10%: 3121 TWh -20%: 3412 TWh -30%: 3766 TWh

\begin{figure*}
    \includegraphics[width=\textwidth]{20231025-zecm/sensitivity-import-volume-any.pdf}
    \caption{\textbf{Sensitivity of import volume on total system cost and composition.}
        The dashed line splits total system cost into domestic and foreign cost.
        Dotted lines represent import cost variations, indicating the respective
        altered profile of total system cost for given prescribed import
        volumes. The black markers denote the maximum cost reductions and
        cost-optimal import volume for a given import cost level (extreme points
        of the profiles). Steel is included in energy terms applying 2.1 kWh/kg
        as released by the oxidation of iron. Cost alterations are uniformly
        applied to all imports opotions but direct electricity imports. }
    \label{fig:sensitivity-volume}
\end{figure*}

\section*{Attainable cost savings for varying import volumes}

What is consistent across all import cost variations is the flat solution space
around the respective cost-optimal import volumes. Increasing or decreasing the
total amount of imports barely affects system costs within $\pm 1000$ TWh. This
is illustrated in \cref{fig:sensitivity-volume}, which shows the system cost as
a function of enforced import volumes and different import costs. A wide range
scenarios with import volumes below 5600~TWh (4000~TWh with +20\% and 7500~TWh
with -20\% import costs) have lower total energy system costs than the scenario
without any imports. These import volumes are roughly twice the cost-optimal
import volumes, which are indicated by the black markers in
\cref{fig:sensitivity-volume} and correspond to the bars previously shown in
\cref{fig:sensitivity-costs:A}. Naturally, the cost-optimal volume of imports
increases as their costs decrease, but the response weakens with lower import
costs.

As we explore the effect of increasing import volumes on system costs, we find
that already 43\% (36-61\% within $\pm$20\% import costs) of the 4.9\%
(1.6-10.2\%) total cost benefit (\bneuro{17}) can be achieved with the first 500
TWh of imports. This corresponds to only 18\% (15-29\%) of the cost-optimal
import volumes, highlighting the diminishing return of large amounts of energy
imports in Europe. While importing 1000 TWh already realises 70\% of the maximum
cost savings with our default assumptions, this maximum is only obtained for
2800~TWh of imports. For these initial 1000~TWh, primary crude steel and
methanol imports are prioritised, followed by hydrogen and, subsequently,
electricity beyond 2000 TWh. Once more than 5000~TWh are imported, less than
half the total system cost would be spent on domestic energy infrastructure.

As imports increase, there is a corresponding decrease in the need for domestic
power-to-X (PtX) production and renewable capacities. A large share of the
hydrogen, methanol and raw steel production is outsourced from Europe, reducing
the need for domestic wind and solar capacities. This trend is further
characterised by the displacement of biogas usage in favour of hydrogen imports
around the 2000~TWh mark as demand for domestic \ce{CO2} utilisation drops. An
increase in the amount of hydrogen imported coincides with an increasing use of
hydrogen fuel cells for electricity and central heat supply in district heating
networks, partially displacing the use of methane. Regarding electricity imports
from the MENA region, \cref{fig:sensitivity-volume} reveals a mix of wind and
solar power to establish favourable feed-in profiles for the European system
integration and higher utilisation rates for the long-distance HVDC links with a
capacity-weighted average of 72\%. Utilisation rates are high because a
considerable share of the import costs of electricity can be attributed to power
transmission.

% FN LCOE for solar and wind in MENA are mostly similar potentials are not
% constraining DZ: solar 34 €/MWh, wind 32 €/MWh TN: solar 30 €/MWh, wind 38
% €/MWh LY: solar 24 €/MWh, wind 37 €/MWh

As import costs are varied, the composition of the domestic system and import
mix is primarily similar (\sfigref{fig:si:volume}). The main differences are a
more prominent role for Fischer-Tropsch fuel imports with lower import costs and
green methane for high import volumes. It should also be noted that the windows
for cost savings are much smaller if only subsets of import options are
available (\sfigref{fig:si:volume-subsets}). However, up to an import volume of
2000~TWh, excluding electricity imports would not diminish the cost-saving
potential substantially.

\begin{figure*}
    \includegraphics[width=\textwidth]{20231025-zecm/infrastructure-map-2x3-A.pdf}
    \caption{\textbf{Layout of European energy infrastructure for different import scenarios.}
        Left column shows the regional electricity supply mix (pies), added HVDC
        and HVAC transmission capacity (lines), and the siting of battery
        storage (choropleth). Right column shows the hydrogen supply (top half
        of pies) and consumption (bottom half of pies), net flow and direction
        of hydrogen in newly built pipelines (lines), and the siting of hydrogen
        storage subject to geological potentials (choropleth). Total volumes of
        transmission expansion are given in TWkm, which is the sum product of
        the capacity and length of individual connections. The half circle in
        the Bay of Biscay indicates the imports of carriers that are not
        spatially resolved: ammonia, steel, methanol, Fischer-Tropsch fuels.
        Hydrogen imports are shown at the entry points. Hydrogen imports in
        Bulgaria and Romania originate from Algeria and Egypt. Maps for more
        scenarios are included in the supplementary material. }
    \label{fig:import-infrastructure}
\end{figure*}

\section*{Interactions of import strategy \& domestic infrastructure}

Across the range of import scenarios analysed, we find that the decision which
import vectors are used strongly affects domestic energy infrastructure needs
(\cref{fig:import-infrastructure}).

%%% self-sufficiency %%%

In the fully self-sufficient European energy supply scenario, we see large
\mbox{power-to-X} production within Europe to cover the demand for hydrogen and
hydrogen derivatives in steelmaking, high-value chemicals, green shipping and
aviation fuels. Production sites are concentrated in Southern Europe for
solar-based electrolysis and the broader North Sea region for wind-based
electrolysis. The steel and ammonia industries relocate to the periphery of
Europe in Spain and Scotland, where hydrogen is cheap and abundant. Electricity
grid reinforcements are focused in Northwestern Europe and long-distance HVDC
connections but are broadly distributed overall. Hydrogen pipeline build-out is
strongest in Spain and France to transport hydrogen from the Southern production
hubs to fuel synthesis sites. Most of these pipelines are used unidirectionally,
with bidirectional usage where pipelines link hydrogen production and low-cost
geological storage sites (for instance, between Greece and Italy and Southern
Spain).

%%% flexible imports %%%

Considering imports of renewable electricity, green hydrogen, and electrofuels
substantially alters the infrastructure buildout in Europe. Imports displace
much of the European power-to-X production capacities and, particularly,
domestic solar energy generation in Southern Europe. In contrast, the British
Isles retain some domestic electrolyser capacities to produce synthetic methane
locally, also leveraging the Sabatier process's waste heat. The electricity
imports are distributed evenly between the North African countries Algeria,
Libya, and Tunisia and across multiple entry points in Spain, France, Italy and
Greece. This facilitates grid integration without strong reinforcement needs.
Electricity imports are also optimised to achieve higher utilisation rates above
70\% for the HVDC import connections. This is realised by mixing wind and solar
generation for seasonal balancing and using some batteries for short-term
storage (\sfigref{fig:si:import-operation}).

While the amount and locations of domestic power grid reinforcements are not
significantly affected by the import of electricity and other fuels, the extent
of the hydrogen network is halved and its routing is significantly altered.
Compared to the self-sufficiency scenario, the cost-benefit of the hydrogen
network shrinks from \bneuro{11} (1.3\%) to \bneuro{3} (0.4\%). This is caused
by substantial amounts of methanol imports that diminish the demand for hydrogen
in Europe and, hence, the need to transport it. In combination with the steel
and ammonia industry relocation, longer hydrogen pipeline connections are then
predominantly built to meet hydrogen CHP demands to bring electricity and heat
to renewables-poor and grid-poor regions in Eastern Europe and Germany.
Moreover, the hydrogen network helps absorb inbound hydrogen in South and
Southeast Europe, transporting some hydrogen, which is not directly used for
fuel synthesis at the entry points, to neighbouring regions.

%%% overarching trends %%%

A further observation is the high value of power-to-X production for system
integration and the role of waste heat in the siting of fuel synthesis plants
(\sfigref{fig:si:infra-b}). Using the process waste heat in district heating
networks with seasonal thermal storage generates notable cost savings of
\bneuro{11-21} (1.3-2.6\%). Consequently, savings are lower when imports
displace domestic PtX infrastructure. To realise these benefits, PtX facilities
tend to be located in densely populated areas (e.g.~Paris or Hamburg), which
drives part of the the hydrogen network. Notably, because of the waste heat
produced in Fischer-Tropsch and Sabatier plants, these tend to locate where
space heating demand is high. This is not the case for methanolisation plants,
which have lower waste heat potential. Alongside the flexible operation of
electrolysis to integrate variable wind and solar feed-in and the broad
availability of industrial and biogenic carbon sources in Europe, waste heat
usage is a key factor that makes electricity and hydrogen imports with
subsequent domestic conversion more attractive relative to the direct import of
derivative products. Infrastructure layouts for further import scenarios are
presented in \sfigref{fig:si:infra-b} to \sfigref{fig:si:infra-d}.
