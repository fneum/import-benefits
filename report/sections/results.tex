
\section*{Results}

\subsection*{Assessment of energy and material import unit costs}

%%% brief methodology TRACE %%%


Green fuel and steel import costs seen by the model are based on an extension of
recent research by Hampp et al.,\cite{hamppImportOptions2023} who assessed the
levelised cost of energy exports for different green energy and material supply
chains from various world regions to Europe. Our selection of exporting regions
comprises all 53 coloured or dotted regions in \cref{fig:options:global}. In the
TRACE optimisation model,\cite{hamppImportOptions2023} regional wind and solar
potentials are assessed based on prevailing weather conditions and land
availability while prioritising projected domestic demand. In combination with
the techno-economic modelling of the various fuel-specific supply chains stages
(\sfigref{fig:si:import-esc-scheme}), the lowest levelised supply cost for each
carrier, exporter and importer are determined for a volume of 500~TWh~a$^{-1}$ (or
100~Mt~a$^{-1}$ of steel/HBI), thus incorporating the trade-off between import cost and
import location (\cref{fig:options:europe}). Unlike domestic electrofuel
synthesis in Europe, which could use captured CO$_2$ from point sources, direct
air capture is assumed to be the only carbon source of imported fuels. Concepts
involving the shipment of captured CO$_2$ from Europe to exporting regions for
carbonaceous fuel synthesis or permanent sequestration of CO$_2$ captured by
direct air capture abroad are not
considered.\cite{treeenergysolutionsGreenCycle2024,fonderSyntheticMethaneClosing2024}

%%% brief methodology TRACE-PyPSA-Eur coupling %%%

The import costs are then included as supply options in the PyPSA-Eur model.
Hydrogen and methane can be imported where there are LNG terminals in operation
or under construction, or where pipeline entry-points exist. Due to higher
volatility, electricity imports are endogenously optimised, meaning that the
capacities and operation of wind and solar generation as well as storage in the
respective exporting regions and the HVDC transmission lines, are co-planned
with the rest of the  European system. Ammonia, carbonaceous fuels, and ferrous
materials are not spatially resolved in the model, assuming they can be
transported within Europe at negligible cost. Thus, their specific import
location is undetermined.  An import limit of 500~TWh per region for the sum of
all exports is imposed to prevent over-reliance on single exporters.

For imports of hydrogen by pipeline, North African regions offer the lowest cost
(ca.~74-88 \euro{}~MWh$^{-1}$, \sfigref{fig:si:worlmap-h2}). Importing hydrogen by ship is
substantially more expensive due to liquefaction and evaporation losses, with a
cost difference of 18\% between each vector's lowest cost supplier
(\sfigref{fig:si:isc-h2}). For hydrogen derivatives, Argentina and Chile offer
additional potential for low-cost imports, for instance, 125-132~\euro{}~MWh$^{-1}$ for
Fischer-Tropsch fuels or 548-566~\euro{}~t$^{-1}$ for steel. These values are similar
to those achieved in the Maghreb region. Further notable cases include Australia
and Canada. Methanol is slightly cheaper than the Fischer-Tropsch route because
it is assumed to be more flexible with a 20\% minimum part load compared to 50\%
for Fischer-Tropsch synthesis.\cite{brownUltralongdurationEnergyStorage2023} The
lower process flexibility shifts the energy mix towards solar electricity and
causes higher levels of curtailment and battery storage, increasing costs
(\sfigref{fig:si:isc-meoh-ft}). The transport costs of \ce{CH4(l)} are lower than
for \ce{H2(l)} since the liquefaction consumes less energy and individual ships
can carry more energy with \ce{CH4(l)}. Pipeline imports of \ce{CH4(g)} were
also considered, but costs were higher than for \ce{CH4(l)} shipping under the
assumption that new pipelines would have to be built or renewed.

\begin{figure}
    \fullwidthfigure{
    \includegraphics[width=\linewidth]{../workflow/notebooks/20240826-z1/sensitivity-bars.pdf}
    \caption{\textbf{Potential for cost reductions with reduced sets of import options.}
        Subsets of available import options are sorted by ascending cost
        reduction potential. Top panel shows profile of total system cost
        savings. Bottom panel shows composition and extent of imports in
        relation to total energy system costs. Percentage numbers in bar plot
        indicate the share of total system costs spent on domestic energy
        infrastructure. Alternative scenarios of this figure with higher and
        lower import cost assumptions are shown in Supplementary Figs.~12 and
        13. }
    \label{fig:sensitivity-bars}
    }
\end{figure}

\subsection*{Cost savings for fuel and material import combinations}

In \cref{fig:sensitivity-bars}, we first explore the cost reduction potential of
various energy and material import options. In the absence of energy imports,
total energy system costs add up to \bneuro{836}. By enabling imports from
outside of Europe and considering all import vectors, we find a potential
reduction of total energy system costs by up to \bneuro{37}. This corresponds to
a relative reduction of 4.4\%. For cost-optimal imports, around 77\% of these
costs are used to develop domestic energy infrastructure. The remaining 23\% are
spent on importing a volume of 50~Mt of green steel and around 1498 TWh of green
energy, which is around 13\% of the system's total energy supply
(\cref{fig:import-shares}). Our results show a cost-effective import mix
consisting primarily of liquid carbon-based fuels, hydrogen and steel imports
with small volumes of ammonia and electricity imports.

Next, we investigate the impact of restricting the available import options to
subsets of import vectors. We find that if only hydrogen can be imported, cost
savings are reduced to \bneuro{20} (2.4\%), with pipeline-based hydrogen imports
being preferred to imports as liquid by ship. By importing a larger volume of
hydrogen as an intermediary carrier (1338~TWh instead of 576~TWh,
\sfigref{fig:si:import-shares-a}), low-cost renewable energy from abroad can
still be leveraged for the synthesis of derivative products in Europe but the
benefit is reduced as domestic CO$_2$ feedstocks from industrial sources are
depleted.

Conversely, when direct hydrogen imports are excluded from the available import
options, cost savings are close to the maximum with \bneuro{34} (4.1\%). This
indicates that the benefit of using domestic captured biogenic or fossil CO$_2$
is similar in magnitude to tapping into low-cost renewable resources abroad.
Focusing imports exclusively on liquid carbonaceous fuels derived from hydrogen,
i.e.~methanol or Fischer-Tropsch fuels, still achieves high cost savings of
\bneuro{31} (3.7\%), which is due to the smaller demand or variety of
applications for ammonia, methane and steel compared to liquid carbonaceous
fuels. Thus, excluding them has a small effect on cost savings, which aligns
with the finding that restricting options to only methane, ammonia or ferrous
material imports yields negligible to small cost savings below \bneuro{5}
(0.6\%). Negligible cost savings were also found for the direct import of
electricity as it poses more challenges for integration into the European
system.

Overall, while varying import costs within $\pm 20\%$ affects the magnitude of
attainable cost savings, the relative impact of restricting certain import
options remains largely consistent
(\sfigref{fig:si:subsets-higher,fig:si:subsets-lower}).

\begin{figure}
    \fullwidthfigure{
    \includegraphics[width=0.58\linewidth]{../workflow/pypsa-eur/results/20240826-z1/graphics/import_shares/s_115_lvopt__imp_2050.pdf}
    \includegraphics[width=0.42\linewidth]{../workflow/pypsa-eur/results/20240826-z1/graphics/import_sankey/s_115_lvopt__imp_2050.pdf}
    \caption{\textbf{Shares of imports and domestic production by carrier and optimised import carrier mix for import scenario with flexible carrier choice.}
        The figure also shows total supply for each carrier. Import shares for
        further import scenarios are included in the supplementary material.
        Steel is included in energy terms applying 2.1 kWh~kg$^{-1}$ as released by the
        oxidation of iron. }
    \label{fig:import-shares}
    }
\end{figure}

\begin{figure}
    \fullwidthfigure{
    \centering
    \footnotesize
    (a) no imports allowed \\
    \includegraphics[width=\linewidth]{../workflow/notebooks/20240826-z1/market-values-noimp.pdf} \\
    (b) only hydrogen imports allowed \\
    \includegraphics[width=\linewidth, trim=0cm 0cm 0cm 0cm,clip]{../workflow/notebooks/20240826-z1/market-values-imp+H2.pdf} \\
    (c) all imports allowed \\
    \includegraphics[width=\linewidth, trim=0cm 0cm 0cm 0cm, clip]{../workflow/notebooks/20240826-z1/market-values-imp.pdf}
    \caption{\textbf{Comparison of domestic synthetic production costs and import costs for varying import scenarios.}
        The three panels (a), (b), and (c) refer to different import scenarios.
        In each panel, the \textit{bar charts} show the production-weighted
        average costs of domestic production of steel, hydrogen and its
        derivatives split into its cost and revenue components. These have been
        computed using the marginal prices of the respective inputs and outputs
        for the production volume of each region and snapshot. Capital
        expenditures are distributed to hours in proportion to the production
        volume. Missing bars indicate that no domestic production occured in the
        scenario, e.g.~for the case of methane where all demand is met by
        biogenic and fossil methane and no synthetic production occured
        (cf.~energy balances in supplementary material). All hydrogen is
        produced from electrolysis; i.e.~the model did not choose to produce
        hydrogen via steam methane reforming with or without carbon capture. For
        each bar, the yellow errorbars show the range of time-averaged domestic
        production costs across all regions. The black error bars show the range
        of import costs across all regions. The \textit{maps} on the right
        relate the hydrogen production volume to the weighted cost of domestic
        hydrogen production (left colorbar). Confer Supplementary Fig.~21 for
        information on the domestic cost supply curves.}
    \label{fig:market-values}
    }
\end{figure}

\subsection*{Import dynamics for different energy carriers}

%%% state results %%%

\cref{fig:import-shares} outlines which carriers are imported in which
quantities in relation to their total supply under default assumptions when the
vector and volume can be flexibly chosen (``all imports allowed'' in
\cref{fig:sensitivity-bars}). In energy terms, cost-optimal imports comprise
around 45\% carbonaceous fuels, 35\% hydrogen, and less than 10\% electricity.
Noticeably, all primary crude steel and ammonia for fertilizers is imported,
whereby imports of steel are preferred over HBI. Around half of the total
hydrogen supply is imported, matching the ratio of the 2030 REPowerEU targets.
Hydrogen is imported so that it can be processed into derivative products
domestically rather than used for direct applications for hydrogen. Smaller
import shares are observed for electricity, which is largely supplied from
domestic resources, and for methane, which is supplied from domestic fossil and
biogenic sources (\sfigref{fig:si:balances-a}).

%%% trade flows from origin to destination %%%

In terms of trade flows (\cref{fig:import-shares}), we observe carbonaceous fuel
imports by ship from South America -- leveraging low transport costs of dense
liquid fuels -- as well as ammonia, steel and hydrogen imports from the Maghreb
region. Hydrogen is mostly received by pipeline in Spain. Moreover, due to its
proximity to Italy, some electricity imports are received by HVDC connections
from Tunisia. While the model suggests these specific trade routes, it should be
noted that alternative origin-destination pairs could yield similar results
(\sfigref{fig:si:isc-h2,fig:si:isc-ch4-nh3,fig:si:isc-meoh-ft,fig:si:isc-hbi-St}).

%%% explanation through market values %%%

To explain the import shares in \cref{fig:import-shares} in more detail, we
compare import costs with average domestic production cost split by cost and
revenue components in \cref{fig:market-values}. First, for the scenario without
imports, imported fuels appear to be substantially cheaper than domestic
production. The high demand for hydrogen derivatives (\sfigref{fig:si:demands})
means that the most attractive domestic potentials for renewable electricity and
captured carbon dioxide have been exhausted. Consequently, power from wind and
solar needs to be produced in regions with worse capacity factors.

%%% hydrogen imports lower pressure on domestic supply chain %%%

Part of this gap is closed when hydrogen imports are allowed. By sourcing
cheaper hydrogen from outside Europe, the domestic costs of derivative fuel
synthesis are reduced. However, the large remaining volume of CO$_2$ handled in
the European system for use and sequestration (\sfigref{fig:si:balances-b})
means that direct air capture is still the price-setting technology for \ce{CO2}
as economic applications for biogenic and industrial carbon capture (i.e.~those
with high full load hours) are depleted.

%%% explanations of low cost differences if all imports allowed %%%

With all import vectors allowed, we see minimal cost differences between
domestic production and imports as the supply curves reach equilibrium
(\sfigref{fig:si:cost-supply-curves}). This is because imports of hydrogen and
derivative products lower the strain on the domestic supply curves for hydrogen
and carbon dioxide. Thereby, domestic production would only ramp up where it
competes with imports and associated infrastructure costs. This was the case for
hydrogen, methanol and Fischer-Tropsch fuels in the British Isles as well as
parts of Southern Europe and Nordic countries (\cref{fig:market-values}).
Consequently, not all hydrogen is imported but some domestic production is
retained.

\begin{figure}
    \fullwidthfigure{
    \begin{subfigure}[t]{0.49\linewidth}
        \caption{cost variations applied to all carriers}
        \label{fig:sensitivity-costs:A}
        \includegraphics[width=\linewidth]{../workflow/notebooks/20240826-z1/sensitivity-bars-all-wAC.pdf}
    \end{subfigure}
    \begin{subfigure}[t]{0.49\linewidth}
        \caption{cost variations applied to all carriers but electricity}
        \label{fig:sensitivity-costs:B}
        \includegraphics[width=\linewidth]{../workflow/notebooks/20240826-z1/sensitivity-bars-all.pdf}
    \end{subfigure}
    \begin{subfigure}[t]{0.49\linewidth}
        \caption{cost variations only applied to carbonaceous fuels}
        \label{fig:sensitivity-costs:C}
        \includegraphics[width=\linewidth]{../workflow/notebooks/20240826-z1/sensitivity-bars-all-C.pdf}
    \end{subfigure}
    \caption{\textbf{Effect of import cost variations on cost savings and import shares with all vectors allowed.}
    In panel (a), indicated relative cost changes are applied uniformly to all
    vectors. In panel (b), cost changes are applied uniformly to all vectors but
    electricity imports. In panel (c), cost changes are only applied to
    carbonaceous fuels (methane, methanol and Fischer-Tropsch). Top subpanels
    show potential cost savings compared to the scenario without imports. Bottom
    subpanels show the share and composition of different import vectors in
    relation to total energy system costs. The information is shown both in
    absolute terms and relative terms compared to the scenario without imports.
    }
    \label{fig:sensitivity-costs}
    }
\end{figure}

\subsection*{Sensitivity of potential cost savings to import costs}

Thus far, the presented findings originate from a central estimate for the
import cost. However, the cost-optimal import mix strongly depends on the
assumed import costs. This uncertainty is addressed in
\cref{fig:sensitivity-costs}. \cref{fig:sensitivity-costs:A} highlights the
extensive range in potential cost reductions if higher or lower import costs
could be attained and underlines the resulting variance in cost-effective import
mixes. Within $\pm 30\%$ of the default import costs applied to all carriers,
total cost savings vary between \bneuro{2} (0.3\%) and \bneuro{112} (13.5\%).
Within this range, import volumes vary between 500 and 2646~TWh. Across most
scenarios, there is a stable role for ammonia and liquid carbonaceous fuel
imports. Within a narrower $\pm 20\%$ range, hydrogen imports also appear in
larger quantities, while steel imports become less attractive with cost
increases of 10\% or more. Electricity imports grow with declining costs.

However, not all carriers are equally affected by technology cost variations.
Fuel synthesis technologies do not influence electricity imports and only
carbon-based fuels are subject to the cost of CO$_2$ supply. We find that when
the relative cost variation is not applied to electricity imports
(\cref{fig:sensitivity-costs:B}), they remain less attractive than other vectors, even when those alternative vectors face a 20\% cost rise.

One central assumption regarding costs for carbon-based fuels is that imported
fuels rely on direct air capture (DAC) as a carbon source. Arguments for this
assumption relate to the potential remoteness of the ideal locations for
renewable fuel production or the absence of industrial point sources in the
exporting region. In contrast, domestic electrofuels can mostly use less
expensive captured biogenic or fossil carbon dioxide from industrial processes.
Therefore, the higher cost for DAC partially cancels out the savings from
utilising better renewable resources abroad. This is one of the reasons why
there is substantial power-to-X production in Europe, even with corresponding
import options. However, the availability of cheaper (biogenic) CO$_2$ in
exporting regions, would lower costs of carbonaceous fuel imports
(\stabref{tab:cost-uncertainty}).

When the relative cost variation is only applied to carbon-based fuels
(\cref{fig:sensitivity-costs:C}), reflecting cost uncertainty in carbon
provision, hydrogen imports are quickly displaced by Fischer-Tropsch and
methanol imports with falling costs. Only when costs rise by 20\% do
domestically produced liquid hydrocarbons -- derived mainly from imported
hydrogen -- become more cost-effective than direct imports. In all three cases
of import cost variations, methane imports become relevant only with strong cost
reductions of 40\%, replacing biogas and residual fossil gas consumption.

\begin{figure}
    \fullwidthfigure{
    \includegraphics[width=\linewidth]{../workflow/notebooks/20240826-z1/sensitivity-import-volume-any.pdf}
    \caption{\textbf{Sensitivity of import volume on total system cost and composition.}
        \textit{Left panel:} Solid lines show the total system cost as a
        function of enforced import volumes for higher (brown scale) or lower
        (blue scale) import costs. The dashed lines indicate the corresponding
        shares of the domestic system cost. The red markers denote the maximum
        cost reductions and cost-optimal import volume for given import cost
        levels (extreme points of the curves). The cost alterations are
        uniformly applied to all imports options but direct electricity imports.
        Steel is included in energy terms applying 2.1 kWh~kg$^{-1}$ as released by the
        oxidation of iron. \textit{Right panel:} Shows the composition of the
        total system cost as a function of enforced import volumes for the
        central import cost estimate. The dashes line splits the system costs
        into costs for imports and the domestic system. Cost compositions for
        the alternative import cost scenarios are presented in Supplementary
        Figs.~14-17.}
    \label{fig:sensitivity-volume}
    }
\end{figure}

\subsection*{Attainable cost savings for varying import volumes}

What is consistent for many scenarios with higher or lower import costs is the
flat solution space around the respective cost-optimal import volumes.
Increasing or decreasing the total amount of imports from the optimum barely
affects system costs within $\pm 1000$~TWh. This is illustrated in
\cref{fig:sensitivity-volume} and extended
\sfigref{fig:si:volume-higher-1,fig:si:volume-higher-2,fig:si:volume-lower-1,fig:si:volume-lower-2},
which show the system cost as a function of enforced import volumes and
different import costs for hydrogen and its derivatives. A wide range scenarios
with import volumes below 4100~TWh (2300~TWh for 20\% higher import costs,
5800~TWh for -20\% lower import costs) have lower total energy system costs than
the no-imports scenario. These ranges of import values are two to three times as
large as the corresponding cost-optimal import volumes, which are indicated by
the red markers in \cref{fig:sensitivity-volume} and correspond to the bars
previously shown in \cref{fig:sensitivity-costs:B}. Naturally, the cost-optimal
volume of imports increases as their costs decrease, but with noticably varying
slopes for system cost savings per unit of additional imported energy.

As we explore the effect of increasing import volumes on system costs, we find
that already 56\% (48-80\% within $\pm$20\% import costs) of the 4.4\%
(1.3-9.0\%) total cost benefit can be achieved with the first 500 TWh of
imports. This corresponds to 31\% (25-49\%) of the cost-optimal import volumes,
highlighting the diminishing return of large amounts of energy imports in
Europe. The initial 1000 TWh realise 90\% (80-100\%) of the highest cost
savings, for which primary crude steel and liquid carbonaceous fuel imports are
prioritised, followed by ammonia and hydrogen and, subsequently, larger volumes
of electricity beyond cost-optimal import levels. Once more than 5500~TWh
(5000-8200~TWh) are imported, less than half the total system cost would be
spent on domestic energy infrastructure.

As imports increase, there is a corresponding decrease in the need for domestic
power-to-X (PtX) production and renewable capacities. A large share of the
hydrogen, methanol and primary steel production is outsourced from Europe,
reducing the need for domestic wind and solar capacities. This trend is further
characterised by the displacement of biogas usage in favour of hydrogen imports
around the 4000~TWh mark (3000-5000~TWh within $\pm$20\% import costs) as demand
for domestic \ce{CO2} utilisation drops and methane use for power and heat
provision is displaced by hydrogen. The increase in hydrogen imports results in
the build-out of more hydrogen fuel cell CHPs for power and heat supply in
district heating networks. Regarding electricity imports from the MENA region,
\cref{fig:sensitivity-volume} reveals a mix of wind and solar power with some
batteries to establish favourable feed-in profiles for the European system
integration and higher utilisation rates for the long-distance HVDC links. For
instance, for imports of 4000~TWh in \cref{fig:sensitivity-volume}, the
capacity-weighted average utilisation rate was 85\%. This is because a
considerable share of the import costs of electricity can be attributed to power
transmission.

As import costs are varied, the composition of the domestic system and import
mix for different import volumes is primarily similar
(\sfigref{fig:si:volume-higher-2,fig:si:volume-higher-1,fig:si:volume-lower-1,fig:si:volume-lower-2}).
The main difference is a less prominent role for steel imports with higher
import costs. What is furthermore noteworthy is that reducing import costs from
-30\% to -50\% only marginally reduces domestic infrastructure costs, indicating
largely saturated import potentials. In terms of available import options, the
windows for cost savings are more limited if only subsets are available
(\sfigref{fig:si:volume-subsets}). However, up to an import volume of 1500~TWh
for the central cost estimate, excluding electricity imports or constraining
imports to methanol and Fischer-Tropsch fuels only would not diminish the
cost-saving potential substantially.

\begin{figure}
    \fullwidthfigure{
    \includegraphics[width=\linewidth]{../workflow/notebooks/20240826-z1/infrastructure-map-2x3-A.pdf}
    \caption{\textbf{Layout of European energy infrastructure for different import scenarios.}
        Left column shows the regional electricity supply mix (pies), added HVDC
        and HVAC transmission capacity (lines), and the siting of battery
        storage (choropleth). Right column shows the hydrogen supply (top half
        of pies) and consumption (bottom half of pies), net flow and direction
        of hydrogen in newly built pipelines (lines), and the siting of hydrogen
        storage subject to geological potentials (choropleth). Total volumes of
        transmission expansion are given in TWkm, which is the sum product of
        the capacity and length of individual connections. The half circle in
        the Bay of Biscay labelled ``Any in Europe'' indicates the imports of
        carriers that are not spatially resolved: ammonia, steel, HBI, methanol,
        Fischer-Tropsch fuels. Hydrogen imports are shown at the entry points.
        Maps for more scenarios are included in Supplementary Figs.~25 to 27. }
    \label{fig:import-infrastructure}
    }
\end{figure}

\subsection*{Interactions of import strategy \& domestic infrastructure}

Across the range of import scenarios analysed, we find that the decision which
import vectors are used strongly affects domestic energy infrastructure needs
(\cref{fig:import-infrastructure}).

%%% self-sufficiency %%%

In the fully self-sufficient European energy supply scenario, we see large PtX
production within Europe to cover the demand for hydrogen and hydrogen
derivatives in steelmaking, fertilizers, high-value chemicals, green shipping
and aviation fuels. Production sites are concentrated mainly in and around the
North Sea and Baltic Sea using wind-based electrolysis plus some additional hubs
in Southern Europe using solar-based electrolysis. Electricity grid
reinforcements, representing around 50\% of the current transmission capacity,
are focused in Northwestern Europe, with numerous long-distance HVDC
connections, but are broadly distributed overall.

%%% industry relocation and hydrogen network %%%

With a total of 57~TWkm, the hydrogen pipeline build-out is smaller, mostly
serving regional connections. For several reasons, it is also considerably
smaller than the 204-306~TWkm observed previously in Neumann et
al.~\cite{neumannPotentialRoleHydrogen2023} or the European Hydrogen Backbone
reports\cite{gasforclimateEuropeanHydrogen2022} which envisioned a similar order
of magnitude. Besides assumed full electrification of heavy-duty road
transport\cite{} and assuming low CO$_2$ transport costs from point sources to
low-cost hydrogen sites,\cite{hofmannH2CO2Network2024} one reason is the
considered relocation of steel and ammonia production to where hydrogen is cheap
and abundant, reducing the need to transport hydrogen
(\sfigref{fig:si:relocation}). Not considering industry relocation would result
in a slightly larger hydrogen network of 69~TWkm (\sfigref{fig:si:infra-b}),
while increasing system costs by \bneuro{22} (2.7\%) in the no-imports scenario.
With relocation permitted, primary steel production shifts to the British Isles
and Spain while ammonia production moves to the Nordic-Baltic region. Both
sectors become more strongly localized, with individual regions capturing a
market share surpassing 30\%.

%%% flexible imports %%%

Considering imports of renewable electricity, green hydrogen, and electrofuels
substantially alters the magnitude of energy infrastructure in Europe. Imports
displace much of the European power-to-X production capacities and,
particularly, domestic solar energy generation in Southern Europe. Much of the
remaining derivative fuel synthesis in Southern Spain uses imported hydrogen
assuming the delivery of captured CO$_2$ from other parts of Europe at low
cost.\cite{hofmannH2CO2Network2024} In contrast, the British Isles retain some
domestic electrolyser capacities to produce synthetic fuels locally. Electricity
imports of 131~TWh, compared to total imports of 1609~TWh, mainly enter from
Tunisia at multiple nodes in Mallorca, Corsica, Sardinia, Sicily and mainland
Italy. This distribution facilitates grid integration without strong
reinforcement needs in the Italian peninsula.

While the broad regions of domestic power grid reinforcements are not
significantly affected by the import of electricity and other fuels, the volume
of power grid expansion reduces by 20\%. The reduction in network infrastructure
is even more pronounced with the hydrogen network; the hydrogen network size is
reduced by 70\% with many of the North and East European connections omitted.
Compared to the self-sufficiency scenario, the cost-benefit of the hydrogen
network shrinks from \bneuro{3} (0.4\%) to less than \bneuro{1} (0.1\%). This is
caused by substantial amounts of hydrogen derivative imports or direct
processing of imported hydrogen at the entry points, which diminishes the demand
for hydrogen in Europe and, hence, the need to transport it. With further 10\%
cheaper carbonaceous fuel imports, the hydrogen network would then shrink to
9~TWkm (\sfigref{fig:si:infra-c}).

%%% backup power and power-to-X flexibility value %%%

Changes in the magnitude of domestic PtX production also affect Europe's backup
capacity needs. Next to energy storage and demand-side management of electric
vehicles and heat pumps, the operational flexibility of electrolysers and
derivative fuel production yields significant benefits for integrating variable
wind and solar feed-in and reduces reserve capacity requriements. In a
theoretical scenario without imports where all PtX processes must run inflexibly
at full capacity, system costs rise by 8.8\%. Between scenarios with and without
imports, we observe that as some flexible domestic power-to-X is displaced by
imported fuels, domestic thermal backup capacities increase from
129~GW$_\text{el.}$ (no imports allowed) to 276~GW$_\text{el.}$ (all imports
allowed).  Instead of curtailing the domestic production of electrofuels, backup
power plants need to be dispatched. Most of these power plants are gas CHPs
providing backup heat alongside backup power when electricity prices are high
during winter months (\sfigref{fig:si:backup-power}). Spatially, these are
distributed across Central Europe, while batteries provide backup power in
Southern Europe (\sfigref{fig:si:backup-power-map}). The model leverages
Europe's extensive power grid to widely distribute centralised backup power,
even though in reality individual nations may prefer maintaining their own
reserve capacities.

%%% waste heat usage potential %%%

A further observation is the high potential value of power-to-X waste heat and
its role siting fuel synthesis plants (\sfigref{fig:si:infra-b},
\sfigref{fig:si:infra-d}). Alongside the flexible operation of electrolysis to
integrate variable wind and solar feed-in and the broad availability of
industrial and biogenic carbon sources in Europe, waste heat usage in district
heating networks is a potential revenue stream that could make electricity and
hydrogen imports with subsequent domestic conversion more attractive relative to
the direct import of derivative products. Our default assumption that only 25\%
of the waste heat can be utilised stems from potential challenges in co-locating
PtX plants with district heating networks within the 115 model regions. If all
waste heat could be leveraged, notable system cost savings of \bneuro{20}
(2.4\%) could be achieved in the no-imports scenario compared to a scenario
where waste heat is fully vented.  To realise these benefits, Fischer-Tropsch
and Haber-Bosch plants tend to be located where space heating demand is high
(e.g.~Paris or Hamburg) (\sfigref{fig:si:infra-b}), which increases hydrogen
network build-out to 98~TWkm (+72\%) compared to the reference scenario with
25\% waste heat utilisation. This is not the case for methanolisation plants,
which have lower waste heat potential.

\subsection*{Causes of import cost variations and their effect}


\begin{table}
    \fullwidthfigure{
    \footnotesize
    \centering
    \begin{tabular}{lrrrr}
        \toprule
        Cost factor & Absolute change & Unit & Relative change & Unit\\
        \midrule
        Higher WACC of 12\% abroad (e.g.~high project risk) & +48.8 & \euro{}~MWh$^{-1}$  &
        +38.0 & \% \\
        Higher WACC of 10\% abroad (e.g.~high project risk) & +28.6 & \euro{}~MWh$^{-1}$  &
        +22.3 & \% \\
        Higher WACC of 8\% abroad (e.g.~high project risk) & +9.2 & \euro{}~MWh$^{-1}$  & +7.2
        & \% \\
        Higher direct air capture investment cost abroad (+200\%) & +44.3 & \euro{}~MWh$^{-1}$
        & +34.5 & \% \\
        Higher direct air capture investment cost abroad (+100\%) & +22.3 & \euro{}~MWh$^{-1}$
        & +17.4 & \% \\
        Higher direct air capture investment cost abroad (+50\%) & +11.2 & \euro{}~MWh$^{-1}$
        & +8.7 & \% \\
        Higher electrolysis investment cost abroad (+50\%) & +17.3 & \euro{}~MWh$^{-1}$  &
        +13.5 & \% \\
        % Argentina and Chile not available for export & +10.1 & \euro{}~MWh$^{-1}$  &
        % +9.2 & \% \\
        \midrule
        Lower WACC of 3\% abroad (e.g.~government guarantees) & -33.4 & \euro{}~MWh$^{-1}$  &
        -26.0 & \% \\
        Lower WACC of 5\% abroad (e.g.~government guarantees) & -17.5 & \euro{}~MWh$^{-1}$  &
        -13.6 & \% \\
        Lower WACC of 6\% abroad (e.g.~government guarantees) & -8.9 & \euro{}~MWh$^{-1}$  &
        -6.9 & \% \\
        Lower electrolysis investment cost abroad (-50\%) & -18.4 & \euro{}~MWh$^{-1}$  &
        -14.3 & \% \\
        Sell excess curtailed electricity at 50 \euro{}~MWh$^{-1}$ abroad & -8.3 & \euro{}~MWh$^{-1}$  &
        -6.5 & \% \\
        Sell excess curtailed electricity at 30 \euro{}~MWh$^{-1}$ abroad & -4.6 & \euro{}~MWh$^{-1}$  &
        -3.6 & \% \\
        Sell excess curtailed electricity at 10 \euro{}~MWh$^{-1}$ abroad & -1.5 & \euro{}~MWh$^{-1}$  &
        -1.2 & \% \\
        Buy available biogenic or cycled \ce{CO2} for 50 \euro{}~t$^{-1}$ abroad & -20.1 &
        \euro{}~MWh$^{-1}$  & -15.6 & \% \\
        Buy available biogenic or cycled \ce{CO2} for 75 \euro{}~t$^{-1}$ abroad & -13.7 &
        \euro{}~MWh$^{-1}$  & -10.7 & \% \\
        Buy available biogenic or cycled \ce{CO2} for 100 \euro{}~t$^{-1}$ abroad & -7.2 &
        \euro{}~MWh$^{-1}$  & -5.6 & \% \\
        Availability of geological hydrogen storage at 2.1 \euro{}/kWh
        (reduction by 95.5\%) & -5.1
        & \euro{}~MWh$^{-1}$  & -4.0 & \% \\
        Sell power-to-X waste heat at 10 \euro{}~MWh$^{-1}$ abroad  &
        -7.8 & \euro{}~MWh$^{-1}$  & -6.1 & \% \\
        Sell power-to-X waste heat at 5 \euro{}~MWh$^{-1}$ abroad &
        -6.9 & \euro{}~MWh$^{-1}$  & -5.4 & \% \\
        Highly flexible operation of Fischer-Tropsch synthesis (20\% minimum
        part-load) & -3.6 & \euro{}~MWh$^{-1}$  & -2.8 & \% \\
        \bottomrule
    \end{tabular}
    \caption{\textbf{Examples for potential import cost increases or decreases.}
    The table presents cost sensitivities in absolute and relative terms based
    on the supply chain for producing Fischer-Tropsch fuels in Southern
    Argentina for export to Europe (Portugal). The reference fuel import cost
    for this case is 128.5~\euro{}~MWh$^{-1}$.}
    \label{tab:cost-uncertainty}
    }
\end{table}

In \cref{tab:cost-uncertainty}, we present a breakdown of some potential causes
for import cost variations compared to domestic supply chains relating to
technology costs, financing costs, excess power and heat revenues, fuel
synthesis flexibility, and the availability of geological hydrogen storage and
alternative sources of CO$_2$.

For example, we show that a higher weighted average cost of capital (WACC) than
the uniformly applied 7\%, e.g.~due to higher project financing risks, and lower
WACC, e.g.~due to the government-backing of projects, strongly affect import
costs.\cite{calcaterraReducingCostCapital2024} An increase or decrease by just
one percentage point already alters the unit costs by around $\pm$7\%. Likewise,
technology cost variations abroad for electrolysers and DAC units have a strong
influence. Biogenic CO$_2$ -- or fossil CO$_2$ from industrial processes that is
largely cycled between use and synthesis and, hence, not emitted to the
atmosphere -- can reduce the levelised fuel cost by 16\% if it can be provided
for 50~\euro{}~t$^{-1}$.

By default, we assume islanded fuel synthesis sites which causes high
curtailment rates of 8\%. If surplus electricity production could be sold and
absorbed by the local power grid in exporting regions, additional cost
reductions could be achieved. Furthermore, process integration with waste heat
usage and flexible operation can also reduce fuel cost by 3-6\%. Import costs
are also reduced by 4\% where geological hydrogen storage is available, by
reducing the need for flexible \mbox{power-to-X} operation.

In contrast, cost rises when the most competitive exporting regions would not
offer to export green energy would be limited. Within a cost premium of 10\% in
relation to the lowest cost eporting region, Chile, ten other regions could
step in if these regions were unavailable for exports
(\sfigref{fig:si:isc-meoh-ft}).
