
% TODO catchy section titles

\begin{figure*}
    \includegraphics[width=\textwidth]{20231025-zecm/sensitivity-bars.pdf}
    \caption{\textbf{Potential for cost reductions with reduced sets of import options.}
        Subsets of available import options are sorted by ascending cost reduction potential. 
        Top panel shows profile of total cost savings.
        Bottom panel shows composition and extent of imports in relation to total energy system costs.
        Percentage numbers in bar plot indicate the share of total system costs spent on domestic energy infrastructure.
        Alternative versions of this figure with higher and lower import cost assumptions are included in the supplementary material.
    }
    \label{fig:sensitivity-bars}
\end{figure*}

\section*{Combinations and volumes of fuel and material imports for highest cost savings}

In \cref{fig:sensitivity-bars}, we first explore the cost reduction potential of
various energy and material import options. Starting from total energy system
costs of \bneuro{768} in the absence of energy imports, we find that enabling
imports from outside of Europe and considering all import vectors can reduce
total energy system costs by up to \bneuro{37}, which corresponds to a reduction
by 4.9\%. In this case, around 71\% of these costs are used to develop domestic
energy infrastructure. The remaining 29\% are spent on importing a volume of
52~Mt of green steel and around 2700 TWh of green energy, which is almost a
quarter of the system's total energy supply.

A more granular inspection of the results indicates varied cost savings when
only subsets of import options are available. The maximum savings are reduced to
\bneuro{22} (3\%) if hydrogen is excluded from the available import options.
This value is similar to the savings of \bneuro{21} (2.8\%) achieved when only
hydrogen can be imported. Here, pipeline-based hydrogen imports are preferred to
imports as liquid fuel by ship. Focusing an import strategy exclusively on
liquid carbonaceous fuels derived from hydrogen, like methanol or
Fischer-Tropsch fuels, consistently achieves cost savings of \bneuro{13-19}
(1.7-2.5\%).

On the contrary, restricting options to only ammonia or methane imports yields
negligible cost savings and small savings below \bneuro{4} (0.6\%) if only
electricity or steel can be imported. This can be explained by the lower variety
and volume of potential usage options compared to energy carriers higher
upstream in the value chain. Generally, our results indicate a preference for
chemicals, like methanol and hydrogen, and materials over electricity imports,
with a strategic mix of imports emerging as the most cost-effective approach.
For insights into how varying import costs affect these findings, refer to
\sfigref{fig:si:subsets}. 

\begin{figure}[!htb]
    \includegraphics[width=\columnwidth]{20231025-zecm/graphics/import_shares/s_110_lvopt__Co2L0-2190SEG-T-H-B-I-S-A-onwind+p0.5-imp_2050.pdf}
    \caption{\textbf{Shares of imports and domestic production by carrier and optimised import carrier mix for import scenario with flexible carrier choice.} 
        Figure also shows total supply for each carrier.
        Import shares for further import scenarios are included in the supplementary material.
        Steel is included in energy terms applying 2.1 kWh/kg as released by the oxidation of iron.
    }
    \label{fig:import-shares}
\end{figure}

\section*{Varying roles of imports for different energy carriers}

\cref{fig:import-shares} outlines which carriers are imported in which
quantities in relation to their total supply when the vector and volume can be
flexibly chosen. Significantly, all methanol, which is used in shipping and
industry, as well as all crude steel, is imported. Also, around three-quarters
of the total hydrogen supply is imported. This is mainly done to subsequently
process the hydrogen into derivative products domestically rather than direct
applications for hydrogen. Smaller import shares are observed for electricity,
Fischer-Tropsch fuels and ammonia, which are mostly domestically produced. For
the example of ammonia, this can be explained by lower levelised cost for
domestic production of 69 \euro{}/MWh compared to 83 \euro{}/MWh for imported
ammonia, which can be attributed to the additional revenue from waste heat
streams in the domestic supply chain. In energy terms, cost-optimal imports
comprise around 50\% hydrogen, more than 20\% electricity and an equal amount of
carbonaceous fuels. It should be noted, however, that different carriers have
undergone further conversion steps abroad with energy losses and that the
cost-optimal import mix also depends on the assumed import costs.


\begin{figure*}
    \begin{subfigure}[t]{\columnwidth}
        \caption{cost reductions applied to all carriers but electricity}
        \label{fig:sensitivity-costs:A}
        \includegraphics[width=\columnwidth]{20231025-zecm/sensitivity-bars-all.pdf}
    \end{subfigure}
    \begin{subfigure}[t]{\columnwidth}
        \caption{cost reductions only applied to carbonaceous fuels}
        \label{fig:sensitivity-costs:B}
        % \includegraphics[trim=0cm 0cm 0cm 3.4cm, clip, width=\columnwidth]{20231025-zecm/sensitivity-bars-all-C.pdf}
        \includegraphics[width=\columnwidth]{20231025-zecm/sensitivity-bars-all-C.pdf}
    \end{subfigure}
    \caption{\textbf{Effect of import cost variations on cost savings and import shares.}
    In panel (a), indicated relative cost changes are applied uniformly to all vectors but electricity imports.
    In panel (b), cost changes are only applied to carbonaceous fuels (methane, methanol and Fischer-Tropsch).
    Top subpanels show potential cost savings compared to the scenario with full self-sufficiency.
    Bottom subpanels show the share and composition of different import vectors in relation to total energy system costs.
    }
    \label{fig:sensitivity-costs}
\end{figure*}

\section*{Sensitivity of potential cost savings to unit costs of energy imports}

This uncertainty of estimates for import costs is addressed in
\cref{fig:sensitivity-costs}. \cref{fig:sensitivity-costs:A} highlights the
extensive range in potential cost reductions if higher or lower import costs
could be attained as well as the resulting variance in cost-effective import
blends. Within $\pm 20\%$ of the default import costs previously presented
applied to all carriers but electricity, total cost savings vary between
\bneuro{12} (1.6\%) and \bneuro{78} (10.2\%) with import volumes ranging between
1700 and 3800~TWh. Across most of these scenarios, there is a stable role for
methanol, steel, electricity and hydrogen imports. One significant difference is
the appearance of Fischer-Tropsch fuel imports starting from cost reductions of
10\%.

Some potential causes for such cost variations are presented in
\cref{tab:cost-uncertainty} and discussed later. Some of these only affect the
cost of carbonaceous fuels. One central assumption for carbon-based fuels is
that imported fuels rely exclusively on direct air capture (DAC) as a carbon
source. For example, arguments for this assumption relate to the potential
remoteness of the ideal locations for renewable fuel production or the absence
of industrial point sources in the exporting country. On the other hand,
domestic electrofuels can mostly use less expensive captured carbon dioxide from
industrial point sources or biogenic origin. Therefore, the higher cost for DAC
partially cancels out the savings from utilising better renewable resources
abroad, which is one of the reasons why there is substantial power-to-X
production in Europe, even with corresponding import options. The availability
of cheaper biogenic CO$_2$ in the exporting country, for instance, would
inevitably make importing carbonaceous fuels more attractive.

When the relative cost variation of $\pm 20\%$ is only applied to carbon-based
fuels (\cref{fig:sensitivity-costs:B}), hydrogen imports are increasingly
displaced by green methane and Fischer-Tropsch imports with lower costs, but it
takes a cost increase of more than 10\% for domestic methanol production to
become more cost-effective than green methanol imports. 

% import volumes in cost optimum
% +20%: 1705 TWh
% +10%: 2391 TWh
% +-0%: 2800 TWh
% -10%: 3121 TWh
% -20%: 3412 TWh
% -30%: 3766 TWh

\begin{figure*}
    \includegraphics[width=\textwidth]{20231025-zecm/sensitivity-import-volume-any.pdf}
    \caption{\textbf{Sensitivity of import volume on total system cost and composition.} 
        Dashed line splits total system cost into domestic and foreign cost.
        Dotted lines indicate the profile of lowest total system cost attainable for given import volumes and different levels of import costs.
        Markers denote the maximum cost reductions and cost-optimal import volume for a given import cost level (extreme points of the profiles).
        Steel is included in energy terms applying 2.1 kWh/kg as released by the oxidation of iron.
        Cost alterations are uniformly applied to all carriers but electricity.
    }
    \label{fig:sensitivity-volume}
\end{figure*}

\section*{Attainable cost savings for varying import volumes}

What is consistent across all import cost variations is the flat solution space around the respective cost-optimal import volumes. Increasing or decreasing the total amount of imports barely affects system costs within $\pm 1000$ TWh. This is illustrated in \cref{fig:sensitivity-volume}, which shows the possible cost reductions as a function of import volumes. A wide range of import volumes below 5600~TWh (4000-7500~TWh within $\pm$20\% import costs) have lower costs than the scenario without any imports. These values are equivalent to just more than twice the cost-optimal volumes, which are indicated by the black markers and correspond to the bars previously shown in \cref{fig:sensitivity-costs:A}. Naturally, the cost-optimal volume of imports increases as their costs decrease, but the response weakens with lower import costs.

As we explore the effect of increasing import volumes on system costs, we find that already 43\% (36-61\% within $\pm$20\% import costs) of the 4.9\% (1.6-10.2\%) total cost benefit (\bneuro{16}) can be achieved with the first 500 TWh of imports. This corresponds to only 18\% (15-29\%) of the cost-optimal import volumes, highlighting the diminishing return of large amounts of energy imports over domestic production in Europe. While importing 1000 TWh already realises 70\% of the maximum cost savings with our default assumptions, this maximum is only obtained for 2800~TWh of imports. For these initial 1000~TWh, primary crude steel and methanol imports are prioritised, followed by hydrogen and, subsequently, electricity beyond 2000 TWh. Once more than 5000~TWh are imported, less than half the total system cost would be spent on domestic energy infrastructure.

As imports increase, there is a corresponding decrease in the need for domestic power-to-X (PtX) production and renewable capacities. A large share of the hydrogen, methanol and raw steel production is outsourced from Europe, reducing domestic wind and solar capacities. This trend is further characterised by the displacement of biogas usage in favour of hydrogen imports around the 2000~TWh mark. An increase in the amount of hydrogen imported coincides with an increasing use of hydrogen fuel cells for electricity and central heat supply in district heating networks, partially displacing the use of methane. Regarding electricity imports, \cref{fig:sensitivity-volume} reveals a predominant reliance on wind-generated electricity over solar for the imports via HVDC links from the MENA region. A considerable share of the costs can be attributed to realising long-distance power transmission.

% Furthermore, large amounts of electricity imports are supported by
% an increase in battery deployment to manage the imported power more effectively.

For deviating levels of import costs, the composition of the domestic system and import mix is primarily similar (\sfigref{fig:si:volume}). The main differences are a more prominent role for Fischer-Tropsch fuel imports with lower import costs and green methane for high import volumes. It should also be noted that the windows for cost savings are much smaller if only subsets of import options are considered (\sfigref{fig:si:volume-subsets}). However, up to an import volume of 2000~TWh, excluding electricity imports does not diminish the cost-saving potential substantially.

\begin{figure*}
    \includegraphics[width=\textwidth]{20231025-zecm/infrastructure-map-2x3-A.pdf}
    \caption{\textbf{Layout of European energy infrastructure for different import scenarios.}
        Left column shows the regional electricity supply mix (pies), added HVDC and HVAC transmission capacity (lines), and the siting of battery storage (choropleth).
        Right column shows the hydrogen supply (top half of pies) and consumption (bottom half of pies), net flow and direction of hydrogen in newly built pipelines (lines), and the siting of hydrogen storage subject to geological potentials (choropleth).
        Total volumes of transmission expansion are given in TWkm, which is the sum product of the capacity and length of individual connections.
    }
    \label{fig:import-infrastructure}
\end{figure*}

\section*{Interdependence of import strategy and domestic energy infrastructures}

Across the range of import scenarios analysed, we find that the strategy taken
on imports strongly affects domestic energy infrastructure needs
(\cref{fig:import-infrastructure}).

% self-sufficiency

In the fully self-sufficient European energy supply scenario, we see large
\mbox{power-to-X} production within Europe to cover the demand for hydrogen
derivatives in steelmaking, high-value chemicals, green shipping and aviation
fuels. Production sites are concentrated in Southern Europe for solar-based
electrolysis and the broader North Sea region for wind-based electrolysis. The
steel and ammonia industries relocate to the periphery of Europe in Spain and
Scotland, where hydrogen is cheap and abundant. Electricity grid reinforcements
are focused in Northwestern Europe and long-distance HVDC connections but are
broadly distributed overall. Hydrogen pipeline build-out is strongest in Spain
and France to transport hydrogen from the Southern production hubs to chosen
fuel synthesis sites. Most of these pipelines are used unidirectionally, with
bidirectional usage where pipelines link hydrogen production and low-cost
geological storage sites (for instance, between Greece and Italy and Southern
Spain).
% waste heat discussion comes later  

% flexible imports

Considering imports of renewable electricity, green hydrogen, and electrofuels substantially alters the infrastructure buildout in Europe. Imports displace much of the European power-to-X production capacities and, particularly, domestic solar energy generation in Southern Europe. In contrast, the British Isles retain some domestic electrolyser capacities to produce synthetic methane locally. The electricity imports are distributed evenly between the North African countries Algeria, Libya, and Tunisia and across multiple entry points in Spain, France, Italy and Greece. This facilitates grid integration without strong reinforcement needs. Electricity imports are also optimised to achieve higher utilisation rates above 60\% for the HVDC import connections. This is realised by mixing wind and solar generation for seasonal balancing and using some batteries for short-term storage (\sfigref{fig:si:import-operation}).

While the amount and locations of domestic power grid reinforcements are not significantly affected by the import of electricity and other fuels, the extent of the hydrogen network is halved. Compared to the self-sufficiency scenario, the cost-benefit of the hydrogen network shrinks from \bneuro{10} (1.3\%) to \bneuro{3} (0.4\%). This is caused by substantial amounts of methanol imports that diminish the demand for hydrogen in Europe and, hence, the need to transport it. In combination with the steel and ammonia industry relocation, longer hydrogen pipeline connections are then predominantly built to meet hydrogen CHP demands to bring electricity and heat to renewables-poor and grid-poor regions in Eastern Europe and Germany. Moreover, the hydrogen network helps absorb inbound hydrogen in South and Southeast Europe, transporting some hydrogen not directly used for fuel synthesis at the entry points to neighbouring regions. 

% overarching trends

A further observation is the high value of power-to-X production for system integration and the role of waste heat in the siting of fuel synthesis plants (\sfigref{fig:si:infra-b}). Using the process waste heat in district heating networks with seasonal thermal storage generates notable cost savings of \bneuro{10-20} (1.3-2.6\%), with lower savings when imports displace domestic PtX infrastructure. To realise these benefits, PtX facilities tend to be located in densely populated areas (e.g.~Paris or Hamburg), which drives part of the the hydrogen network. Alongside the flexible operation of electrolysis to integrate variable wind and solar feed-in and the broad availability of industrial and biogenic carbon sources in Europe, waste heat usage is a key factor that raises the attractiveness of electricity and hydrogen imports with subsequent domestic conversion relative to the direct import of derivative products. Infrastructure layouts for further import scenarios are presented in \sfigref{fig:si:infra-b} to \sfigref{fig:si:infra-d}.

\section*{Causes of import cost uncertainty and their severity}

In \cref{tab:cost-uncertainty}, we vary some of the techno-economic assumptions
for evaluating green fuel supply chains in the exporting countries to justify
the range of import cost deviations from the defaults. These relate to
technology costs, financing costs, excess power and heat handling, fuel
synthesis flexibility, and the availability of geological hydrogen storage and
alternative sources of CO$_2$. For all the following sensitivities, it should be
noted that they may not be additive owing to correlation.

A higher weighted average cost of capital (WACC) in developing or economically
unstable countries than the uniformly applied 7\%, e.g.~due to higher project
risks, and lower WACC, e.g.~due to the government-backing of a project, have a
substantial effect on the import cost calculations; an increase or decrease by
just one percentage point already alters the costs per unit of energy by more
than
7\%.\cite{egliBiasEnergy2019,bogdanovReplyBias2019,lonerganImprovingRepresentation2023a,schyskaHowRegional2020,steffenDeterminantsCost2022}

Likewise, a failure to achieve the anticipated cost reductions for electrolysers
and DAC systems would also result in far-reaching cost increases for green
energy imports, especially if the fuel contains carbon. The availability of
biogenic CO$_2$ (or CO$_2$ from industrial processes that is largely cycled
between use and synthesis and, hence, not emitted to the atmosphere) can reduce
the green fuel cost by 20\% if it can be provided for 60 \euro{}/t and by 10\%
if made available for 100 \euro{}/t.

The default assumptions for export supply chains assume islanded fuel synthesis
sites that are not connected to the local electricity system. The isolation in
consideration of the investment costs of the various components drives the
system into high curtailment rates of 8\%. If surplus electricity production
could be sold and absorbed by the local power grid, considerable cost reductions
could be achieved depending on the average selling price, reducing curtailment
(refer to \cref{tab:cost-uncertainty}). 

Besides integration with the local energy system, process integration using
waste heat streams from power-to-X plants for direct air capture and flexible
Fischer-Tropsch synthesis similar to methanolisation can also reduce fuel cost
by 3-5\% each. Conditions that would allow for geological hydrogen storage
reduce the need for flexible synthesis plant operation and could reduce import
costs by more than 7\% %. However, even though many countries considered to be
exporting possess geological hydrogen storage potential, the sites are not
always co-located with the countries' best renewable potentials.
\cite{hevinUndergroundStorage2019}

Finally, cost rises can also be expected if the most competitive exporting
countries are not offering to export green energy. Argentina and Chile have a
margin of 9.5 \euro{}/MWh over the next cheapest exporting country
(i.e.~Australia, Algeria and Libya with 113 \euro{}/MWh). If these countries
were unavailable for import, costs would rise by almost 10\%.

\begin{table*}
    \small
    \centering
    \begin{tabular}{lrrrr}
        \toprule
        Factor & Change & Unit & Change & Unit\\
        \midrule
        higher WACC of 12\% (e.g.~high project risk) & +40.6 & \euro{}/MWh  & +39.3 & \% \\
        higher WACC of 10\% (e.g.~high project risk) & +23.8 & \euro{}/MWh  & +23.0 & \% \\
        higher WACC of 8\% (e.g.~high project risk) & +7.7 & \euro{}/MWh  & +7.4 & \% \\
        higher direct air capture investment cost (+200\%) & +52.6 & \euro{}/MWh  & +50.8 & \% \\
        higher direct air capture investment cost (+100\%) & +26.5 & \euro{}/MWh  & +25.6 & \% \\
        higher direct air capture investment cost (+50\%) & +13.3 & \euro{}/MWh  & +12.9 & \% \\
        higher direct air capture investment cost (+25\%) & +6.7 & \euro{}/MWh  & +6.5 & \% \\
        higher electrolysis investment cost (+200\%) & +27.5 & \euro{}/MWh  & +26.6 & \% \\
        higher electrolysis investment cost (+100\%) & +15.7 & \euro{}/MWh  & +15.2 & \% \\
        higher electrolysis investment cost (+50\%) & +8.5 & \euro{}/MWh  & +8.2 & \% \\
        higher electrolysis investment cost (+25\%) & +4.4 & \euro{}/MWh  & +4.3 & \% \\
        Argentina and Chile not available for export & +9.5 & \euro{}/MWh  & +9.2 & \% \\
        % pay border adjustment tax of X \euro{}/t for methane leakage$^\star$ & +? & \euro{}/MWh  & +? & \% \\
        \midrule
        lower WACC of 3\% (e.g.~government guarantees) & -27.8 & \euro{}/MWh  & -26.8 & \% \\
        lower WACC of 5\% (e.g.~government guarantees) & -14.6 & \euro{}/MWh  & -14.1 & \% \\
        lower WACC of 6\% (e.g.~government guarantees) & -7.5 & \euro{}/MWh  & -7.2 & \% \\
        sell excess curtailed electricity at 40€/MWh & -23.3 & \euro{}/MWh  & -22.6 & \% \\
        sell excess curtailed electricity at 30€/MWh & -14.7 & \euro{}/MWh  & -14.2 & \% \\
        sell excess curtailed electricity at 20€/MWh & -7.5 & \euro{}/MWh  & -7.2 & \% \\
        % option to use available biogenic or cycled \ce{CO2} for 50€/t & -22.9 & \euro{}/MWh  & -22.2 & \% \\
        option to use available biogenic or cycled \ce{CO2} for 60€/t & -20.4 & \euro{}/MWh  & -19.7 & \% \\
        option to use available biogenic or cycled \ce{CO2} for 80€/t & -15.2 & \euro{}/MWh  & -14.7 & \% \\
        option to use available biogenic or cycled \ce{CO2} for 100€/t & -10.0 & \euro{}/MWh  & -9.7 & \% \\
        option to build geological hydrogen storage at 2.25 \euro{}/kWh (reduction by 95\%) & -7.7 & \euro{}/MWh  & -7.4 & \% \\
        option to use power-to-X waste heat streams for direct air capture & -3.6 & \euro{}/MWh  & -3.4 & \% \\
        highly flexible operation of fuel synthesis plant (20\% minimum part-load instead of 70\%) & -5.1 & \euro{}/MWh  & -4.9 & \% \\
        \bottomrule
    \end{tabular}
    \caption{\textbf{Examples for potential import cost increases or decreases.}
    The table presents cost sensitivities in absolute and relative terms based
    on the supply chain for producing Fischer-Tropsch fuels in Argentina for
    export to Europe. The reference fuel import cost for this case is 103.5
    \euro{}/MWh. Responses to changes in the input assumptions may not
    necessarily be additive.}
    \label{tab:cost-uncertainty}
\end{table*}
