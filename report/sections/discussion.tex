
Our analysis offers insights into how energy imports might interact with
European energy infrastructures and what economic benefit they can bring. Our
results show that imports of green energy reduce costs of a carbon-neutral
European energy system by 5\% (\bneuro{37}), noting however that the uncertainty
range is considerable and potential cost savings strongly depend on the
assumptions made.

Nevertheless, our results present a quantitative basis for further discussions
about the trade-offs between system cost, carbon neutrality, public acceptance, 
and energy security pertaining to the import of low-carbon fuels and assess the
extent to which infrastructure policy decisions depend on the path taken on
energy imports.

This is particularly relevant as other factors than pure costs might rather
drive the import strategy. Given the relatively limited cost benefit of imports
and value chain reordering, there could be a case against pursuing this avenue.
Geopolitical considerations could shift the focus towards energy sovereignty and
independence through self-sufficient supply, but also towards imports to
strengthen trade relations and to diversify energy supply chains. Low reliance
on imports from few countries and rigid infrastructures like pipelines could
mitigate the risks of sudden supply disruptions and enhance the resilience of
energy supply. An important consideration is also the economic outflow caused by
imports, shifting value creation to other parts to the world with potential
repercussions on local jobs. Furthermore, while we assessed the import and
relocation of steel production, we did not consdier the import of intermediate
products like sponge iron, which has undergone the most energy-intensive part of
the value chain and simultaneously presents a good that is transportable at low
cost.

There is also a social dimension to the import strategy and the question how
fast the aspired infrastructure can get built. Therefore, policymaking might
prefer easy-to-implement systems, featuring for instance lower domestic
infrastructure requirements, reuse of existing infrastructure, lower technology
ris, reduced land usage for broader public support than the most cost-effective
solution. Ultimately, Europe's energy strategy needs to balance cost savings
from green energy imports with broader concerns like geopolitics and public
opinion, ensuring a secure, swift and sustainable energy future.
