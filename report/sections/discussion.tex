Our analysis offers insights into how renewable energy imports might reduce
overall systems costs and interact with European energy infrastructure. Our
results show that imports of green energy reduce costs of a carbon-neutral
European energy system by \bneuro{37} (4.4\%), noting, however, that the
uncertainty range is considerable. While we find that some imports are robustly
beneficial, system cost savings range between 1\% and 10\% within a $\pm$20\%
variation of import costs. What is consistent, however, are the diminishing
return of energy imports for larger quantities, with peak cost savings below
imports of 3000~TWh/a (equivalent to 90~Mt H$_2$). We also find that there is
value in pursuing some \mbox{power-to-X} production in Europe as a source of
flexibility for wind and solar integration and as a potential source of waste
heat for district heating networks. Another location factor in favour of
European \mbox{power-to-X} is the  wide availability of sustainable biogenic and
industrial carbon sources, which helps to reduce reliance on more costly direct
air capture.

\todo[inline]{comparison to other studies}

Several limitations of the study should be noted. First, the optimization
results represent a long-term equilibrium that disregards potential
transition-related infrastructure lock-ins or mid-term ramp-up constraints of
export capacities or domestic infrastructure development. A further limitation
is that our purely cost-based analysis, which best reflects long-term bilateral
purchase agreements, neglects price impacts of intensifying global competition
for green fuel imports and exports. Besides unclear market developments, local
challenges in exporting regions such as public acceptance for export-oriented
energy projects\cite{ishmamMappingLocalGreen2024} and potential water
scarcity\cite{franzmannGreenHydrogenCostpotentials2023,terlouwFutureHydrogenEconomies2024} to produce large amounts of
hydrogen in renewable-rich but arid regions are not addressed. Finally, the
model's lack of spatial resolution for CO$_2$ means that carbonaceous hydrogen
derivatives are sited where H$_2$ is cheapest, implicitly assuming CO$_2$ from
biogenic or industrial CO$_2$ sources would be transported there. However, such
required CO$_2$ pipelines could be built at relatively low additional system
cost.\cite{hofmannH2CO2Network2024}

Overall, we find that the import vectors used strongly affect domestic
infrastructure needs. For example, only a smaller hydrogen network would be
required if hydrogen derivatives were largely imported and domestic industry
decides to relocate. We also identify higher backup requirements in the absence
of large power-to-X flexibilities. These findings underscore the importance of
coordination between energy import strategies and infrastructure policy
decisions. Our results present a quantitative basis for further discussions
about the trade-offs between system cost, carbon neutrality, public acceptance,
energy security, infrastructure buildout and imports.

The small differences in cost observed between some scenarios is particularly
relevant because factors other than pure costs might then drive the designs of
import strategies. The relatively limited cost benefit of imports and offshoring
of industrial production, may speak against imports. Concerns about energy
sovereignty would motivate more domestic supply and diversified imports. For
instance, ship-bourne imports would reduce pipeline lock-in and mitigate the
risks of sudden supply disruptions and abuse of market power.  From a practical
perspective, it may also be more appealing to focus on carriers that are already
globally traded commodities and to prefer infrastructure offering quick
deployment.

Policymaking in Europe might prefer such easy-to-implement systems featuring,
for instance, lower domestic infrastructure requirements, reuse of existing
infrastructure, lower technology risk, and reduced land usage for broader public
support than the most cost-effective solution. Moreover, policies favoring local
energy supply chains and importing intermediary products like sponge iron could
preserve European jobs while outsourcing only the most energy-intensive
processes. However, in shifting potential land use and infrastructure conflicts
to abroad where population densities are often lower, potential exporting
countries must weigh the prospect of economic development against internal
social and environmental concerns, particularly in countries with a history of
colonial exploitation.\cite{tunnGreenHydrogenTransitions2024} Ultimately,
Europe's energy strategy must balance cost savings from green energy and
material imports with broader concerns like geopolitics, economic development,
public opinion and the willingness of potential exporting countries in order to
ensure a swift, secure and sustainable energy future. Our research shows that
there is maneuvering space to accommodate such non-cost concerns.
