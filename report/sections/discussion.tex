Our analysis offers insights into how renewable energy imports might reduce
overall systems costs and interact with European energy infrastructure. Our
results show that imports of green energy reduce costs of a carbon-neutral
European energy system by \bneuro{39} (5\%), noting, however, that the
uncertainty range is considerable. While we find that some imports are robustly
beneficial, system cost savings range between 1\% and 14\% depending on the
import costs. What is consistent, however, are the diminishing return of energy
imports for larger quantities, with peak cost savings below imports of
4000~TWh/a. We also find that there is value in pursuing some \mbox{power-to-X}
production in Europe as a source of flexibility for wind and solar integration
and as a source of waste heat for district heating networks. Another location
factor in favour of European \mbox{power-to-X} is the  wide availability of
sustainable biogenic and industrial carbon sources, which helps to reduce
reliance on costly direct air capture.

Overall, we find that the import vectors used strongly affect domestic
infrastructure needs. For example, only a smaller hydrogen network would be
required if hydrogen derivatives were largely imported. This underscores the
importance of coordination between energy import strategies and infrastructure
policy decisions. Our results present a quantitative basis for further
discussions about the trade-offs between system cost, carbon neutrality, public
acceptance, energy security, infrastructure buildout and imports.

The small differences in cost between some scenarios is particularly relevant
because factors other than pure costs might then drive the designs of import
strategies. The relatively limited cost benefit of imports and value chain
reordering, may speak against pursuing this avenue. A desire for energy
sovereignty would motivate more domestic supply and diversified imports. For
instance, focusing on ship-bourne imports would reduce pipeline lock-in and
mitigate the risks of sudden supply disruptions and abuse of market power.
Focussing on carriers that are already a globally traded commodity may also be
more appealing. Producing renewable energy locally would bring value creation
and jobs to Europe that are currently outsourced to fossil fuel producers
abroad. For similar reasons, the import of intermediary raw products like sponge
iron, which represents the most energy-intensive part of the steel value chain,
could also be a relevant option.

There is also a social dimension to the import strategy and the question of how
fast the associated infrastructure can get built. Policymaking in Europe might
prefer easy-to-implement systems featuring, for instance, lower domestic
infrastructure requirements, reuse of existing infrastructure, lower technology
risk, and reduced land usage for broader public support than the most
cost-effective solution. However, in outsourcing potential land use and
infrastructure conflicts to abroad, potential exporting countries must weigh the
prospect of economic development against internal social and environmental
issues. Ultimately, Europe's energy strategy must balance cost savings from
green energy and material imports with broader concerns like geopolitics,
economic development, public opinion and the willingness of potential exporting
countries in order to ensure a swift, secure and sustainable energy future. Our
research shows that there is maneuvering space to accommodate such non-cost
concerns.
