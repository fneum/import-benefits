Our analysis offers insights into how renewable energy imports might reduce
overall systems costs and interact with European energy infrastructure. Our
results show that imports of green energy reduce the costs of a carbon-neutral
European energy system by \bneuro{37} (4.4\%), noting, however, that the
uncertainty range is considerable. While we find that some imports are
beneficial within a $\pm$20\% variation of import costs, system cost savings
range between 1\% and 10\%. However, what is consistent within this range are
the diminishing returns of energy imports for larger quantities, with peak cost
savings below imports of 3000~TWh/a (equivalent to 90~Mt of hydrogen). We also find
that there is value in pursuing some \mbox{power-to-X} production in Europe as a
source of flexibility for wind and solar integration and as a potential source
of waste heat for district heating networks. Another siting factor favouring
European \mbox{power-to-X} is the wide availability of sustainable biogenic and
industrial carbon sources, which helps reduce reliance on more costly direct air
capture.

With reference to the meta-study by Genge et al.~\cite{gengeSupplyCostsGreen2023},
our import costs to Europe for lowest-cost to median-level cost exporters
mostly conform to the review's interquartile ranges for different carriers and
the technology projection years 2030 and 2050. Some higher costs, e.g.~for
ammonia, can be attributed to our updated electrolyser cost assumptions
(1100/950/800\euro{}~kW$_\text{el.}^{-1}$ in 2030/2040/2050), which reflect
recent market developments.\cite{ieaGlobalHydrogenReview2024} Also for crude
steel imports, our central cost estimate of 531~\euro{}~t$^{-1}$ for the
lowest-cost exporter is positioned between studies with
lower\cite{lopezDefossilisedSteel2023} and
higher\cite{verpoortImpactGlobalHeterogeneity2024} cost estimates. Among other
studies investigating the relationship between energy imports and the European
energy system, several analyses report lower import shares in the range of
10-20\% of total hydrogen
supply.\cite{seckHydrogenDecarbonization2022,frischmuthHydrogenSourcingStrategies2022,kountourisUnifiedEuropeanHydrogen2024}
For instance, Kountouris et al.~\cite{kountourisUnifiedEuropeanHydrogen2024} see
limited hydrogen imports of 182~TWh~a$^{-1}$ from the Maghreb region and Ukraine
despite favourable import costs of 33~\euro{}~MWh$_{H_2}^{-1}$. Conversely,
Wetzel et al.~\cite{wetzelGreenEnergy2023a} find higher import shares of 53\%
for methane and 43\% for hydrogen. The latter closely aligns with our 49\%
import share for hydrogen. Wetzel et al.~\cite{wetzelGreenEnergy2023a} find that
imports reduce system costs by 2.8\%, which is also comparable to our 2.4\%
system cost reduction when only direct hydrogen imports are considered. The most
pronounced import dependency we found in Schmitz et
al.\cite{schmitzImplicationsHydrogenImport2024a}, with import shares beyond 90\%
for Germany. Results in Kountouris et
al.~\cite{kountourisUnifiedEuropeanHydrogen2024} further substantiate our
finding that derivative imports and demand relocation could diminish hydrogen
network benefits.

Several limitations of our study should be noted. First, the optimization
results represent a long-term equilibrium that disregards potential
transition-related infrastructure lock-ins or mid-term ramp-up constraints of
export capacities or domestic infrastructure development. The development speed
of key technologies is also uncertain and could affect cost-optimal
infrastructure and import strategies. A further limitation is that our
cost-based analysis of imports, which best reflects long-term bilateral purchase
agreements, neglects price impacts of intensifying global competition for green
fuel imports and exports.\cite{galimovaGlobalTrading2023a} Besides unclear
market developments, local challenges in exporting regions such as public
acceptance for export-oriented energy projects\cite{ishmamMappingLocalGreen2024}
and potential water
scarcity\cite{franzmannGreenHydrogenCostpotentials2023,terlouwFutureHydrogenEconomies2024}
to produce large amounts of hydrogen in renewable-rich but arid regions are not
addressed. We also do not assess potential impacts on the regional economy and
local employment effects within Europe as some energy-intensive manufacturing
relocate in the model. Furthermore, the model's lack of spatial resolution for
CO$_2$ means that carbonaceous hydrogen derivatives are sited where H$_2$ is
cheapest, implicitly assuming that the CO$_2$ from biogenic or industrial
sources can be transported there. However, such required CO$_2$ pipelines could
be built at relatively low additional system cost.\cite{hofmannH2CO2Network2024}
In the context of carbon management, more lenient assumptions on sustainable
biofuel potentials, allowed levels of geological carbon sequestration, or
plastic landfill could alter the results, shifting the system away from
synthetic electrofuels towards more fossil fuel use with carbon capture or
carbon dioxide
removal.\cite{hofmannH2CO2Network2024,millingerDiversityBiomassUsage2023}

Overall, we find that the import vectors used strongly affect domestic
infrastructure needs. For example, only a smaller hydrogen network would be
required if hydrogen derivatives were largely imported and the domestic ammonia
and crude steel industry is allowed to relocate. We also identify higher electricity
backup requirements in the absence of large power-to-X flexibilities. These
findings underscore the importance of coordination between energy import
strategies and infrastructure policy decisions. Our results present a
quantitative basis for further discussions about the trade-offs between system
cost, carbon neutrality, public acceptance, energy security, infrastructure
buildout, and imports.

The small differences in cost observed between some scenarios are particularly
relevant because factors other than pure costs, which are not reflected in our
infrastructure optimisation model, might then drive import strategies. To some,
the relatively limited cost benefit of imports and offshoring of industrial
production may speak against imports. Concerns about energy sovereignty could
motivate more domestic supply and diversified imports. For instance, shipborne
imports of hydrogen derivatives could be preferred to reduce pipeline lock-in
and to mitigate the risks of sudden supply disruptions and exercise of market
power. From a practical perspective, it may also be more appealing to focus on
carriers that are already globally traded commodities and to prefer
infrastructure offering quick deployment.

Policymakers in Europe might prefer alternative systems featuring, for instance,
lower domestic infrastructure requirements, reuse of existing infrastructure,
lower technology risk, and reduced land usage for broader public support than
the most cost-effective solution. Moreover, policies favoring local energy
supply chains and importing intermediary products like sponge iron could be
favoured to preserve European jobs while outsourcing only the most
energy-intensive processes. However, in shifting potential land use and
infrastructure conflicts to abroad, where population densities are often lower,
potential exporting countries would need to weigh the prospect of economic
development against internal social and environmental concerns, particularly in
countries with a history of colonial
exploitation.\cite{tunnGreenHydrogenTransitions2024} Ultimately, Europe's energy
strategy would likely seek to balance cost savings from green energy and
material imports with broader concerns like geopolitics, economic development,
public opinion, and the willingness of potential exporting countries in order to
ensure a swift, secure, and sustainable energy future. Our research shows that
there is maneuvering space around Europe's energy import strategies to
accommodate such non-cost concerns.
