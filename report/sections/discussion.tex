Our analysis offers insights into how energy imports might interact with
European energy infrastructures and what cost reductions they can yield. Our
results show that imports of green energy reduce costs of a carbon-neutral
European energy system by \bneuro{39} (5\%), noting, however, that the
uncertainty range is considerable. While we find that some imports are robustly
beneficial, system cost savings range between 1\% and 14\% depending on the
import costs. What is consistent, however, are the diminishing return of energy
imports for larger quantities and peak cost savings below imports of 4000~TWh.
We also find that there is value in pursuing some \mbox{power-to-X} production
in Europe as a source of flexibility for wind and solar integration and due to
the presence of district heating networks as offtaker for waste heat. Another
location factor in favour of European \mbox{power-to-X} is the availability of
biogenic and industrial carbon sources.

Overall, we find that which import vectors are used strongly affect domestic
infrastructure needs; for instance, that only a much smaller hydrogen network
would be required if hydrogen derivatives were largely imported. This
underscores the importance of coordination between energy import strategies and
infrastructure policy decisions. Our results present a quantitative basis for
further discussions about the trade-offs between system cost, carbon neutrality,
public acceptance, energy security, infrastructure buildout and imports.

This is particularly relevant as factors other than pure costs might drive the
designs of import strategies. The relatively limited cost benefit of imports and
value chain reordering, may speak against pursuing this avenue. A desire for
energy sovereignty would motivate more domestic supply and diversified imports.
Thereby, focusing on ship-bourne imports would reduce pipeline lock-in and
mitigate the risks of sudden supply disruptions. Focussing on carriers that are
already a globally traded commodity may also be more appealing. Besides
questions of resilience, a further consideration is the economic outflow caused
by imports, shifting value creation to other parts of the world with potential
repercussions on local jobs. Therefore, the import of intermediary raw products
like sponge iron, which represents the most energy-intensive part of the steel
value chain, could be another relevant option.

There is also a social dimension to the import strategy and the question of how
fast the aspired infrastructure can get built. Therefore, policymaking might
prefer easy-to-implement systems featuring, for instance, lower domestic
infrastructure requirements, reuse of existing infrastructure, lower technology
risk, and reduced land usage for broader public support than the most
cost-effective solution. Ultimately, Europe's energy strategy must balance cost
savings from green energy and material imports with broader concerns like
geopolitics, economic development and public opinion to ensure a swift, secure
and sustainable energy future.
