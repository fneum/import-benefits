
Our analysis offers insights into how energy imports might interact with
European energy infrastructures and what economic benefit they can bring. Our
results show that imports of green energy reduce costs of a carbon-neutral
European energy system by 5\%, and that European infrastructure requirements
strongly depend on the strategy taken on energy imports.

stimulate further discussions about
trade-offs between public acceptance, system cost, and energy security
pertaining to the import of low-carbon fuels and sensitise policymakers to the
extent to which infrastructure policy decisions depend on the path taken on
energy imports.

This is particularly relevant as other factors than pure costs
might rather drive import strategy, such as geopolitical considerations,
preferences for easy-to-implement systems, reuse of existing infrastructure,
resilience of supply chains, technology risk, diversification and land usage.

Could make case against imports and value chain reordering because of
relatively limited cost benefit of imports.

what other factors than cost might drive imports?
- energy costs
- abundance of high-yield renewable potentials in other parts of the world
- land use / landscape impact
- lower infrastructure requirement (wind, transmission) may mean higher public acceptance
- energy independence / energy security / geopolitical considerations / diversification of supply (to mitigate risks associated with supply disruptions in specific regions, reduced reliance in volatile regions of the world)
- energy sovereignty / self-sufficiency
- less rigid infrastructure (flexible supply, see sabotage of pipelines in Baltic sea)
- strengthened trade relations are a peacemaker?
- economic outflow by imports (value creation outside of Europe) with effect on local jobs
- wasted energy due to unnecessary conversion and transport of goods/fuels

hydrogen network \& comparison to Joule paper
- how much more costly without H2 network in different scenarios?

strong dependence on assumptions


