% 150 words
% 5 things a good abstract needs:

% 1. Introduce the topic,
Importing renewable energy to Europe offers many potential benefits, including
reduced energy costs, lower pressure on infrastructure development, and
less land-use within Europe.
% 2. State the unknown,
However, there remain many open questions: on the achievable cost reductions,
how much should be imported, whether the energy vector should be electricity,
hydrogen or hydrogen derivatives like ammonia or steel, and their impact on
Europe's domestic energy infrastructure needs.
% 3. Outline the method used to answer the question,
This study integrates the TRACE global energy supply chain model with the
sector-coupled energy system model for Europe, PyPSA-Eur, to explore net-zero
emission scenarios with varying import volumes, costs, and vectors.
% 4. Preview the findings, and
We find system cost reductions of 1-10\%, within import cost variations of
$\pm20\%$ around our central estimate, with diminishing returns for larger
import volumes and a preference for methanol, steel and hydrogen imports.
Keeping some domestic power-to-X production is beneficial for integrating
variable renewables, leveraging local sustainable carbon sources and utilising
some waste heat from fuel synthesis.
% 5. Tell us what your work teaches us.
Our findings highlight the need for coordinating import strategies with
infrastructure policy and reveal maneuvering space for incorporating non-cost
decision factors.
