Importing renewable energy to Europe may offer many potential benefits,
including reduced energy costs, lower pressure on infrastructure development,
and less land-use within Europe. However, there remain many open questions: on
the achievable cost reductions, how much should be imported, whether the energy
vector should be electricity, hydrogen or hydrogen derivatives like ammonia or
steel, and their impact on Europe's domestic energy infrastructure needs. This
study integrates the TRACE global energy supply chain model with the
sector-coupled energy system model for Europe, PyPSA-Eur, to explore net-zero
emission scenarios with varying import volumes, costs, and vectors. We find
system cost reductions of 1-11\%, within import cost variations of $\pm20\%$
around our central estimate, with diminishing returns for larger import volumes
and a preference for methanol, steel and hydrogen imports. Keeping some domestic
power-to-X production is beneficial for integrating variable renewables,
leveraging local sustainable carbon sources and utilising some waste heat from
fuel synthesis. Across scenarios, power grid reinforcements are more stable than
hydrogen pipeline expansion. Our findings highlight the need for coordinating
import strategies with infrastructure policy and reveal maneuvering space for
incorporating non-cost decision factors. 