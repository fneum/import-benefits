\documentclass[12pt,3p]{elsarticle}

\journal{ECEMP 2022}

\bibliographystyle{elsarticle-num}
\biboptions{numbers,sort&compress}

% format hacks
\usepackage{libertine}
\usepackage{libertinust1math}
% \usepackage{geometry}
% \geometry{
%     top=30mm,
%     bottom=35mm,
% }

\usepackage{amsmath}
\usepackage{bbold}
\usepackage{graphicx}
\usepackage{eurosym}
\usepackage{mathtools}
\usepackage{url}
\usepackage{booktabs}
\usepackage{epstopdf}
\usepackage{xfrac}
\usepackage{tabularx}
\usepackage{bm}
\usepackage{subcaption}
\usepackage{blindtext}
\usepackage{longtable}
\usepackage{multirow}
\usepackage{threeparttable}
\usepackage{pdflscape}
\usepackage[export]{adjustbox}
\usepackage[version=4]{mhchem}
\usepackage[colorlinks]{hyperref}
\usepackage[parfill]{parskip}
\usepackage[nameinlink,sort&compress,capitalise,noabbrev]{cleveref}
\usepackage[leftcaption,raggedright]{sidecap}
\usepackage[prependcaption,textsize=footnotesize]{todonotes}

\usepackage{siunitx}
\sisetup{
	range-units = single,
	per-mode = symbol
}
\DeclareSIUnit\year{a}
\DeclareSIUnit{\tco}{t_{\ce{CO2}}}
\DeclareSIUnit{\sieuro}{\mbox{\euro}}
\DeclareSIUnit{\twh}{{\tera\watt\hour}}
\DeclareSIUnit{\mwh}{{\mega\watt\hour}}
\DeclareSIUnit{\kwh}{{\kilo\watt\hour}}

\usepackage{lipsum}

\usepackage[resetlabels,labeled]{multibib}
\newcites{S}{Supplementary References}
\bibliographystyleS{elsarticle-num}

\graphicspath{
    {static/graphics/},
}

% \usepackage[
% 	type={CC},
% 	modifier={by},
% 	version={4.0},
% ]{doclicense}

\newcommand{\abs}[1]{\left|#1\right|}
\newcommand{\norm}[1]{\left\lVert#1\right\rVert}
\newcommand{\set}[1]{\left\{#1\right\}}
\DeclareMathOperator*{\argmin}{\arg\!\min}
\newcommand{\R}{\mathbb{R}}
\newcommand{\B}{\mathbb{B}}
\newcommand{\N}{\mathbb{N}}
\newcommand{\co}{\ce{CO2}~}
\def\el{${}_{\textrm{el}}$}
\def\th{${}_{\textrm{th}}$}
\def\deg{${}^\circ$}


\begin{document}

\begin{frontmatter}

	\title{Energy Imports and Infrastructure in a\\Climate-Neutral European Energy System}
    
	\author[tub]{Fabian Neumann\corref{correspondingauthor}}
	\ead{f.neumann@tu-berlin.de}
	\author[jlu]{Johannes Hampp}
    \author[tub]{Tom Brown}
	
	\address[jlu]{Center for International Development and Environmental Research, Justus-Liebig-University Gießen, Gießen, Germany}
	\address[tub]{Department of Digital Transformation in Energy Systems, Institute of Energy Technology, Technische Universität Berlin, Fakultät III, Einsteinufer 25 (TA 8), 10587 Berlin, Germany}

\end{frontmatter}

\section*{Long Abstract -- Introduction}

The transformation of the European energy system towards climate-neutrality
demands unrivalled technological change. Whereas the development of renewables
energy sources in Europe and supporting measures like reinforcing the
electricity grid do not always meet the level of acceptance required for a swift
transition, other parts of the world have cheap and abundant renewable energy
supply potentials to offer to global energy markets. They could become key
partners for a cost-effective and socially accepted energy transition in Europe.

However, even if they are economically attractive, a strong dependence on energy
imports can be a double-edged sword, as Europe is currently experiencing owing
to its heavy reliance on Russian fossil energy. On one side, next to decreasing
costs, energy imports can increase energy security by offering a way to mitigate
weather-induced energy droughts. On the other hand, energy imports might tie
European energy supply to few exporters or markets outside the control of
European actors.

In this contribution, we explore the dipole between full self-sufficiency and
wide-ranging energy imports into Europe. We investigate how the infrastructure
requirements of a self-sufficent European energy system that exclusively
leverages local resources from the continent may differ from a system that
relies on energy imports from outside of Europe. For our analysis, we integrate
and existing model of global energy carrier trade by Hampp et al.
(2021) \cite{hamppImportOptions2021} with an existing spatially and temporally
resolved sector-coupled energy system model, PyPSA-Eur-Sec, to investigate the
impact of imports on European energy infrastructure
needs \cite{PyPSAEurSecSectorCoupled}. We evaluate potential import locations and
carriers, the economic impetus for such imports, and how their inclusion affects
deployed infrastructure.

\section*{European infrastructure needs depend on import choices}

What infrastructure is needed to support the system depends on the levels of
clean energy imported. This relates to the geographic distribution of energy
resources tapped as well as directions and magnitudes of energy flows which have
to be supported.

Today, European energy infrastructure is built around the heavy imports of
fossil oil and gas. The transition to a system that exploits the best wind and
solar sites across the continent would offer manifold opportunities to develop a
self-sufficient system without imports
\cite{pickeringDiversityOptions2022,brownSynergiesSector2018}. Developing
transmission infrastructure, like reinforcing the power grid and building a
hydrogen network that partially repurposes an increasingly unused gas network is
consistently beneficial in such scenarios \cite{neumannBenefitsHydrogen2022,wetzelGreenEnergy2022}. A
\textit{European Hydrogen Backbone} was also recently envisioned by a consortium
of the European gas industry
\cite{gasforclimateEuropeanHydrogen2020,gasforclimateEuropeanHydrogen2022}.
However, a hydrogen backbone may not be needed if imports of renewable energy
carriers are considered.

Since most hydrogen is used to produce synthetic fuels and ammonia, if these
were imported at scale, much of the hydrogen demand would fall away. This would
reduce the need for hydrogen transport infrastructure. Even if there is high
demand for direct hydrogen imports, the optimised topology of a hydrogen network
might differ significantly as new import locations need to be connected rather
than domestic production.  The network's role changes from distributing energy
from North Sea hydrogen production hubs to incorporating inbound hydrogen
pipelines from North-Africa.

In this contribution, we investigate the potential benefit of importing energy
into the European energy system in scenarios with high shares of wind and solar
electricity and net-zero carbon emissions. In our analysis, we explore what
level of energy imports would lead to the lowest system costs, which types of
energy carriers are preferred, how import choices affect European infrastructure
requirements, and how much more expensive a fully self-sufficient energy system
would be. For this purpose, we perform sensitivity analyses interpolating
between cost-optimal levels of imports and no imports at all.

\section*{Characteristics of energy carriers carry advantages and challenges}

As possible import options we consider imports of electricity, hydrogen,
methane, ammonia and Fischer-Tropsch fuels. Each carrier has different
characteristics which leads to trade-offs regarding how, where and under which
circumstances they may be imported.

Electricity, the most versatile carrier, is challenging to store and requires
variability management if directly sourced from renewable sources. Hydrogen is
easier to store and transport in large quantities than electricity but at the
expense of being less versatile. Hydrogen has attracted significant interest
with plans of the European Commission under REPowerEU \cite{europeancommissionRepowerEUPlan} to import 10 Mt hydrogen
and derivatives by 2030. Furthermore, hydrogen from electrolysis and
climate-neutral electricity is considered a replacement for hydrogen from fossil
sources as a chemical feedstock in the future. Synthetic, carbon-neutral methane
could benefit from existing infrastructure but requires a sustainable carbon
source and leakage prevention. Ammonia does not require a carbon source and is
simple and cheap to store and transport over long distances. However, it suffers
from acceptance problems due to its toxicity and lower energy density. Lastly,
Fischer-Tropsch fuels are easy to store, transport and reuse existing
infrastructure, but the synthesis is energy-intensive due to high conversion
losses. Like methane, a sustainable carbon source is required.

For each energy carrier we identify locations with existing or planned import
infrastructure where the respective carrier may enter the European energy
system. We consider import options for electricity by transmission line,
hydrogen as gas by pipeline and liquid by ship, methane as gas by pipeline and
liquid by ship, ammonia as liquid by ship, and Fischer-Tropsch fuels by ship.
Moreover, we compute scenarios where only a subset of carriers can be imported
to probe the flatness of the near-optimal solution space and assess how the
carrier choice affects intra-European import infrastructure.

\section*{Sector-coupling in the European energy system with PyPSA-Eur-Sec}

To model the European energy system, we use the open-source energy system
optimisation model PyPSA-Eur-Sec \cite{PyPSAEurSecSectorCoupled}, which combines a fully
sector-coupled approach with high spatial and temporal resolution and detailed
transmission infrastructure representation. The model co-optimises the
investment and operation of generation, storage, conversion and transmission
infrastructures in a single linear optimisation problem. It covers 181 regions
and uses a 3-hourly time resolution for a full year. With these settings, the
model is detailed enough to capture existing grid bottlenecks and the
variability of renewables and requirements for seasonal storage. The model
includes regional demands from the electricity, industry, buildings, agriculture
and transport sectors, including shipping and aviation as well as non-energy
feedstock demands in the chemicals industry.  Furthermore, the model covers
transmission infrastructure for electricity, gas and hydrogen as well as
candidate entry points for energy imports like existing and prospective LNG
terminals and cross-continental pipelines.

\section*{Energy import options and costs modelled for each entry-point}

Import costs seen by our energy system model are based on recent research by
Hampp et al. (2021) \cite{hamppImportOptions2021}, who assessed the cost
importing energy across different energy supply chains for the afromentioned
energy carriers from various regions of the world. Based on this research we
determine for each energy carrier and model entry point the regional-specific
lowest import cost, thus, incoporating the potential trade-off between import
cost and import location.

\section*{Impact}

Our analysis offers insights into how energy imports might interact with
European energy infrastructures and what economic benefit they can bring. We
seek to stimulate further discussions about trade-offs between public
acceptance, system cost, and energy security pertaining to the import of
low-carbon fuels and sensitise to what extent infrastructure policy decisions
depend on the path taken on energy imports.

% tidy with https://flamingtempura.github.io/bibtex-tidy/
\addcontentsline{toc}{section}{References}
\renewcommand{\ttdefault}{\sfdefault}
\bibliography{/home/fneum/zotero}
% \bibliography{import-benefits}


\begin{figure}[ht!]
    \centering
\begin{subfigure}[t]{0.49\textwidth}
    \centering
    % \caption{clustered electricity network}
    \includegraphics[width=\textwidth]{power-network.pdf}
\end{subfigure}
\begin{subfigure}[t]{0.49\textwidth}
    \centering
    % \caption{gas network}
    \includegraphics[width=\textwidth]{gas-network.pdf}
\end{subfigure}
\begin{subfigure}{\textwidth}
	\centering
    \includegraphics[height=0.27\textheight]{total-annual-demand.pdf}
    \includegraphics[height=0.27\textheight]{ts-demand.pdf}
\end{subfigure}
\begin{subfigure}{\textwidth}
    \centering
    \includegraphics[height=0.28\textheight]{demand-by-sector-carrier.pdf}
    \includegraphics[height=0.28\textheight]{demand-by-carrier-sector.pdf}
\end{subfigure}
\caption{Overview of spatio-temporal resolution and included demand sectors in the energy system model PyPSA-Eur-Sec \cite{PyPSAEurSecSectorCoupled}. Clustered electricity and gas transmission networks (top row). Annual final energy and non-energy demand and system-level time series of demand by carrier (middle row). Annual final energy and non-energy demand by carrier and sector (bottom row).}
\label{fig:teaser}
\end{figure}

\end{document}