\documentclass[5p,9pt,draft]{elsarticle}
%DIF LATEXDIFF DIFFERENCE FILE
%DIF DEL draft-1/report/import-benefits.tex   Thu Mar 28 17:58:26 2024
%DIF ADD report/import-benefits.tex           Wed Mar 27 14:09:01 2024

\journal{Nature Energy}

% Length – up to 3,000 words, excluding abstract, Methods, references and figure legends.
% Abstract – up to 150 words, unreferenced.
% Display items – up to 8 items (figures and/or tables).
% Article should be divided as follows:
%     Introduction (without heading)
%     Main text
%     Discussion/Conclusions
%     Methods. ​
% Main text and Methods should be divided by topical subheadings; the Discussion/Conclusions does not contain subheadings.

% https://twitter.com/mwclimatesci/status/1111773593656918016?s=46&t=p6Kals2l9yy3e5W8zExYbQ

% \usepackage{natbib}
\bibliographystyle{elsarticle-num}
\biboptions{numbers,sort&compress,super}

% format hacks
\usepackage{libertine}
\usepackage{libertinust1math}
\usepackage{geometry}
\geometry{
    top=10mm,
    bottom=15mm,
	left=10mm,
	right=10mm,
}

% \usepackage[final]{microtype}
\usepackage{amsmath}
\usepackage{bbold}
\usepackage{graphicx}
\usepackage{eurosym}
\usepackage{mathtools}
\usepackage{url}
\usepackage{booktabs}
\usepackage{epstopdf}
\usepackage{xfrac}
\usepackage{tabularx}
\usepackage{bm}
\usepackage{subcaption}
\usepackage{longtable}
\usepackage{multirow}
\usepackage{threeparttable}
\usepackage[export]{adjustbox}
\usepackage[version=4]{mhchem}
\usepackage[colorlinks]{hyperref}
% \usepackage[parfill]{parskip}
\usepackage[nameinlink,sort&compress,capitalise]{cleveref}
\usepackage[leftcaption,raggedright]{sidecap}
\usepackage[prependcaption,textsize=footnotesize]{todonotes}
\usepackage{blindtext}
% \usepackage{pdflscape}

\usepackage{xr}
\externaldocument{import-benefits-si}

\urlstyle{sf}

% Adjust the hyphenation penalty
\hyphenpenalty=1000

% Adjust the line breaking tolerance
\tolerance=5000

%DIF 69a69-70
\renewcommand{\thefootnote}{\Alph{footnote}} %DIF > 
 %DIF > 
%DIF -------
\usepackage{lipsum}

\graphicspath{
    {../workflow/notebooks/},
    {../workflow/pypsa-eur/resources/},
    {../workflow/pypsa-eur/results/},
}

% \usepackage[
% 	type={CC},
% 	modifier={by},
% 	version={4.0},
% ]{doclicense}

\usepackage{siunitx}
\sisetup{range-units=single, per-mode=symbol}
\DeclareSIUnit\year{a}
\DeclareSIUnit{\tco}{t_{\ce{CO2}}}
\DeclareSIUnit{\sieuro}{\mbox{\euro}}
\DeclareSIUnit{\twh}{{\tera\watt\hour}}
\DeclareSIUnit{\mwh}{{\mega\watt\hour}}
\DeclareSIUnit{\kwh}{{\kilo\watt\hour}}
\newcommand{\co}{\ce{CO2}~}
\def\el{${}_{\textrm{el}}$}
\def\th{${}_{\textrm{th}}$}

\newcommand{\bneuro}[1]{#1\,bn\euro{}/a}

\renewcommand{\ttdefault}{\sfdefault}

\newcommand{\sfigref}[1]{%
    \begingroup%
    \crefname{figure}{Supplementary Fig.}{Supplementary Figs.}%
    \cref{#1}%
    \endgroup%
}

\newcommand{\Sfigref}[1]{%
    \begingroup%
    \Crefname{figure}{Supplementary Fig.}{Supplementary Figs.}%
    \Cref{#1}%
    \endgroup%
}

\newcommand{\stabref}[1]{%
    \begingroup%
    \crefname{table}{Supplementary Table}{Supplementary Tables}%
    \cref{#1}%
    \endgroup%
}

\newcommand{\Stabref}[1]{%
    \begingroup%
    \Crefname{table}{Supplementary Table}{Supplementary Tables}%
    \Cref{#1}%
    \endgroup%
}
%DIF PREAMBLE EXTENSION ADDED BY LATEXDIFF
%DIF UNDERLINE PREAMBLE %DIF PREAMBLE
\RequirePackage[normalem]{ulem} %DIF PREAMBLE
\RequirePackage{color}\definecolor{RED}{rgb}{1,0,0}\definecolor{BLUE}{rgb}{0,0,1} %DIF PREAMBLE
\providecommand{\DIFaddtex}[1]{{\protect\color{blue}\uwave{#1}}} %DIF PREAMBLE
\providecommand{\DIFdeltex}[1]{{\protect\color{red}\sout{#1}}}                      %DIF PREAMBLE
%DIF SAFE PREAMBLE %DIF PREAMBLE
\providecommand{\DIFaddbegin}{} %DIF PREAMBLE
\providecommand{\DIFaddend}{} %DIF PREAMBLE
\providecommand{\DIFdelbegin}{} %DIF PREAMBLE
\providecommand{\DIFdelend}{} %DIF PREAMBLE
\providecommand{\DIFmodbegin}{} %DIF PREAMBLE
\providecommand{\DIFmodend}{} %DIF PREAMBLE
%DIF FLOATSAFE PREAMBLE %DIF PREAMBLE
\providecommand{\DIFaddFL}[1]{\DIFadd{#1}} %DIF PREAMBLE
\providecommand{\DIFdelFL}[1]{\DIFdel{#1}} %DIF PREAMBLE
\providecommand{\DIFaddbeginFL}{} %DIF PREAMBLE
\providecommand{\DIFaddendFL}{} %DIF PREAMBLE
\providecommand{\DIFdelbeginFL}{} %DIF PREAMBLE
\providecommand{\DIFdelendFL}{} %DIF PREAMBLE
%DIF HYPERREF PREAMBLE %DIF PREAMBLE
\providecommand{\DIFadd}[1]{\texorpdfstring{\DIFaddtex{#1}}{#1}} %DIF PREAMBLE
\providecommand{\DIFdel}[1]{\texorpdfstring{\DIFdeltex{#1}}{}} %DIF PREAMBLE
\newcommand{\DIFscaledelfig}{0.5}
%DIF HIGHLIGHTGRAPHICS PREAMBLE %DIF PREAMBLE
\RequirePackage{settobox} %DIF PREAMBLE
\RequirePackage{letltxmacro} %DIF PREAMBLE
\newsavebox{\DIFdelgraphicsbox} %DIF PREAMBLE
\newlength{\DIFdelgraphicswidth} %DIF PREAMBLE
\newlength{\DIFdelgraphicsheight} %DIF PREAMBLE
% store original definition of \includegraphics %DIF PREAMBLE
\LetLtxMacro{\DIFOincludegraphics}{\includegraphics} %DIF PREAMBLE
\newcommand{\DIFaddincludegraphics}[2][]{{\color{blue}\fbox{\DIFOincludegraphics[#1]{#2}}}} %DIF PREAMBLE
\newcommand{\DIFdelincludegraphics}[2][]{% %DIF PREAMBLE
\sbox{\DIFdelgraphicsbox}{\DIFOincludegraphics[#1]{#2}}% %DIF PREAMBLE
\settoboxwidth{\DIFdelgraphicswidth}{\DIFdelgraphicsbox} %DIF PREAMBLE
\settoboxtotalheight{\DIFdelgraphicsheight}{\DIFdelgraphicsbox} %DIF PREAMBLE
\scalebox{\DIFscaledelfig}{% %DIF PREAMBLE
\parbox[b]{\DIFdelgraphicswidth}{\usebox{\DIFdelgraphicsbox}\\[-\baselineskip] \rule{\DIFdelgraphicswidth}{0em}}\llap{\resizebox{\DIFdelgraphicswidth}{\DIFdelgraphicsheight}{% %DIF PREAMBLE
\setlength{\unitlength}{\DIFdelgraphicswidth}% %DIF PREAMBLE
\begin{picture}(1,1)% %DIF PREAMBLE
\thicklines\linethickness{2pt} %DIF PREAMBLE
{\color[rgb]{1,0,0}\put(0,0){\framebox(1,1){}}}% %DIF PREAMBLE
{\color[rgb]{1,0,0}\put(0,0){\line( 1,1){1}}}% %DIF PREAMBLE
{\color[rgb]{1,0,0}\put(0,1){\line(1,-1){1}}}% %DIF PREAMBLE
\end{picture}% %DIF PREAMBLE
}\hspace*{3pt}}} %DIF PREAMBLE
} %DIF PREAMBLE
\LetLtxMacro{\DIFOaddbegin}{\DIFaddbegin} %DIF PREAMBLE
\LetLtxMacro{\DIFOaddend}{\DIFaddend} %DIF PREAMBLE
\LetLtxMacro{\DIFOdelbegin}{\DIFdelbegin} %DIF PREAMBLE
\LetLtxMacro{\DIFOdelend}{\DIFdelend} %DIF PREAMBLE
\DeclareRobustCommand{\DIFaddbegin}{\DIFOaddbegin \let\includegraphics\DIFaddincludegraphics} %DIF PREAMBLE
\DeclareRobustCommand{\DIFaddend}{\DIFOaddend \let\includegraphics\DIFOincludegraphics} %DIF PREAMBLE
\DeclareRobustCommand{\DIFdelbegin}{\DIFOdelbegin \let\includegraphics\DIFdelincludegraphics} %DIF PREAMBLE
\DeclareRobustCommand{\DIFdelend}{\DIFOaddend \let\includegraphics\DIFOincludegraphics} %DIF PREAMBLE
\LetLtxMacro{\DIFOaddbeginFL}{\DIFaddbeginFL} %DIF PREAMBLE
\LetLtxMacro{\DIFOaddendFL}{\DIFaddendFL} %DIF PREAMBLE
\LetLtxMacro{\DIFOdelbeginFL}{\DIFdelbeginFL} %DIF PREAMBLE
\LetLtxMacro{\DIFOdelendFL}{\DIFdelendFL} %DIF PREAMBLE
\DeclareRobustCommand{\DIFaddbeginFL}{\DIFOaddbeginFL \let\includegraphics\DIFaddincludegraphics} %DIF PREAMBLE
\DeclareRobustCommand{\DIFaddendFL}{\DIFOaddendFL \let\includegraphics\DIFOincludegraphics} %DIF PREAMBLE
\DeclareRobustCommand{\DIFdelbeginFL}{\DIFOdelbeginFL \let\includegraphics\DIFdelincludegraphics} %DIF PREAMBLE
\DeclareRobustCommand{\DIFdelendFL}{\DIFOaddendFL \let\includegraphics\DIFOincludegraphics} %DIF PREAMBLE
%DIF COLORLISTINGS PREAMBLE %DIF PREAMBLE
\RequirePackage{listings} %DIF PREAMBLE
\RequirePackage{color} %DIF PREAMBLE
\lstdefinelanguage{DIFcode}{ %DIF PREAMBLE
%DIF DIFCODE_UNDERLINE %DIF PREAMBLE
  moredelim=[il][\color{red}\sout]{\%DIF\ <\ }, %DIF PREAMBLE
  moredelim=[il][\color{blue}\uwave]{\%DIF\ >\ } %DIF PREAMBLE
} %DIF PREAMBLE
\lstdefinestyle{DIFverbatimstyle}{ %DIF PREAMBLE
	language=DIFcode, %DIF PREAMBLE
	basicstyle=\ttfamily, %DIF PREAMBLE
	columns=fullflexible, %DIF PREAMBLE
	keepspaces=true %DIF PREAMBLE
} %DIF PREAMBLE
\lstnewenvironment{DIFverbatim}{\lstset{style=DIFverbatimstyle}}{} %DIF PREAMBLE
\lstnewenvironment{DIFverbatim*}{\lstset{style=DIFverbatimstyle,showspaces=true}}{} %DIF PREAMBLE
%DIF END PREAMBLE EXTENSION ADDED BY LATEXDIFF

\begin{document}


\begin{frontmatter}

	\title{Energy Imports and Infrastructure in a Carbon-Neutral European Energy System}

	\author[tub]{Fabian Neumann\corref{correspondingauthor}}
	\ead{f.neumann@tu-berlin.de}
	\author[pik]{Johannes Hampp}
	\author[tub]{Tom Brown}

	\address[tub]{Department of Digital Transformation in Energy Systems, Institute of Energy Technology,\\Technische Universität Berlin, Fakultät III, Einsteinufer 25 (TA 8), 10587 Berlin, Germany}
	\address[pik]{Potsdam Institute for Climate Impact Research (PIK), Member of the Leibniz Association, P.O.~Box 60 12 03, 14412 Potsdam, Germany}

	\begin{abstract}
		% 150 words
% 5 things a good abstract needs:

% 1. Introduce the topic,
\DIFaddbegin \DIFadd{Importing renewable energy to Europe offers many potential benefits, including
reduced energy costs, bypassed delays in infrastructure development, and
decreased land-use pressure by leveraging the abundant renewable resources of
various potential export countries.
}\DIFaddend % 2. State the unknown,
\DIFaddbegin \DIFadd{However, the extent of achievable cost reductions, favourable import volumes and
vectors, and their impact on Europe's domestic energy infrastructure needs
remain uncertain.
}\DIFaddend % 3. Outline the method used to answer the question,
\DIFaddbegin \DIFadd{This study integrates the TRACE global energy supply chain model with the
sector-coupled energy system model PyPSA-Eur to explore scenarios with varying
import volumes, costs, and vectors.
}\DIFaddend % 4. Preview the findings, and
\DIFaddbegin \DIFadd{We find system cost reductions of 1-14\%, depending on assumed import costs,
with diminishing returns for larger import volumes and a preference for
methanol, steel and hydrogen imports. Nevertheless, keeping some domestic
power-to-X production is beneficial for integrating variable renewables,
utilising waste heat and leveraging local carbon sources.
}\DIFaddend % 5. Tell us what your work teaches us.
\DIFaddbegin \DIFadd{Our findings highlight the need for coordinating import strategies with
infrastructure policy and reveal maneuvering space for non-cost decision
factors.
}

	\DIFaddend \end{abstract}

	% \begin{keyword}
	% 	TODO
	% \end{keyword}

	% \begin{graphicalabstract}
	% \end{graphicalabstract}

	% \begin{highlights}
	% 	\item A
	% 	\item B
	% 	\item C
	% \end{highlights}

\end{frontmatter}

% \listoftodos[TODOs]

% \tableofcontents

% \section*{Introduction}
% \label{sec:intro}

%DIF <  promise of imports
%DIF > %% promise of imports %%%

\DIFdelbegin \DIFdel{The transformation of the European energy system towards climate neutrality
demands unrivalled technological change. Whereas the development of renewable energy sources in Europe and supporting measures like reinforcing the
electricity grid do not always meet the level of acceptance required for a swift
transition, other }\DIFdelend \DIFaddbegin \DIFadd{Importing renewable energy to Europe promises several advantages for achieving a
swift energy transition. It could lower costs, help circumvent the slow domestic
deployment of renewable energy infrastructure and reduce pressure on land usage
in Europe. Many }\DIFaddend parts of the world have cheap and abundant renewable energy
supply potentials \DIFdelbegin \DIFdel{to offer to }\DIFdelend \DIFaddbegin \DIFadd{they could offer to existing or emerging }\DIFaddend global energy
markets.\DIFdelbegin \DIFdel{\mbox{%DIFAUXCMD
\cite{irenaGlobalHydrogen2022,luxSupplyCurves2021,vanderzwaanTimmermansDream2021,fasihiLongTermHydrocarbon2017,reichenbergDeepDecarbonization2022,galvanExportingSunshine2022,armijoFlexibleProduction2020,pfennigGlobalGISbased2022}
}\hskip0pt%DIFAUXCMD
These regions could become key partners for a cost-effective and socially
accepted energy transition in Europe , especially when the production of large
quantities of domestic green fuels and materials falters}\DIFdelend \DIFaddbegin \DIFadd{\mbox{%DIFAUXCMD
\cite{irenaGlobalHydrogen2022,luxSupplyCurves2021,vanderzwaanTimmermansDream2021,fasihiLongTermHydrocarbon2017,reichenbergDeepDecarbonization2022,galvanExportingSunshine2022,armijoFlexibleProduction2020,pfennigGlobalGISbasedPotential2023}
}\hskip0pt%DIFAUXCMD
Partnering with these regions could help Europe reach its carbon neutrality
goals while stimulating economic development in exporting countries}\DIFaddend .

%DIF <  repulsion of imports
%DIF > %% dangers of imports %%%

However, even if energy imports are economically attractive \DIFaddbegin \DIFadd{for Europe}\DIFaddend , a strong
\DIFdelbegin \DIFdel{dependence
on them }\DIFdelend \DIFaddbegin \DIFadd{reliance }\DIFaddend may not be \DIFdelbegin \DIFdel{preferred }\DIFdelend \DIFaddbegin \DIFadd{desired }\DIFaddend due to energy security concerns\DIFdelbegin \DIFdel{, as European
countries recently experienced owing to their reliance on Russian natural
gas. \mbox{%DIFAUXCMD
\cite{pedersenLongtermImplications2022} }\hskip0pt%DIFAUXCMD
In 2021,}\DIFdelend \DIFaddbegin \DIFadd{. Awareness of energy
security has risen since Russia throttled gas supplies to Europe in
2022,\mbox{%DIFAUXCMD
\cite{pedersenLongtermImplications2022} }\hskip0pt%DIFAUXCMD
at a time when }\DIFaddend the EU27 \DIFdelbegin \DIFdel{nations sourced
}\DIFdelend \DIFaddbegin \DIFadd{imported
}\DIFaddend around two-thirds of \DIFdelbegin \DIFdel{their energy needsthrough
imports,\mbox{%DIFAUXCMD
\cite{eurostatCompleteEnergy2023}
}\hskip0pt%DIFAUXCMD
, and accordingly}\DIFdelend \DIFaddbegin \DIFadd{its energy needs.\mbox{%DIFAUXCMD
\cite{eurostatCompleteEnergy2023}
}\hskip0pt%DIFAUXCMD
Accordingly}\DIFaddend , much of the European energy infrastructure is built around the
imports of fossil oil and gas. Continued energy imports and associated \DIFaddbegin \DIFadd{rigid
}\DIFaddend infrastructure might tie European energy supply to a small number of exporters
\DIFdelbegin \DIFdel{or markets }\DIFdelend with market power. \DIFdelbegin \DIFdel{These }\DIFdelend \DIFaddbegin \DIFadd{Such lock-in }\DIFaddend risks must be weighed against the potential
benefits of \DIFdelbegin \DIFdel{decreasing energy supply
costs, reducing land usage in Europeand increasing energy security by supplying
storable fuels that can mitigate energy droughts for systems with high shares of
weather-dependent energy supply. }%DIFDELCMD < 

%DIFDELCMD < %%%
%DIF <  feasibility of no imports
\DIFdelend \DIFaddbegin \DIFadd{imports.
}\DIFaddend 

\DIFdelbegin \DIFdel{The transition to a system that exploits the best wind and solar sites across
the continent would offer }\DIFdelend %DIF > %% dependence of energy imports on energy infrastructure %%%
\DIFaddbegin 

\DIFadd{Europe's strategy for clean energy imports will also strongly affect the
requirements for domestic energy infrastructure. Previous research found }\DIFaddend many
ways to develop a self-sufficient \DIFdelbegin \DIFdel{system without
imports.\mbox{%DIFAUXCMD
\cite{pickeringDiversityOptions2022,trondleHomemadeImported2019,brownSynergiesSector2018}
}\hskip0pt%DIFAUXCMD
For instance}\DIFdelend \DIFaddbegin \DIFadd{energy
system.\mbox{%DIFAUXCMD
\cite{pickeringDiversityOptions2022,trondleHomemadeImported2019,brownSynergiesSector2018}
}\hskip0pt%DIFAUXCMD
To support such scenarios without energy imports into Europe}\DIFaddend , reinforcing the
power grid or building a hydrogen network \DIFdelbegin \DIFdel{that may
repurpose parts of the gas network was consistently identified as
beneficial in
such autarkic
scenarios}\DIFdelend \DIFaddbegin \DIFadd{was often identified as
constructive}\DIFaddend .\cite{neumannPotentialRole2023,victoriaSpeedTechnological2022}
However, \DIFdelbegin \DIFdel{what energy infrastructure is needed might strongly depend on the levels of clean energy imported. For instance, scaling up Europe 's energy
transmission infrastructure may not be necessary. Since most hydrogen would be
used to produce synthetic fuels }\DIFdelend \DIFaddbegin \DIFadd{depending on the vectors and volumes of imports, Europe might not need
to expand its hydrogen transport infrastructure. Most hydrogen is used to make
derivative products }\DIFaddend (e.g.~, \DIFaddbegin \DIFadd{Fischer-Tropsch fuels or methanol for }\DIFaddend high-value
chemicals, aviation and shipping \DIFdelbegin \DIFdel{) and steel,\mbox{%DIFAUXCMD
\cite{neumannPotentialRole2023} }\hskip0pt%DIFAUXCMD
if these derivatives were
imported }\DIFdelend \DIFaddbegin \DIFadd{or ammonia for
fertilisers).\mbox{%DIFAUXCMD
\cite{neumannPotentialRole2023} }\hskip0pt%DIFAUXCMD
If Europe imported these products
}\DIFaddend at scale, much of the hydrogen demand would fall away\DIFdelbegin \DIFdel{, hence, reducing
}\DIFdelend \DIFaddbegin \DIFadd{. In consequence, this
would reduce }\DIFaddend the need for hydrogen transport. \DIFdelbegin \DIFdel{Even if there were a high demand for }\DIFdelend \DIFaddbegin \DIFadd{Furthermore, substantial }\DIFaddend direct
hydrogen imports \DIFdelbegin \DIFdel{, the optimised topology would differ if it needed to absorb
inbound hydrogen }\DIFdelend \DIFaddbegin \DIFadd{would require a different pipeline network topology, tailored
to accommodate hydrogen arriving }\DIFaddend from North Africa \DIFdelbegin \DIFdel{rather than domestic production}\DIFdelend \DIFaddbegin \DIFadd{or maritime shipping routes
to Northern Europe}\DIFaddend .

%DIF <  policy strategies
\DIFdelbegin %DIFDELCMD < 

%DIFDELCMD < %%%
%DIF <  TODO more recent news / strategies?
\DIFdelend %DIF > %% review of policy strategies %%%

\DIFaddbegin \DIFadd{Policy has reflected these different visions for imports in various ways. }\DIFaddend In
particular, hydrogen imports have recently attracted considerable interest, with
plans of the European Commission under
\mbox{REPowerEU}\cite{europeancommissionRepowerEUPlan} to import 10~Mt
(333~TWh\DIFaddbegin \footnote{\DIFadd{All mass-energy conversion is based on the lower heating value
(LHV). Steel is included in energy terms applying 2.1 kWh/kg as released by the
oxidation of iron.}}\DIFaddend ) hydrogen and derivatives by \DIFdelbegin \DIFdel{2030 and reflections of this vision in the German
national
hydrogen
strategy.\mbox{%DIFAUXCMD
\cite{bundesministeriumfuerwirtschaftundklimaschutzFortschreibungNationalen2023}
}\hskip0pt%DIFAUXCMD
}%DIFDELCMD < 

%DIFDELCMD < %%%
%DIF <  literature review of studies focusing on the cost of energy imports
%DIF <  literature review focusing on the European energy system with/without considering imports
\DIFdelend \DIFaddbegin \DIFadd{2030. Desire to import hydrogen
and derivative products is also present in various national
strategies.\mbox{%DIFAUXCMD
\cite{corbeauNationalHydrogenStrategies2024} }\hskip0pt%DIFAUXCMD
In particular, Germany
seeks to cover up to 70\% of its hydrogen consumption through imports by 2030
and pursues bilateral partnerships to accomplish
this.\mbox{%DIFAUXCMD
\cite{bundesministeriumfuerwirtschaftundklimaschutzFortschreibungNationalenWasserstoffstrategie2023}
}\hskip0pt%DIFAUXCMD
Conversely, hydrogen roadmaps of
Denmark,\mbox{%DIFAUXCMD
\cite{danishministryofclimateenergyandutilitiesRegeringensStrategiPowertoX2021}
}\hskip0pt%DIFAUXCMD
Ireland,\mbox{%DIFAUXCMD
\cite{departmentoftheenvironmentclimateandcommunicationsgovernmentofirelandNationalHydrogenStrategy2023}
}\hskip0pt%DIFAUXCMD
Spain,\mbox{%DIFAUXCMD
\cite{marcoestrategicodeenergiayclimaRutaHidrogenoApuesta2020} }\hskip0pt%DIFAUXCMD
and the
United
Kingdom,\mbox{%DIFAUXCMD
\cite{ukdepartmentforenergysecurity&netzeroHydrogenStrategyUpdate2023}
}\hskip0pt%DIFAUXCMD
recognise these countries' potential to become major exporters of renewable
energy, whereas France's strategy focuses on local hydrogen production to meet
domestic needs.\mbox{%DIFAUXCMD
\cite{frenchgovernmentStrategieNationalePour2023} }\hskip0pt%DIFAUXCMD
Additionally,
in the recent TYNDP 2024,\mbox{%DIFAUXCMD
\cite{entso-eTYNDP2024Project2024}  }\hskip0pt%DIFAUXCMD
European grid
development plans reveal renewed enthusiasm for electricity imports via
ultra-long HVDC cables, evolving from early
DESERTEC\mbox{%DIFAUXCMD
\cite{desertecfoundationDESERTECSustainableWealth2024} }\hskip0pt%DIFAUXCMD
ideas to
contemporary proposals like the Morocco-UK Xlinks
project.\mbox{%DIFAUXCMD
\cite{xlinksMoroccoUKPower2023}
}\hskip0pt%DIFAUXCMD
}\DIFaddend 

%DIF <  TODO stronger novelty statement?
%DIF > %% literature review %%%

While many previous academic studies have evaluated the cost of green energy and
material imports in the form of
electricity,\cite{lilliestamEnergySecurity2011,triebSolarElectricity2012,lilliestamVulnerabilityTerrorist2014,bogdanovNorthEastAsian2016,benaslaTransitionSustainable2019,reichenbergDeepDecarbonization2022}% (some with reference to the DESERTEC idea),
hydrogen,\DIFdelbegin \DIFdel{\mbox{%DIFAUXCMD
\cite{timmerbergHydrogenRenewables2019,ishimotoLargescaleProduction2020,brandleEstimatingLongterm2021,luxSupplyCurves2021,galvanExportingSunshine2022,collisDeterminingProduction2022,galimovaImpactInternational2023}
}\hskip0pt%DIFAUXCMD
}\DIFdelend \DIFaddbegin \DIFadd{\mbox{%DIFAUXCMD
\cite{timmerbergHydrogenRenewables2019,ishimotoLargescaleProduction2020,brandleEstimatingLongterm2021,luxSupplyCurves2021,galvanExportingSunshine2022,collisDeterminingProduction2022,galimovaImpactInternational2023,schmitzImplicationsHydrogenImport2024}
}\hskip0pt%DIFAUXCMD
}\DIFaddend ammonia,\cite{nayak-lukeTechnoeconomicViability2020,armijoFlexibleProduction2020,galimovaFeasibilityGreen2023}
methane,\cite{luxSupplyCurves2021,agoraenergiewendeHydrogenImport2022}
steel,\cite{trollipHowGreen2022a,devlinRegionalSupply2022,lopezDefossilisedSteel2023}
carbon-based
fuels,\cite{fasihiLongTermHydrocarbon2017,sherwinElectrofuelSynthesis2021} or a
broader variety of power-to-X
fuels,\cite{vanderzwaanTimmermansDream2021,pfennigGlobalGISbased2022,irenaGlobalHydrogen2022,solerEFuelsTechno2022,hamppImportOptions2023,gengeSupplyCosts2023,galimovaGlobalTrading2023a}
these do not address \DIFaddbegin \DIFadd{the }\DIFaddend interactions of imports with European energy
infrastructure requirements. On the other hand, among studies dealing with the
detailed planning of net-zero energy systems in Europe, some do not consider
energy
imports,\cite{pickeringDiversityOptions2022,brownSynergiesSector2018,victoriaSpeedTechnological2022}
while others only consider hydrogen imports or a limited set of alternative
import
vectors.\cite{gilsInteractionHydrogen2021,seckHydrogenDecarbonization2022,wetzelGreenEnergy2023,kountourisUnifiedEuropean2023,neumannPotentialRole2023}
Only \DIFaddbegin \DIFadd{a }\DIFaddend few consider at least elementary cost
uncertainties,\DIFdelbegin \DIFdel{\mbox{%DIFAUXCMD
\cite{frischmuthHydrogenSourcing2022} }\hskip0pt%DIFAUXCMD
}\DIFdelend \DIFaddbegin \DIFadd{\mbox{%DIFAUXCMD
\cite{frischmuthHydrogenSourcing2022,schmitzImplicationsHydrogenImport2024} }\hskip0pt%DIFAUXCMD
}\DIFaddend and none investigate a
larger range of potential import volumes across subsets of available import
vectors.


%DIF <  main paper idea - scenarios
%DIF > %% main paper idea - scenarios %%%

In this \DIFdelbegin \DIFdel{contribution}\DIFdelend \DIFaddbegin \DIFadd{study}\DIFaddend , we explore the full range between the two poles of complete
self-sufficiency and wide-ranging \DIFaddbegin \DIFadd{renewable }\DIFaddend energy imports into Europe in
scenarios with high shares of wind and solar electricity and net-zero carbon
emissions. We investigate how the infrastructure requirements of a
self-sufficient European energy system that exclusively leverages domestic
resources from the continent may differ from a system that relies on energy
imports from outside of Europe. For our analysis, we integrate an open model of
global energy supply chains, TRACE,\cite{hamppImportOptions2023} with a
spatially and temporally resolved sector-coupled open-source energy system
model, PyPSA-Eur,\cite{PyPSAEurSecSectorCoupled} to investigate the impact of
imports on European energy infrastructure needs. We evaluate potential import
locations and costs for different supply vectors, \DIFdelbegin \DIFdel{the economic impetus for such
}\DIFdelend \DIFaddbegin \DIFadd{by how much system costs can
be reduced through }\DIFaddend imports, and how their inclusion affects deployed transport
networks and storage. For this purpose, we perform sensitivity analyses
interpolating between very high levels of imports and no imports at all,
\DIFaddbegin \DIFadd{exploring }\DIFaddend low and high costs for imports to account for associated
uncertainties, and system responses to the exclusion of subsets of import
vectors, all to probe the flatness of the solution space.


\begin{figure*}
    \begin{subfigure}[t]{\textwidth}
        \caption{\DIFdelbeginFL \DIFdelFL{global }\DIFdelendFL \DIFaddbeginFL \DIFaddFL{Global }\DIFaddendFL perspective \DIFaddbeginFL \DIFaddFL{for energy imports into Europe}\DIFaddendFL }
        \label{fig:options:global}
        \includegraphics[width=\textwidth]{20231025-zecm/graphics/import_world_map.pdf}
    \end{subfigure}
    \DIFdelbeginFL %DIFDELCMD < \begin{subfigure}[t]{\textwidth}
%DIFDELCMD <         %%%
\DIFdelendFL \DIFaddbeginFL \begin{subfigure}[t]{0.7\textwidth}
        \DIFaddendFL \caption{European perspective \DIFaddbeginFL \DIFaddFL{for inbound energy imports}\DIFaddendFL }
        \label{fig:options:europe}
        \centering
        \includegraphics[width=\textwidth]{20231025-zecm/graphics/import_options_s_110_lvopt__Co2L0-2190SEG-T-H-B-I-S-A-onwind+p0.5-imp_2050.pdf}
    \end{subfigure}
    \DIFaddbeginFL \begin{subfigure}[t]{0.3\textwidth}
        \DIFaddendFL \caption{\DIFdelbeginFL \textbf{\DIFdelFL{Overview of considered import options.}}
        %DIFAUXCMD
\textit{\DIFdelFL{Panel (a)}} %DIFAUXCMD
\DIFdelFL{shows the regional differences in the cost to deliver
        green methanol to Europe (choropleth layer), the cost composition of
        different import vectors (bar charts), an illustration of the wind and
        solar availability in Morocco, and an illustration of the land
        eligibility analysis for wind turbine development in the region of
        Buenos Aires in Argentina. }\textit{\DIFdelFL{Panel (b)}} %DIFAUXCMD
\DIFdelFL{depicts potential entry
        points for energy imports into Europe like the location of existing and
        planned LNG terminals and gas pipeline entry points, the costs of
        hydrogen imports in different European regions (choropleth layer), the
        considered connections for long-distance HVDC import links from the MENA
        region, Kazakhstan, Turkey and Ukraine, and the }\DIFdelendFL \DIFaddbeginFL \DIFaddFL{Cost }\DIFaddendFL distribution \DIFdelbeginFL \DIFdelFL{and range
        of import costs for different energy carriers and entry points with
        indications for selected countries of }\DIFdelendFL \DIFaddbeginFL \DIFaddFL{by }\DIFaddendFL origin\DIFdelbeginFL \DIFdelFL{(violin charts). }\DIFdelendFL \DIFaddbeginFL \DIFaddFL{, entrypoint and carrier}\DIFaddendFL }
        \DIFaddbeginFL \label{fig:options:distribution}
        \centering
        \includegraphics[width=\textwidth]{20231025-zecm/graphics/import_options_s_110_lvopt__Co2L0-2190SEG-T-H-B-I-S-A-onwind+p0.5-imp_2050-distribution.pdf}
    \end{subfigure}
    \caption{\textbf{\DIFaddFL{Overview of considered import options.}}
        \textit{\DIFaddFL{Panel (a)}} \DIFaddFL{shows the regional differences in the cost to deliver
        green methanol to Europe (choropleth layer), the cost composition of
        different import vectors (bar charts), an illustration of the wind and
        solar availability in Morocco, and an illustration of the land
        eligibility analysis for wind turbine placement in the region of Buenos
        Aires in Argentina. }\textit{\DIFaddFL{Panel (b)}} \DIFaddFL{depicts considered potential
        entry points for energy imports into Europe like the location of
        existing and planned LNG terminals and gas pipeline entry points, the
        lowest costs of hydrogen imports in different European regions
        (choropleth layer), and the considered connections for long-distance
        HVDC import links from the MENA region, Kazakhstan, Turkey and Ukraine.
        }\textit{\DIFaddFL{Panel (c)}} \DIFaddFL{displays the distribution and range of import costs
        for different energy carriers and entry points with indications for
        selected countries of origin from the TRACE model (violin charts),
        i.e.~differences in identically coloured markers are due to regional
        differences in the transport costs to entrypoints.}}
    \DIFaddendFL \label{fig:options}
\end{figure*}

%DIF <  TODO give sources for carrier statements
%DIF >  JH Draw connections such that they intersect less, e.g. MA-IE around ES, not
%DIF >  through ES and PT.

%DIF <  discussion of different import vectors
%DIF > %% technical discussion of import vectors %%%

As possible import options, we consider electricity by transmission line,
hydrogen as gas by pipeline and liquid by ship, methane as gas by pipeline and
liquid by ship, \DIFdelbegin \DIFdel{ammonia as liquid by ship, and }\DIFdelend \DIFaddbegin \DIFadd{liquid ammonia, }\DIFaddend steel, methanol and Fischer-Tropsch fuels by
ship. Each \DIFdelbegin \DIFdel{vector not only varies in European demand
levels but also presents unique characteristics }\DIFdelend \DIFaddbegin \DIFadd{energy vector has unique characteristics with regards to its
production, transport and consumption
(}\sfigref{fig:si:balances-a,fig:si:balances-b}\DIFadd{)}\DIFaddend . Electricity offers the most
flexible usage but is challenging to store and requires variability management
if sourced from wind or solar energy. Hydrogen is easier to store and transport
in large quantities but at the expense of conversion losses and less versatile
applications. \DIFaddbegin \DIFadd{Large quantities could be used for backup power and heat, steel
production, and the domestic synthesis of shipping and aviation fuels. }\DIFaddend Synthetic
carbonaceous fuels like methane, methanol and Fischer-Tropsch fuels \DIFdelbegin \DIFdel{are easy to store and transport and could benefit from
existing
infrastructure }\DIFdelend \DIFaddbegin \DIFadd{could
largely substitute the need for domestic synthesis. There is much more
experience with storing and transporting these fuels and part of the existing
infrastructure could potentially be leveraged}\DIFaddend . However, \DIFdelbegin \DIFdel{these }\DIFdelend \DIFaddbegin \DIFadd{they }\DIFaddend require a
sustainable carbon source and, particularly for methane, effective \DIFaddbegin \DIFadd{carbon
management and }\DIFaddend leakage prevention.\DIFaddbegin \DIFadd{\mbox{%DIFAUXCMD
\cite{shirizadehImpactMethaneLeakage2023}
}\hskip0pt%DIFAUXCMD
}\DIFaddend Ammonia is similarly \DIFdelbegin \DIFdel{straightforward to handle and, while not needing }\DIFdelend \DIFaddbegin \DIFadd{easier to handle than hydrogen but does not require }\DIFaddend a
carbon source\DIFdelbegin \DIFdel{, faces
acceptance issues }\DIFdelend \DIFaddbegin \DIFadd{. However, it faces safety and acceptance concerns }\DIFaddend due to its
toxicity \DIFaddbegin \DIFadd{and potentially adverse effects on the global nitrogen
cycle.\mbox{%DIFAUXCMD
\cite{bertagniMinimizingImpactsAmmonia2023,wolframUsingAmmoniaShipping2022}
}\hskip0pt%DIFAUXCMD
Its demand in Europe is mostly driven by fertiliser usage}\DIFaddend . Steel represents the
import of energy-intensive materials and offers low transport costs\DIFdelbegin \DIFdel{compared to its other
cost factors.
}\DIFdelend \DIFaddbegin \DIFadd{.
}

\DIFaddend Further conversion of imported fuels is also possible once they have arrived in
Europe, e.g.~hydrogen could be used to synthesise carbon-based fuels, \DIFdelbegin \DIFdel{and methane could be converted to hydrogen
}\DIFdelend \DIFaddbegin \DIFadd{ammonia
could be cracked to hydrogen, methane and methanol could be reformed to hydrogen
or combusted for power generation with or without carbon capture}\DIFaddend . However,
conversion losses \DIFaddbegin \DIFadd{can }\DIFaddend make it less \DIFdelbegin \DIFdel{likely }\DIFdelend \DIFaddbegin \DIFadd{attractive economically }\DIFaddend to use a high-value
hydrogen derivative merely as a transport \DIFaddbegin \DIFadd{and storage }\DIFaddend vessel only to reconvert
it back to hydrogen or electricity.

%DIF <  brief methodology of TRACE and PyPSA-Eur
%DIF > %% brief methodology PyPSA-Eur %%%

The PyPSA-Eur\cite{PyPSAEurSecSectorCoupled} model co-optimises the investment
and operation of generation, storage, conversion and transmission
infrastructures, as well as the relocation of some industries \DIFaddbegin \DIFadd{within
Europe}\DIFaddend ,\cite{verpoortEstimatingRenewables2023,samadiRenewablesPull2023} in a
single linear optimisation problem. We resolve 110 regions \DIFdelbegin \DIFdel{and use }\DIFdelend \DIFaddbegin \DIFadd{comprising the
European Union without Cyprus and Malta as well as the United Kingdom, Norway,
Switzerland, Albania, Bosnia and Herzegovina, Montenegro, North Macedonia,
Serbia, and Kosovo. In combination with }\DIFaddend a 4-hourly equivalent time resolution
for one year\DIFdelbegin \DIFdel{. Thereby}\DIFdelend , grid bottlenecks, renewable variability, and seasonal storage
requirements are efficiently captured. \DIFaddbegin \DIFadd{Weather variations between years are not
considered for computational reasons. }\DIFaddend The model includes regional demands from
the electricity, industry, buildings, agriculture and transport sectors,
international shipping and aviation, and non-energy feedstock demands in the
chemicals industry. Transmission infrastructure for electricity, gas and
hydrogen and candidate entry points like existing and prospective LNG terminals
and cross-continental pipelines are also represented. \DIFdelbegin \DIFdel{In the scenarios shown below, we enforce }\DIFdelend \DIFaddbegin \DIFadd{We utilize techno-economic
assumptions for 2030\mbox{%DIFAUXCMD
\cite{dea2019}}\hskip0pt%DIFAUXCMD
, reflecting that infrastructure required for
achieving carbon neutrality must be built well in advance of reaching this goal.
While enforcing }\DIFaddend net-zero emissions for carbon dioxide, \DIFdelbegin \DIFdel{take technology and assumptions for the year 2030,\mbox{%DIFAUXCMD
\cite{dea2019}
}\hskip0pt%DIFAUXCMD
and limit the }\DIFdelend \DIFaddbegin \DIFadd{we also limit the annual
}\DIFaddend carbon sequestration potential to 200~Mt$_{\text{CO}_2}$/a\DIFdelbegin \DIFdel{, which
}\DIFdelend \DIFaddbegin \DIFadd{. This }\DIFaddend suffices to
offset unabatable industrial process emissions \DIFaddbegin \DIFadd{of around
140~Mt$_{\text{CO}_2}$/a and limited use of fossil fuels beyond that, either
through capturing emissions at source or via carbon dioxide removal.
}

\DIFadd{More details are included in the }\nameref{sec:methods} \DIFadd{section}\DIFaddend .


%DIF >  \section*{Results}
%DIF >  \label{sec:results}
\DIFaddbegin 


\section*{\DIFadd{Cost assessment of energy and material import vectors}}

%DIF > %% brief methodology TRACE %%%

\DIFaddend Green fuel and steel import costs seen by the model are based on an extension of
recent research by Hampp et al.,\cite{hamppImportOptions2023} who assessed the
levelised cost of energy exports for different green energy and material supply
chains in various world regions (\cref{fig:options:global}). Our selection of
exporting countries comprises Australia, Argentina, Chile, Kazakhstan, Namibia,
Turkey, Ukraine, the Eastern United States and Canada, mainland China, and the
MENA region. Regional supply cost curves for these countries are developed based
on renewable resources, land availability and prioritised domestic demand.
\DIFdelbegin \DIFdel{Other
than }\DIFdelend \DIFaddbegin \DIFadd{Unlike }\DIFaddend domestic electrofuel synthesis in Europe, which could use captured CO$_2$
from point sources, direct air capture is assumed to be the \DIFdelbegin \DIFdel{sole }\DIFdelend \DIFaddbegin \DIFadd{only }\DIFaddend carbon source
of imported fuels. \DIFdelbegin \DIFdel{For hydrogen derivatives, the lowest-cost suppliers are Argentina and Chile.}\DIFdelend \DIFaddbegin \DIFadd{Concepts involving the recurrent shipment of captured CO$_2$
from Europe to exporting countries for carbonaceous fuel synthesis are not
considered.\mbox{%DIFAUXCMD
\cite{treeenergysolutionsGreenCycle2024,fonderSyntheticMethaneClosing2024}
}\hskip0pt%DIFAUXCMD
}

%DIF > %% brief methodology TRACE-PyPSA-Eur scoupling %%%
\DIFaddend 

We use these supply curves to determine the region-specific lowest import cost
for each carrier, thus incorporating the potential trade-off between import cost
and import location (\cref{fig:options:europe}). \DIFaddbegin \DIFadd{For hydrogen derivatives, the
lowest-cost suppliers are Argentina and Chile for all entry points into Europe.
}\DIFaddend Electricity imports are endogenously optimised, meaning that the capacities and
operation of wind and solar generation as well as storage in the respective
exporting countries and the HVDC transmission lines, are co-planned with the
rest of the system. Hydrogen and methane can be imported where there are
existing or planned LNG terminals or pipeline entry-points (excluding
connections through Russia). This results in lower hydrogen import costs, where
it can be imported by pipeline. \DIFdelbegin \DIFdel{Imports of ammonia}\DIFdelend \DIFaddbegin \DIFadd{Ammonia}\DIFaddend , carbonaceous fuels and steel are not
spatially resolved \DIFaddbegin \DIFadd{in the model}\DIFaddend , assuming they can be transported within Europe
at \DIFdelbegin \DIFdel{low }\DIFdelend \DIFaddbegin \DIFadd{negligible additional }\DIFaddend cost.

\DIFdelbegin \DIFdel{More details are included in the }%DIFDELCMD < \nameref{sec:methods} %%%
\DIFdel{section.
}%DIFDELCMD < 

%DIFDELCMD < %%%
%DIF <  unused text snippets
%DIFDELCMD < 

%DIFDELCMD < %%%
%DIF <  , alongside equal amounts of domestic hydrogen production (compared to hydrogen imports REPowerEU)
%DIFDELCMD < 

%DIFDELCMD < %%%
%DIF <  repel renewal of such dependencies with green fuels and goods
%DIFDELCMD < 

%DIFDELCMD < %%%
%DIF <  Furthermore, green hydrogen could offer
%DIF <  a replacement for hydrogen from fossil sources as a chemical feedstock in the
%DIF <  future.
%DIFDELCMD < 

%DIFDELCMD < %%%
%DIF <  It is not
%DIF <  a question of technical feasibility as the renewable potential to fully satisfy
%DIF <  its own energy demands would be sufficient.
%DIFDELCMD < 

%DIFDELCMD < %%%
%DIF <  how the different import scenarios affect the energy infrastructure inside
%DIF <  Europe.
%DIFDELCMD < 

%DIFDELCMD < %%%
%DIF <  which import routes are selected, as well as directions and magnitudes
%DIF <  of energy flows which have to be supported.
%DIFDELCMD < 

%DIFDELCMD < %%%
%DIF <  \section*{Results}
%DIF <  \label{sec:results}
%DIFDELCMD < 

%DIFDELCMD < %%%
%DIF <  TODO catchy section titles
%DIFDELCMD < 

%DIFDELCMD < %%%
\DIFdelend \begin{figure*}
    \includegraphics[width=\textwidth]{20231025-zecm/sensitivity-bars.pdf}
    \caption{\textbf{Potential for cost reductions with reduced sets of import options.}
        Subsets of available import options are sorted by ascending cost
        reduction potential. Top panel shows profile of total cost savings.
        Bottom panel shows composition and extent of imports in relation to
        total energy system costs. Percentage numbers in bar plot indicate the
        share of total system costs spent on domestic energy infrastructure.
        Alternative versions of this figure with higher and lower import cost
        assumptions are included in the supplementary material. }
    \label{fig:sensitivity-bars}
\end{figure*}

\section*{\DIFdelbegin \DIFdel{Combinations and volumes of }\DIFdelend \DIFaddbegin \DIFadd{Cost savings for }\DIFaddend fuel and material \DIFdelbegin \DIFdel{imports for highest cost savings}\DIFdelend \DIFaddbegin \DIFadd{import combinations}\DIFaddend }

In \cref{fig:sensitivity-bars}, we first explore the cost reduction potential of
various energy and material import options. \DIFdelbegin \DIFdel{Starting from total energy system
costs of }%DIFDELCMD < \bneuro{768} %%%
\DIFdel{in }\DIFdelend \DIFaddbegin \DIFadd{In }\DIFaddend the absence of energy imports,
\DIFdelbegin \DIFdel{we find that }\DIFdelend \DIFaddbegin \DIFadd{total energy system costs add up to }\bneuro{815}\footnote{\DIFadd{All currency values
are given in }\euro{}\DIFadd{$_{2020}$.}}\DIFadd{. By }\DIFaddend enabling imports from outside of Europe and
considering all import vectors\DIFdelbegin \DIFdel{can reduce
}\DIFdelend \DIFaddbegin \DIFadd{, we find a potential reduction of }\DIFaddend total energy
system costs by up to \DIFdelbegin %DIFDELCMD < \bneuro{37}%%%
\DIFdel{, which }\DIFdelend \DIFaddbegin \bneuro{39}\DIFadd{. This }\DIFaddend corresponds to a \DIFdelbegin \DIFdel{reduction
by }\DIFdelend \DIFaddbegin \DIFadd{relative reduction of
}\DIFaddend 4.9\%. \DIFdelbegin \DIFdel{In this case}\DIFdelend \DIFaddbegin \DIFadd{For cost-optimal imports}\DIFaddend , around 71\% of these costs are used to develop
domestic energy infrastructure. The remaining 29\% are spent on importing a
volume of 52~Mt of green steel and around 2700 TWh of green energy, which is
almost a quarter of the system's total energy supply \DIFaddbegin \DIFadd{(\mbox{%DIFAUXCMD
\cref{fig:import-shares}}\hskip0pt%DIFAUXCMD
)}\DIFaddend .

\DIFdelbegin \DIFdel{A more granular inspection of the results indicates varied cost savings when
only }\DIFdelend \DIFaddbegin \DIFadd{Next, we investigate the impact of restricting the available import options to
}\DIFaddend subsets of import \DIFdelbegin \DIFdel{options are available. The maximum }\DIFdelend \DIFaddbegin \DIFadd{vectors. We find that if only hydrogen can be imported, cost
}\DIFaddend savings are reduced to \bneuro{22} (\DIFdelbegin \DIFdel{3\%) if hydrogen is excluded from the available import options. This value is
similar to the savings of }%DIFDELCMD < \bneuro{21} %%%
\DIFdel{(2.8\%) achieved when only
hydrogen can be imported. Here}\DIFdelend \DIFaddbegin \DIFadd{2.8\%) but still substantial. This is
because by using hydrogen as intermediary carrier, low-cost renewable
electricity from abroad can still be leveraged for the synthesis of derivative
products in Europe. For this purpose}\DIFaddend , pipeline-based hydrogen imports are
preferred to \DIFaddbegin \DIFadd{ship-based }\DIFaddend imports as liquid\DIFdelbegin \DIFdel{fuel by ship. Focusing an import strategy }\DIFdelend \DIFaddbegin \DIFadd{. When direct hydrogen imports are
excluded from the available import options, cost savings are similar with
}\bneuro{23} \DIFadd{(3\%). Focusing imports }\DIFaddend exclusively on liquid carbonaceous
fuels derived from hydrogen, \DIFdelbegin \DIFdel{like }\DIFdelend methanol or Fischer-Tropsch fuels, consistently
achieves cost savings of \DIFdelbegin %DIFDELCMD < \bneuro{13-19}
%DIFDELCMD < %%%
\DIFdelend \DIFaddbegin \bneuro{14-20} \DIFaddend (1.7-2.5\%).

On the contrary, restricting options to only ammonia or methane imports yields
negligible cost savings\DIFdelbegin \DIFdel{and small savings below }%DIFDELCMD < \bneuro{4} %%%
\DIFdelend \DIFaddbegin \DIFadd{. Small savings below }\bneuro{5} \DIFaddend (0.6\%) \DIFaddbegin \DIFadd{can be reached
}\DIFaddend if only electricity or steel can be imported. This \DIFdelbegin \DIFdel{can be explained by the lower variety
and volume
of potential usage options compared to energy carriers higher
upstream in the value chain}\DIFdelend \DIFaddbegin \DIFadd{is due to the lower volume
and variety of usage options for ammonia, methane and steel compared to
hydrogen, methanol and Fischer-Tropsch fuels. Furthermore, the direct import of
electricity poses more challenges for system integration}\DIFaddend . Generally, our results
indicate a preference for \DIFdelbegin \DIFdel{chemicals, like methanoland hydrogen, and materials }\DIFdelend \DIFaddbegin \DIFadd{methanol, hydrogen and steel imports }\DIFaddend over electricity
imports, with a \DIFdelbegin \DIFdel{strategic mix of imports }\DIFdelend \DIFaddbegin \DIFadd{mix }\DIFaddend emerging as the most cost-effective approach.
\DIFdelbegin \DIFdel{For }\DIFdelend \DIFaddbegin \sfigref{fig:si:subsets} \DIFadd{show additional }\DIFaddend insights into how varying import costs
affect these findings\DIFdelbegin \DIFdel{, refer to
}%DIFDELCMD < \sfigref{fig:si:subsets}%%%
\DIFdelend .

\DIFdelbegin %DIFDELCMD < \begin{figure}[!htb]
%DIFDELCMD <     %%%
\DIFdelendFL \DIFaddbeginFL \begin{figure}%DIF > [!htb]
    \DIFaddendFL \includegraphics[width=\columnwidth]{20231025-zecm/graphics/import_shares/s_110_lvopt__Co2L0-2190SEG-T-H-B-I-S-A-onwind+p0.5-imp_2050.pdf}
    \caption{\textbf{Shares of imports and domestic production by carrier and optimised import carrier mix for import scenario with flexible carrier choice.}
        \DIFdelbeginFL \DIFdelFL{Figure }\DIFdelendFL \DIFaddbeginFL \DIFaddFL{The figure }\DIFaddendFL also shows total supply for each carrier. Import shares for
        further import scenarios are included in the supplementary material.
        Steel is included in energy terms applying 2.1 kWh/kg as released by the
        oxidation of iron. }
    \label{fig:import-shares}
\end{figure}

\DIFaddbegin \begin{figure*}
    \centering
    \footnotesize
    \DIFaddFL{(a) no imports allowed }\\
    \includegraphics[width=\textwidth]{20231025-zecm/market-values-noimp.pdf} \\
    \DIFaddFL{(b) only hydrogen imports allowed }\\
    \includegraphics[width=\textwidth, trim=0cm 0cm 0cm 1.5cm,
    clip]{20231025-zecm/market-values-imp+H2.pdf} \\
    \DIFaddFL{(c) all imports allowed }\\
    \includegraphics[width=\textwidth, trim=0cm 0cm 0cm 1.5cm, clip]{20231025-zecm/market-values-imp.pdf}
    \caption{\textbf{\DIFaddFL{Comparison of domestic production costs and import costs for varying import scenarios.}}
        \DIFaddFL{The three panels (a), (b), and (c) refer to different import scenarios.
        In each panel, the }\textit{\DIFaddFL{bar charts}} \DIFaddFL{show the production-weighted
        average costs of domestic production of steel, hydrogen and its
        derivatives split into its cost and revenue components. These have been
        computed using the marginal prices of the respective inputs and outputs
        for the production volume of each region and snapshot. Missing bars
        indicate that no domestic production occured in the scenario. For each
        bar, the yellow errorbars show the range of time-averaged domestic
        production costs across all regions. The black error bars show the range
        of import costs across all regions. The }\textit{\DIFaddFL{maps}} \DIFaddFL{relate the
        hydrogen production volume to the weighted cost of domestic hydrogen
        production (left colorbar). The }\textit{\DIFaddFL{time series}} \DIFaddFL{indicate the
        variance of the domestic production cost over time for hydrogen and
        Fischer-Tropsch fuel (FTF) including the regional spectrum as shaded area.}}
    \label{fig:market-values}
\end{figure*}

\DIFaddend \section*{\DIFdelbegin \DIFdel{Varying roles of imports }\DIFdelend \DIFaddbegin \DIFadd{Import dynamics }\DIFaddend for different energy carriers}

%DIF > %% state results %%%
\DIFaddbegin 

\DIFaddend \cref{fig:import-shares} outlines which carriers are imported in which
quantities in relation to their total supply \DIFaddbegin \DIFadd{under default assumptions }\DIFaddend when the
vector and volume can be flexibly chosen \DIFdelbegin \DIFdel{. Significantly, all methanol, which is used in shipping and industry, as well as }\DIFdelend \DIFaddbegin \DIFadd{(``all imports allowed'' in
\mbox{%DIFAUXCMD
\cref{fig:sensitivity-bars}}\hskip0pt%DIFAUXCMD
). In energy terms, cost-optimal imports comprise
around 50\% hydrogen, more than 20\% electricity, and an equal amount of
carbonaceous fuels. Noticeably, }\DIFaddend all crude steel \DIFdelbegin \DIFdel{, }\DIFdelend \DIFaddbegin \DIFadd{and methanol for shipping and
industry }\DIFaddend is imported. Also, around three-quarters of the total hydrogen supply
is imported. This is mainly done to subsequently process the hydrogen into
derivative products domestically rather than direct applications for hydrogen.
Smaller import shares are observed for electricity, Fischer-Tropsch fuels\DIFaddbegin \DIFadd{, }\DIFaddend and
ammonia, which are mostly domestically produced.
\DIFdelbegin \DIFdel{For
the example of ammonia, this can be explained by lower levelised cost for
domestic production of 69 }\DIFdelend \DIFaddbegin 

%DIF > %% explanation through market values %%%

\DIFadd{To explain the import shares in \mbox{%DIFAUXCMD
\cref{fig:import-shares} }\hskip0pt%DIFAUXCMD
in more detail, we
compare import costs with average domestic production cost split by cost and
revenue components in \mbox{%DIFAUXCMD
\cref{fig:market-values}}\hskip0pt%DIFAUXCMD
. First, for the scenario without
imports, imported fuel appear to be substantially cheaper than domestic
production. The high demand for hydrogen and derivative products
(}\sfigref{fig:si:balances-a,fig:si:balances-b}\DIFadd{) means that the most attractive
domestic potentials for renewable electricity and carbon dioxide have been
exhausted. Power from wind and solar needs to be produced in regions with worse
capacity factors and direct air capture becomes the price-setting technology for
\ce{CO2} as biogenic and industrial sources ($\approx$600~Mt$_\text{\ce{CO2}}$)
are depleted.
}

%DIF > %% hydrogen imports lower pressure on domestic supply chain %%%

\DIFadd{Part of this gap is closed when hydrogen imports are allowed. By sourcing
cheaper hydrogen from outside Europe, the domestic costs of derivative fuel
synthesis are reduced. This hybrid approach has the largest effect on
Fischer-Tropsch production due to its higher hydrogen demand compared to
methanolisation and the Haber-Bosch process. Hydrogen imports also decouple the
synthesis from the seasonal variation of domestic hydrogen production costs.
}

%DIF > %% waste heat integration %%%

\DIFadd{The potential for waste heat utilisation within Europe adds further appeal to
this hybrid approach. By importing hydrogen rather than the derivative product,
heat supply into district heating networks can create an additional revenue
stream of up to 10 }\euro{}\DIFadd{/MWh, depending on the process. Taking ammonia as
example, the levelised cost is of 73 }\DIFaddend \euro{}/MWh \DIFdelbegin \DIFdel{compared to 83 }\DIFdelend \DIFaddbegin \DIFadd{for domestic production compared
to 88 }\DIFaddend \euro{}/MWh for imported ammonia\DIFdelbegin \DIFdel{, which can be attributed to the additional revenue from waste heat
streams in the domestic supply chain }\DIFdelend .
\DIFdelbegin \DIFdel{In energy terms, cost-optimal importscomprise around 50\% hydrogen , more than 20\% electricity and
an equal amount of
carbonaceous fuels. It should be noted, however,
that different carriers have
undergone further conversion steps abroad with energy losses and
that the
cost-optimal import mix also depends on the assumed import costs. }\DIFdelend \DIFaddbegin 

\DIFadd{The waste heat integration is also the reason why in \mbox{%DIFAUXCMD
\cref{fig:import-shares}}\hskip0pt%DIFAUXCMD
,
with all import vectors allowed, all methanol is imported, whereas
Fischer-Tropsch fuels and ammonia are produced mainly domestically using high
shares of imported hydrogen. Because the thermal discharge from the methanol
synthesis is primarily used for the distillation of the methanol-water output
mix, its waste heat potential is considered much lower compared to
Fischer-Tropsch, Haber-Bosch and Sabatier processes. Therefore, it is less
attractive to retain this part of the value chain within Europe.
}

%DIF > %% explanations of low cost differences if all imports allowed %%%

\DIFadd{With all import vectors allowed, we see minimal cost differences between
domestic production and imports. This is because imports of hydrogen and
derivative products lower the strain on the domestic supply chain. Thereby,
domestic production would only be ramped up where it competes with imports and
associated infrastructure costs. Such regions exist in Southern Europe or the
British Isles and, therefore, not all hydrogen is imported
(}\sfigref{fig:si:cost-supply-curves}\DIFadd{).
}\DIFaddend 

\begin{figure*}
    \begin{subfigure}[t]{\columnwidth}
        \caption{cost reductions applied to all carriers but electricity}
        \label{fig:sensitivity-costs:A}
        \includegraphics[width=\columnwidth]{20231025-zecm/sensitivity-bars-all.pdf}
    \end{subfigure}
    \begin{subfigure}[t]{\columnwidth}
        \caption{cost reductions only applied to carbonaceous fuels}
        \label{fig:sensitivity-costs:B}
        %DIF <  \includegraphics[trim=0cm 0cm 0cm 3.4cm, clip, width=\columnwidth]{20231025-zecm/sensitivity-bars-all-C.pdf}
        \includegraphics[width=\columnwidth]{20231025-zecm/sensitivity-bars-all-C.pdf}
    \end{subfigure}
    \caption{\DIFdelbeginFL \textbf{\DIFdelFL{Effect of import cost variations on cost savings and import shares.}}
    %DIFAUXCMD
\DIFdelendFL \DIFaddbeginFL \textbf{\DIFaddFL{Effect of import cost variations on cost savings and import shares with all vectors allowed.}}
    \DIFaddendFL In panel (a), indicated relative cost changes are applied uniformly to all
    vectors but electricity imports. In panel (b), cost changes are only applied
    to carbonaceous fuels (methane, methanol and Fischer-Tropsch). Top subpanels
    show potential cost savings compared to the scenario \DIFdelbeginFL \DIFdelFL{with full self-sufficiency}\DIFdelendFL \DIFaddbeginFL \DIFaddFL{without imports}\DIFaddendFL . Bottom
    subpanels show the share and composition of different import vectors in
    relation to total energy system costs. \DIFaddbeginFL \DIFaddFL{The information is shown both in
    absolute terms and relative terms compared to the scenario without imports.
    }\DIFaddendFL }
    \label{fig:sensitivity-costs}
\end{figure*}

\section*{Sensitivity of potential cost savings to \DIFdelbegin \DIFdel{unit }\DIFdelend \DIFaddbegin \DIFadd{import }\DIFaddend costs\DIFdelbegin \DIFdel{of energy imports}\DIFdelend }

\DIFdelbegin \DIFdel{This uncertainty of estimates for import costs}\DIFdelend \DIFaddbegin \DIFadd{It should be noted, however, that the cost-optimal import mix also strongly
depends on the assumed import costs. This uncertainty }\DIFaddend is addressed in
\cref{fig:sensitivity-costs}. \cref{fig:sensitivity-costs:A} highlights the
extensive range in potential cost reductions if higher or lower import costs
could be attained \DIFdelbegin \DIFdel{as well as }\DIFdelend \DIFaddbegin \DIFadd{and underlines }\DIFaddend the resulting variance in cost-effective import
\DIFdelbegin \DIFdel{blends}\DIFdelend \DIFaddbegin \DIFadd{mixes}\DIFaddend . Within $\pm 20\%$ of the default import costs \DIFdelbegin \DIFdel{previously presented
}\DIFdelend applied to all carriers but
electricity, total cost savings vary between \DIFdelbegin %DIFDELCMD < \bneuro{12} %%%
\DIFdelend \DIFaddbegin \bneuro{13} \DIFaddend (1.6\%) and \DIFdelbegin %DIFDELCMD < \bneuro{78} %%%
\DIFdelend \DIFaddbegin \bneuro{83}
\DIFaddend (10.2\%)\DIFdelbegin \DIFdel{with import volumes ranging }\DIFdelend \DIFaddbegin \DIFadd{. Within this range, import volumes vary }\DIFaddend between 1700 and 3800~TWh.
Across most \DIFdelbegin \DIFdel{of these }\DIFdelend scenarios, there is a stable role for methanol, steel, electricity
and hydrogen imports. One significant difference\DIFdelbegin \DIFdel{is
the appearance of }\DIFdelend \DIFaddbegin \DIFadd{, however, are }\DIFaddend Fischer-Tropsch
fuel imports starting from cost reductions of 10\% \DIFaddbegin \DIFadd{and their absence at cost
increases beyond 10\%}\DIFaddend .

Some potential causes for such cost variations are presented in
\DIFdelbegin \DIFdel{\mbox{%DIFAUXCMD
\cref{tab:cost-uncertainty} }\hskip0pt%DIFAUXCMD
and discussed later}\DIFdelend \DIFaddbegin \stabref{tab:cost-uncertainty}\DIFaddend . Some of these only affect the cost of
carbonaceous fuels. One central assumption for carbon-based fuels is that
imported fuels rely exclusively on direct air capture (DAC) as a carbon source.
\DIFdelbegin \DIFdel{For example, arguments }\DIFdelend \DIFaddbegin \DIFadd{Arguments }\DIFaddend for this assumption relate to the potential remoteness of the ideal
locations for renewable fuel production or the absence of industrial point
sources in the exporting country. On the other hand, domestic electrofuels can
mostly use less expensive captured carbon dioxide from industrial point sources
or biogenic origin. Therefore, the higher cost for DAC partially cancels out the
savings from utilising better renewable resources abroad\DIFdelbegin \DIFdel{, which }\DIFdelend \DIFaddbegin \DIFadd{. This }\DIFaddend is one of the
reasons why there is substantial power-to-X production in Europe, even with
corresponding import options. The availability of cheaper biogenic CO$_2$ in the
exporting country, for instance, would \DIFdelbegin \DIFdel{inevitably }\DIFdelend make importing carbonaceous fuels more
attractive \DIFaddbegin \DIFadd{(\mbox{%DIFAUXCMD
\cref{tab:cost-uncertainty}}\hskip0pt%DIFAUXCMD
)}\DIFaddend .

When the relative cost variation of $\pm 20\%$ is only applied to carbon-based
fuels (\cref{fig:sensitivity-costs:B}), hydrogen imports are increasingly
displaced by \DIFdelbegin \DIFdel{green }\DIFdelend methane and Fischer-Tropsch imports with \DIFdelbegin \DIFdel{lower costs, but }\DIFdelend \DIFaddbegin \DIFadd{falling costs. However, }\DIFaddend it
takes a cost increase of more than 10\% for domestic methanol production to
become more cost-effective than \DIFdelbegin \DIFdel{green }\DIFdelend methanol imports. \DIFaddbegin \DIFadd{This underlines the robust
benefit of importing methanol.
}\DIFaddend 

%DIF <  import volumes in cost optimum
%DIF <  +20%: 1705 TWh
%DIF <  +10%: 2391 TWh
%DIF <  +-0%: 2800 TWh
%DIF <  -10%: 3121 TWh
%DIF <  -20%: 3412 TWh
%DIF <  -30%: 3766 TWh
%DIF >  FN import volumes in cost optimum +20%: 1705 TWh +10%: 2391 TWh +-0%: 2800 TWh
%DIF >  -10%: 3121 TWh -20%: 3412 TWh -30%: 3766 TWh

\begin{figure*}
    \includegraphics[width=\textwidth]{20231025-zecm/sensitivity-import-volume-any.pdf}
    \caption{\textbf{Sensitivity of import volume on total system cost and composition.}
        \DIFdelbeginFL \DIFdelFL{Dashed }\DIFdelendFL \DIFaddbeginFL \DIFaddFL{The dashed }\DIFaddendFL line splits total system cost into domestic and foreign cost.
        Dotted lines indicate the profile of lowest total system cost attainable
        for given import volumes and different levels of import costs. \DIFdelbeginFL \DIFdelFL{Markers }\DIFdelendFL \DIFaddbeginFL \DIFaddFL{The black
        markers }\DIFaddendFL denote the maximum cost reductions and cost-optimal import
        volume for a given import cost level (extreme points of the profiles).
        Steel is included in energy terms applying 2.1 kWh/kg as released by the
        oxidation of iron. Cost alterations are uniformly applied to all \DIFdelbeginFL \DIFdelFL{carriers }\DIFdelendFL \DIFaddbeginFL \DIFaddFL{imports
        opotions }\DIFaddendFL but \DIFaddbeginFL \DIFaddFL{direct }\DIFaddendFL electricity \DIFaddbeginFL \DIFaddFL{imports}\DIFaddendFL . }
    \label{fig:sensitivity-volume}
\end{figure*}

\section*{Attainable cost savings for varying import volumes}

What is consistent across all import cost variations is the flat solution space
around the respective cost-optimal import volumes. Increasing or decreasing the
total amount of imports barely affects system costs within $\pm 1000$ TWh. This
is illustrated in \cref{fig:sensitivity-volume}, which shows the possible cost
reductions as a function of \DIFaddbegin \DIFadd{enforced }\DIFaddend import volumes. A wide range \DIFdelbegin \DIFdel{of }\DIFdelend \DIFaddbegin \DIFadd{scenarios with
}\DIFaddend import volumes below 5600~TWh (\DIFdelbegin \DIFdel{4000-7500~TWh within $\pm$}\DIFdelend \DIFaddbegin \DIFadd{4000~TWh with +}\DIFaddend 20\% \DIFaddbegin \DIFadd{and 7500~TWh with -20\%
}\DIFaddend import costs) have lower \DIFaddbegin \DIFadd{total energy system }\DIFaddend costs than the scenario without any
imports. These \DIFdelbegin \DIFdel{values are equivalent to just more than }\DIFdelend \DIFaddbegin \DIFadd{import volumes are roughly }\DIFaddend twice the cost-optimal \DIFaddbegin \DIFadd{import }\DIFaddend volumes,
which are indicated by the black markers \DIFaddbegin \DIFadd{in \mbox{%DIFAUXCMD
\cref{fig:sensitivity-volume} }\hskip0pt%DIFAUXCMD
}\DIFaddend and
correspond to the bars previously shown in \cref{fig:sensitivity-costs:A}.
Naturally, the cost-optimal volume of imports increases as their costs decrease,
but the response weakens with lower import costs.

As we explore the effect of increasing import volumes on system costs, we find
that already 43\% (36-61\% within $\pm$20\% import costs) of the 4.9\%
(1.6-10.2\%) total cost benefit (\DIFdelbegin %DIFDELCMD < \bneuro{16}%%%
\DIFdelend \DIFaddbegin \bneuro{17}\DIFaddend ) can be achieved with the first 500
TWh of imports. This corresponds to only 18\% (15-29\%) of the cost-optimal
import volumes, highlighting the diminishing return of large amounts of energy
imports over domestic production in Europe. While importing 1000 TWh already
realises 70\% of the maximum cost savings with our default assumptions, this
maximum is only obtained for 2800~TWh of imports. For these initial 1000~TWh,
primary crude steel and methanol imports are prioritised, followed by hydrogen
and, subsequently, electricity beyond 2000 TWh. Once more than 5000~TWh are
imported, less than half the total system cost would be spent on domestic energy
infrastructure.

As imports increase, there is a corresponding decrease in the need for domestic
power-to-X (PtX) production and renewable capacities. A large share of the
hydrogen, methanol and raw steel production is outsourced from Europe, reducing
\DIFaddbegin \DIFadd{the need for }\DIFaddend domestic wind and solar capacities. This trend is further
characterised by the displacement of biogas usage in favour of hydrogen imports
around the 2000~TWh mark \DIFaddbegin \DIFadd{as demand for domestic \ce{CO2} utilisation drops}\DIFaddend . An
increase in the amount of hydrogen imported coincides with an increasing use of
hydrogen fuel cells for electricity and central heat supply in district heating
networks, partially displacing the use of methane. Regarding electricity imports
\DIFaddbegin \DIFadd{from the MENA region}\DIFaddend , \cref{fig:sensitivity-volume} reveals a \DIFdelbegin \DIFdel{predominant reliance on wind-generated electricity over solar for the imports via HVDC links from the MENA region. A }\DIFdelend \DIFaddbegin \DIFadd{mix of wind and
solar power to establish favourable feed-in profiles for the European system
integration and higher utilisation rates for the long-distance HVDC links with a
capacity-weighted average of 72\%. This is relevant since a }\DIFaddend considerable share
of the \DIFdelbegin \DIFdel{costs }\DIFdelend \DIFaddbegin \DIFadd{import costs of electricity }\DIFaddend can be attributed to \DIFdelbegin \DIFdel{realising long-distance }\DIFdelend power transmission.

%DIF <  Furthermore, large amounts of electricity imports are supported by
%DIF <  an increase in battery deployment to manage the imported power more effectively.
%DIF >  FN LCOE for solar and wind in MENA are mostly similar potentials are not
%DIF >  constraining DZ: solar 34 €/MWh, wind 32 €/MWh TN: solar 30 €/MWh, wind 38
%DIF >  €/MWh LY: solar 24 €/MWh, wind 37 €/MWh

For deviating levels of import costs, the composition of the domestic system and
import mix is primarily similar (\sfigref{fig:si:volume}). The main differences
are a more prominent role for Fischer-Tropsch fuel imports with lower import
costs and green methane for high import volumes. It should also be noted that
the windows for cost savings are much smaller if only subsets of import options
are \DIFdelbegin \DIFdel{considered }\DIFdelend \DIFaddbegin \DIFadd{available }\DIFaddend (\sfigref{fig:si:volume-subsets}). However, up to an import volume
of 2000~TWh, excluding electricity imports \DIFdelbegin \DIFdel{does }\DIFdelend \DIFaddbegin \DIFadd{would }\DIFaddend not diminish the cost-saving
potential substantially.

\begin{figure*}
    \includegraphics[width=\textwidth]{20231025-zecm/infrastructure-map-2x3-A.pdf}
    \caption{\textbf{Layout of European energy infrastructure for different import scenarios.}
        Left column shows the regional electricity supply mix (pies), added HVDC
        and HVAC transmission capacity (lines), and the siting of battery
        storage (choropleth). Right column shows the hydrogen supply (top half
        of pies) and consumption (bottom half of pies), net flow and direction
        of hydrogen in newly built pipelines (lines), and the siting of hydrogen
        storage subject to geological potentials (choropleth). Total volumes of
        transmission expansion are given in TWkm, which is the sum product of
        the capacity and length of individual connections. \DIFaddbeginFL \DIFaddFL{The half circle in
        the Bay of Biscay indicates the imports of carriers that are not
        spatially resolved: ammonia, steel, methanol, Fischer-Tropsch fuels.
        Maps for more scenarios are included in the supplementary material. }\DIFaddendFL }
    \label{fig:import-infrastructure}
\end{figure*}

\section*{\DIFdelbegin \DIFdel{Interdependence }\DIFdelend \DIFaddbegin \DIFadd{Interactions }\DIFaddend of import strategy \DIFdelbegin \DIFdel{and }\DIFdelend \DIFaddbegin \DIFadd{\& }\DIFaddend domestic \DIFdelbegin \DIFdel{energy infrastructures}\DIFdelend \DIFaddbegin \DIFadd{infrastructure}\DIFaddend }

Across the range of import scenarios analysed, we find that the \DIFdelbegin \DIFdel{strategy taken
on imports }\DIFdelend \DIFaddbegin \DIFadd{decision which
import vectors are used }\DIFaddend strongly affects domestic energy infrastructure needs
(\cref{fig:import-infrastructure}).

%DIF <  self-sufficiency
%DIF > %% self-sufficiency %%%

In the fully self-sufficient European energy supply scenario, we see large
\mbox{power-to-X} production within Europe to cover the demand for hydrogen \DIFaddbegin \DIFadd{and
hydrogen }\DIFaddend derivatives in steelmaking, high-value chemicals, green shipping and
aviation fuels. Production sites are concentrated in Southern Europe for
solar-based electrolysis and the broader North Sea region for wind-based
electrolysis. The steel and ammonia industries relocate to the periphery of
Europe in Spain and Scotland, where hydrogen is cheap and abundant. Electricity
grid reinforcements are focused in Northwestern Europe and long-distance HVDC
connections but are broadly distributed overall. Hydrogen pipeline build-out is
strongest in Spain and France to transport hydrogen from the Southern production
hubs to chosen fuel synthesis sites. Most of these pipelines are used
unidirectionally, with bidirectional usage where pipelines link hydrogen
production and low-cost geological storage sites (for instance, between Greece
and Italy and Southern Spain).
%DIF <  waste heat discussion comes later  

%DIF <  flexible imports
%DIF > %% flexible imports %%%

Considering imports of renewable electricity, green hydrogen, and electrofuels
substantially alters the infrastructure buildout in Europe. Imports displace
much of the European power-to-X production capacities and, particularly,
domestic solar energy generation in Southern Europe. In contrast, the British
Isles retain some domestic electrolyser capacities to produce synthetic methane
locally\DIFaddbegin \DIFadd{, also leveraging the Sabatier process's waste heat}\DIFaddend . The electricity
imports are distributed evenly between the North African countries Algeria,
Libya, and Tunisia and across multiple entry points in Spain, France, Italy and
Greece. This facilitates grid integration without strong reinforcement needs.
Electricity imports are also optimised to achieve higher utilisation rates above
\DIFdelbegin \DIFdel{60}\DIFdelend \DIFaddbegin \DIFadd{70}\DIFaddend \% for the HVDC import connections. This is realised by mixing wind and solar
generation for seasonal balancing and using some batteries for short-term
storage (\sfigref{fig:si:import-operation}).

While the amount and locations of domestic power grid reinforcements are not
significantly affected by the import of electricity and other fuels, the extent
of the hydrogen network is halved \DIFaddbegin \DIFadd{and its routing is significantly altered}\DIFaddend .
Compared to the self-sufficiency scenario, the cost-benefit of the hydrogen
network shrinks from \DIFdelbegin %DIFDELCMD < \bneuro{10} %%%
\DIFdelend \DIFaddbegin \bneuro{11} \DIFaddend (1.3\%) to \bneuro{3} (0.4\%). This is caused
by substantial amounts of methanol imports that diminish the demand for hydrogen
in Europe and, hence, the need to transport it. In combination with the steel
and ammonia industry relocation, longer hydrogen pipeline connections are then
predominantly built to meet hydrogen CHP demands to bring electricity and heat
to renewables-poor and grid-poor regions in Eastern Europe and Germany.
Moreover, the hydrogen network helps absorb inbound hydrogen in South and
Southeast Europe, transporting some hydrogen\DIFaddbegin \DIFadd{, which is }\DIFaddend not directly used for
fuel synthesis at the entry points\DIFaddbegin \DIFadd{, }\DIFaddend to neighbouring regions.

%DIF <  overarching trends
%DIF > %% overarching trends %%%

A further observation is the high value of power-to-X production for system
integration and the role of waste heat in the siting of fuel synthesis plants
(\sfigref{fig:si:infra-b}). Using the process waste heat in district heating
networks with seasonal thermal storage generates notable cost savings of
\DIFdelbegin %DIFDELCMD < \bneuro{10-20} %%%
\DIFdelend \DIFaddbegin \bneuro{11-21} \DIFaddend (1.3-2.6\%)\DIFdelbegin \DIFdel{, with lower savings }\DIFdelend \DIFaddbegin \DIFadd{. Consequently, savings are lower }\DIFaddend when imports
displace domestic PtX infrastructure. To realise these benefits, PtX facilities
tend to be located in densely populated areas (e.g.~Paris or Hamburg), which
drives part of the the hydrogen network. \DIFaddbegin \DIFadd{Notably, because we expect lower waste
heat potential for methanolisation compared to other power-to-X processes,
Fischer-Tropsch and Sabatier plants tend to locate where space heating demand is
higher. }\DIFaddend Alongside the flexible operation of electrolysis to integrate variable
wind and solar feed-in and the broad availability of industrial and biogenic
carbon sources in Europe, waste heat usage is a key factor that \DIFdelbegin \DIFdel{raises the attractiveness of }\DIFdelend \DIFaddbegin \DIFadd{makes
}\DIFaddend electricity and hydrogen imports with subsequent domestic conversion \DIFaddbegin \DIFadd{more
attractive }\DIFaddend relative to the direct import of derivative products. Infrastructure
layouts for further import scenarios are presented in \sfigref{fig:si:infra-b}
to \sfigref{fig:si:infra-d}.


\DIFdelbegin \section*{\DIFdel{Causes of import cost uncertainty and their severity}}
%DIFAUXCMD
%DIFDELCMD < 

%DIFDELCMD < %%%
\DIFdel{In \mbox{%DIFAUXCMD
\cref{tab:cost-uncertainty}}\hskip0pt%DIFAUXCMD
, we vary some of the techno-economic assumptions
for evaluating green fuel supply chains in the exporting countries to justify
the range of import cost deviations from the defaults. These relate to
technology costs, financing costs, excess power and heat handling, fuel
synthesis flexibility, and the availability of geological hydrogen storage and
alternative sources of CO$_2$. For all the following sensitivities, it should be
noted that they may not be additive owing to correlation.
}%DIFDELCMD < 

%DIFDELCMD < %%%
\DIFdel{A higher weighted average cost of capital (WACC) in developing or economically
unstable countries than the uniformly applied 7\%, e.g.~due to higher project
risks, and lower WACC, e.g.~due to the government-backing of a project, have a
substantial effect on the import cost calculations; an increase or decrease by
just one percentage point already alters the costs per unit of energy by more
than
7\%.\mbox{%DIFAUXCMD
\cite{egliBiasEnergy2019,bogdanovReplyBias2019,lonerganImprovingRepresentation2023a,schyskaHowRegional2020,steffenDeterminantsCost2022}
}\hskip0pt%DIFAUXCMD
}%DIFDELCMD < 

%DIFDELCMD < %%%
\DIFdel{Likewise, a failure to achieve the anticipated cost reductions for electrolysers
and DAC systems would also result in far-reaching cost increases for green
energy imports, especially if the fuel contains carbon. The availability of
biogenic CO$_2$ (or CO$_2$ from industrial processes that is largely cycled
between use and synthesis and, hence, not emitted to the atmosphere) can reduce
the green fuel cost by 20\% if it can be provided for 60 }%DIFDELCMD < \euro{}%%%
\DIFdel{/t and by 10\%
if made available for 100 }%DIFDELCMD < \euro{}%%%
\DIFdel{/t.
}%DIFDELCMD < 

%DIFDELCMD < %%%
\DIFdel{The default assumptions for export supply chains assume islanded fuel synthesis
sites that are not connected to the local electricity system. The isolation in
consideration of the investment costs of the various components drives the
system into high curtailment rates of 8\%. If surplus electricity production
could be sold and absorbed by the local power grid, considerable cost reductions
could be achieved depending on the average selling price, reducing curtailment
(refer to \mbox{%DIFAUXCMD
\cref{tab:cost-uncertainty}}\hskip0pt%DIFAUXCMD
). 
}%DIFDELCMD < 

%DIFDELCMD < %%%
\DIFdel{Besides integration with the local energy system, process integration using
waste heat streams from power-to-X plants for direct air capture and flexible
Fischer-Tropsch synthesis similar to methanolisation can also reduce fuel cost
by 3-5\% each. Conditions that would allow for geological hydrogen storage
reduce the need for flexible synthesis plant operation and could reduce import
costs by more than 7\% %DIF < . However, even though many countries considered to be
exporting possess geological hydrogen storage potential, the sites are not
always co-located with the countries' best renewable potentials.
\mbox{%DIFAUXCMD
\cite{hevinUndergroundStorage2019}
}\hskip0pt%DIFAUXCMD
}%DIFDELCMD < 

%DIFDELCMD < %%%
\DIFdel{Finally, cost rises can also be expected if the most competitive exporting
countries are not offering to export green energy. Argentina and Chile have a
margin of 9.5 }%DIFDELCMD < \euro{}%%%
\DIFdel{/MWh over the next cheapest exporting country
(i.e.~Australia, Algeria and Libya with 113 }%DIFDELCMD < \euro{}%%%
\DIFdel{/MWh). If these countries
were unavailable for import, costs would rise by almost 10\%.
}%DIFDELCMD < 

%DIFDELCMD < \begin{table*}
%DIFDELCMD <     \small
%DIFDELCMD <     \centering
%DIFDELCMD <     \begin{tabular}{lrrrr}
%DIFDELCMD <         \toprule
%DIFDELCMD <         %%%
\DIFdelFL{Factor }%DIFDELCMD < & %%%
\DIFdelFL{Change }%DIFDELCMD < & %%%
\DIFdelFL{Unit }%DIFDELCMD < & %%%
\DIFdelFL{Change }%DIFDELCMD < & %%%
\DIFdelFL{Unit}%DIFDELCMD < \\
%DIFDELCMD <         \midrule
%DIFDELCMD <         %%%
\DIFdelFL{higher WACC of 12\% (e.g.~high project risk) }%DIFDELCMD < & %%%
\DIFdelFL{+40.6 }%DIFDELCMD < & \euro{}%%%
\DIFdelFL{/MWh  }%DIFDELCMD < & %%%
\DIFdelFL{+39.3 }%DIFDELCMD < & %%%
\DIFdelFL{\% }%DIFDELCMD < \\
%DIFDELCMD <         %%%
\DIFdelFL{higher WACC of 10\% (e.g.~high project risk) }%DIFDELCMD < & %%%
\DIFdelFL{+23.8 }%DIFDELCMD < & \euro{}%%%
\DIFdelFL{/MWh  }%DIFDELCMD < & %%%
\DIFdelFL{+23.0 }%DIFDELCMD < & %%%
\DIFdelFL{\% }%DIFDELCMD < \\
%DIFDELCMD <         %%%
\DIFdelFL{higher WACC of 8\% (e.g.~high project risk) }%DIFDELCMD < & %%%
\DIFdelFL{+7.7 }%DIFDELCMD < & \euro{}%%%
\DIFdelFL{/MWh  }%DIFDELCMD < & %%%
\DIFdelFL{+7.4 }%DIFDELCMD < & %%%
\DIFdelFL{\% }%DIFDELCMD < \\
%DIFDELCMD <         %%%
\DIFdelFL{higher direct air capture investment cost (+200\%) }%DIFDELCMD < & %%%
\DIFdelFL{+52.6 }%DIFDELCMD < & \euro{}%%%
\DIFdelFL{/MWh  }%DIFDELCMD < & %%%
\DIFdelFL{+50.8 }%DIFDELCMD < & %%%
\DIFdelFL{\% }%DIFDELCMD < \\
%DIFDELCMD <         %%%
\DIFdelFL{higher direct air capture investment cost (+100\%) }%DIFDELCMD < & %%%
\DIFdelFL{+26.5 }%DIFDELCMD < & \euro{}%%%
\DIFdelFL{/MWh  }%DIFDELCMD < & %%%
\DIFdelFL{+25.6 }%DIFDELCMD < & %%%
\DIFdelFL{\% }%DIFDELCMD < \\
%DIFDELCMD <         %%%
\DIFdelFL{higher direct air capture investment cost (+50\%) }%DIFDELCMD < & %%%
\DIFdelFL{+13.3 }%DIFDELCMD < & \euro{}%%%
\DIFdelFL{/MWh  }%DIFDELCMD < & %%%
\DIFdelFL{+12.9 }%DIFDELCMD < & %%%
\DIFdelFL{\% }%DIFDELCMD < \\
%DIFDELCMD <         %%%
\DIFdelFL{higher direct air capture investment cost (+25\%) }%DIFDELCMD < & %%%
\DIFdelFL{+6.7 }%DIFDELCMD < & \euro{}%%%
\DIFdelFL{/MWh  }%DIFDELCMD < & %%%
\DIFdelFL{+6.5 }%DIFDELCMD < & %%%
\DIFdelFL{\% }%DIFDELCMD < \\
%DIFDELCMD <         %%%
\DIFdelFL{higher electrolysis investment cost (+200\%) }%DIFDELCMD < & %%%
\DIFdelFL{+27.5 }%DIFDELCMD < & \euro{}%%%
\DIFdelFL{/MWh  }%DIFDELCMD < & %%%
\DIFdelFL{+26.6 }%DIFDELCMD < & %%%
\DIFdelFL{\% }%DIFDELCMD < \\
%DIFDELCMD <         %%%
\DIFdelFL{higher electrolysis investment cost (+100\%) }%DIFDELCMD < & %%%
\DIFdelFL{+15.7 }%DIFDELCMD < & \euro{}%%%
\DIFdelFL{/MWh  }%DIFDELCMD < & %%%
\DIFdelFL{+15.2 }%DIFDELCMD < & %%%
\DIFdelFL{\% }%DIFDELCMD < \\
%DIFDELCMD <         %%%
\DIFdelFL{higher electrolysis investment cost (+50\%) }%DIFDELCMD < & %%%
\DIFdelFL{+8.5 }%DIFDELCMD < & \euro{}%%%
\DIFdelFL{/MWh  }%DIFDELCMD < & %%%
\DIFdelFL{+8.2 }%DIFDELCMD < & %%%
\DIFdelFL{\% }%DIFDELCMD < \\
%DIFDELCMD <         %%%
\DIFdelFL{higher electrolysis investment cost (+25\%) }%DIFDELCMD < & %%%
\DIFdelFL{+4.4 }%DIFDELCMD < & \euro{}%%%
\DIFdelFL{/MWh  }%DIFDELCMD < & %%%
\DIFdelFL{+4.3 }%DIFDELCMD < & %%%
\DIFdelFL{\% }%DIFDELCMD < \\
%DIFDELCMD <         %%%
\DIFdelFL{Argentina and Chile not available for export }%DIFDELCMD < & %%%
\DIFdelFL{+9.5 }%DIFDELCMD < & \euro{}%%%
\DIFdelFL{/MWh  }%DIFDELCMD < & %%%
\DIFdelFL{+9.2 }%DIFDELCMD < & %%%
\DIFdelFL{\% }%DIFDELCMD < \\
%DIFDELCMD <         %%%
%DIF <  pay border adjustment tax of X \euro{}/t for methane leakage$^\star$ & +? & \euro{}/MWh  & +? & \% \\
        %DIFDELCMD < \midrule
%DIFDELCMD <         %%%
\DIFdelFL{lower WACC of 3\% (e.g.~government guarantees) }%DIFDELCMD < & %%%
\DIFdelFL{-27.8 }%DIFDELCMD < & \euro{}%%%
\DIFdelFL{/MWh  }%DIFDELCMD < & %%%
\DIFdelFL{-26.8 }%DIFDELCMD < & %%%
\DIFdelFL{\% }%DIFDELCMD < \\
%DIFDELCMD <         %%%
\DIFdelFL{lower WACC of 5\% (e.g.~government guarantees) }%DIFDELCMD < & %%%
\DIFdelFL{-14.6 }%DIFDELCMD < & \euro{}%%%
\DIFdelFL{/MWh  }%DIFDELCMD < & %%%
\DIFdelFL{-14.1 }%DIFDELCMD < & %%%
\DIFdelFL{\% }%DIFDELCMD < \\
%DIFDELCMD <         %%%
\DIFdelFL{lower WACC of 6\% (e.g.~government guarantees) }%DIFDELCMD < & %%%
\DIFdelFL{-7.5 }%DIFDELCMD < & \euro{}%%%
\DIFdelFL{/MWh  }%DIFDELCMD < & %%%
\DIFdelFL{-7.2 }%DIFDELCMD < & %%%
\DIFdelFL{\% }%DIFDELCMD < \\
%DIFDELCMD <         %%%
\DIFdelFL{sell excess curtailed electricity at 40€/MWh }%DIFDELCMD < & %%%
\DIFdelFL{-23.3 }%DIFDELCMD < & \euro{}%%%
\DIFdelFL{/MWh  }%DIFDELCMD < & %%%
\DIFdelFL{-22.6 }%DIFDELCMD < & %%%
\DIFdelFL{\% }%DIFDELCMD < \\
%DIFDELCMD <         %%%
\DIFdelFL{sell excess curtailed electricity at 30€/MWh }%DIFDELCMD < & %%%
\DIFdelFL{-14.7 }%DIFDELCMD < & \euro{}%%%
\DIFdelFL{/MWh  }%DIFDELCMD < & %%%
\DIFdelFL{-14.2 }%DIFDELCMD < & %%%
\DIFdelFL{\% }%DIFDELCMD < \\
%DIFDELCMD <         %%%
\DIFdelFL{sell excess curtailed electricity at 20€/MWh }%DIFDELCMD < & %%%
\DIFdelFL{-7.5 }%DIFDELCMD < & \euro{}%%%
\DIFdelFL{/MWh  }%DIFDELCMD < & %%%
\DIFdelFL{-7.2 }%DIFDELCMD < & %%%
\DIFdelFL{\% }%DIFDELCMD < \\
%DIFDELCMD <         %%%
%DIF <  option to use available biogenic or cycled \ce{CO2} for 50€/t & -22.9 & \euro{}/MWh  & -22.2 & \% \\
        \DIFdelFL{option to use available biogenic or cycled \ce{CO2} for 60€/t }%DIFDELCMD < & %%%
\DIFdelFL{-20.4 }%DIFDELCMD < & \euro{}%%%
\DIFdelFL{/MWh  }%DIFDELCMD < & %%%
\DIFdelFL{-19.7 }%DIFDELCMD < & %%%
\DIFdelFL{\% }%DIFDELCMD < \\
%DIFDELCMD <         %%%
\DIFdelFL{option to use available biogenic or cycled \ce{CO2} for 80€/t }%DIFDELCMD < & %%%
\DIFdelFL{-15.2 }%DIFDELCMD < & \euro{}%%%
\DIFdelFL{/MWh  }%DIFDELCMD < & %%%
\DIFdelFL{-14.7 }%DIFDELCMD < & %%%
\DIFdelFL{\% }%DIFDELCMD < \\
%DIFDELCMD <         %%%
\DIFdelFL{option to use available biogenic or cycled \ce{CO2} for 100€/t }%DIFDELCMD < & %%%
\DIFdelFL{-10.0 }%DIFDELCMD < & \euro{}%%%
\DIFdelFL{/MWh  }%DIFDELCMD < & %%%
\DIFdelFL{-9.7 }%DIFDELCMD < & %%%
\DIFdelFL{\% }%DIFDELCMD < \\
%DIFDELCMD <         %%%
\DIFdelFL{option to build geological hydrogen storage at 2.25 }%DIFDELCMD < \euro{}%%%
\DIFdelFL{/kWh (reduction by 95\%) }%DIFDELCMD < & %%%
\DIFdelFL{-7.7 }%DIFDELCMD < & \euro{}%%%
\DIFdelFL{/MWh  }%DIFDELCMD < & %%%
\DIFdelFL{-7.4 }%DIFDELCMD < & %%%
\DIFdelFL{\% }%DIFDELCMD < \\
%DIFDELCMD <         %%%
\DIFdelFL{option to use power-to-X waste heat streams for direct air capture }%DIFDELCMD < & %%%
\DIFdelFL{-3.6 }%DIFDELCMD < & \euro{}%%%
\DIFdelFL{/MWh  }%DIFDELCMD < & %%%
\DIFdelFL{-3.4 }%DIFDELCMD < & %%%
\DIFdelFL{\% }%DIFDELCMD < \\
%DIFDELCMD <         %%%
\DIFdelFL{highly flexible operation of fuel synthesis plant (20\% minimum part-load instead of 70\%) }%DIFDELCMD < & %%%
\DIFdelFL{-5.1 }%DIFDELCMD < & \euro{}%%%
\DIFdelFL{/MWh  }%DIFDELCMD < & %%%
\DIFdelFL{-4.9 }%DIFDELCMD < & %%%
\DIFdelFL{\% }%DIFDELCMD < \\
%DIFDELCMD <         \bottomrule
%DIFDELCMD <     \end{tabular}
%DIFDELCMD <     %%%
%DIFDELCMD < \caption{%
{%DIFAUXCMD
\textbf{\DIFdelFL{Examples for potential import cost increases or decreases.}}
    %DIFAUXCMD
\DIFdelFL{The table presents cost sensitivities in absolute and relative terms based
    on the supply chain for producing Fischer-Tropsch fuels in Argentina for
    export to Europe. The reference fuel import cost for this case is 103.5
    }%DIFDELCMD < \euro{}%%%
\DIFdelFL{/MWh. Responses to changes in the input assumptions may not
    necessarily be additive.}}
    %DIFAUXCMD
%DIFDELCMD < \label{tab:cost-uncertainty}
%DIFDELCMD < \end{table*}
%DIFDELCMD < 

%DIFDELCMD < %%%
\DIFdelend \section*{Discussion and conclusions}
\label{sec:discussion}

%DIF <  TODO need to expand dicussion?!
\DIFdelbegin %DIFDELCMD < 

%DIFDELCMD < %%%
\DIFdelend Our analysis offers insights into how energy imports might interact with
European energy infrastructures and what \DIFdelbegin \DIFdel{economic benefit they can bring}\DIFdelend \DIFaddbegin \DIFadd{cost reductions they can yield}\DIFaddend . Our
results show that imports of green energy reduce costs of a carbon-neutral
European energy system by \DIFaddbegin \bneuro{39} \DIFadd{(}\DIFaddend 5\%\DIFdelbegin \DIFdel{(}%DIFDELCMD < \bneuro{37}%%%
\DIFdelend ), noting, however, that the
uncertainty range is considerable\DIFdelbegin \DIFdel{and potential cost savings strongly depend on the
assumptions made. Nevertheless, our }\DIFdelend \DIFaddbegin \DIFadd{. While we find that some imports are robustly
beneficial, system cost savings range between 1\% and 14\% depending on the
import costs. What is consistent, however, are the diminishing return of energy
imports for larger quantities and peak cost savings below imports of 4000~TWh.
We also find that there is value in pursuing some \mbox{power-to-X} production
in Europe as a source of flexibility for wind and solar integration and due to
the presence of district heating networks as offtaker for waste heat. Another
location factor in favour of European \mbox{power-to-X} is the availability of
biogenic and industrial carbon sources.
}

\DIFadd{Overall, we find that which import vectors are used strongly affect domestic
infrastructure needs; for instance, that only a much smaller hydrogen network
would be required if hydrogen derivatives were largely imported. This
underscores the importance of coordination between energy import strategies and
infrastructure policy decisions. Our }\DIFaddend results present a quantitative basis for
further discussions about the trade-offs between system cost, carbon neutrality,
public acceptance, \DIFdelbegin \DIFdel{and energy securitypertaining to the import of low-carbon
fuels and assess the extent to which infrastructure policy decisions depend on
the path taken on energy }\DIFdelend \DIFaddbegin \DIFadd{energy security, infrastructure buildout and }\DIFaddend imports.

This is particularly relevant as factors other than pure costs might drive the
\DIFdelbegin \DIFdel{import strategy. Given the }\DIFdelend \DIFaddbegin \DIFadd{designs of import strategies. The }\DIFaddend relatively limited cost benefit of imports and
value chain reordering, \DIFdelbegin \DIFdel{there could be a case }\DIFdelend \DIFaddbegin \DIFadd{may speak }\DIFaddend against pursuing this avenue. \DIFdelbegin \DIFdel{Geopolitical considerations could shift the focus towards energy sovereignty and
independence through self-sufficient supply , but also towards imports to
strengthen trade relations and diversify energy supply chains.
Low reliance on imports from few countries and
rigid infrastructures like pipelines could
}\DIFdelend \DIFaddbegin \DIFadd{A desire for
energy sovereignty would motivate more domestic supply and diversified imports.
Thereby, focusing on ship-bourne imports would reduce pipeline lock-in and
}\DIFaddend mitigate the risks of sudden supply disruptions\DIFdelbegin \DIFdel{and enhance the resilience of
the energy supply. An important consideration is also }\DIFdelend \DIFaddbegin \DIFadd{. Focussing on carriers that are
already a globally traded commodity may also be more appealing. Besides
questions of resilience, a further consideration is }\DIFaddend the economic outflow caused
by imports, shifting value creation to other parts of the world with potential
repercussions on local jobs. \DIFdelbegin \DIFdel{Furthermore, while we assessed the import
and relocation of steel production, we did not consider }\DIFdelend \DIFaddbegin \DIFadd{Therefore, }\DIFaddend the import of \DIFdelbegin \DIFdel{intermediate }\DIFdelend \DIFaddbegin \DIFadd{intermediary raw }\DIFaddend products
like sponge iron, which \DIFdelbegin \DIFdel{has undergone }\DIFdelend \DIFaddbegin \DIFadd{represents }\DIFaddend the most energy-intensive part of the \DIFdelbegin \DIFdel{value chain and simultaneously presents a good that
is transportable at low cost}\DIFdelend \DIFaddbegin \DIFadd{steel
value chain, could be another relevant option}\DIFaddend .

%DIF <  \cite{hauserDoesMore2021,eickeGreenHydrogen2022a}
\DIFdelbegin %DIFDELCMD < 

%DIFDELCMD < %%%
\DIFdelend There is also a social dimension to the import strategy and the question of how
fast the aspired infrastructure can get built. Therefore, policymaking might
prefer easy-to-implement systems featuring, for instance, lower domestic
infrastructure requirements, reuse of existing infrastructure, lower technology
risk, and reduced land usage for broader public support than the most
cost-effective solution. Ultimately, Europe's energy strategy must balance cost
savings from green energy and material imports with broader concerns like
geopolitics, economic development and public opinion to ensure a swift, secure
and sustainable energy future.


\section*{Methods}

\label{sec:methods}



\subsection*{Modelling of the European energy system}

For our analysis, we use the European sector-coupled high-resolution energy
system model PyPSA-Eur\cite{horschPyPSAEurOpen2018a} based on the open-source
modelling framework PyPSA\cite{brownPyPSAPython2018} (Python for Power System
Analysis) in a setup similar to Neumann et al.~\cite{neumannPotentialRole2023},
covering the energy demands of all sectors including electricity, heat,
transport, industry, agriculture, as well as international shipping and
aviation.

The model simultaneously optimises spatially explicit investments and the
operation of generation, storage, conversion and transmission assets to minimise
total system costs in a linear optimisation problem, which is solved with
\textit{Gurobi}.\cite{gurobi} The capacity expansion is based on technology cost
and efficiency projections for the year 2030, \DIFdelbegin \DIFdel{most }\DIFdelend \DIFaddbegin \DIFadd{many }\DIFaddend of which are taken from the
technology catalogue of the Danish Energy Agency.\cite{DEA} \DIFaddbegin \DIFadd{Choosing projections
for the year 2030 for a net-zero carbon emission scenarios more likely to be
reached by mid-century acknowledges that much of the required infrastructure
must be constructed well in advance of reaching this goal. }\DIFaddend Spatially, the model
resolves 110 regions in Europe,\cite{frysztackiStrongEffect2021} covering the
European Union, the United Kingdom, Norway, Switzerland and the Balkan countries
without Malta and Cyprus. Temporally, the model is solved with an uninterrupted
4-hourly equivalent resolution for the weather year 2013, using a segmentation
clustering approach implemented in the \textit{tsam}
toolbox.\cite{hoffmannParetooptimalTemporal2022} In terms of investment periods,
no pathway optimisation is conducted, but a greenfield approach is pursued
except for existing hydro-electricity and transmission infrastructure targeting
net-zero CO$_2$ emissions.

Networks are considered for electricity, methane and hydrogen
transport.\cite{ENTSOE,plutaSciGRIDGas2022a} However, other than in Neumann et
al.,\cite{neumannPotentialRole2023} pipeline retrofitting to hydrogen is
disabled for computational reasons. Data on the gas transmission system is
further supplemented by the locations of fossil gas extraction sites and gas
storage facilities based on SciGRID\_gas,\cite{plutaSciGRIDGas2022a} as well as
investment costs and capacities of existing and planned LNG
terminals\cite{instituteforenergyeconomicsandfinancialanalysisEuropeanLNG2023}
Moreover, \DIFdelbegin \DIFdel{no option for }\DIFdelend a carbon dioxide network is \DIFdelbegin \DIFdel{included }\DIFdelend \DIFaddbegin \DIFadd{not explicitly co-optimised }\DIFaddend since as CO$_2$ is
not spatially resolved in this model version.\cite{hofmannDesigningCO22023}

The overall annual sequestration of CO$_2$ is limited to 200
Mt$_{\text{CO}_2}$/a. This number allows for sequestering the industry's
unabated fossil emissions (e.g. in the cement industry) while minimising
reliance on carbon removal technologies. The carbon management features of the
model trace the carbon cycles through various conversion stages: industrial
emissions, biomass and gas combustion, carbon capture, storage or long-term
sequestration, direct air capture, electrofuels, recycling, and waste-to-energy
plants.

Renewable potentials and time series for wind and solar electricity generation
are calculated with \textit{atlite},\cite{hofmannAtliteLightweight2021}
considering land eligibility constraints like nature reserves or distance
criteria to settlements. Given low onshore wind expansion in many European
countries in recent years,\cite{ourworldindataInstalledWind2023} restrictive
onshore wind expansion potentials are applied, using a 1.5 MW/km$^2$ factor for
the eligible land area. Geological potentials for hydrogen storage are taken
from Caglayan et al.\cite{caglayanTechnicalPotential2020} Biomass potentials are
restricted to residues from agriculture and forestry, as well as waste and
manure, based on the medium potentials specified for 2030 in the JRC-ENSPRESO
database.\cite{ruizENSPRESOOpen2019} The finite biomass resource can be employed
for low-temperature heat provision in industrial applications, biomass boilers
and CHPs, and biofuel production for use in aviation, shipping and the chemicals
industry. Additionally, we allow biogas upgrading, including the capture of the
CO$_2$ contained in biogas. \DIFaddbegin \DIFadd{The total assumed bioenergy potentials are 1569~TWh
with a carbon content corresponding to 546~Mt$_{\text{CO}_2}$/a.
}\DIFaddend 

Heating supply technologies like heat pumps, electric boilers, gas boilers, and
combined heat and power (CHP) plants are endogenously optimised separately for
decentral use and central district heating. District heating networks can
further be supplemented with waste heat from various power-to-X processes
(electrolysis, methanation, methanolisation, ammonia synthesis, Fischer-Tropsch
fuel synthesis).

While the shipping sector is assumed to use methanol as fuel, land-based
transport, including heavy-duty vehicles, is deemed fully electrified in the
presented scenarios. Aviation can decide to use green kerosene derived from
Fischer-Tropsch fuels or methanol. Besides methanol-to-kerosene, \DIFdelbegin \DIFdel{which was not
chosen over steam cracking of Fischer-Tropsch fuels in our results, }\DIFdelend further
usage options for methanol have been added.
These include
methanol-to-olefins/aromatics for the production of green plastics,
methanol-to-power\DIFdelbegin \DIFdel{\mbox{%DIFAUXCMD
\cite{brownUltralongdurationEnergy2023} }\hskip0pt%DIFAUXCMD
}\DIFdelend \DIFaddbegin \DIFadd{\mbox{%DIFAUXCMD
\cite{brownUltralongdurationEnergyStorage2023} }\hskip0pt%DIFAUXCMD
}\DIFaddend in open-cycle gas
turbines or Allam cycle turbines, and steam reforming of methanol with or
without carbon capture. For the synthesis of electrofuels, we also account for
potential operational restrictions by considering a minimum part load of 30\%
for methanolisation and methanation compared to 70\% for Fischer-Tropsch
synthesis, both within Europe and abroad.

A further core improvement of the model regards the physical representation of
energy transport over long distances. For gas and hydrogen pipelines, we
incorporate electricity demands for compression \DIFdelbegin \DIFdel{in the order of 1-2}\DIFdelend \DIFaddbegin \DIFadd{of 1\% and 2}\DIFaddend \% per 1000km
of the transported energy\DIFaddbegin \DIFadd{, respectively}\DIFaddend .\cite{gasforclimateEuropeanHydrogen2021}
For HVDC transmission lines, we assume 2\% static losses at the substations and
additional losses of 3\% per 1000km. The losses of high-voltage AC transmission
lines are estimated using a piecewise linear approximation as proposed in
Neumann et al.,\cite{neumannAssessmentsLinear2022} in addition to the linearised
power flow equations.\cite{horschLinearOptimal2018} Up to a maximum capacity
increase of 30\%, we consider dynamic line rating (DLR), leveraging the cooling
effect of wind and low ambient temperatures to exploit existing transmission
assets fully.\cite{glaumLeveragingExisting2023} To approximate N-1 resilience,
transmission lines may only be used up to 70\% of their rated dynamic capacity.
To prevent excessive expansion of single connections, the expansion of power
transmission lines between two regions is limited to 15 GW for HVAC and 25 GW
for HVDC lines, while a similar constraint of 50.7 GW is placed on hydrogen
pipelines, which corresponds to three parallel 48-inch
pipelines.\cite{gasforclimateEuropeanHydrogen2021}

Finally, we also developed the possibility for the model to relocate the steel
and ammonia industry within Europe, mainly to level the playing field between
non-European green steel imports and domestic production. This is achieved by
explicitly modelling the cost, efficiency and operation of hydrogen direct iron
reduction (H2-DRI) and electric arc furnaces (EAF), which can be sited all over
Europe, and the cost to procure iron ore. We further allow the oversizing of
steelmaking plants to allow flexible production in response to the renewables
supply conditions.

\subsection*{Modelling of import supply chains and costs}

\begin{figure*}
    \centering
    \DIFdelbeginFL %DIFDELCMD < \includegraphics[width=.82\textwidth]{static/graphics/sketch2.drawio-1.pdf}
%DIFDELCMD <     %%%
\DIFdelendFL \DIFaddbeginFL \includegraphics[width=.82\textwidth]{static/graphics/sketch2.drawio-2.pdf}
    \DIFaddendFL \caption{\textbf{Schematic overview of the import supply chains.} The
    illustration includes key input-output ratios of the different conversion
    processes and the transport efficiencies for the different import vectors.}
    \label{fig:import-esc-scheme}
%DIF <  TODO not all colors for right-hand side are correct
\end{figure*}

The European energy system model is extended with data from the TRACE model
\cite{hamppImportOptions2023} to assess the costs of different vectors for
importing green energy and material into Europe from various world regions. For
each vector, we identify locations with existing or planned import
infrastructure where the respective carrier may enter the European energy
system.

Starting from the methodology by Hampp et al.\cite{hamppImportOptions2023}, some
adjustments were made to the original TRACE model. Namely, land availability and
wind and solar time-series are determined using
\textit{atlite}\cite{hofmannAtliteLightweight2021} instead of
\textit{GEGIS},\cite{mattssonAutopilotEnergy2021}. Techno-economic assumptions
were aligned with those used in the European model, and steel was included as an
energy-intensive material import vector. The exporting countries comprise
Australia, Argentina, Chile, Kazakhstan, Namibia, Turkey, Ukraine, the Eastern
United States and Canada, mainland China, and counties in the MENA region.

To determine the levelised cost of energy for exports, the methodology first
assesses the regional potentials for wind and solar energy. A regional
electricity cost supply curve is determined, from which projected future energy
demand is subtracted. Thereby, domestic consumption is prioritised and supplied
by the countries' best renewable resources \DIFaddbegin \DIFadd{even though we do not model the
energy transition in exporting countries in detail}\DIFaddend . The remaining wind and solar
electricity supply can then be used to produce the specific energy or material
vector using water electrolysis for \ce{H2}, direct air capture (DAC) for
\ce{CO2}, air separation units (ASU) for \ce{N2}, synthesis of methane,
methanol, ammonia or Fischer-Tropsch fuels from \ce{H2} with \ce{CO2} or
\ce{N2}, and \ce{H2} direct iron reduction (DRI) with subsequent processing in
electric arc furnaces (EAF) for the processing of iron ore (\DIFdelbegin \DIFdel{97.73 }\DIFdelend \DIFaddbegin \DIFadd{103.7 }\DIFaddend \euro{}/t)
into green steel. Other CO$_2$ sources than DAC are not considered in the
exporting countries, a notable difference from the European model. Liquid
organic hydrogen carriers (LOHC) are not considered \DIFdelbegin \DIFdel{an export vector . }\DIFdelend \DIFaddbegin \DIFadd{as export vector due to
their lower technology readiness level (TRL) compared to other
vectors.\mbox{%DIFAUXCMD
\cite{irenaGlobalHydrogen2022} }\hskip0pt%DIFAUXCMD
}\DIFaddend Further details on the energy and
feedstock flow and process efficiencies are outlined in
\cref{fig:import-esc-scheme}.

For each vector, an annual reference import demand of 500~TWh (lower heating
value, LHV) or 100~Mt of steel is assumed, mirroring large-scale energy and
material infrastructures and export volumes, corresponding to approximately 40\%
of current LNG
imports\cite{instituteforenergyeconomicsandfinancialanalysisEuropeanLNG2023} and
66\% of European steel
production.\cite{eurofer-theeuropeansteelassociationEuropeanSteel2023}

%DIF <  capacity expansion
%DIF > %% capacity expansion %%%

Based on these supply chain definitions, a capacity expansion optimisation is
performed to determine the cost-optimal combination of infrastructure and
process capacities for all intermediary products and delivering the final
carrier either through pipelines (\DIFdelbegin \DIFdel{\ce{H2_{(g)}}, \ce{CH4_{(g)}}}\DIFdelend \DIFaddbegin \DIFadd{\ce{H2(g)}, \ce{CH4(g)}}\DIFaddend ) or by ship
(\DIFdelbegin \DIFdel{\ce{H2_{(l)}}, \ce{CH4_{(l)}}, \ce{NH3_{(l)}}}\DIFdelend \DIFaddbegin \DIFadd{\ce{H2(l)}, \ce{CH4(l)}, \ce{NH3(l)}}\DIFaddend , \ce{MeOH}, Fischer-Tropsch fuel,
and steel). Exports from each of the regions shown in \cref{fig:options:global}
are modelled to each of twelve European import locations based on large port
locations, determining the levelised costs of energy or steel the European entry
point will see for each supply chain. All energy supply chains are assumed to
consume their energy vector as fuel for transport to Europe, except for steel,
which uses externally bought methanol as shipping fuel.

%DIF <  import constraints in European model
%DIF > %% import constraints in European model %%%

The resulting levelised cost of exported energy specific to the respective
importing regions is added as a constant marginal \DIFdelbegin \DIFdel{fuel }\DIFdelend import cost for all
chemical energy carriers and steel. For the import of hydrogen and methane,
candidate entry points are identified based on where existing and prospective
LNG terminals and cross-continental pipelines are located. This includes new LNG
import terminals in Europe in response to ambitions to phase out Russian gas
supply in 2022.
\cite{instituteforenergyeconomicsandfinancialanalysisEuropeanLNG2023} To achieve
regional diversity in potential gas and hydrogen imports and avoid vulnerable
singular import locations, we allow the expansion beyond the reported capacities
only up to a factor of 2.5, taking the median value of reported investment costs
for LNG terminals.\cite{GlobalGas2022} A surcharge of 20\% is added for hydrogen
import terminals due to the lack of practical experience. Carbonaceous fuels,
ammonia, and steel imports are not spatially resolved due to their low transport
costs and, therefore, are not constrained by the availability of suitable entry
points. To present energy and material imports in a common unit, the embodied
energy in steel is approximated with the 2.1 kWh/kg released in iron oxide
reduction, i.e.~energy released by combustion.\cite{kuhnIronRecyclable2022}

%DIF <  special handling of electricity imports
%DIF > %% special handling of electricity imports %%%

Owing to the variability of wind and solar electricity, the supply chain of
electricity imports is endogenously optimised with the rest of the European
system rather than using a constant levelised cost of exported electricity. This
comprises the optimisation of wind and solar capacities, batteries and hydrogen
storage in steel tanks, and the size and operation of HVDC link connection into
Europe based on the availability time series in neighbouring countries as
\DIFdelbegin \DIFdel{presented }\DIFdelend \DIFaddbegin \DIFadd{illustrated }\DIFaddend in \cref{fig:options:europe}. Underground hydrogen storage options
are not considered due to the limited availability of salt caverns in many of
the best renewable resource regions in the countries that are considered
exporting.\cite{hevinUndergroundStorage2019} We also assume that the energy
supply chains dedicated to exports will be islanded from the rest of the local
energy system. Europe's connection options with exporting countries are confined
to the 5\% nearest regions, with additional ultra-long distance connection
options to Ireland, Cornwall and Brittany following the vision of the Xlinks
project between Morocco and the United Kingdom.\cite{xlinksMoroccoUKPower2023}
Connections through Russia or Belarus are excluded, and thus, some connections
are affected by additional detours beyond the regularly applied \DIFdelbegin \DIFdel{distance }\DIFdelend \DIFaddbegin \DIFadd{detour }\DIFaddend factor of
125\% of the \DIFdelbegin \DIFdel{crow-fly distance. Transmission losses of }\DIFdelend \DIFaddbegin \DIFadd{as-the-crow-flies distance. Similar to intra-European HVDC
transmission, a }\DIFaddend 3\% \DIFaddbegin \DIFadd{loss }\DIFaddend per 1000km \DIFdelbegin \DIFdel{are
applied}\DIFdelend \DIFaddbegin \DIFadd{and a 2\% converter station loss are assumed}\DIFaddend .

%DIF <  more detailed results
%DIF > %% more detailed results %%%

As illustrated in \cref{fig:options}, for imports of hydrogen by pipeline,
nearby countries like Algeria and Egypt emerged as lowest cost exporters (ca.~\DIFdelbegin \DIFdel{54
}\DIFdelend \DIFaddbegin \DIFadd{57
}\DIFaddend \euro{}/MWh). Importing hydrogen by ship is substantially more expensive due to
liquefaction and evaporation losses. Algeria could offer supply through this
vector at \DIFdelbegin \DIFdel{79 }\DIFdelend \DIFaddbegin \DIFadd{84 }\DIFaddend \euro{}/MWh. For all other hydrogen derivatives, Argentina and
Chile offer the lowest cost imports between \DIFdelbegin \DIFdel{83 and 104}\DIFdelend \DIFaddbegin \DIFadd{88 and 110}\DIFaddend ~\euro{}/MWh or
\DIFdelbegin \DIFdel{472}\DIFdelend \DIFaddbegin \DIFadd{501}\DIFaddend ~\euro{}/t for steel. Methanol is found to be cheaper than the
Fischer-Tropsch route because it is assumed to be more flexible (30\% minimum
part load compared to 70\% for
Fischer-Tropsch).\cite{brownUltralongdurationEnergy2023} The lower process
flexibility shifts the energy mix towards solar electricity and causes higher
levels of curtailment, increasing costs. The transport costs of \DIFdelbegin \DIFdel{\ce{CH4_{(l)}}
}\DIFdelend \DIFaddbegin \DIFadd{\ce{CH4(l)} }\DIFaddend are
lower than for \DIFdelbegin \DIFdel{\ce{H2_{(l)}} }\DIFdelend \DIFaddbegin \DIFadd{\ce{H2(l)} }\DIFaddend since the liquefaction consumes less energy and
individual ships can carry more energy with \DIFdelbegin \DIFdel{\ce{CH4_{(l)}}}\DIFdelend \DIFaddbegin \DIFadd{\ce{CH4(l)}}\DIFaddend . Pipeline imports of
\DIFdelbegin \DIFdel{\ce{CH4_{(g)}} }\DIFdelend \DIFaddbegin \DIFadd{\ce{CH4(g)} }\DIFaddend were also considered, but costs were higher than for \DIFdelbegin \DIFdel{\ce{CH4_{(l)}} }\DIFdelend \DIFaddbegin \DIFadd{\ce{CH4(l)}
}\DIFaddend shipping under the assumption that new pipelines would have to be built.
\DIFaddbegin \DIFadd{Consequently, the model preferred LNG imports over pipeline imports.
}\DIFaddend 

\end{document}
