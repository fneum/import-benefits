\documentclass[10pt,5p,reversenotenum,lefttitle]{elsarticle}

\journal{International Energy Workshop 2023}

\bibliographystyle{elsarticle-num}
\biboptions{numbers,sort&compress,super}

\usepackage{libertine}
\usepackage{libertinust1math}
% \usepackage{geometry} \geometry{ top=30mm, bottom=35mm,
% }

\renewcommand{\ttdefault}{\sfdefault}

\usepackage{amsmath}
\usepackage{graphicx}
\usepackage{eurosym}
\usepackage{url}
\usepackage{booktabs}
\usepackage{epstopdf}
\usepackage[colorlinks]{hyperref}
\usepackage[nameinlink,sort&compress,capitalise,noabbrev]{cleveref}
% \usepackage[parfill]{parskip}

\graphicspath{ {static/graphics/}, {../results/}, {../workflow/results/v8/},
    {../workflow/subworkflows/pypsa-eur-sec/results/v8/maps/} }

\makeatletter
\long\def\MaketitleBox{%
  \resetTitleCounters \def\baselinestretch{1}%
   \def\baselinestretch{1}%
    \Large\@title\par\vskip18pt \normalsize\elsauthors\par\vskip10pt
    \footnotesize\itshape\elsaddress\par\vskip36pt }
\makeatother


\begin{document}

\begin{frontmatter}

	\title{Energy Imports and Infrastructure in a Climate-Neutral European Energy System}
    
	\author[jlu]{Johannes Hampp}
    \author[tub]{Tom Brown}
	\author[tub]{Fabian Neumann}
	\ead{f.neumann@tu-berlin.de}
	\address[jlu]{Center for International Development and Environmental Research, Justus-Liebig-University Gießen, Gießen, Germany}
	\address[tub]{Department of Digital Transformation in Energy Systems, Institute of Energy Technology, Technische Universität Berlin, Fakultät III, Einsteinufer 25 (TA 8), 10587 Berlin, Germany}

\end{frontmatter}

\section*{Long Abstract -- Introduction}

The transformation of the European energy system towards climate-neutrality
demands unrivalled technological change. Whereas the development of renewables
energy sources in Europe and supporting measures like reinforcing the
electricity grid do not always meet the level of acceptance required for a swift
transition, other parts of the world have cheap and abundant renewable energy
supply potentials to offer to global energy
markets.\cite{irenaGlobalHydrogen2022,luxSupplyCurves2021,vanderzwaanTimmermansDream2021,fasihiLongTermHydrocarbon2017,reichenbergDeepDecarbonization2022,galvanExportingSunshine2022,armijoFlexibleProduction2020,pfennigGlobalGISbased2022}
They could become key partners for a cost-effective and socially accepted energy
transition in Europe.

However, even if they are economically attractive, a strong dependence on energy
imports can be detrimental to energy security, as Europe is currently
experiencing owing to its reliance on Russian fossil
energy.\cite{pedersenLongtermImplications2022} Energy imports and associated
infrastructure might tie European energy supply to a small number of exporters
or markets who then have market power. These risks must be weighed against the
potential benefits of decreasing energy supply costs, reducing land usage and
increasing energy security by supplying storable fuels that can mitigate energy
droughts for systems with high shares weather-dependent energy supply.

In this contribution, we explore the full range between the two poles of full
self-sufficiency and wide-ranging energy imports into Europe in scenarios with
high shares of wind and solar electricity and net-zero carbon emissions. We
investigate how the infrastructure requirements of a self-sufficent European
energy system that exclusively leverages local resources from the continent may
differ from a system that relies on energy imports from outside of Europe. For
our analysis, we integrate a model of global energy supply chains by Hampp et
al.\cite{hamppImportOptions2023} with a spatially and temporally resolved
sector-coupled energy system model,
PyPSA-Eur-Sec,\cite{PyPSAEurSecSectorCoupled} to investigate the impact of
imports on European energy infrastructure needs. We evaluate potential import
locations and carriers, the economic impetus for such imports, and how their
inclusion affects deployed transport networks and storage. For this purpose, we
perform sensitivity analyses interpolating between very high levels of imports
and no imports at all.

\section*{How do import choices affect infrastructure needs?}

What infrastructure is needed to support the system depends on the levels of
clean energy imported. This relates to the geographic distribution of energy
resources tapped as well as directions and magnitudes of energy flows which have
to be supported.

Today, European energy infrastructure is built around the imports of fossil oil
and gas. The transition to a system that exploits the best wind and solar sites
across the continent would offer many ways to develop a self-sufficient system
without imports.\cite{pickeringDiversityOptions2022,brownSynergiesSector2018}
Developing transmission infrastructure, like reinforcing the power grid and
building a hydrogen network that partially repurposes an increasingly unused gas
network is consistently beneficial in such autarkic
scenarios.\cite{neumannBenefitsHydrogen2022a,wetzelGreenEnergy2022,victoriaSpeedTechnological2022}
This aligns with the vision of a \textit{European Hydrogen Backbone} of the
European gas industry.
\cite{gasforclimateEuropeanHydrogen2020,gasforclimateEuropeanHydrogen2022}

However, scaling up Europe's energy transmission infrastructure may not be
necessary if imports of renewable energy carriers are considered. Since most
hydrogen would be used to produce synthetic fuels (e.g.~for high-value
chemicals), steel and ammonia,\cite{neumannBenefitsHydrogen2022a} if these
hydrogen derivatives were imported at scale, much of the hydrogen demand would
fall away. This would reduce the need for hydrogen transport infrastructure.
Even if there is high demand for direct hydrogen imports, the optimised topology
of a hydrogen network might differ significantly as new import locations need to
be connected rather than domestic production. The network's role could change
from distributing energy from North Sea hydrogen production hubs to
incorporating inbound hydrogen pipelines from North Africa.

\begin{figure*}[ht]
    \centering
    \includegraphics[width=.8\textwidth]{elec_s_128_lv2.0__Co2L0-3H-T-H-B-I-A-solar+p3-linemaxext15-imp_2030/import-option}
    \caption{Overview of model setup comprising spatial resolution, electricity network topology, electricity import options, regional hydrogen import costs, and the distribution of LNG terminals and pipeline entry-points.}
    \label{fig:import-option}
\end{figure*}

\section*{Which energy carriers should be imported?}

As possible import options we consider imports of electricity, hydrogen,
methane, ammonia and Fischer-Tropsch fuels. Thereby, each carrier has different
characteristics which leads to trade-offs regarding how and where they may be
imported.

Electricity, the most versatile carrier, is challenging to store and requires
variability management if directly sourced from renewable sources. Hydrogen is
easier to store and transport in large quantities than electricity but at the
expense of conversion losses and less versatile usage. Hydrogen has also
attracted considerable interest with plans of the European Commission under
\mbox{REPowerEU}\cite{europeancommissionRepowerEUPlan} to import 10~Mt (333~TWh)
hydrogen and derivatives by 2030, alongside equal amounts of domestic hydrogen
production. Furthermore, green hydrogen could offer a replacement for hydrogen
from fossil sources as a chemical feedstock in the future. Synthetic, green
methane could benefit from existing infrastructure but requires a sustainable
carbon source and leakage prevention. Ammonia does not require a carbon source
and is simple and cheap to store and transport over long distances. However, it
suffers from acceptance problems due to its toxicity and lower energy density.
Finally, synthetic green Fischer-Tropsch fuels and other liquid hydrocarbons are
easy to store, transport and reuse existing infrastructure, but the synthesis is
energy-intensive due to high conversion losses. Like methane, a sustainable
carbon source is required.

For each energy carrier we identify locations with existing or planned import
infrastructure where the respective carrier may enter the European energy
system. We consider import options for electricity by transmission line,
hydrogen as gas by pipeline and as liquid by ship, methane as gas by pipeline
and as liquid by ship, ammonia as liquid by ship, and Fischer-Tropsch fuels by
ship. Further conversion of imported fuels is also possible once fuels have
arrived in Europe, e.g.~hydrogen can be used for the synthesis of carbon-based
fuels and methane can be converted to hydrogen. Moreover, we compute scenarios
where only a subset of carriers can be imported to probe the flatness of the
near-optimal solution space and assess how the different import scenarios affect
the energy infrastructure inside Europe.

\section*{Sector-coupled energy system at high spatial\\and temporal resolution with PyPSA-Eur-Sec}

To model the European energy system, we use the open energy system
optimisation model PyPSA-Eur-Sec,\cite{PyPSAEurSecSectorCoupled} that combines a
fully sector-coupled approach with high spatial and temporal resolution and
detailed transmission infrastructure representation. The model co-optimises the
investment and operation of generation, storage, conversion and transmission
infrastructures in a single linear optimisation problem. It covers 128
individual regions and uses a 3-hourly time resolution for a full year. With
these settings, the model is detailed enough to capture existing grid
bottlenecks and the variability of renewables and requirements for seasonal
storage. The model includes regional demands from the electricity, industry,
buildings, agriculture and transport sectors, including shipping and aviation as
well as non-energy feedstock demands in the chemicals industry. Furthermore, the
model covers transmission infrastructure for electricity, gas and hydrogen as
well as candidate entry points for energy imports like existing and prospective
LNG terminals and cross-continental pipelines. 

In the scenarios shown below we enforce net-zero emissions for carbon dioxide,
allow a doubling of today's power grid infrastructure, take technology and
assumptions for the year 2030\cite{dea2019}, and limit the carbon sequestration
potential to 200~Mt$_{CO_2}$/a. This is sufficient to sequester unabated fossil process
emissions in industry, but limits the system's reliance on carbon-dioxide
capture and removal technologies. Detailed carbon management is central to the model, as it
tracks the carbon cycles between industrial process emissions, the combustion of
biomass and gas, carbon capture, direct air capture, synfuel production,
sequestration, recycling and waste-to-energy conversion. 

\section*{Import options and costs modelled at each entry-point}

Fuel import costs seen by our energy system model are based on recent research
by Hampp et al.\cite{hamppImportOptions2023}, who assessed the cost importing
energy across different global green energy supply chains for the afromentioned
energy carriers from various regions of the world (\cref{tab:costs}). Hampp et
al.\cite{hamppImportOptions2023} developed regional supply cost curves based on
the countries' renewable resources and land availability. To determine the
levelised cost of energy for exports, any domestic electricity demand was assumed to be
supplied with the countries' cheapest potentials. Other than domestic
electrofuel synthesis in Europe, which could use captured CO$_2$ from point
sources, direct air capture is assumed as the sole carbon source of import
fuels. For hydrogen derivatives, the cheapest suppliers are Argentina and Chile.

We use these supply curves to determine for each energy carrier and model entry
point the region-specific lowest import cost, thus, incoporating the potential
trade-off between import cost and import location (\cref{fig:import-option}).
Our selection of exporting countries comprises Australia, Argentina, Chile,
Kazakhstan, Turkey, Ukraine, the Eastern United States and Canada, Western and
Eastern mainland China, and the MENA region. Electricity imports are
endogenously optimised, meaning that the capacities and operation of wind and
solar generation as well as storage in the respective exporting countries and
the HVDC transmission lines to Europe (with losses) are co-planned with the rest
of the system. Hydrogen and methane can be imported where there are existing or
planned LNG terminals or pipeline entry-points. As \cref{fig:import-option}
shows, the resulting hydrogen import costs are much lower in Southern and
Eastern Europe where it can be imported via pipeline \mbox{($\geq$ 55\euro/MWh)}
rather than by ship \mbox{($\geq$ 83\euro/MWh)}. The imports of ammonia and liquid hydrocarbons are not
spatially resolved, assuming they can be transported within Europe at low cost.

\begin{table}
    \centering\small
    \begin{tabular}[pos]{lrr}
        \toprule
        Carrier & Import Costs [\euro/MWh] \\
        \midrule
        Electricity & 32-57 \\
        Hydrogen & 55-88 \\
        Ammonia & 85 \\ 
        Methane (LNG) & 88-90 \\
        Methane (pipeline) & 100 \\
        Fischer-Tropsch fuel & 115 \\
        \bottomrule
    \end{tabular}
    \caption{Fuel import costs depending on their import location. Import costs
    of methane and hydrogen are location-dependent as shown for H$_2$ in
    \cref{fig:import-option} contingent on pipeline or ship-based import
    possibilities. The supply chain for electricity imports is endogenous
    optimised; range shows time-averaged market prices.}
    \label{tab:costs}
\end{table}

\section*{Preliminary results indicate flat near-optimal solution\\space between autarky and high e-fuel imports}

We find that allowing energy imports from outside of Europe reduces total energy
system costs by up to 7\%. In this case, around 30\% of the system cost would be
spent on energy imports. But as \cref{fig:sensitivity-bars} outlines, the level
of cost reductions depends on the available import options. For now, our
preliminary results indicate a preference for hydrogen and electricity imports.
Half of the cost reductions can be achieved with exclusive hydrogen imports,
whereas 70\% of the reduction (for a total reduction of 5\%) can be achieved
with exclusive electricity imports.

If the carrier mix for energy imports can be flexibly chosen, we find a
cost-optimal import volume of 3750~TWh, which is more than a third of the
system's final energy demand (\cref{fig:sensitivity-import-volume-any}). Out of
these imports, roughly 59\% is imported in the form of electricity. Another 39\%
are imported as hydrogen. Imports of liquid hydrocarbons, ammonia and methane
are marginal. A large share of power-to-hydrogen production is outsourced from
from Europe. As we explore the effect of increasing import volumes on system
costs, we find that half of the 7\% cost benefit can be achieved with imports
below 1000~TWh. This corresponds to only a quarter of cost-optimal import
volumes, highlighting the diminishing return of energy imports over domestic
production in Europe. From this sensitivity analysis we further learn that the
solution space is relatively flat in a wide range between imports of 0 and
8000~TWh. Importing around 7500~TWh (twice as much as cost-optimal) is just as
expensive as the scenario without any imports.

\section*{Preliminary results show drastically altered\\infrastructure buildout with e-fuel imports}

Allowing imports of electricity, green hydrogen and electrofuels changes the
infrastructure buildout in Europe completely (\cref{fig:import-result}). In the
scenario with fully self-sufficient European energy supply, we see large
\mbox{power-to-X} production within Europe to cover the demand for hydrogen
derivatives in steelmaking, high-value chemicals, as well as shipping and
aviation fuels. Production sites are concentrated in Southern Europe for
solar-based electrolysis as well as in the broader North Sea region for
wind-based electrolysis. Electricity grid reinforcements are mostly focused in
Northwestern Europe. However, the import of electricity and hydrogen displaces
much of the European power-to-X production capacities and diverts some of the
electricity grid reinforcements to Southern Europe in order to absorb the
inbound power. The electricity imports distribute evenly between the nearby
exporting countries to facilitate their grid integration and best exploit the
limited grid expansion volume. We see both wind and solar generation for better
seasonal balancing but no storage in the exporting countries, such that power is
directly transfered to the European continent. How increased energy imports
change the role of network infrastructure is also shown for hydrogen transport
in \cref{fig:hydrogen-flows}. With imports, the hydrogen network is built for
transporting imports from North Africa rather than distributing hydrogen from
the North Sea area.

\begin{figure}
    \centering
    \includegraphics[width=\columnwidth]{sensitivity-bars}
    \caption{Total energy system cost and share spent on fuel imports with restricted import carrier options.}
    \label{fig:sensitivity-bars}
\end{figure}

\begin{figure*}
    \centering
    \includegraphics[width=.8\textwidth]{sensitivity-import-volume-any}
    \vspace{-0.3cm}
    \caption{Total energy system cost and system composition with increasing energy imports.}
    \label{fig:sensitivity-import-volume-any}
\end{figure*}

\begin{figure*}
    \centering
    \includegraphics[height=.33\textheight,trim=0cm 0cm 7cm 0cm,clip]{elec_s_128_lv2.0__Co2L0-3H-T-H-B-I-A-solar+p3-linemaxext15_2030/import-result}
    \includegraphics[height=.33\textheight]{elec_s_128_lv2.0__Co2L0-3H-T-H-B-I-A-solar+p3-linemaxext15-imp_2030/import-result}
    \caption{Distribution of power grid reinforcements, synthetic fuel production and import locations for scenario without imports (left) and with imports (right).}
    \label{fig:import-result}
\end{figure*}

\begin{figure*}
    \centering\small
    \begin{tabular}{cc}
        (a) without imports & (b) with imports \\
        \includegraphics[width=.8\columnwidth]{elec_s_128_lv2.0__Co2L0-3H-T-H-B-I-A-solar+p3-linemaxext15_2030/h2-flow-map-backbone}
        &
        \includegraphics[width=.8\columnwidth]{elec_s_128_lv2.0__Co2L0-3H-T-H-B-I-A-solar+p3-linemaxext15-imp_2030/h2-flow-map-backbone}
    \end{tabular}
    \caption{Net energy flows in hydrogen network with and without energy imports. Width proportional to energy flow.}
    \label{fig:hydrogen-flows}
\end{figure*}



\section*{Impact and Outlook}

Our analysis offers insights into how energy imports might interact with
European energy infrastructures and what economic benefit they can bring. Our
preliminary results show that imports of green energy reduce costs of a
climate-neutral European energy system by 7\%, and that European infrastructure
requirements strongly depend on the strategy taken on energy imports. To
understand the dependency of the results on the assumptions, we have planned
more work on running import cost sensitivities, including losses in European
energy networks, and considering industry relocation as well as secondary
material imports like green steel. With our research, we seek to stimulate
further discussions about trade-offs between public acceptance, system cost, and
energy security pertaining to the import of low-carbon fuels and sensitise
policymakers to the extent to which infrastructure policy decisions depend on
the path taken on energy imports. This is particularly relevant as other factors
than pure costs might rather drive import strategy, such as geopolitical
considerations, preferences for easy-to-implement systems, reuse of existing
infrastructure, resilience of supply chains, technology risk, diversification
and land usage.

\addcontentsline{toc}{section}{References}
\renewcommand{\ttdefault}{\sfdefault}
\bibliography{/home/fneum/zotero}

\end{document}