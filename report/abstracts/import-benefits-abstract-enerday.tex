\documentclass[10pt,5p,reversenotenum,lefttitle]{elsarticle}

\journal{ENERDAY 2023}

\usepackage{libertine}
\usepackage{libertinust1math}
\renewcommand{\ttdefault}{\sfdefault}
\usepackage{amsmath}
\usepackage{eurosym}

\makeatletter
\long\def\MaketitleBox{%
  \resetTitleCounters \def\baselinestretch{1}%
   \def\baselinestretch{1}%
    \Large\@title\par\vskip18pt \normalsize\elsauthors\par\vskip10pt
    \footnotesize\itshape\elsaddress\par\vskip36pt }
\makeatother


\begin{document}

\begin{frontmatter}

	\title{Energy Imports and Infrastructure in a Climate-Neutral European Energy System}

	\author[jlu]{Johannes Hampp}
    \author[tub]{Tom Brown}
	\author[tub]{Fabian Neumann}
	\ead{f.neumann@tu-berlin.de}
	\address[jlu]{Center for International Development and Environmental Research, Justus-Liebig-University Gießen, Gießen, Germany}
	\address[tub]{Department of Digital Transformation in Energy Systems, Institute of Energy Technology, Technische Universität Berlin, Fakultät III, Einsteinufer 25 (TA 8), 10587 Berlin, Germany}

\end{frontmatter}

\section*{Motivation}

The transformation of the European energy system towards climate-neutrality
demands unrivalled technological change. Whereas the development of renewables
energy sources in Europe and supporting measures like reinforcing the
electricity grid do not always meet the level of acceptance required for a swift
transition, other parts of the world have cheap and abundant renewable energy
supply potentials to offer to global energy markets. However, even if
economically attractive, a strong dependence on energy imports can be
detrimental to energy security. Risks must be weighed against the potential
benefits of decreasing energy supply costs, reducing land usage and increasing
energy security by supplying storable fuels that can mitigate energy droughts
for systems with high shares weather-dependent energy supply.

Here, we explore the full range between the two poles of full self-sufficiency
and wide-ranging energy imports into Europe in scenarios with high shares of
wind and solar electricity and net-zero carbon emissions. We investigate how the
infrastructure requirements of a self-sufficent European energy system that
exclusively leverages local resources from the continent may differ from a
system that relies on energy imports from outside of Europe. We integrate a
model of global energy supply chains by Hampp et al. (2023) with a
sector-coupled energy system model, PyPSA-Eur-Sec, to investigate the impact of
imports on European energy infrastructure needs. We evaluate potential import
locations and carriers, the economic impetus for such imports, and how their
inclusion affects deployed transport networks and storage. As possible import
options we consider imports of electricity, hydrogen, methane, ammonia,
methanol, steel/iron sponge and Fischer-Tropsch fuels. Moreover, we compute
scenarios where only a subset of carriers can be imported to assess how the
different import scenarios affect the energy infrastructure inside Europe.

\section*{Methods}

We use the open European energy system model PyPSA-Eur-Sec, that combines a
fully sector-coupled approach with high spatial and temporal resolution and
detailed transmission infrastructure representation. The model co-optimises the
investment and operation of generation, storage, conversion and transmission
infrastructures in a single linear optimisation problem. It covers 128
individual regions and uses a 3-hourly time resolution for a year. Thereby, the
model is detailed enough to capture existing grid bottlenecks, the
variability of renewables and seasonal storage. It
includes regional demands from the electricity, industry, buildings, agriculture
and transport sectors, including shipping and aviation as well as non-energy
feedstock demands in the chemicals industry. Furthermore, it covers
transmission infrastructure for electricity, gas and hydrogen as well as
candidate entry points for energy imports like existing and prospective LNG
terminals and cross-continental pipelines.

We enforce net-zero emissions for CO2, allow a
doubling of today's power grid infrastructure, take technology and assumptions
for the year 2030, and limit the carbon sequestration potential to 200Mt CO2
per year. Fuel import costs are based on Hampp et al., who assessed the cost importing various energy carriers across
different global green energy supply chains from various regions of the world.
We use these supply curves to determine for each energy carrier and model entry
point the region-specific lowest import cost, thus, incoporating the potential
trade-off between import cost and import location.
Our selection of exporting countries comprises Australia, Argentina, Chile,
Kazakhstan, Turkey, Ukraine, the United States and Canada, China, and the MENA region. The hydrogen import costs are lower where they can be imported via pipeline
rather than by ship. The imports of ammonia and liquid hydrocarbons are not
spatially resolved. Electricity imports are endogenously optimised.

\section*{Results}

We find that allowing energy imports from outside of Europe reduces total energy
system costs by up to 7\%. In this case, around 30\% of the system cost would be
spent on energy imports. But the cost reductions depend on the available import
options. Our preliminary results indicate that half of the cost reductions can
be achieved with exclusive hydrogen imports, whereas 70\% of the reduction (for
a total reduction of 5\%) can be achieved with exclusive electricity imports.

If the carrier mix for energy imports can be flexibly chosen, we find a
cost-optimal import volume of 3750~TWh (a third total final energy demand).
Roughly 59\% is imported in the form of electricity. Another 39\% are imported
as hydrogen. A large share of power-to-hydrogen production is outsourced from
Europe. We find that half of the 7\% cost benefit can be achieved with imports
below 1000~TWh (a quarter of cost-optimal import volumes). Results also show
that the solution space is flat in a wide range between imports of 0 and
8000~TWh. Importing around 7500~TWh is just as expensive as the scenario without
any imports.

Allowing imports of electricity, green e-fuels alters the infrastructure
buildout in Europe. With fully self-sufficient European energy supply, we see
large PtX production within the European periphery to cover the demand for
hydrogen derivatives in steelmaking, high-value chemicals, as well as shipping
and aviation fuels. Electricity grid reinforcements are mostly focused in
Northwestern Europe. However, the import of electricity and hydrogen displaces
much of the European PtX production capacities and diverts some of the
electricity grid reinforcements to Southern Europe. Increased energy imports
change the hydrogen network's role to transporting imports from North
Africa rather than distributing hydrogen from the North Sea area.

\end{document}
